\documentclass{article}
\title{Week 4 lecture B}
\author{Danny Topete}
\date{August 20, 2025}

\begin{document}
\maketitle

\section{Wed lecture}

\subsection{Profit max with market power mathematically}
\begin{itemize}
  \item Suppose demand Q = 200-0.2P; MC = 200
  \item P = 1000 - 5Q
  \item MR = [demand with twice the slope] \rightarrow{} MR = 1000 - 10Q
  \item Profit-maximizing quantity:

    MR = MC \rightarrow{} 200 = 1000 - 10Q \rightarrow{} Q* = 80
  \item Monopolist price

    Q* = 80 \rightarrow{} demand \rightarrow{} P = 1000 - 5(80) = 600
\end{itemize}

\subsection{Learner Index}
\begin{itemize}
  \item Markup is \% of firm's price above its marginal cost
  \item Under PC, there should be no markups
  \item A monopolist will have MC above
  \item Learner Index is a measurement of firm's markup and market power

    (P-MC)/P = -1/E$_D$, where E$_D$ is elasticity of demand
  \item The amount that a monopolist is able markup their quantity
  \item When price is inelastic (-1 < E$_D$ < 0) then (PC)...
\end{itemize}

\subsection{Response to demand increases}
\begin{itemize}
  \item P=1000-5Q \rightarrow{} P=1200 - 5Q
  \item MR \rightarrow{} MR' = 1200 - 10Q
  \item MC = MR \rightarrow{} 200 = 1200-10Q
  \item Q* = 100
  \item P*=1200 - 5(100) = 700
\end{itemize}

\subsection{Customer price sensitivity}
\begin{itemize}
  \item Firms love elastic demand
  \item Allows them for markups
\end{itemize}

\subsection{Consumer and producer surplus}
\begin{itemize}
  \item Producer surplus is the price they are able to receive under MC
  \item Under PC, there is no long run surplus and barely break even
    \begin{itemize}
      \item CS is all above MC
      \item PS not existent (they barely break even)
    \end{itemize}
  \item Under Monopoly
    \begin{itemize}
      \item CS is above Pm and left of Am slope
      \item PS is below CS and above MC
      \item Deadweight Loss (DWL) right of Qm and above MC

        This is inefficiency in the market. This is bad for society
      \item There are some times where market can not hold multiple firms,
        therefore, this may be the only option

        Especially when fixed costs are huge
      \item It is possible that if only a single firm exists, they can supply
        water to all consumers more efficiently. Meaning it is best to have
        a monopolist
    \end{itemize}
\end{itemize}

\subsection{Deadweight Loss (DWL)}
\begin{itemize}
  \item DWL (efficiency loss) is a reduction in total economic surplus resulting
    from inefficiencies
  \item Last slide, surplus could be all 3 colors under PC, but under monopoly, they
    lose the blue triangle, that's DWL
  \item Oftentimes this term is used in relation to tax inefficiencies
\end{itemize}

\end{document}
