\documentclass{article}

\title{ECON102: Microeconomics}
\author{Danny Topete}
\date{July 28, 2025}

\begin{document}
\maketitle

\section{Lecture 1}

\subsection{Markets and Models}
\begin{itemize}
  \item Markets are defined by the individual product,
    all the brands are equivalent given they sale the same product.
    All products are direct substitutes; like oranges.
\end{itemize}


\subsection{Supply Curve}
\begin{itemize}
  \item It is sloped positively upward
  \item Similar to the demand curve
  \item Choke price is where supply dies; here it would be 1,
    vertical intercept of the supply curve (x = 0; x-axis)
  \item A new technology in supply would shift the supply curve
    downwards (to the right), making it cheaper to produce
\end{itemize}

\subsection{Market equilibrium}
\begin{itemize}
  \item Where supply and demand curves crross
  \item Typically demoted Q*
  \item SOlving for market equilibrium:
    \begin{itemize}
      \item Look for equilibrium (Q*, P*)
      \item See point where two curves intersect
      \item You could set the two equations equal to each other
        (demand = supply) and solve for P
    \end{itemize}
  \item Excess supply is the point where supply is greater
    than demand; this is also known as a surplus.

    When Qd > Qs, there is excess demand,
    Qd demand is over Qs point of supply.
  \item The excess demand is the point where demand is greater
    than supply; this is also known as a shortage.
  \item Excess supply is the point where supply is greater
    than demand; this is also known as a surplus.
\end{itemize}

\subsection{Demand/Supply shifts}
\textbf{Demand Shift}
\begin{itemize}
  \item When there's a new equilibrium, the demand curve shifts
    to the right (upward; more demand) or left (downward; less demand)
\end{itemize}

\textbf{Supply Shift}
\begin{itemize}
  \item Positive supply shift move rightward, (downward; more supply)
  \item P $\downarrow{}$, Q $\uparrow{}$
\end{itemize}

\section*{PollEv}
\begin{itemize}
  \item Qd = 100-200P
  \item Qs' = 200P + 200
  \item What is the equilibrium price and quantity?
    \begin{itemize}
      \item Set Qd = Qs'
      \item 100 - 200P = 200P + 200
      \item 100 - 200 = 400P
      \item P = -0.25
      \item Qd = 100 - 200(-0.25) = 150
      \item Qs' = 200(-0.25) + 200 = 150
    \end{itemize}
\end{itemize}

\subsection{Magnitude of effects}
\begin{itemize}
  \item We can predict P*' and Q*' after a shift in either supply or demand,
    but can we infer the magnitude of the shift?
  \item Large shift \rightarrow{} large change in P* and Q*
  \item Steep curve \rightarrow{} larger change in P, relatively small change in Q
  \item The magnitude (steepness) of the curve matters
\end{itemize}

\textbf{What if we shift both curves?}
\begin{itemize}
  \item Double shifts; supply and demand curves shift at the same time.
  \item Use the same process as before, but now we have to consider
    both shifts at the same time.
  \item Given the example
    \begin{itemize}
      \item The supply shifts leftward (upward; less supply)
      \item Increase in price, quantity decreases
      \item Now the demand curve shifts downward (left; less demand)
      \item It pulls the price up and down, and quantity down
      \item Q $\downarrow{}$, P (??; we don't know, unless we have the equations. price ambiguous)
    \end{itemize}
\end{itemize}

\subsection{Elasticity}
\begin{itemize}
  \item P and Q can have a different sized effects.
  \item Elasticity is the ratio of percentage change in one value
    on the percentage change in another
  \item Specifically, price elasticity of demand (supply) is the percentage change
    in quantity demeand (supplied) resulting from a given change in price
  \item Elasticity $\neq$ slope,

    Elasticity has no units, it is a ratio of two percentages.

    Slopes will change depending on the units, for example.
    change the units from dollars to cents.
  \item If elasticity is low, it is inelastic.
    If elasticity is high, it is elastic.
%  \item Price Elasticity of Demand (PED) is defined as:
%    $$\text{PED} = \frac{\% \Delta Q^D}{\% \Delta P}$$
  \item Price Elasticity of Supply (PES) is defined as:
%    $$\text{PES} = \frac{\% \Delta Q^S}{\% \Delta P}$$  \item Price Elasticity of Demand (PED) is defined as:
    $\frac{\% change in quantity demanded}{\% change in price}$
%    $\frac{\%\Delta Q$^D$}{\%\Delta  P}$
  \item Price Elasticity of Supply (PES) is defined as:
    $\frac{\% change in quantity supplied}{\% change in price}$
%    $\frac{\%\Delta Q$^S$}{\%\Delta  P}$
\end{itemize}

\subsection{Responsiveness to Price Change}
\begin{itemize}
  \item Price Elasticty of Demand
  \item Large price elasticity demand \rightarrow{} consumers are highly responsive to price changes,
    they will respond with large (and opposite) chaned in Qd
  \item Small price elasticity demand \rightarrow{} consumers are not very responsive to price changes,
    they will respond with small (and opposite) changes in Qd
\end{itemize}

\subsection{Responsiveness to Price Change}
\begin{itemize}
  \item Price Elasticity of Supply:
    Large price elasticity of supply \rightarrow{} producers are highly responsive to price changes,
  \item Small price elasticity of supply \rightarrow{} producers are not very responsive to price changes,
    they will respond with small (and opposite) changes in Qs
  \item How should we logically expect elasticity to behave over time?
  \item Demand SR vs LR: elassticity is higher in the long run than in the short run.
    In the short run, consumers may not be able to change their behavior
    as quickly as they can in the long run.
  \item Supply SR vs LR: elasticity is higher in the long run than in the short run.
    In the short run, producers may not be able to change their production
    as quickly as they can in the long run.
  \item Elasticity = \% \Delta Q / \% \Delta P
  \item good pric elastic when its elasticity is greater than 1 in absolute value
  \item good price inelastic when its elastic is less than 1 in abs val
  \item We call good \textbf{unit price elastic} when its elastic equal to 1


  \item good is \textbf{perfectly price elastic} when a change in price
    corresponds with an infinite change in quantity,

    You buy pepsi because coke is \$0.01 more expensive,
  \item a good is \textbf{perfectly price inelastic} when a change in price corresponds with no change in quantity

    Insulain is a good example, you need it regardless of price.
    Weird because if price goes down, you still buy the same amount,
\end{itemize}

\subsection{Linear Demand Elasticity}
\begin{itemize}
  \item For a linear demnd curve,
    portion of the demand curve above the unit elastic point is elastic
    - Qd in this region is very responsive to price changes
  \item Potion below is inelastic,
    Qd in this region is not very responsive to price changes
%  \item E = \frac{\% \Delta Q$^D$}{\% \Delta P} = \frac{\Delta Q$^D$ / Q$^D$}{\Delta P / P}
  \item Perfectly inelastic demand, they buy how ever much they need,
    regardless of price. A vertical \Delta Q / \Delta P line.
  \item Perfectly elastic is \Delta P / \Delta Q = \infty,
    a horizontal line, where refuse to purchase if above a set price
  \item Gas stations understand this,
    they know that if they raise the price of gas, people will go to the next station.
  \item Good elastic when demand is responsive, inelasitc is when its not responsive
\end{itemize}

\subsection{Other forms of Elasticity}
\begin{itemize}
  \item Measure income elasticity of demand
    E$^D_I$ = \%\Delta Q$^D$ / \%\Delta I
    \Rightarrow{} \Delta Q$^D$ / \Delta I \times{} I / Q$^D$
    - Income elasticity of demand
  \item Measure quantity-responsiveness of consuders faced with
    change in the price of something else, we use cross-price elasticity of demand

    E$^{D}_{XY}$ = \%\Delta Q$^D_X$ / \%\Delta P$_Y$
    \Rightarrow{} \Delta Q$^D_X$ / \Delta P$_Y$ \times{} P$_Y$ / Q$^D_X$
  \item Left and right shoes are perfect complements,
    if the price of left shoes goes up, the demand for right shoes goes down.
    Thus selling less right shoes if the price of the left shoe goes up.
\end{itemize}

\newpage

\section*{Wednesday Lecture}

\section{Consumer Preference}
\begin{itemize}
  \item Consumption bundle is a collection of goods and services
    that a consumer can choose to consume.
  \item Assumptions:
    \begin{itemize}
      \item Completeness: consumers can rank all consumption bundles
        from best to worst.
      \item Transitivity: if bundle A is preferred to B, and B is preferred to C,
        then A is preferred to C.
      \item Non-satiation: more is better, consumers prefer more of a good
        to less of it.; 100k hats > 100 hats > 1 hat
    \end{itemize}
  \item Oftentimes, we also assume Inada Condition:
    \begin{itemize}
      \item Diminishing Marginal Returns - As you consume more of a good, the additional satisfaction
        (marginal utility) you get from consuming one more unit of that good
        decreases.

        You can buy 100,000 hats, but the worth of each additional hat is less than the previoous one,
        but still worth above 0. But the value of the 1st item is greater than the 100,000th item.
      \item Boundary Behavior: as you consume more of a good,
        the marginal utility approaches zero, but never reaches it.

        People hate having zero, and they care very little about infinite amounts.
    \end{itemize}
\end{itemize}

\subsection{Utility}
\begin{itemize}
  \item Utility is a measure of consumer satisfaction or happiness
    derived from consuming goods and services.
  \item Utility function is a mathematical (numerical) representation of the relationship
    between consumption bundles and the utility they provide.

    "I like apples twice as much as oranges", we would like to quantify that.

    These equations can take on nearly any mathematical form, it is a formula
\end{itemize}

\subsection{Comparing Utility}
\begin{itemize}
  \item We want to be able to compare consumption bundles and their utility levels,
    along with predicting how consumers will choose between them.
  \item Utility theory, we make $ordinal$ comparisons (not cardinal),
    meaning we can rank consumption bundles based on their utility levels.

    Cardinal utility would be a numerical value vs a ranking items (ordianal; having an order).
\end{itemize}

\subsection{Marginal Utility}
\begin{itemize}
  \item Marginal utility is the additional satisfaction or happiness
    derived from consuming one more unit of a good or service.
\end{itemize}
$$MU_k = \frac{\Delta U_k}{\Delta Q_k}$$

\subsection{Indifference Curves; IC}
\begin{itemize}
  \item Consumer is indifferent between two consumption bundles
    if they provide the same level of utility. (they are equal competitors)
  \item two indifference curves cannot cross,
    because if they did, it would imply that the same bundle provides
    two different levels of utility, which is a contradiction.
  \item Treat these slopes parallel to each other,
    they are not the same, but they are equal in utility.
\end{itemize}

\section{IC Characteristics}
   There are three assumptions
\begin{enumerate}
  \item We can always draw an IC
  \item We can always poiunt to which IC the consumer prefers
    (higher IC is preferred to lower IC)
  \item ICs are convex to origin due to diminishing marginal returns -
    the more R you receive, the less you care about the next R.
\end{enumerate}

\subsection{Marginal Rate of Substitution; MRS}
\begin{itemize}
  \item Marginal rate of substitution is the rate at which a consumer is willing to give up one good
    in exchange for another while maintaining the same level of utility.
  \item This would be the IROC of the indifference curve,
    the slope of the indifference curve at a given point.
\end{itemize}
$$ MRS_{XY} = \frac{MU_X}{MU_Y} = -\frac{\Delta Q_Y}{\Delta Q_X} $$

\section*{MRS, and marginal utility; MU}
\begin{itemize}
  \item Moving bteween two consumption bundles along an indifference curve,
    there should be a \Delta R as well as a \Delta M.

    the consumer is willing to give up some amount of good Y for some amount of good X
    while maintaining the same level of utility.
  \item Therefore: \Delta U = MU$_X$ \Delta Q$_X$ + MU$_Y$ \Delta Q$_Y$
\end{itemize}

\section*{IC Steepness}
\begin{itemize}
  \item A steeper IC curve is preferable to a flatter IC curve,
    because it means that the consumer is willing to give up more of good Y
    for a given amount of good X, while maintaining the same level of utility.
  \item The curve closer to horizontal vs the curve closer to vertical,
    losing value faster and the convex approaching x-axis sooner.
\end{itemize}

\section*{IC Curvature}
\begin{itemize}
  \item More substitutable two goods are, the straighter the IC.
  \item Perfect substitutes have a linear IC,
    meaning that the consumer is willing to give up a constant amount of good Y
    for a constant amount of good X, while maintaining the same level of utility.
\end{itemize}

\section*{Perfect substitutes \& Complements}
\begin{itemize}
  \item Perfect complements always have a ratio between the two goods,
    there will not always be a 1:1 ratio, sometimes it is 1:2 or 1:3, and so on.
  \item Perfect complements have L-shaped ICs,
    meaning that the consumer is willing to give up a constant amount of good Y
    for a constant amount of good X, while maintaining the same level of utility.
\end{itemize}

\section*{Differential Preferences}
\begin{itemize}
  \item Recall: different consumers can have different preferences
    for the same goods and services. Giving unique ICs
  \item Some ppeoplep mayt refer eating chicken or waffles; substitutes
  \item Others prefer chichen and waffles; complements
  \item Can we tell who is who based on the ICs?

    The complements will have a more L-shaped (convex) IC, substitute
    will have a straighter IC.
\end{itemize}

\section*{The Budget Constraint; BC}
\begin{itemize}
  \item Moving on from preferences, we need a constraint (or loss function)
  \item Budget Constaint: line that represents the combinations of goods and services
    that a consumer can afford given their income and the prices of those goods and services.
  \item Assumptions:
    \begin{itemize}
      \item Consumer has a fixed income (I)
      \item Prices of goods and services are fixed (P$_X$, P$_Y$)
      \item Consumer spends all their income on goods and services
    \end{itemize}
  \item Feasable budget is anything on or below the budget constraint,
    meaning that the consumer can afford those goods and services.
  \item Infeasable is anything outside the budget constraint,
    meaning that the consumer cannot afford those goods and services.
\end{itemize}

\section*{Slope of the BC}
\begin{itemize}
  \item Mathematicall,y a BC is typically depicted as:
    Income = P$_X$ \times Q$_X$ + P$_Y$ \times Q$_Y$
  \item We can rewrite this into y = mx+b form too, of course

    I = P$_Y$ \times Y + P$_X$ \times X
\end{itemize}


\section*{Non-standard BC}
\begin{itemize}
  \item Closer to real world
  \item Maddie has a Coffee rewards card, and once she has purchased 5 coffees,
    she receives a 50\% discount.
  \item This effectively means that C1-C5 are \$5, and C6+ are \$2.50
  \item What does this mean? Both max(C) will increase, and the slope will
    pivot after C = 5.
  \item The IC ends up becoming semi-piece wise linear,
    with a kink at the point where the discount applies.
\end{itemize}

\section*{Optimal Consumer Choice}
\begin{itemize}
  \item Graphical explanation, now for a little math
  \item Optimally, we can see the slope of BC ($\frac{-P_X}{P_Y}$)
    and the slope of the IC, MRS = $\frac{-MU_X}{MU_Y}$
  \item Top: The rate the consumer is willing to trade X and Y is equal to
    the rate that the market is willing to trade X and Y, price ratio.
  \item Bottom: Per-dollar (P$_X$) the marginal utility of X is equal to the per-dollar (P$_Y$)
    marginal utility of Y, meaning that the consumer is maximizing their utility
    given their budget constraint.
  \item Suppose we are not optimal, MU$_X$ / P$_X$ > MU$_Y$ / P$_Y$
    then the consumer is not maximizing their utility, and should consume more of good X
    and less of good Y.
  \item Dollar for dollar, this person can get more happiness from good X than good Y,
    so they should consume more of good X and less of good Y.

    Trading Y for X will increase their utility.
  \item Opposite is true if MU$_X$ / P$_X$ < MU$_Y$ / P$_Y$,
    then the consumer should consume more of good Y and less of good X.
\end{itemize}

\section*{Corner Solutions}
\begin{itemize}
  \item Are the solutions which lie on the X and Y axis
  \item Optimal choice is to buy as much Y as possible
  \item Imagine are neutral to X
\end{itemize}

\section*{Utility maximization \Rightarrow{} Optimal Solution}
\begin{itemize}
  \item U(X,Y) = X$^a$Y$^{1-a}$
  \item MU$_X$ = a \times X$^{a-1}$Y$^{1-a}$
  \item MU$_Y$ = (1-a) \times X$^a$Y$^{-a}$
  \item MRS$_{XY}$ = MU$_X$ / MU$_Y$ = a \times Y / X
  \item MRS$_{XY}$ = -P$_X$ / P$_Y$
  \item We have MRS$_{XY}$ = a \times Y / (1-a)X
  \item Setting this equal to the price eation (our optimal condition) yields:
  \item aY / (1-a)X = -P$_X$ / P$_Y$
  \item aY = $\frac{P_X}{P_Y (1-a)X}$
  \item Y* = $\frac{1-a}{a (PX/PY)}$
  \item X* = $\frac{aI}{PX}$
\end{itemize}



\end{document}

