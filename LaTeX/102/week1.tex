\documentclass{article}

\title{ECON102: Microeconomics}
\author{Danny Topete}
\date{July 28, 2025}

\begin{document}
\maketitle

\section{Lecture 1}

\subsection{Markets and Models}
\begin{itemize}
  \item Markets are defined by the individual product,
    all the brands are equivalent given they sale the same product.
    All products are direct substitutes; like oranges.
\end{itemize}


\subsection{Supply Curve}
\begin{itemize}
  \item It is sloped positively upward
  \item Similar to the demand curve
  \item Choke price is where supply dies; here it would be 1,
    vertical intercept of the supply curve (x = 0; x-axis)
  \item A new technology in supply would shift the supply curve
    downwards (to the right), making it cheaper to produce
\end{itemize}

\subsection{Market equilibrium}
\begin{itemize}
  \item Where supply and demand curves crross
  \item Typically demoted Q*
  \item SOlving for market equilibrium:
    \begin{itemize}
      \item Look for equilibrium (Q*, P*)
      \item See point where two curves intersect
      \item You could set the two equations equal to each other
        (demand = supply) and solve for P
    \end{itemize}
  \item Excess supply is the point where supply is greater
    than demand; this is also known as a surplus.

    When Qd > Qs, there is excess demand,
    Qd demand is over Qs point of supply.
  \item The excess demand is the point where demand is greater
    than supply; this is also known as a shortage.
  \item Excess supply is the point where supply is greater
    than demand; this is also known as a surplus.
\end{itemize}

\subsection{Demand/Supply shifts}
\textbf{Demand Shift}
\begin{itemize}
  \item When there's a new equilibrium, the demand curve shifts
    to the right (upward; more demand) or left (downward; less demand)
\end{itemize}

\textbf{Supply Shift}
\begin{itemize}
  \item Positive supply shift move rightward, (downward; more supply)
  \item P $\downarrow{}$, Q $\uparrow{}$
\end{itemize}

\section*{PollEv}
\begin{itemize}
  \item Qd = 100-200P
  \item Qs' = 200P + 200
  \item What is the equilibrium price and quantity?
    \begin{itemize}
      \item Set Qd = Qs'
      \item 100 - 200P = 200P + 200
      \item 100 - 200 = 400P
      \item P = -0.25
      \item Qd = 100 - 200(-0.25) = 150
      \item Qs' = 200(-0.25) + 200 = 150
    \end{itemize}
\end{itemize}

\subsection{Magnitude of effects}
\begin{itemize}
  \item We can predict P*' and Q*' after a shift in either supply or demand,
    but can we infer the magnitude of the shift?
  \item Large shift \rightarrow{} large change in P* and Q*
  \item Steep curve \rightarrow{} larger change in P, relatively small change in Q
  \item The magnitude (steepness) of the curve matters
\end{itemize}

\textbf{What if we shift both curves?}
\begin{itemize}
  \item Double shifts; supply and demand curves shift at the same time.
  \item Use the same process as before, but now we have to consider
    both shifts at the same time.
  \item Given the example
    \begin{itemize}
      \item The supply shifts leftward (upward; less supply)
      \item Increase in price, quantity decreases
      \item Now the demand curve shifts downward (left; less demand)
      \item It pulls the price up and down, and quantity down
      \item Q $\downarrow{}$, P (??; we don't know, unless we have the equations. price ambiguous)
    \end{itemize}
\end{itemize}

\subsection{Elasticity}
\begin{itemize}
  \item P and Q can have a different sized effects.
  \item Elasticity is the ratio of percentage change in one value
    on the percentage change in another
  \item Specifically, price elasticity of demand (supply) is the percentage change
    in quantity demeand (supplied) resulting from a given change in price
  \item Elasticity $\neq$ slope,

    Elasticity has no units, it is a ratio of two percentages.

    Slopes will change depending on the units, for example.
    change the units from dollars to cents.
  \item If elasticity is low, it is inelastic.
    If elasticity is high, it is elastic.
%  \item Price Elasticity of Demand (PED) is defined as:
%    $$\text{PED} = \frac{\% \Delta Q^D}{\% \Delta P}$$
  \item Price Elasticity of Supply (PES) is defined as:
%    $$\text{PES} = \frac{\% \Delta Q^S}{\% \Delta P}$$  \item Price Elasticity of Demand (PED) is defined as:
    $\frac{\% change in quantity demanded}{\% change in price}$
%    $\frac{\%\Delta Q$^D$}{\%\Delta  P}$
  \item Price Elasticity of Supply (PES) is defined as:
    $\frac{\% change in quantity supplied}{\% change in price}$
%    $\frac{\%\Delta Q$^S$}{\%\Delta  P}$
\end{itemize}

\subsection{Responsiveness to Price Change}
\begin{itemize}
  \item Price Elasticty of Demand
  \item Large price elasticity demand \rightarrow{} consumers are highly responsive to price changes,
    they will respond with large (and opposite) chaned in Qd
  \item Small price elasticity demand \rightarrow{} consumers are not very responsive to price changes,
    they will respond with small (and opposite) changes in Qd
\end{itemize}

\subsection{Responsiveness to Price Change}
\begin{itemize}
  \item Price Elasticity of Supply:
    Large price elasticity of supply \rightarrow{} producers are highly responsive to price changes,
  \item Small price elasticity of supply \rightarrow{} producers are not very responsive to price changes,
    they will respond with small (and opposite) changes in Qs
  \item How should we logically expect elasticity to behave over time?
  \item Demand SR vs LR: elassticity is higher in the long run than in the short run.
    In the short run, consumers may not be able to change their behavior
    as quickly as they can in the long run.
  \item Supply SR vs LR: elasticity is higher in the long run than in the short run.
    In the short run, producers may not be able to change their production
    as quickly as they can in the long run.
  \item Elasticity = \% \Delta Q / \% \Delta P
  \item good pric elastic when its elasticity is greater than 1 in absolute value
  \item good price inelastic when its elastic is less than 1 in abs val
  \item We call good \textbf{unit price elastic} when its elastic equal to 1


  \item good is \textbf{perfectly price elastic} when a change in price
    corresponds with an infinite change in quantity,

    You buy pepsi because coke is \$0.01 more expensive,
  \item a good is \textbf{perfectly price inelastic} when a change in price corresponds with no change in quantity

    Insulain is a good example, you need it regardless of price.
    Weird because if price goes down, you still buy the same amount,
\end{itemize}

\subsection{Linear Demand Elasticity}
\begin{itemize}
  \item For a linear demnd curve,
    portion of the demand curve above the unit elastic point is elastic
    - Qd in this region is very responsive to price changes
  \item Potion below is inelastic,
    Qd in this region is not very responsive to price changes
%  \item E = \frac{\% \Delta Q$^D$}{\% \Delta P} = \frac{\Delta Q$^D$ / Q$^D$}{\Delta P / P}
  \item Perfectly inelastic demand, they buy how ever much they need,
    regardless of price. A vertical \Delta Q / \Delta P line.
  \item Perfectly elastic is \Delta P / \Delta Q = \infty,
    a horizontal line, where refuse to purchase if above a set price
  \item Gas stations understand this,
    they know that if they raise the price of gas, people will go to the next station.
  \item Good elastic when demand is responsive, inelasitc is when its not responsive
\end{itemize}

\subsection{Other forms of Elasticity}
\begin{itemize}
  \item Measure income elasticity of demand
    E$^D_I$ = \%\Delta Q$^D$ / \%\Delta I
    \Rightarrow{} \Delta Q$^D$ / \Delta I \times{} I / Q$^D$
    - Income elasticity of demand
  \item Measure quantity-responsiveness of consuders faced with
    change in the price of something else, we use cross-price elasticity of demand

    E$^{D}_{XY}$ = \%\Delta Q$^D_X$ / \%\Delta P$_Y$
    \Rightarrow{} \Delta Q$^D_X$ / \Delta P$_Y$ \times{} P$_Y$ / Q$^D_X$
  \item Left and right shoes are perfect complements,
    if the price of left shoes goes up, the demand for right shoes goes down.
    Thus selling less right shoes if the price of the left shoe goes up.
\end{itemize}

\end{document}

