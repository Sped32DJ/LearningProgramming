\documentclass{article}
\title{Week 2 Lectures Notes}
\author{Danny Topete}
\date{August 8th, 2025}


\begin{document}
\maketitle

\section{Wednesday Notes}
\subsection{Simplifying Assumptions}
\begin{itemize}
  \item Firm produces a single final good, and they already picked what it is
  \item For what ever quantity they make, firm's goal is to minimize their costs.
  \item Firm only uses two inputs: Labor (L) and Capital (K)
    \begin{itemize}
      \item Labor is variable
      \item Capital is fixed
      \item Future capital and labor; skilled and unskilled labor; different types of capital
    \end{itemize}
  \item The more inputs the firm uses, the more output they can produce.
    (not realistic since the too many cooks problem)
  \item A firm's production exhibits diminishing marginal returns to labor.
    \begin{itemize}
      \item As you add more labor, holding capital fixed, the increase in output
        from each additional unit of labor gets smaller and smaller.
      \item This is because of the too many cooks problem.
      \item The first worker you buy will be the most productive, the second
        worker will be less productive, and so on.
    \end{itemize}
  \item The firm can buy as many capital or labor inputs as they want, but market prices are fixed.
\end{itemize}

\subsection{Production Function}
\begin{itemize}
  \item A production function is a mathematical relationship that describes how much output can be made from different
    combinations of inputs.
    \begin{itemize}
      \item At thsi stage, we are fixing output (Q)
      \item In many respects then, this is like consumer's Budget Constraint
        What will our objective be? Some level of costs which we want to minimize.
      \item The firm, in general wants to minimize costs s.t.s their fixed output, (more on this later).
    \end{itemize}
  \item In general, Q = f(K,L); quantity produced is a function of capital and labor. For example:
    \begin{itemize}
      \item Q = 2K + 3L
      \item Q = K$^{1/2}$L$^{1/2}$
      \item Q = K$^{1/3}$L$^{2/3}$
    \end{itemize}
  \item In the short run, we have production functions of labor (Q = f(L))
\end{itemize}

\subsection{Marginal and average product}
\begin{itemize}
  \item Marginal product: additional output that a firm can produce by using an additional unit of an input (holding the other constant)
    \begin{itemize}
      \item Sounds like marginal utility
      \item MP$_L$ = $\frac{\Delta Q}{\Delta L}$
      \item MP$_K$ = $\frac{\Delta Q}{\Delta K}$
    \end{itemize}
  \item Diminishing Marginal Product: reduction of incremental output
    obtained from adding more labor or capital
  \item Average Product: total Q of O divided by \# of units of inputs used
    \begin{itemize}
      \item AP$_L$ = $\frac{Q}{L}$
      \item AP$_K$ = $\frac{Q}{K}$
    \end{itemize}
\end{itemize}

\subsection{MP and PA}
\begin{itemize}
  \item Let Q = f(K,L) = K0.5 L0.5
  \item In SR, fixed capital is K = 4
  \item Therefore, SR production becomes:
    QSR = f(L) = 40.5L0.5 = 2L0.5
  \item These tables show that Capital (K) being constant,
    increasing Labor (L) increases output (Q) at a decreasing rate (diminishing marginal product of output).
  \item These are calculated as follows:
    \begin{itemize}
      \item MP$_L$ = $\frac{\Delta Q}{\Delta L}$
      \item AP$_L$ = $\frac{Q}{L}$
    \end{itemize}
  \item \pi{} is often used as profit
  \item The graph shows a slope that decreases over time
\end{itemize}

\subsection{Production in the Long Run}
\begin{itemize}
  \item LR, firm can change all inputs; labor and capital
  \item Naturally brings with some important benefits:
    \begin{itemize}
      \item Firm can choose the best combination of inputs to produce any
        given level of output.
      \item Firm can choose the scale of production that minimizes costs.
      \item They can adjust K to alter their MP$_L$ (which is diminishing).
      \item They tend to be able to substitute some K for L, or L for K.
    \end{itemize}
\end{itemize}

\subsection{The Firm's Cost Minization Problem}
\begin{itemize}
  \item Cost Minimization: Goal: Producing specific Q of O at min cost
  \item Typically, we think about prices being fixed by the market,
    this means that if you produce Q, you receive Q*P in revenue.
  \item Profit = Revenue - Cost
  \item Firm has little to no power to control their revenue (outside of Q choice)
\end{itemize}

\subsection{Isoquants}
\begin{itemize}
  \item Isoquant: curve that shows all combinations of inputs
    that yield the same level of output.
  \item Isolation curve representing all combinations of inputs that allow
    a firm to make a particular Q of O.
  \item Similar to an indifference curve.
\end{itemize}

\subsection{Marginal Rate of Technical Substitution (MRTS)}
\begin{itemize}
  \item Rate at which the firm can trade input X for input Y, holding
    output Q constant.
  \item MRTS$_{KL}$ = -$\frac{\Delta K}{\Delta L}$ | Q constant
\end{itemize}

\subsection{Substitutes \& Complement Isoquants}
\begin{itemize}
  \item Subsitutes: inputs that can be used in place of one another.
    \begin{itemize}
      \item Isoquants are straight lines
      \item MRTS is constant
    \end{itemize}
  \item Complements: inputs that are used together.
    \begin{itemize}
      \item Isoquants are L-shaped
      \item MRTS is either 0 or infinite
    \end{itemize}
  \item Meanwhile, when they are both almost perfect, subst and complements,
    isoquants are convex to the origin.
\end{itemize}

\subsection{Isocost Lines:}
\begin{itemize}
  \item Fixed lines the consumer must obey,
    therefore, choosing within their budget constraint.
  \item Consumers have limited budget, firm has to target output here
  \item Now lets see our new objective function,
    where consumer wanted to maximize utility, firm wants to minimize costs.
  \item Isocost is a curve that shows all the input combinations that yield the same cost.
    Isocost lines for higher expenditures are further from the origin.

    Isocost lines are parallel, due to fixed input prices.
  \item What should look nearly identical to budget constraint, an isocost line.

    Cost = rK + wL; (in income = P$_X$X + P$_Y$Y))

    r = rental rate of capital (price of capital; think about it as buying capital)
    Firm rents capital since they will eventually want to sell it.

    w = wage rate of labor (price of labor)
\end{itemize}

\subsection{Identifying Min(Cost)}
\begin{itemize}
  \item Firm's objective is to minimize costs s.t.s the constraint
    of producing some leve of output, $\bar{Q}$.
\end{itemize}

\subsection{Input Price Change}
\begin{itemize}
  \item One common way isocost line can change
  \item Similar Consumer problem, this is when price changes - w or r
  \item Wages goes down
\end{itemize}

\subsection{Return To Scale (RTS)}
\begin{itemize}
  \item Returns to scale: how output Q changes as all inputs change
  \item Can doubling inputs lead to more than double outputs?
  \item Typicall a Good idea for firm experiencing increasing returns to scale
    should increase their RTS to ramp-up production. (implies average costs are decreasing)
\end{itemize}


\subsection{Factors Affecting RTS}
\begin{itemize}
  \item Constant returns to scale are probably easiest to recognize
  \item Increasing returns to scale typically involve a fixed cost, or learning-by-doing
  \item Fixed cost: input cost that does not vary
  \item Learning-by-doing: as firm produces more, they learn how to produce more efficiently
  \item Decreasing returns are possible, but at least in the long run they are unlikely

    Unlikely for a firm to increase production here, it is likely ineffiient
\end{itemize}

\subsection{Technological Change}
\begin{itemize}
  \item Total Factor Productivity Growth: refer to improvements in technology
    that allow a firm to produce more output from the same amount of inputs.
  \item Q = AK$^a$L$^{1-a}$, A refers to TFP
\end{itemize}

\subsection{A Solution "Blueprint"}
\begin{itemize}
  \item The Consumer wants to maximize utility s.t. their budget
    \begin{itemize}
      \item Solution is defined by tangencty condition: MRS = Px / Py (price ratio)
      \item Find MRS, set it equal to price ratio
      \item Yields X* = f(Y)
      \item Substitute this into budget constraint to find Y*
      \item Plug Y* (or X*) into X* = f(Y) (or Y* = f(X)) to find X*) this yields X* (or Y*)
    \end{itemize}
  \item The Producer: wants to minimize costs s.t.s their production target
    \begin{itemize}
      \item Solution defined by tangency condition: MRTS = wage / rental rate (price ratio)
      \item Find MRTS, set equal to price ratio
      \item Yields K* = f(L) (Equivalent to L* = f(K)))
      \item Subsitute into production function \Rightarrow{} find L* (or K*)
      \item Plug L* (or K*) into K* = f(L) (or L* = f(K)) to find K* (or L*)
    \end{itemize}
\end{itemize}

\subsection{Solving the Firm's Problem}
\begin{itemize}
  \item Cobb Douglass: X$^a$y$^{1-a}$, C = D
  \item For now, lets assume the useful Cobb-Douglass production function:

    Q = AK$^{\alpha}$L$^{1-\alpha}$
  \item min(cost) occurs when:

    MRTS$_{LS}$ = wage / rental rate = w / r
\end{itemize}
\begin{enumerate}
  \item Find MP, MP$_L$ = (1-a)K$^a$L$^{-a}$, MP$_K$ = aK$^{a-1}$L$^{1-a}$
  \item Divide twom, we get MRTS$_{KL}$ = MP$_L$/MP$_K$ = $\frac{(1-a)}{a} \cdot \frac{K}{L}$
  \item Simplified version of MRTS$_{KL}$ = $\frac{(1-a)}{a} \cdot \frac{K}{L}$ = $\frac{w}{r}$
\end{enumerate}

\end{document}

