\documentclass{article}
\title{Week 2 Lectures Notes}
\author{Danny Topete}
\date{August 8th, 2025}


\begin{document}
\maketitle

\section{Monday Notes}
\subsection{Income Effect}
\begin{itemize}
  \item IE: describes change in consumption resulting from change in the purchasing power of their income.

    Graph moves left if there is an increase in inconme, right if decrease in income.
  \item Substitution effect (SE) and total effect (TE) will show up later
  \item IE, more money doesn't always mean more goods
  \item There are normal and inferior goods

    $$ \epsilon{}_{Inc} = \frac{\%\Delta Q_o}{\%\Delta Income} $$
  \item Calculating IE:
    \begin{itemize}
      \item Normal goods: as income increases, consumption increases

        $\epsilon{}_{Inc} > 0$
      \item Inferior goods: as income increases, consumption decreases

        $\epsilon{}_{Inc} < 0$
    \end{itemize}
\end{itemize}

\subsection{IE and Good Types}
\begin{itemize}
  \item IE is percentage of change in Qd of good response to given percentage change in income.
    (similar to PE)
    $$ E_{I}^D  = \frac{\%\Delta Q}{\% \Delta I} $$
  \item Normal good, good with positive3 income elasticity of demand
    \begin{itemize}
      \item Necessity good, subset of normal goods, income elasticity between 0 and 1
      \item Luxury good, subset normal goods with income elasticity greater than 1
    \end{itemize}
  \item inferior good, good with negative income elasticity of demand
\end{itemize}

\subsection{Income Expansion Path (IEP)}
\begin{itemize}
  \item IEP: curve that connects optimal consumption bundles at different income levels, holding prices constant.
  \item Shows how consumer's optimal consumption bundle changes as income changes.
  \item Similar to Engel Curve, but IEP shows combinations of goods.
  \item As consumer income goes up, the curve goes up
  \item Path of consumption bundles as income changes, holding prices constant.
  \item Engel Curve: graph showing relationship between income and quantity demanded of a good, holding prices constant.
\end{itemize}

\subsection{Engel Curve}
\begin{itemize}
  \item Y axis is income, X axis is quantity of good
\end{itemize}

\subsection{Deriving a demand curve from IEP}
\begin{itemize}
  \item Slope: - Py / Px
  \item You can derive real demand curves, it just
    requires a lot of data.
  \item Lets say the consumer has less demand for Px, now what happens?
    \begin{itemize}
      \item Budget constraint pivots out
      \item New optimal consumption bundle
      \item New quantity demanded of good X
      \item If there is less demanded for good X, then our slope is steeper,
        goes to purchase less products of X and more products Y.
    \end{itemize}
\end{itemize}

\newpage
\subsection{Substitution and Income effects}
\begin{itemize}
  \item Substitution effect: describes change in conusmer choice resulting
    from change in relative price of two goods, holding utility constant.

    If apples become more expensive, you may buy more peaches
    (Reach the substitute if a good if the relateive price changes, reach more of the cheaper good)
  \item Income Effect (IE): describes change in consumer choice resulting from a change in purchasing power of their income.

    If you get a raise, you might buy more peaches.
    (You get more income, you buy more of more expensive good)
  \item Total Effect (TE): overall change in consumer choice resulting from change in price of a good.

    $$TE = IE + SE$$
  \item You have a budget, and two items to buy and you
    can check A* for the optimal consumption bundle.
  \item Slope is the income level contraint, there is an Indifference
    Curve tangent to the budget constraint at A*. Each IC shows utility levels.

    A* is the optimal consumption bundle since it is tangent to the budget constraint
    with the highest possible utility level (shown by slope).

    Looking for the highest IC that is tangent to the budget constraint.
  \item New graph, price of chicken drops from \$4 to \$2.
    \begin{itemize}
      \item Slope pivots out to Chicken-axis, favoring more chicken.
      \item New budget constraint pivots out
      \item New optimal consumption bundle at B*
      \item Total effect is the change from A* to B*
      \item TE in this example is how much more chicken is bought when price drops.
      \item Despite Rice being constant, the consumer buys more rice as well, since
        chickne is cheaper. Since they have more spending power, they buy more of both goods.
    \end{itemize}
  \item SE refers to change in Q resulting from relative price change, holding utility constant.
    \begin{itemize}
      \item To isolate SE, we need to create a hypothetical budget constraint
        that has the new relative prices, but allows the consumer to reach
        their original utility level (A*).
      \item This is done by pivoting the new budget constraint inwards
        until it is tangent to the original IC at point C*.
      \item SE is the change from A* to C*
    \end{itemize}
  \item If you have the IC and IE, you don't need to calculate SE, you can just subtract IE from TE. (basic math)
  \item When the price of X goes down, people buy more X
  \item If price of X goes down, Y is relatively more expensive, so people buy less Y, (substitution effect).
    The SE is negative for Y.
  \item If income goes up, people buy more of both goods (normal goods).
    The IE is positive for both goods.
  \item When income and price changes, you have to consider both effects.
    \begin{itemize}
      \item If both effects go in the same direction, then total effect is easy to determine.
      \item If effects go in opposite directions, then you have to calculate the size of each effect to determine total effect.
    \end{itemize}

    You can calculate which effect is larger by looking at the steepness of the IC.
  \item Suppose consumer is optimal at point A, consuming (10,5)
    \begin{itemize}
      \item Py drops and A* \Rightarrow{} B* (8,9)
      \item Can we tell whether IE or SE is larger?
      \item (TE$_x$, TE$_y$) = (X, 8-10 = -2), (Y, 5-9 = +4))
      \item Sub effect: X more expensive relative to Y, so buy more X, less Y. SE$_x$ < 0, negative

       Y is now cheaper reltative to X, so SE$_y$ > 0, positive
      \item Income Effect: Py dropped, so purhcasing power went up. This means
        IE$_x$ and IE$_y$ > 0, both positive.
    \end{itemize}
  \item New tools to solve
    \begin{itemize}
      \item Good X: -SE, +IE, -TE

        SE dominates IE
      \item Good Y: +SE, +IE, +TE

        You can't tell, they are both positive
    \end{itemize}

\end{itemize}

\subsection{Size of Substitution Effects}
\begin{itemize}
  \item SE is about price ratios
  \item Straighter line IC: better substitutes, larger SE
  \item Consumer preferences decides size of SE, deciding how great
    of substituets two goods are.
  \item If they are quite curved, SE is relatively small.
  \item If left shoe increases in price, you buy less right shoes.
  \item Income effect has to do with the fraction of the price comes
    from your budget.
\end{itemize}

\subsection{IE and SE with an inferior good}
\begin{itemize}
  \item Graph contains ramen and steak
  \item There is a change in the price of ramen, it goes down
    \begin{itemize}
      \item Ramen is cheaper
      \item The SE is positive
      \item income effect is negative since ramen is an inferior good
      \item SE$_R$ > 0
      \item IE$_R$ < 0 \Rightarrow{} inferior good
      \item TE > 0
      \item Normal good, when you have more income, you buy more of it
      \item IE is zero if price of gumballs goes down, and I never buy gumballs. Therefore is no income effect.

        Since I already have non-budget for gumballs.
    \end{itemize}
  \item Since IE is a percentage, you can see how much \% of your budget you spend on a good.
    If it 1\% of your budget and price goes up, then a higher percentage of your income (budget) is spent on that good.
\end{itemize}

\subsection{Substitute good price change}
\begin{itemize}
  \item Coke and Pepsi both exists as good substitutes
  \item Suppose \$P$_C$ drops
  \item Crossprice elasticity of demand measures how much Qd of one good
    $$ \epsilon_{cross} = \frac{\%\Delta Q_p}{\%\Delta P_c} $$
  \item People change A to B, more Coke, less Pepsi
  \item \$P$_C$ decreases, Qp decreases, as well as people switch to Coke
  \item IE says both should go up, SE says that Pepsi should go down. We see that overall
    Qp decreases, so SE > IE; substitutes.
\end{itemize}

\subsection{Complement good price change}
\begin{itemize}
  \item Ice cream and Cones!
  \item Suppose \$P$_I$ increases
  \item People switch to point A to point B, buying less of both goods.
  \item Price of cones relative to ice cream decreases, yet people buy less cones.
    (Cones are relatively cheaer, bnut people buy less cones)
  \item They aren't substitutes, they are purchased together; complements.
\end{itemize}

\subsection{Impact on the Demand Curve}
\begin{itemize}
  \item Complements and substitutes,
    we should know how the demand curve shifts if
    goods A or B change in price.
  \item Substitutes: Price of A increases, A shifts demand right

    Price A decreases, sub will shift left.
  \item Complements: Price of A increases, comp will shift demand left

    Price A decreases, comp will shift right.
\end{itemize}

\subsection{Finding Market Demand}
\begin{itemize}
  \item Deriving Demand Curve
  \item There is a lower price, higher quanntity demanded,
    and a choke price for these graphs
  \item Consumer A's and B's demand curve is just zoomed in from the Market Demand curve.
  \item There is a kink at the point where one consumer stops buying the good.
  \item There are different curves for easch consumer, and the market demand curve
    is the horizontal summation of each consumer's demand curve.
  \item Q1: 5-0.05P
  \item Q2: 13-0.25P
  \item Qm = Q1 + Q2 = (5 - 0.05P) + (13 - 0.25P) = 18 - 0.30P

    This holds the market demand curve.
  \item Two consumers have their own choke price, that would be the y-intercept
    of their individual demand curves.
  \item 52 is the choke price of conmsumer 2, 100 is the choke price of consumer 1.
  \item To find choke price, just get Q1 or Q2, and set it to 0, then solve for P (y-intercept).
  \item Q2 is not part of the market demand curve when P > 52,
    since consumer 2 will not buy any of the good.
  \item The first part of the graph until 52 is just consumer 1's demand curve.
  \item Then when Q2 starts buying the good, the market demand curve
    becomes the horizontal summation of both consumers. Qm = Q1 + Q2.
\end{itemize}


\newpage
\section{Wednesday Notes}
\subsection{Simplifying Assumptions}
\begin{itemize}
  \item Firm produces a single final good, and they already picked what it is
  \item For what ever quantity they make, firm's goal is to minimize their costs.
  \item Firm only uses two inputs: Labor (L) and Capital (K)
    \begin{itemize}
      \item Labor is variable
      \item Capital is fixed
      \item Future capital and labor; skilled and unskilled labor; different types of capital
    \end{itemize}
  \item The more inputs the firm uses, the more output they can produce.
    (not realistic since the too many cooks problem)
  \item A firm's production exhibits diminishing marginal returns to labor.
    \begin{itemize}
      \item As you add more labor, holding capital fixed, the increase in output
        from each additional unit of labor gets smaller and smaller.
      \item This is because of the too many cooks problem.
      \item The first worker you buy will be the most productive, the second
        worker will be less productive, and so on.
    \end{itemize}
  \item The firm can buy as many capital or labor inputs as they want, but market prices are fixed.
\end{itemize}

\subsection{Production Function}
\begin{itemize}
  \item A production function is a mathematical relationship that describes how much output can be made from different
    combinations of inputs.
    \begin{itemize}
      \item At thsi stage, we are fixing output (Q)
      \item In many respects then, this is like consumer's Budget Constraint
        What will our objective be? Some level of costs which we want to minimize.
      \item The firm, in general wants to minimize costs s.t.s their fixed output, (more on this later).
    \end{itemize}
  \item In general, Q = f(K,L); quantity produced is a function of capital and labor. For example:
    \begin{itemize}
      \item Q = 2K + 3L
      \item Q = K$^{1/2}$L$^{1/2}$
      \item Q = K$^{1/3}$L$^{2/3}$
    \end{itemize}
  \item In the short run, we have production functions of labor (Q = f(L))
\end{itemize}

\subsection{Marginal and average product}
\begin{itemize}
  \item Marginal product: additional output that a firm can produce by using an additional unit of an input (holding the other constant)
    \begin{itemize}
      \item Sounds like marginal utility
      \item MP$_L$ = $\frac{\Delta Q}{\Delta L}$
      \item MP$_K$ = $\frac{\Delta Q}{\Delta K}$
    \end{itemize}
  \item Diminishing Marginal Product: reduction of incremental output
    obtained from adding more labor or capital
  \item Average Product: total Q of O divided by \# of units of inputs used
    \begin{itemize}
      \item AP$_L$ = $\frac{Q}{L}$
      \item AP$_K$ = $\frac{Q}{K}$
    \end{itemize}
\end{itemize}

\subsection{MP and PA}
\begin{itemize}
  \item Let Q = f(K,L) = K0.5 L0.5
  \item In SR, fixed capital is K = 4
  \item Therefore, SR production becomes:
    QSR = f(L) = 40.5L0.5 = 2L0.5
  \item These tables show that Capital (K) being constant,
    increasing Labor (L) increases output (Q) at a decreasing rate (diminishing marginal product of output).
  \item These are calculated as follows:
    \begin{itemize}
      \item MP$_L$ = $\frac{\Delta Q}{\Delta L}$
      \item AP$_L$ = $\frac{Q}{L}$
    \end{itemize}
  \item \pi{} is often used as profit
  \item The graph shows a slope that decreases over time
\end{itemize}

\subsection{Production in the Long Run}
\begin{itemize}
  \item LR, firm can change all inputs; labor and capital
  \item Naturally brings with some important benefits:
    \begin{itemize}
      \item Firm can choose the best combination of inputs to produce any
        given level of output.
      \item Firm can choose the scale of production that minimizes costs.
      \item They can adjust K to alter their MP$_L$ (which is diminishing).
      \item They tend to be able to substitute some K for L, or L for K.
    \end{itemize}
\end{itemize}

\subsection{The Firm's Cost Minization Problem}
\begin{itemize}
  \item Cost Minimization: Goal: Producing specific Q of O at min cost
  \item Typically, we think about prices being fixed by the market,
    this means that if you produce Q, you receive Q*P in revenue.
  \item Profit = Revenue - Cost
  \item Firm has little to no power to control their revenue (outside of Q choice)
\end{itemize}

\subsection{Isoquants}
\begin{itemize}
  \item Isoquant: curve that shows all combinations of inputs
    that yield the same level of output.
  \item Isolation curve representing all combinations of inputs that allow
    a firm to make a particular Q of O.
  \item Similar to an indifference curve.
\end{itemize}

\subsection{Marginal Rate of Technical Substitution (MRTS)}
\begin{itemize}
  \item Rate at which the firm can trade input X for input Y, holding
    output Q constant.
  \item MRTS$_{KL}$ = -$\frac{\Delta K}{\Delta L}$ | Q constant
\end{itemize}

\subsection{Substitutes \& Complement Isoquants}
\begin{itemize}
  \item Subsitutes: inputs that can be used in place of one another.
    \begin{itemize}
      \item Isoquants are straight lines
      \item MRTS is constant
    \end{itemize}
  \item Complements: inputs that are used together.
    \begin{itemize}
      \item Isoquants are L-shaped
      \item MRTS is either 0 or infinite
    \end{itemize}
  \item Meanwhile, when they are both almost perfect, subst and complements,
    isoquants are convex to the origin.
\end{itemize}

\subsection{Isocost Lines:}
\begin{itemize}
  \item Fixed lines the consumer must obey,
    therefore, choosing within their budget constraint.
  \item Consumers have limited budget, firm has to target output here
  \item Now lets see our new objective function,
    where consumer wanted to maximize utility, firm wants to minimize costs.
  \item Isocost is a curve that shows all the input combinations that yield the same cost.
    Isocost lines for higher expenditures are further from the origin.

    Isocost lines are parallel, due to fixed input prices.
  \item What should look nearly identical to budget constraint, an isocost line.

    Cost = rK + wL; (in income = P$_X$X + P$_Y$Y))

    r = rental rate of capital (price of capital; think about it as buying capital)
    Firm rents capital since they will eventually want to sell it.

    w = wage rate of labor (price of labor)
\end{itemize}

\subsection{Identifying Min(Cost)}
\begin{itemize}
  \item Firm's objective is to minimize costs s.t.s the constraint
    of producing some leve of output, $\bar{Q}$.
\end{itemize}

\subsection{Input Price Change}
\begin{itemize}
  \item One common way isocost line can change
  \item Similar Consumer problem, this is when price changes - w or r
  \item Wages goes down
\end{itemize}

\subsection{Return To Scale (RTS)}
\begin{itemize}
  \item Returns to scale: how output Q changes as all inputs change
  \item Can doubling inputs lead to more than double outputs?
  \item Typicall a Good idea for firm experiencing increasing returns to scale
    should increase their RTS to ramp-up production. (implies average costs are decreasing)
\end{itemize}


\subsection{Factors Affecting RTS}
\begin{itemize}
  \item Constant returns to scale are probably easiest to recognize
  \item Increasing returns to scale typically involve a fixed cost, or learning-by-doing
  \item Fixed cost: input cost that does not vary
  \item Learning-by-doing: as firm produces more, they learn how to produce more efficiently
  \item Decreasing returns are possible, but at least in the long run they are unlikely

    Unlikely for a firm to increase production here, it is likely ineffiient
\end{itemize}

\subsection{Technological Change}
\begin{itemize}
  \item Total Factor Productivity Growth: refer to improvements in technology
    that allow a firm to produce more output from the same amount of inputs.
  \item Q = AK$^a$L$^{1-a}$, A refers to TFP
\end{itemize}

\subsection{A Solution "Blueprint"}
\begin{itemize}
  \item The Consumer wants to maximize utility s.t. their budget
    \begin{itemize}
      \item Solution is defined by tangencty condition: MRS = Px / Py (price ratio)
      \item Find MRS, set it equal to price ratio
      \item Yields X* = f(Y)
      \item Substitute this into budget constraint to find Y*
      \item Plug Y* (or X*) into X* = f(Y) (or Y* = f(X)) to find X*) this yields X* (or Y*)
    \end{itemize}
  \item The Producer: wants to minimize costs s.t.s their production target
    \begin{itemize}
      \item Solution defined by tangency condition: MRTS = wage / rental rate (price ratio)
      \item Find MRTS, set equal to price ratio
      \item Yields K* = f(L) (Equivalent to L* = f(K)))
      \item Subsitute into production function \Rightarrow{} find L* (or K*)
      \item Plug L* (or K*) into K* = f(L) (or L* = f(K)) to find K* (or L*)
    \end{itemize}
\end{itemize}

\subsection{Solving the Firm's Problem}
\begin{itemize}
  \item Cobb Douglass: X$^a$y$^{1-a}$, C = D
  \item For now, lets assume the useful Cobb-Douglass production function:

    Q = AK$^{\alpha}$L$^{1-\alpha}$
  \item min(cost) occurs when:

    MRTS$_{LS}$ = wage / rental rate = w / r
\end{itemize}
\begin{enumerate}
  \item Find MP, MP$_L$ = (1-a)K$^a$L$^{-a}$, MP$_K$ = aK$^{a-1}$L$^{1-a}$
  \item Divide twom, we get MRTS$_{KL}$ = MP$_L$/MP$_K$ = $\frac{(1-a)}{a} \cdot \frac{K}{L}$
  \item Simplified version of MRTS$_{KL}$ = $\frac{(1-a)}{a} \cdot \frac{K}{L}$ = $\frac{w}{r}$
\end{enumerate}

\end{document}

