\documentclass{article}
\title{Week 4: Econ 102}
\author{Danny Topete}
\date{August 18, 2025}

\begin{document}
\maketitle
\section{Monday Lecture 7; Ch 9: Supply in a Competitive Market}

\subsection{Market Structures}
\begin{itemize}
  \item Market Structures:
    Describes the competitive environment in which firms operate.
    \begin{itemize}
        \item Perfect Competition
        \item Monopolistic Competition
        \item Oligopoly
        \item Monopoly
    \end{itemize}
  \item Categorized by \# of firms, product differentiation, and barriers to entry.
\end{itemize}

\subsection{Perfect Competition}
\begin{itemize}
  \item PC: Market in which many firms produce an identical product and no barriers to entry exist.
  \item Large number of firms \Rightarrow{} price takers. No one influences the market price.
  \item Identical products \Rightarrow{} no product differentiation, "Pepsi = Coke".
  \item No barriers to entry \Rightarrow{} firms can enter/exit freely when it becomes unprofitable.
  \item PC is rare, but useful to know of.
    The perfect competition model is a benchmark for other market structures.
  \item Closest example: Agriculture.
    \begin{itemize}
        \item Many farmers produce identical crops.
        \item No single farmer can influence the price of wheat, corn, etc.
    \end{itemize}
\end{itemize}

\subsection{Price-Taking Demand Curve}
\begin{itemize}
  \item With the Market,
    as supply increases, the price decreases.

    Then there exists a point where Price = Marginal Cost (MC).
  \item With a firm, their graph of MC keeps growing as you increase quantity.
    Then there comes a point where it hits the demand line.
\end{itemize}

\subsection{Total Revenue, Total Cost, and Profit}
\begin{itemize}
  \item Profit = Total Revenue - Total Cost
  \item Marginal Cost is the additional cost of producing one more unit.
  \item Marginal Revenue is the additional revenue from selling one more unit.
  \item Under PC, no firm can control the price \Rightarrow{} Revenue = P*Q \Rightarrow{} MR = P
  \item Having the market set prices is easier than when the firm sets prices.
  \item You \textbf{maximize profit} by producing where \textbf{MR = MC}.
\end{itemize}

\subsection{PC firm maximizing profit}
\begin{itemize}
  \item MR = P due to PC
  \item P > MC: increase production, increase profits
  \item P < MC : decrease production, increase profits
  \item PC = MC profits maximized (goal point)
  \item \textbf{Positive Profits} P < ATC
    \begin{itemize}
      \item Each unit sold at a profit.
        \item Firm makes profit, new firms enter market.
        \item Supply increases, price decreases.
        \item Long-run equilibrium: P = ATC, zero economic profit.
    \end{itemize}
  \item \textbf{Negative Profits} P > ATC
    \begin{itemize}
      \item Each unit sold at a loss.
        \item Firm incurs losses, some firms exit market.
        \item Supply decreases, price increases.
        \item Long-run equilibrium: P = ATC, zero economic profit.
    \end{itemize}
  \item \textbf{Zero Profits} P = ATC
    \begin{itemize}
      \item Each unit sold at break-even.
        \item Firm covers all costs, no incentive to enter/exit.
        \item Long-run equilibrium: P = ATC, zero economic profit.
    \end{itemize}
\end{itemize}

\subsection{Negative Profits = Shutdown?}
\begin{itemize}
  \item If a firm shuits down, they earn.
    \begin{itemize}
      \item $\sqcap_{s.d.}^{} = $ TR - TC = TR - (FC + TVC)
      \item $\sqcap_{s.d.}^{} = $ \$0 - (FC + \$0) = -FC, so they lose their fixed costs.
      \item Zero revenue, but still incur fixed costs.
      \item Short-run decision: shut down if P < AVC (Average Variable Cost).
    \end{itemize}
  \item If firm continues, they earn.
    \begin{itemize}
      \item $\sqcap_{c.}^{} = $ TR - TC = TR - (FC + TVC)
      \item $\sqcap_{c.}^{} = $ P*Q - (FC + TVC)
      \item Positive revenue, but still incur fixed costs.
      \item Short-run decision: continue if P > AVC.
    \end{itemize}
  \item So, they should continue if $\sqcap_{cont.} > \sqcap_{s.d.}$, or if P > AVC.
  \item Continue if TR - TVC > \$0 \rightarrow{} TR > TVC.

    TR/Q > TVC/Q

    PC/Q > TVC/Q

    SD here is the shutdown price

    BE is the break-even price

    A is the max $\sqcap-max$ decision
\end{itemize}

\subsection{Short Run supply under PC:}
\begin{itemize}
  \item Firm willing to sell Q when P=MC
  \item Market tells firm P \Rightarrow{} Firm sets Q where P = MC
  \item Another example of 4 firms: A B C D
    \begin{itemize}
      \item Firm A: P = \$10, Q = 5
      \item Firm B: P = \$10, Q = 7
      \item Firm C: P = \$10, Q = 3
      \item Firm D: P = \$10, Q = 4
      \item Between P1 and P2, only A and B operate.
      \item P2+, firm C enters
      \item P3+, firm D enters
    \end{itemize}
\end{itemize}

\subsection{Producer surplus in the SR:}
\begin{itemize}
  \item Surplus is the "extra benefit above the agents of minimum requirement"
  \item Producer surplus, extra money firm earns above their maginal cost.
  \item Consumer surplus is extra savings the consumers retains below their "willingness to pay"
  \item Graphically, producer surplus is:
    \begin{itemize}
      \item The area between the price and MC curve
      \item The area between the price and AVC at q*

        (these two things are equivalent)
    \end{itemize}
\end{itemize}

\subsection{Producer surplus and profit}
\begin{itemize}
  \item PS and Profit are different
    \begin{itemize}
      \item PS = TR - TVC
      \item $\sqcap$ = TR - TVC - FC
    \end{itemize}
  \item Firm may choose to operate with negative profits in the short run if they cover their variable costs.
    (not likely to operate long term))
  \item Firm will never operate with negative surplus.
  \item Ignoring fixed costs, each unit sells for less than what it costs.
\end{itemize}

\section{Wed lecture}

\subsection{Profit max with market power mathematically}
\begin{itemize}
  \item Suppose demand Q = 200-0.2P; MC = 200
  \item P = 1000 - 5Q
  \item MR = [demand with twice the slope] \rightarrow{} MR = 1000 - 10Q
  \item Profit-maximizing quantity:

    MR = MC \rightarrow{} 200 = 1000 - 10Q \rightarrow{} Q* = 80
  \item Monopolist price

    Q* = 80 \rightarrow{} demand \rightarrow{} P = 1000 - 5(80) = 600
\end{itemize}

\subsection{Learner Index}
\begin{itemize}
  \item Markup is \% of firm's price above its marginal cost
  \item Under PC, there should be no markups
  \item A monopolist will have MC above
  \item Learner Index is a measurement of firm's markup and market power

    (P-MC)/P = -1/E$_D$, where E$_D$ is elasticity of demand
  \item The amount that a monopolist is able markup their quantity
  \item When price is inelastic (-1 < E$_D$ < 0) then (PC)...
\end{itemize}

\subsection{Response to demand increases}
\begin{itemize}
  \item P=1000-5Q \rightarrow{} P=1200 - 5Q
  \item MR \rightarrow{} MR' = 1200 - 10Q
  \item MC = MR \rightarrow{} 200 = 1200-10Q
  \item Q* = 100
  \item P*=1200 - 5(100) = 700
\end{itemize}

\subsection{Customer price sensitivity}
\begin{itemize}
  \item Firms love elastic demand
  \item Allows them for markups
\end{itemize}

\subsection{Consumer and producer surplus}
\begin{itemize}
  \item Producer surplus is the price they are able to receive under MC
  \item Under PC, there is no long run surplus and barely break even
    \begin{itemize}
      \item CS is all above MC
      \item PS not existent (they barely break even)
    \end{itemize}
  \item Under Monopoly
    \begin{itemize}
      \item CS is above Pm and left of Am slope
      \item PS is below CS and above MC
      \item Deadweight Loss (DWL) right of Qm and above MC

        This is inefficiency in the market. This is bad for society
      \item There are some times where market can not hold multiple firms,
        therefore, this may be the only option

        Especially when fixed costs are huge
      \item It is possible that if only a single firm exists, they can supply
        water to all consumers more efficiently. Meaning it is best to have
        a monopolist
    \end{itemize}
\end{itemize}

\subsection{Deadweight Loss (DWL)}
\begin{itemize}
  \item DWL (efficiency loss) is a reduction in total economic surplus resulting
    from inefficiencies
  \item Last slide, surplus could be all 3 colors under PC, but under monopoly, they
    lose the blue triangle, that's DWL
  \item Oftentimes this term is used in relation to tax inefficiencies
\end{itemize}


\end{document}
