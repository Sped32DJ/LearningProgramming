% https://www.overleaf.com/learn/latex/Main_Page <-- latex resources
% <--- percent sign starts a comment line in Latex

%----------------------------------------------------------------------------------
%
% This is a sample assignment .tex file. Put your name, assignment number and the
% due date below, as shown.
% If you work with a partner, create another line with command \student{his/her name}
%
% Before you typeset your own assignment try to preview and print this one. If you
% have LaTeX installed on your machine, you need to:
%   1. Save this in a file, say hw.tex
%   2. Run "pdflatex hw"
%	3. LaTeX produces a pdf file that you can view with/print any pdf viewer
%
% Alternatively you can also compile and preview it in overleaf.com, the online
% LaTeX editor.
%

%----------------------------------------------------------------------------------

\documentclass[11pt]{article}

%----------------------------------------------------------------------------------
% this is a list of latex packages that may be useful in this document
%----------------------------------------------------------------------------------

\usepackage{fullpage}
\usepackage{graphicx}
\usepackage{amsfonts,amsmath,latexsym,amssymb,amsthm}
\usepackage[nothing]{algorithm}
\usepackage{algorithmicx}
\usepackage[noend]{algpseudocode}
\usepackage{hyperref}
%----------------------------------------------------------------------------------
% some macros below. don't worry about these.
%----------------------------------------------------------------------------------


\newcommand{\student}[1]{{\noindent\Large\em {#1} \hfill}\vskip 0.1in}

\newcommand{\assignment}[1]{\centerline{\large\bf CS111 Homework {#1}}}

\newcommand{\duedate}[1]{{\centerline{due {#1}}}}

\newcommand{\assign}{\,\leftarrow\,}

\newcounter{prnum}
\setcounter{prnum}{0}
\newenvironment{problem}{{\vskip 0.2in\noindent\bf Problem
       \addtocounter{prnum}{1} \arabic{prnum}.}}{\vskip 0.1in}

%----------------------------------------------------------------------------------
% the actual source for the document starts here
%----------------------------------------------------------------------------------

\begin{document}

\student{Danny Topete} % <-- Replace with your name
\student{Kat} % <-- Cat's name
\vskip 0.1in\noindent\hrule\vskip 0.2in
\assignment{1}                           % <-- ASSIGNMENT NUMBER ******
\duedate{Monday, Oct 14th}              % <-- DUE DATE ***************

%----------------------------------------------------------

\begin{problem}
Give an asymptotic estimate for the number $h(n)$ of "Hello's" printed by Algorithm PRINT-HELLOS below. 
Your solution \textit{must} consist of the following steps:

(a) First express $h(n)$
using the summation notation \Sigma \\

\begin{equation*}
  h(n) = \sum_{i=2}^{3n+1} (\sum_{j=i+1}^{2i}(print("A"))) 
  + \sum_{i=1}^{2n}(\sum_{j=1}^{(3i+1)^2}(print("A")))
\end{equation*}

(b) Next, we give a closed-form expression$^1$ for $h(n)$ \\

Becomes
\begin{equation}
  h(n) = \sum_{i=2}^{3n+1}(2i)  
  + \sum_{i=1}^{2n}((3i+1)^2)
\end{equation}

simplifies into
\begin{equation}
  h(n) = \sum_{i=1}^{3n}(2i)  
  + \sum_{i=1}^{2n}(9i^2)
  + \sum_{i=1}^{2n}(6i)
  + \sum_{i=1}^{2n}(1)
\end{equation}

Using formulas for the sums we can conclude
\begin{equation}
  h(n) = 2(\frac{(3n)(3n+1)}{6})
  + \frac{2n(2n+1)(4n+1)}{6} 
  + \frac{2n(2n+2)}{2} 
  + 2n 
\end{equation}

\begin{equation}
  h(n) = (n)(3n+1)
  + \frac{n(2n+1)(4n+1)}{3} 
  + n(2n+2) 
  + 2n 
\end{equation}

\smallskip
Then simplifying these values
\begin{equation}
  h(n) = (3n^2 + n)
  + (8n^3 + 6n^2 + n)
  + (2n^2 + n)
  + 2n 
\end{equation}

(c) Finally, give the asymptotic value of $h(n)$ using the \Theta-notation \\

Turning the product of into their asymptotic values we get
\begin{equation}
  h(n) = \Theta(n^2 + n)
  + \Theta(n^3 + n^2 + n)
  + \Theta(n^2 + n)
  + \Theta(n) 
\end{equation}

We then get our final asymptotic value of
\begin{equation}
h(n) = \Theta(n^2 + 5)
\end{equation}

\end{problem}


%----------------------------------------------------------

\begin{problem} \\

  (a) Use properties of quadratic functions to prove that 3$x^2$ \geq $(x + 1)^2$ for all real x \geq 4. \\
  \smallskip

  We can expand the quadratic to 
  \smallskip

  $3x^2$ \geq $x^2+2x+1$ \\
  
  You can solve for the roots of this re-arranged equation
  \smallskip

  $2x^2 - 2x -1$ \geq 0 \\

  The two roots are -0.366 and 1.366, we can exclude the negative value.
  We can conclude that when x is any value greater than 1.3, the inequality will hold true.
  Meaning that the inequality holds true for all real x \geq 2. \\
  

  (b) Use mathematical induction and the inequality from part (a) to prove that $3^n$ \geq $2^n$ + $3n^2$ for all integers n \geq  4.\\
  \smallskip

  Base Case: $3^4 \geq 2^4 + 3 \cdot 4^2$ 
  \to $81 \geq 64$ (statement holds true) \\

  Inductive Hypothesis: Assuming that $3^k \geq 2^k + 3k^2$ is true for k \geq 4.
  We can prove this holds true for k+1
  \smallskip
 
  $3^{k+1} \geq 2^{k+1} + 3(k+1)^2$ \\
  

  Given by inductive hypothesis that $3^k = 2^k + 3k^2$
  \smallskip

  $3(2^k + 3k^2) \geq 2^{k+1} + 3(k+1)^2$ \\

  Further simplifying this equation
  $2^k (3-2) + 6k^2 - 6k - 3 \geq 0$ 
  \smallskip

  $2^k + 3(k^2 - 2k -1) \leq 0$ \\

  Given, part (a) where we proved $n^2 - 2n -1$ held true for x \geq 4.
   Therefore, this identical quadratic will also hold true for x \geq 4. \\
 
  (c) Let $g(n)$ = $2^n + 3n^2$ and $h(n) = 3^n$. Using the inequality from part (b), prove that $g(n) = O(h(n)).$ \\

  \begin{proof}
  Proof for $g(n) = \Theta(h(n))$

    \item Recall inequality (b) where we establish
    \begin{enumerate}
        \item First, we establish the base case:
        \begin{itemize}
          \item $3^n \geq 2^n + 3n^2$ for all integers $n \geq 4$ 
        \end{itemize}
      \item $g(n)$ in terms of $h(n)$: Since $g(n) = 2^2 + 3^2$,
        we can substitue for the inequality
        \begin{equation}
          g(n) \leq 3^n for all (n\geq4)
        \end{equation}

      \item Choose Constants:
        \begin{itemize}
          \item C = 1
          \item $n_0 = 4$
        \end{itemize}
        Then becomes: 
        \begin{equation*}
          g(n) \geq 1\cdot3^n for all (n\geq4)
        \end{equation*}
      \item Conclude the proof: We established that $g(n)$ 
        is bound by C\cdot h(n) for all $n\geq n_0$, which leads us to 
        \begin{equation} 
          g(n) = \Theta(h(n))
        \end{equation}
With the use of the Big-\Theta notation definition, we have proved that:
\begin{equation}
  g(n) = \Theta(h(n)) where g(n) = 2^n + 3n^2
  and 
  h(n) = 3^n
\end{equation}
    \end{enumerate} 
  \end{proof}
\end{problem}


%----------------------------------------------------------
\begin{problem}
    Give asymptotic estimates, using \Theta-notation for the following functions\\
   
    \noindent(a) $3n^3 - 15n^2 + 2n +4$\\

    One of the first steps would be to remove all negatives and remove all coefficients. \\

     \Theta ($n^3 + n + 1$)
     \\

     Then $n^3$ is the highest growing term
     \smallskip

     \Theta($n^3$) \\

   \noindent(b) $n^2 log^2(n) + n^{1.5} log^4 (n) + 5n^2 log_4 (n^3)$  \\

   First we start out by removing all coefficients 
   \smallskip

   \Theta($n^2 log^2(n) + n^{1.5} log^4 (n) + n^2 log(n)$) 
      \\

      After factoring out $log(n)$
      \smallskip

   \Theta($n^2 log(n) + n^{1.5} log^3 (n) + n^2 $)  \\

   We can conclude that $n^2$, the quadratic is the faster growing term.
   Since it derived from $5n^2log_4(n^3)$, our \Theta(n) will be 
   \smallskip

   \Theta($n^2log(n)$) \\



    \noindent(c) $nlog^3n - 5n^2 log^3 n + 2n^2 \sqrt n$  \\

      \Theta ($nlog^3n + n^2 \sqrt n$)
      \\

      We can conclude that quadratic grows faster than logarithmic
      \smallskip

      \Theta($n^2 \sqrt n$) \to \Theta(n) \\

    \noindent(d) 7\cdot $n^5 + n^4 log^3 n + 1.5^n  $ \\

      \Theta($n^5 + n^4 log^3 n + 1.5^n $) 
    \\

    The highest growing expression term is the exponential
    \smallskip

    \Theta($1.5^n$) \\

    \noindent(e) $log^7 n + n^5 4^n + n5^n $   \\

    \Theta($log^7 n + n^5 4^n + n5^n$) \\

    Since we know that logoritms grow slower than exponentials, we can rule them out
    \smallskip

    \Theta($n^5 4^n + n5^n$) \\

    Exponentials grow faster than quadratics and 
    $5^n > 4^n$
    \smallskip

    \Theta($n 5^n$)
    
\end{problem}

%----------------------------------------------------------

\vskip 0.2in
\paragraph{Academic integrity declaration.}

I individually did the homework, mostly because I did it over the weekend
and the labs for other classes took a lot of time out the week. \\

Resource used: geeksforgeeks.org article called "Big O Notation Tutorial – A Guide to Big O Analysis" \\
I used \href{http://polysum.tripod.com/Sum_Formulas.gif}{this picture} 
$"(http://polysum.tripod.com/Sum_Formulas.gif)"$
look up the summation formulas to remember what I forgot when I took Calculs II.\\

I also had to look up the law of logs to remember the base of the log is equivalent to two logs being divided. 
Along with a \LaTeX  
cheat sheet for every time I needed a specific character. 
I am not sure if this must be stated, but I used the vim-tex plugin to my IDE to assist with writing in \LaTeX \\



\end{document}
