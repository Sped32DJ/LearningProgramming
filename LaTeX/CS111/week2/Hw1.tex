% https://www.overleaf.com/learn/latex/Main_Page <-- latex resources
% <--- percent sign starts a comment line in Latex

%----------------------------------------------------------------------------------
%
% This is a sample assignment .tex file. Put your name, assignment number and the
% due date below, as shown.
% If you work with a partner, create another line with command \student{his/her name}
%
% Before you typeset your own assignment try to preview and print this one. If you
% have LaTeX installed on your machine, you need to:
%   1. Save this in a file, say hw.tex
%   2. Run "pdflatex hw"
%	3. LaTeX produces a pdf file that you can view with/print any pdf viewer
%
% Alternatively you can also compile and preview it in overleaf.com, the online
% LaTeX editor.
%

%----------------------------------------------------------------------------------

\documentclass[11pt]{article}

%----------------------------------------------------------------------------------
% this is a list of latex packages that may be useful in this document
%----------------------------------------------------------------------------------

\usepackage{fullpage}
\usepackage{graphicx}
\usepackage{amsfonts,amsmath,latexsym,amssymb,amsthm}
\usepackage[nothing]{algorithm}
\usepackage{algorithmicx}
\usepackage[noend]{algpseudocode}

%----------------------------------------------------------------------------------
% some macros below. don't worry about these.
%----------------------------------------------------------------------------------


\newcommand{\student}[1]{{\noindent\Large\em {#1} \hfill}\vskip 0.1in}

\newcommand{\assignment}[1]{\centerline{\large\bf CS111 Homework {#1}}}

\newcommand{\duedate}[1]{{\centerline{due {#1}}}}

\newcommand{\assign}{\,\leftarrow\,}

\newcounter{prnum}
\setcounter{prnum}{0}
\newenvironment{problem}{{\vskip 0.2in\noindent\bf Problem
       \addtocounter{prnum}{1} \arabic{prnum}.}}{\vskip 0.1in}

%----------------------------------------------------------------------------------
% the actual source for the document starts here
%----------------------------------------------------------------------------------

\begin{document}

\student{Danny Topete} % <-- Replace with your name
\student{Tora} % <-- Cat's name
\vskip 0.1in\noindent\hrule\vskip 0.2in
\assignment{1}                           % <-- ASSIGNMENT NUMBER ******
\duedate{Friday, Oct 10th}              % <-- DUE DATE ***************

%----------------------------------------------------------

\begin{problem}
\noindent(a)
The inner loop of the first \textbf{for} loop prints $i^2$ letters for each $i = 1,2,...,2n+1$. The inner loop of the second \textbf{for} loop prints $2i$ letters for each $i = 1,2,...,n^2$.
Thus denoting $f(n)$ the number of letters ``A" printed we have:
\begin{equation*}
    f(n) = \sum_{i=1}^{2n+1} i^2 + \sum_{i=1}^{n^2}2i.
\end{equation*}
%
\noindent (b)
Using formulas for the sum of the first $k$ terms of an arithmetic series and the sum of squares of $k$ first integers, we can simplify the above formula as follows:
%
\begin{align*}
    f(n) &= \sum_{i=1}^{2n+1} i^2 + \sum_{i=1}^{n^2}2i.
    \\
    &= \frac16 (2n+1)(2n+2)(4n+3) + 2 \cdot \frac12 n^2(n^2+1)
    \\
    &= \frac16 (16n^3+36n^2+26n+6) + (n^4+n^2)
    \\
    &= n^4 + \frac83 n^3 + 7n^2+ \frac{13}{3}n + 1
\end{align*}
%
\noindent (c)
$f(n) = \Omega(n^4)$, since $n^4 + 8/3 \, n^3 + 7n^2+ 13/3 \, n + 1 \geq n^4$ for $n \ge 0$

$f(n) = O(n^4)$, since
$n^4 + 8/3 \, n^3 + 7n^2+ 13/3 \, n + 1 \leq 7(n^4 + n^4 +n^4 +n^4 + n^4) = 35 n^4$ for $n \ge 1$\\
We conclude that $f(n) = \Theta(n^4)$.
\end{problem}

%----------------------------------------------------------
\begin{problem}
    Use induction to prove the formula for the sum of a geometric sequence:
$$\sum_{i=0}^{n} a_i = \frac{a^{n+1}-1}{a-1}$$
	\begin{itemize}
		\item Base case: $n=0$: $\text{LHS} =a^0 = 1$ and $\text{RHS}=\frac{a-1}{a-1} = 1$. So it is true for base case.
		\item Inductive Step: Assume the identity holds for some for $n=k$ that is:
		$$\sum_{i=0}^{k} a_i = \frac{a^{k+1}-1}{a-1}$$
	    Prove that it is true for $n=k+1$:
		$$\sum_{i=0}^{k+1} a_i = \frac{a^{k+2}-1}{a-1}$$
		We have:
		\begin{align*}
			\text{LHS} &= \sum_{i=0}^{k+1} a_i = \sum_{i=0}^{k} a_i + a^{k+1} \;\;\text{(separate last term from the sum)}
			\\
			&= \frac{a^{k+1}-1}{a-1} + a^{k+1} \quad\text{(apply inductive assumption)}
			\\
			&= \frac{a^{k+1}-1+ a^{k+1}(a-1)}{a-1}
			\\
			&= \frac{a\cdot a^{k+1}-1}{a-1} = \frac{a^{k+2}-1}{a-1} = \text{RHS}
		\end{align*}
		\item Conclusion: The claim holds for $n=k+1$.
		From the base case and the inductive step, it holds for $n \ge 0$
	\end{itemize}
\end{problem}

%----------------------------------------------------------

\begin{problem}
    Give asymptotic values for this function using $\Theta$-notation:
$f(n) = \dfrac{n^2 3^n}{4} + n^4 2^n$

\smallskip
\begin{itemize}
	\item $\dfrac14 n^2 3^n + n^4 2^n \ge \dfrac14 n^2 3^n$ for $n\ge 0$, so  $f(n) = \Omega(n^2 3^n)$
	\item We also have:
		\begin{flalign*}
			f(n) &= \frac14 n^2 3^n + n^4 2^n &
			\\
			&= O(n^2 3^n) + n^2 \cdot n^2 \cdot 2^n &
			\\
			&= O(n^2 3^n) + n^2 \cdot O(1.5^n) \cdot 2^n  \quad (\text{because} \; n^2 = O(1.5^n)) &
			\\
			&= O(n^2 3^n) + O(n^2 3^n) = O(n^2 3^n) &
		\end{flalign*}
\end{itemize}
We have shown that $f(n) = \Omega(n^2 3^n)$ and $f(n) = O(n^2 3^n)$. Therefore $f(n) = \Theta(n^2 3^n)$
\end{problem}

%----------------------------------------------------------

\vskip 0.2in
\paragraph{Academic integrity declaration.}
I did the assignment with my cat Tora.

\end{document}
