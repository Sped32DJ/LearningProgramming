% https://www.overleaf.com/learn/latex/Main_Page <-- latex resources
% <--- percent sign starts a comment line in Latex

%----------------------------------------------------------------------------------
%
% This is a sample assignment .tex file. Put your name, assignment number and the
% due date below, as shown.
% If you work with a partner, create another line with command \student{his/her name}
%
% Before you typeset your own assignment try to preview and print this one. If you
% have LaTeX installed on your machine, you need to:
%   1. Save this in a file, say hw.tex
%   2. Run "pdflatex hw"
%	3. LaTeX produces a pdf file that you can view with/print any pdf viewer
%
% Alternatively you can also compile and preview it in overleaf.com, the online
% LaTeX editor.
%

%----------------------------------------------------------------------------------

\documentclass[11pt]{article}

%----------------------------------------------------------------------------------
% this is a list of latex packages that may be useful in this document
%----------------------------------------------------------------------------------

\usepackage{fullpage}
\usepackage{graphicx}
\usepackage{amsfonts,amsmath,latexsym,amssymb,amsthm}
\usepackage[nothing]{algorithm}
\usepackage{algorithmicx}
\usepackage[noend]{algpseudocode}
\usepackage{hyperref}
%----------------------------------------------------------------------------------
% some macros below. don't worry about these.
%----------------------------------------------------------------------------------


\newcommand{\student}[1]{{\noindent\Large\em {#1} \hfill}\vskip 0.1in}

\newcommand{\assignment}[1]{\centerline{\large\bf CS111 Homework {#1}}}

\newcommand{\duedate}[1]{{\centerline{due {#1}}}}

\newcommand{\assign}{\,\leftarrow\,}

\newcounter{prnum}
\setcounter{prnum}{0}
\newenvironment{problem}{{\vskip 0.2in\noindent\bf Problem
       \addtocounter{prnum}{1} \arabic{prnum}.}}{\vskip 0.1in}

%----------------------------------------------------------------------------------
% the actual source for the document starts here
%----------------------------------------------------------------------------------

\begin{document}

\student{Danny Topete} % <-- Replace with your name
%\student{Kat} % <-- Cat's name
\vskip 0.1in\noindent\hrule\vskip 0.2in
\assignment{2}                           % <-- ASSIGNMENT NUMBER ******
\duedate{Monday, Oct 28th}              % <-- DUE DATE ***************

%----------------------------------------------------------

\begin{problem}\\
Prove the following statement:\\
If $p$ > 5 and $gcd(p,20)$ = 1, then $(p^2 - 21)(p^2 + 16) \equiv 0 (mod$ 20)
\begin{enumerate}
  \item Simplifying (mod 20) \\
    Given that p>5 and $gcd(p,20)$ = 1, we know that p is odd.
    We also know that if \\
    $gcd(p,20)$ = 1, then p is not divisible by 2 or 5.
    Giving us $p^2 \equiv 1 (mod$ 4) and \\$p^2 \equiv 1 (mod$ 5);
    given from the statement $gcd(p, 5\cdot2^2)= 1.$\\

    \begin{enumerate}
      \item (Mod 4)\\
    For odd integers, p \equiv 1 (mod 2):
    \begin{equation*}
      p^2 \equiv 1 (mod 4)
    \end{equation*}
    Therefore we have:
        \begin{equation*}
          (p^2 - 21)\equiv 1-21 \equiv -20 \equiv 0 (mod 4)
        \end{equation*}
        \begin{equation*}
          (p^2 +16)\equiv 1 + 16 \equiv 17 \equiv 1 (mod 4)
        \end{equation*}
        By applying (Mod 4)
        \begin{equation*}
          (p^2 - 21)(p^2 + 16) \equiv 0\cdot1 \equiv 0 (mod 4)
        \end{equation*}
      \item (Mod 5)
        Given that we have $gcd(p,5) = 1$, we know $p \not\equiv 0(mod 5)$
        \begin{equation*}
          p^2 \equiv 1 (mod 5)
        \end{equation*}
        or
        \begin{equation*}
         p^2 \equiv 4 (mod 5) 
        \end{equation*}
        This grants us two cases:
        \begin{enumerate}
          \item If $p^2$ \equiv 1 (mod 5) then:\\
            \begin{equation*}
              p^2 - 21 \equiv 1 - 21 \equiv -20 \equiv 0 (mod 5)
            \end{equation*}
            \begin{equation*}
              p^2 + 16 \equiv 1 + 16 \equiv 17 \equiv 2(mod 5)
            \end{equation*}
            This case would result in:
            \begin{equation*}
              (p^2 - 21) (p^2 + 16) \equiv 0\cdot2\equiv 0(mod 5)
            \end{equation*}
          \item If $p^2$ \equiv 4 (mod 5), then:\\
            \begin{equation*}
              (p^2 - 21) \equiv 4 - 21 \equiv -17 \equiv 3 (mod 5)
            \end{equation*}
            \begin{equation*}
              (p^2 + 16)\equiv 4 + 16 \equiv 20 \equiv 0 (mod 5)
            \end{equation*}
            This case would result in:
            \begin{equation*}
              (p^2 - 21)(p^2 + 16) \equiv 3\cdot0 = 0(mod 5)
            \end{equation*}
            Both cases would result in 0(mod 5)
        \end{enumerate}
    \end{enumerate}
  \item Combining Results
    We have:
    \begin{equation*}
      (p^2 - 21) (p^2 + 16)\equiv 0 (mod 4)
    \end{equation*}
    and
    \begin{equation*}
      (p^2 - 21)(p^2 + 16)\equiv 0 (mod 5)
    \end{equation*}
    Since $gcd(4,5) = 1$, we can conclude:
    \begin{equation*}
      (p^2 - 21)(p^2 + 16) \equiv 0 (mod 20)
    \end{equation*}

  \item Conclusion:\\
    We have proven if $p>5$ and $gcd(p,20)$ = 1, then:
    \begin{equation*}
      (p^2 - 21) (p^2 + 16) \equiv 0 (mod 20)
    \end{equation*}
\end{enumerate}
  
\end{problem} 
\clearpage


%----------------------------------------------------------

\begin{problem} \\
  Alive's RSA public key is P = (e,n) = (7,4453). 
  Bob sends Alice the message by encoding it follows.
  First he assigns numbers to characteres: A is 7... Z is 32, blank is 33,
  quotation marks: 34, a comma: 35, a period: 36,
  an apostraphe: 37. Then uses RSA to encode each number seperately.
  \begin{enumerate}
    \item Describe step by stop how you arrived at the solution:
      show how to find p and q, \phi(n) and d.
      \begin{enumerate}
        \item d = (e,n)
      \end{enumerate}
  \end{enumerate}
\end{problem}

\clearpage

%----------------------------------------------------------
\begin{problem}
  test
\end{problem}

%----------------------------------------------------------

\vskip 0.2in
\paragraph{Academic integrity declaration.}
I did this myself, trust me.
\end{document}
