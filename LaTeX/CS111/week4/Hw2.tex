% https://www.overleaf.com/learn/latex/Main_Page <-- latex resources
% <--- percent sign starts a comment line in Latex

%----------------------------------------------------------------------------------
%
% This is a sample assignment .tex file. Put your name, assignment number and the
% due date below, as shown.
% If you work with a partner, create another line with command \student{his/her name}
%
% Before you typeset your own assignment try to preview and print this one. If you
% have LaTeX installed on your machine, you need to:
%   1. Save this in a file, say hw.tex
%   2. Run "pdflatex hw"
%	3. LaTeX produces a pdf file that you can view with/print any pdf viewer
%
% Alternatively you can also compile and preview it in overleaf.com, the online
% LaTeX editor.
%

%----------------------------------------------------------------------------------

\documentclass[11pt]{article}

%----------------------------------------------------------------------------------
% this is a list of latex packages that may be useful in this document
%----------------------------------------------------------------------------------

\usepackage{fullpage}
\usepackage{graphicx}
\usepackage{amsfonts,amsmath,latexsym,amssymb,amsthm}
\usepackage[nothing]{algorithm}
\usepackage{algorithmicx}
\usepackage[noend]{algpseudocode}
\usepackage{hyperref}
%----------------------------------------------------------------------------------
% some macros below. don't worry about these.
%----------------------------------------------------------------------------------


\newcommand{\student}[1]{{\noindent\Large\em {#1} \hfill}\vskip 0.1in}

\newcommand{\assignment}[1]{\centerline{\large\bf CS111 Homework {#1}}}

\newcommand{\duedate}[1]{{\centerline{due {#1}}}}

\newcommand{\assign}{\,\leftarrow\,}

\newcounter{prnum}
\setcounter{prnum}{0}
\newenvironment{problem}{{\vskip 0.2in\noindent\bf Problem
       \addtocounter{prnum}{1} \arabic{prnum}.}}{\vskip 0.1in}

%----------------------------------------------------------------------------------
% the actual source for the document starts here
%----------------------------------------------------------------------------------

\begin{document}

\student{Danny Topete} % <-- Replace with your name
%\student{Kat} % <-- Cat's name
\vskip 0.1in\noindent\hrule\vskip 0.2in
\assignment{2}                           % <-- ASSIGNMENT NUMBER ******
\duedate{Monday, Oct 28th}              % <-- DUE DATE ***************

%----------------------------------------------------------

\begin{problem}\\
Prove the following statement:\\
If $p$ > 5 and $gcd(p,20)$ = 1, then $(p^2 - 21)(p^2 + 16) \equiv 0 (mod$ 20)
\begin{enumerate}
  \item Simplifying (mod 20) \\
    Given that p>5 and $gcd(p,20)$ = 1, we know that p is odd.
    We also know that if \\
    $gcd(p,20)$ = 1, then p is not divisible by 2 or 5.
    Giving us $p^2 \equiv 1 (mod$ 4) and \\$p^2 \equiv 1 (mod$ 5);
    given from the statement $gcd(p, 5\cdot2^2)= 1.$\\

    \begin{enumerate}
      \item (Mod 4)\\
    For odd integers, p \equiv 1 (mod 2):
    \begin{equation*}
      p^2 \equiv 1 (mod 4)
    \end{equation*}
    Therefore we have:
        \begin{equation*}
          (p^2 - 21)\equiv 1-21 \equiv -20 \equiv 0 (mod 4)
        \end{equation*}
        \begin{equation*}
          (p^2 +16)\equiv 1 + 16 \equiv 17 \equiv 1 (mod 4)
        \end{equation*}
        By applying (Mod 4)
        \begin{equation*}
          (p^2 - 21)(p^2 + 16) \equiv 0\cdot1 \equiv 0 (mod 4)
        \end{equation*}
      \item (Mod 5)
        Given that we have $gcd(p,5) = 1$, we know $p \not\equiv 0(mod 5)$
        \begin{equation*}
          p^2 \equiv 1 (mod 5)
        \end{equation*}
        or
        \begin{equation*}
         p^2 \equiv 4 (mod 5) 
        \end{equation*}
        This grants us two cases:
        \begin{enumerate}
          \item If $p^2$ \equiv 1 (mod 5) then:\\
            \begin{equation*}
              p^2 - 21 \equiv 1 - 21 \equiv -20 \equiv 0 (mod 5)
            \end{equation*}
            \begin{equation*}
              p^2 + 16 \equiv 1 + 16 \equiv 17 \equiv 2(mod 5)
            \end{equation*}
            This case would result in:
            \begin{equation*}
              (p^2 - 21) (p^2 + 16) \equiv 0\cdot2\equiv 0(mod 5)
            \end{equation*}
          \item If $p^2$ \equiv 4 (mod 5), then:\\
            \begin{equation*}
              (p^2 - 21) \equiv 4 - 21 \equiv -17 \equiv 3 (mod 5)
            \end{equation*}
            \begin{equation*}
              (p^2 + 16)\equiv 4 + 16 \equiv 20 \equiv 0 (mod 5)
            \end{equation*}
            This case would result in:
            \begin{equation*}
              (p^2 - 21)(p^2 + 16) \equiv 3\cdot0 = 0(mod 5)
            \end{equation*}
            Both cases would result in 0(mod 5)
        \end{enumerate}
    \end{enumerate}
  \item Combining Results
    We have:
    \begin{equation*}
      (p^2 - 21) (p^2 + 16)\equiv 0 (mod 4)
    \end{equation*}
    and
    \begin{equation*}
      (p^2 - 21)(p^2 + 16)\equiv 0 (mod 5)
    \end{equation*}
    Since $gcd(4,5) = 1$, we can conclude:
    \begin{equation*}
      (p^2 - 21)(p^2 + 16) \equiv 0 (mod 20)
    \end{equation*}

  \item Conclusion:\\
    We have proven if $p>5$ and $gcd(p,20)$ = 1, then:
    \begin{equation*}
      (p^2 - 21) (p^2 + 16) \equiv 0 (mod 20)
    \end{equation*}
\end{enumerate}
  
\end{problem} 
\clearpage


%----------------------------------------------------------

\begin{problem} \\
  Alive's RSA public key is P = (e,n) = (7,4453). 
  Bob sends Alice the message by encoding it follows.
  First he assigns numbers to characters A is 7... Z is 32, blank is 33,
  quotation marks: 34, a comma: 35, a period: 36,
  an apostraphe: 37. Then uses RSA to encode each number seperately.
  \begin{enumerate}
    \item Describe step by stop how you arrived at the solution:
      show how to find p and q, \phi(n) and d.
      \begin{enumerate}
        \item We know that e * d = 1 (mod $\phi(n)$) \Rightarrow\space d = $e^{-1}$ (mod $\phi(n)$)
        \item $\phi(n) = (p-1)(q-1)$
      \end{enumerate}
    \item Given (M = 764), show the expression, the decrypted integer, and the char it is mapped to.
      \begin{itemize}
        \item Decrypt: M = C$^d$ (mod n)
        \item d = 3703
        \item n = 4453
      \item M = 764$^{3703}$ (mod 4453) = 29
        \begin{itemize}
          \item We know that n is prime and not divisible by d
          \item The prime factors of 764 = $2^2 * 191$\\
          Using Eucledian's Extended Algorithm:
          \item 3703 = 2048 + 1024 + 512 + 64 + 32 + 16 + 4 + 2 + 1
          \item 764(mod 4453) = 764
          \item $764^{2}$ = 583696(mod 4453) = 353
          \item $764^{4}$ = $764^2$(mod 4453)\cdot$764^2$(mod 4453) = 353$^2$(mod 4453) = 4378
          \item $764^{16}$ = 4378$^4$(mod 4453) = 2060
          \item $764^{32}$ = 2060$^2$(mod 4453) = 4344
          \item $764^{64}$ = 4344$^2$(mod 4453) = 2975
          \item $764^{512}$ = 2975$^8$(mod 4453) = 1172
          \item $764^{1024}$ = 1172$^2$(mod 4453) = 2060
          \item $764^{2048}$ = 2060$^2$(mod 4453) = 4344
          \item $353\cdot764\cdot1172\cdot2060^2\cdot2975\cdot4344^2\cdot4378\cdot$(mod 4453) = 29
        \end{itemize}
      \item 29 is mapped to (29 - 7) = 22, which is W (it is indexed at 22 in the alphabet; 29 in our mapping)
      \end{itemize}

      \pagebreak
    \item Decoded Integers: 34 29 14 11 20 33 31 21 27 33 14 7 28 11 33 11 18 15 19 15 20 7 26 
      11 10 33 7 18 18 33 29 14 15 9 14 33 15 25 33 15 19 22 21 25 25 15 8 18 11 35 33 26 
      14 11 20 33 29 14 7 26 11 28 11 24 33 24 11 19 7 15 20 25 35 33 14 21 29 11 28 11 24 33 
      15 19 22 24 21 8 7 8 18 11 35 33 19 27 25 26 33 8 11 33 26 14 11 33 26 24 27 26 14 36 34  
    \item Decoded Message: "WHEN YOU HAVE ELIMINATED ALL 
      WHICH IS IMPOSSIBLE, THEN WHATEVER REMAINS, HOWEVER IMPROBABLE, MUST BE THE TRUTH." 
      \begin{itemize}
        \item This is a quote from "The Adventure of Blanched Soldier", 
          a 1926 Sherlock Holmes short story.

          If you want to say something is true, you must show everything that is
          not possible. Once you have eliminated all the impossibilities.
          You then have the result of all that is true, which helps backup your claim.
          This is a method similar to that of proof by contradiction.
          The issue with this statement is that 
          it help guide to the most probable truth rather than
          the most absolute truth.
      \end{itemize}
  \end{enumerate}
\end{problem}

\clearpage

%----------------------------------------------------------
\begin{problem}
  \begin{enumerate}
    \item Compute $7^{1529}$ (mod 51)
    \item Compute $9^{-1}$ (mod 19) by listing the multiples.
      \begin{itemize}
        \item Due to Fermat's Little Theorem:\\
          $9^{18}\cdot9^{-1} \equiv 1\cdot9^{-1}$ (mod 19)
        \item $9^{17}$ (mod 19) \Rightarrow 9 $(9)^{16}$ (mod 19) \Rightarrow  9 $(9^{4})^{4}$ (mod 19)
        \item $3^{2} \cdot (3^{8})^{4}$ (mod 19)
        \item $3^3$ = 27 (mod 19) \equiv 8 (mod 19) \Rightarrow $2^3$ (mod 19)
        \item $3^2\cdot(3^3 3^3 3^2)^8 $ (mod 19) \equiv $3^2 \cdot (2^6\cdot3^2)^8$ (mod 19)
        \item $(3\cdot 2^3)^2$ (mod 19) \equiv $(5)^2$ (mod 19) \equiv 6 (mod 19)
        \item $3^2 (6)^8$ (mod 19)
        \item $3^2 (36)^4$ (mod 19) \equiv\space $3^2 (17)^4$ (mod 19)
        \item $3^2 (17^2)^2$ (mod 19) \equiv\space $3^2 (4)^2$ (mod 19) \Rightarrow 9\cdot 16 (mod 19)
        \item 144(mod 19) \equiv 11(mod 19)
      \end{itemize}
    \item Compute $9{-1}$ (mod 19) using Fermat's Little Theorem
    \item Compute $9^{-11}$ (mod 19) using Fermat's Little Theorem
    \item Find an integer $x, 0 \leq x \leq 18$, that satisfies
      the following congruence: $9x + 13 \equiv 10$ (mod 19). (NOTE: Don't brute force)
  \end{enumerate}
\end{problem}

%----------------------------------------------------------

\vskip 0.2in
\paragraph{Academic integrity declaration.}
I did this myself, trust me. To check my answers, I used a live python3 command line prompt to check my answers.
\begin{verbatim}
  I used the video https://youtu.be/kYasb426Yjk?si=bs6HySLzAuPUiycR 
  to help me out in the seconid problem.
  To help with the decoded message:
  https://rationalwiki.org/wiki/Holmesian_fallacy
\end{verbatim}
\end{document}
