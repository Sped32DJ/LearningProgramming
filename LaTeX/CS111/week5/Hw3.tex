% https://www.overleaf.com/learn/latex/Main_Page <-- latex resources
% <--- percent sign starts a comment line in Latex

%----------------------------------------------------------------------------------
%
% This is a sample assignment .tex file. Put your name, assignment number and the
% due date below, as shown.
% If you work with a partner, create another line with command \student{his/her name}
%
% Before you typeset your own assignment try to preview and print this one. If you
% have LaTeX installed on your machine, you need to:
%   1. Save this in a file, say hw.tex
%   2. Run "pdflatex hw"
%	3. LaTeX produces a pdf file that you can view with/print any pdf viewer
%
% Alternatively you can also compile and preview it in overleaf.com, the online
% LaTeX editor.
%

%----------------------------------------------------------------------------------

\documentclass[11pt]{article}

%----------------------------------------------------------------------------------
% this is a list of latex packages that may be useful in this document
%----------------------------------------------------------------------------------

\usepackage{fullpage}
\usepackage{graphicx}
\usepackage{amsfonts,amsmath,latexsym,amssymb,amsthm,enumitem}
\usepackage[nothing]{algorithm}
\usepackage{algorithmicx}
\usepackage[noend]{algpseudocode}
\usepackage{hyperref}
%----------------------------------------------------------------------------------
% some macros below. don't worry about these.
%----------------------------------------------------------------------------------


\newcommand{\student}[1]{{\noindent\Large\em {#1} \hfill}\vskip 0.1in}

\newcommand{\assignment}[1]{\centerline{\large\bf CS111 Homework {#1}}}

\newcommand{\duedate}[1]{{\centerline{due {#1}}}}

\newcommand{\assign}{\,\leftarrow\,}

\newcounter{prnum}
\setcounter{prnum}{0}
\newenvironment{problem}{{\vskip 0.2in\noindent\bf Problem
       \addtocounter{prnum}{1} \arabic{prnum}.}}{\vskip 0.1in}

%----------------------------------------------------------------------------------
% the actual source for the document starts here
%----------------------------------------------------------------------------------

\begin{document}

\student{Danny Topete} % <-- Replace with your name
\vskip 0.1in\noindent\hrule\vskip 0.2in
\assignment{2}                           % <-- ASSIGNMENT NUMBER ******
\duedate{Monday, Nov 4th}              % <-- DUE DATE ***************

%----------------------------------------------------------

\begin{problem}
\begin{enumerate}[label=\alph*)]
    \item Consider the following linear homogeneous recurrence relation: $R_n = 4R_{n-1} - 3R_{n-2}.$
      It is known that: $R_0 = 1, R_2 = 5.$ Find $R_3$\\

      Characteristic Equation: 
      \begin{equation*}
        x^2 - 4x + 3 = 0
      \end{equation*}
      \begin{itemize}
        \item Roots: 1, 3
        \item General Solution: \\
         Given $R_0 = 1$ \Rightarrow $\alpha_1(1)^0 + \alpha_2(3)^0$ = 1\\
         Given $R_2 = 5$ \Rightarrow $\alpha_1(1)^2 + \alpha_2(3)^2$ = 5


      \end{itemize}

        $\begin{bmatrix} 1 & 1 \\ 1 & 9 \end{bmatrix}^{-1} \begin{bmatrix} 1 \\ 5 \end{bmatrix}$ \Rightarrow 
        $\frac{1}{8} \begin{bmatrix}9 & -1 \\ -1 & 1\end{bmatrix} \begin{bmatrix}1\\5 \end{bmatrix}$
        $\Rightarrow \frac{1}{8} \begin{bmatrix} 9 - 5 \\ -1 + 5 \end{bmatrix} \Rightarrow \frac{1}{8} \begin{bmatrix} 4 \\ 4 \end{bmatrix} $
        \Rightarrow $\begin{bmatrix} \alpha_1 = 0.5\\ \alpha_2 = 0.5 \end{bmatrix}$\\

      Solution to recurrence: 
      \begin{equation}
        R_n = \frac{1}{2} (1)^n + \frac{1}{2} (3)^n
      \end{equation}

    \item Determine the general solution of the recurrence equation if its characteristic
      equation has the following roots: 1,1,-2,-2,7.\\


        General Solution: 

      \begin{equation}
        \alpha_1(-2)^n+\alpha_2n(-2)^n+\alpha_3n^2(2)^n+\alpha_4(1)^n+\alpha_5n(1)^n+\alpha_6(7)^n
      \end{equation}

    \item Determine the general solution of the recurrence relation $A_n = 256A_{n-4}$
      \begin{itemize}
        \item $x^4 = 256$
        \item 256 = $4^4$
        \item \Rightarrow $\sqrt[4]{x} = \sqrt[4]{256}$
        \item x = \pm 4\\
         General Solution: 
         \begin{equation}
         \alpha_1(-4)^n+\alpha_2(4)^n
         \end{equation}
      \end{itemize}

      \pagebreak
    \item Solve the homogeneous recurrence question:
      \begin{itemize}
      \item $f_n = f_{n-1} + 4 f_{n-2} + 2 f_{n-3}$
        \item $f_0 = 0$
        \item $f_1$ = 1
        \item $f_2$ = 4
        \item Characteristic Equation: 
        \item \Rightarrow $x^3 - x^2 - 4x - 2$ = 0 
        \item \Rightarrow Roots: $-1, 1 - \sqrt{3}, 1 + \sqrt{3}$\\

        General Solution:
          \begin{equation}
        \alpha_{1} (-1)^{n} + \alpha_{2} (1-\sqrt{3})^{n} + \alpha_{3} (1 + \sqrt{3})^n
          \end{equation}
      \end{itemize}
  \end{enumerate}
\end{problem} 


%----------------------------------------------------------

\begin{problem} 
\begin{enumerate}[label=\alph*)]
  \item  Find the general solution of the recurrence
    \begin{equation*}
      f_n = 13f_{n-2} + 12f_{n-3} + 2n + 1
    \end{equation*}
    \begin{enumerate}
      \item $r^3 = 13r + 12 + 2n + 1$\\

        Find the general solution of the recurrence
      \item $r^3 - 13r - 12 = 0$
      \item Solving for candidate roots: \pm 1, \pm2, \pm3, and \pm4,
        we find Roots: -1, -3, 4
      \item Looking for Particular Solution for the non-homogenous equation: $g(n) = 2n + 1$
      \item $An + B$ = $13(A(n-2) + B) + 12(A(n-3) + B) + 2n + 1$
      \item $An + B = (25A + 2)n + (25B - 62A + 1)$
      \item \Rightarrow $An = (25A + 2)n$ \Rightarrow $A = (25A + 2)$ \Rightarrow 0 = 24A + 2
       \Rightarrow A = $-\frac{1}{12}$
      \item \Rightarrow B = (25B - 62A + 1); given A = $-\frac{1}{12}$ \Rightarrow B = (25B + $\frac{62}{12}$ + 1)  \\

        \Rightarrow 0 = 24B + $\frac{31}{6}$ + 1 \Rightarrow 0 = 24B + $\frac{37}{6}$ \Rightarrow B = $\frac{-37}{144}$\\

        General solution of the non-homogenous recurrence:
        \begin{equation}
          f_n = \alpha_1(-3)^n + \alpha_2(-1)^n + \alpha_3(4)^n - \frac{1}{12}n - \frac{37}{144}
        \end{equation}

    \end{enumerate}

    \pagebreak
  \item Find a particular solution of the recurrence
    \begin{equation*}
      t_n = 4t_{n-1} - 4t_{n-2} + 2^{n}
    \end{equation*}
    \begin{enumerate}
      \item Characteristic Equation: $x^2 - 4x +4 = 0$ \Rightarrow $(x-2)^2$ = 0
      \item roots: 2, 2 (repeated root)
      \item General solution to Homogeneous part: $t_n = \alpha_1(2)^n + \alpha_2n(2)^n$
      \item Particular Solution: $t_n = An(2)^n$\\

          Particular Solution of the recurrence:
        \begin{equation}
          % FIX: Check this
          t_n = \alpha_1(2)^n + \alpha_2n(2)^n + 2^n
        \end{equation}
    \end{enumerate}
\end{enumerate}
  
\end{problem}


%----------------------------------------------------------
\begin{problem}
  We want to tile a $n \times 1$ strip with $1 \times 1$ Green tiles (G), Blue (B), and Red (R), \newline
  2 \times\space 1 purple (P) and 2 \times\space 1 orange (O) tiles, and 3 \times\space 1 lavender (L) tiles. 
  Green, blue, and purple tiles cannot be next to each other, 
  and there should be no two purple or three blue or green tiles in a row. 
  To clarify, GG and BB allowed, but GGGOBR, GROPP, and PBOBR are not.
  Give a formula for the number of such tilings. Your
  solution must include a recurrence equation (with initial conditions!), 
  and a full justification. You do not need to solve it.
\begin{enumerate}
  \item What is not allowed: 
    \begin{itemize}
      \item Green, Blue, and Purple tiles cannot be next to each other
      \item No two purple or three blue or green tiles in a row
    \end{itemize}
  \item What is allowed:
    \begin{itemize}
      \item GG and BB allowed
    \end{itemize}
  \item Defining Variables:
    \begin{enumerate}
      \item $T(n)$ = Number of ways to tile a $n \times 1$ strip
      \item $G(n)$ = Number of ways to tile a $n \times 1$ strip ending in a green tile
      \item $B(n)$ = Number of ways to tile a $n \times 1$ strip ending in a blue tile
      \item $R(n)$ = Number of ways to tile a $n \times 1$ strip ending in a red tile
      \item $P(n)$ = Number of ways to tile a $n \times 1$ strip ending in a purple tile
      \item $O(n)$ = Number of ways to tile a $n \times 1$ strip ending in an orange tile
      \item $L(n)$ = Number of ways to tile a $n \times 1$ strip ending in a lavender tile
    \end{enumerate}
    \pagebreak

  \item Recurrence per state:
    \begin{enumerate}
      \item $G(n)$ = $R(n-1) + O(n-1) + L(n-1) + R(n-2) + O(n-2) + L(n-2)$
      \item $B(n)$ = $R(n-1) + O(n-1) + L(n-1) + R(n-2) + O(n-2) + L(n-2)$\\
        When it comes to Green and Blue, they can be next to each other, but only twice
      \item $P(n)$ = $R(n-1) + O(n-1) + L(n-1)$\\
        When it comes to Green, Blue, and Purple tiles
      They can only be placed next to Red, Orange or Lavender tiles or themselves

      \item $R(n)$ = $G(n-1) + R(n-1) + P(n-1) + O(n-1) + B(n-1) + L(n-1)$


      \item $O(n)$ = $G(n-1) + R(n-1) + P(n-1) + O(n-1) + B(n-1) + L(n-1)$
      \item $L(n)$ = $G(n-1) + R(n-1) + P(n-1) + O(n-1) + B(n-1) + L(n-1)$\\
        When it comes to Red, Orange, and Lavender tiles, they can be placed next to any tile
    \end{enumerate}
  \item Combine all states:
    \begin{itemize}
      \item $T(n)$ = $G(n) + B(n) + R(n) + P(n) + O(n) + L(n)$
    \end{itemize}
  \item Initial Conditions:
    \begin{enumerate}
      \item $n$ = 1
    \begin{itemize}
      \item $G(1)$ = 1
      \item $B(1)$ = 1
      \item $R(1)$ = 1
      \item $P(1)$ = 0
      \item $O(1)$ = 0
      \item $L(1)$ = 0
    \end{itemize}
  \item $n$ = 2 
    \begin{itemize}
      \item $G(2)$ = $R(1) + O(1) + L(1)$ = 1 OR $G(2) = 1$
      \item $B(2)$ = $R(1) + O(1) + L(1)$ = 1 OR $B(2) = 1$
        \begin{itemize}
          \item GG and BB are allowed
        \end{itemize}
      \item $P(2)$ = 1
      \item $R(2)$ = $G(1) + R(1) + P(1) + O(1) + B(1) + L(1)$ = 3
      \item $O(2)$ = 1
      \item $L(2)$ = 0 \Rightarrow\space Since Lavender is a 3x1 tile it cannot be placed in a 2x1 strip
        \begin{itemize}
          \item The OR is if they are next to each other
        \end{itemize}
        \end{itemize}
      \item $n$ = 3
        \begin{itemize}
          \item $G(3)$ = $R(2) + O(2) + L(2) + R(1) + O(1) + L(1)$ = 5
          \item $B(3)$ = $R(2) + O(2) + L(2) + R(1) + O(1) + L(1)$ = 5
          \item $P(3)$ = $R(2) + O(2) + L(2)$ = 4
          \item $R(3)$ = $G(2) + R(2) + P(2) + O(2) + B(2) + L(2)$ = 7
          \item $O(3)$ = $G(2) + R(2) + P(2) + O(2) + B(2) + L(2)$ = 7
          \item $L(3)$ = 1
            \begin{itemize}
              \item since Lavender is a 3x1 tile, it will be placed alone in a 3x1 strip
            \end{itemize}
    \end{itemize}
      
    \end{enumerate}
\end{enumerate}

\end{problem}
  

%----------------------------------------------------------

\vskip 0.2in
\paragraph{Academic integrity declaration.}

For this homework, I did this alone. I looked at my notes from my 
Math 045 (Applied Differential Equations for Electrical Engineering) class
to help with particular and general solutions along with Characteristic Equations.

I also used MatLab to check my work for the first problem to realize I forgot how to inverse a matrix
when I was doing the problem by hand, I had to look at my EE 020B (Applied Linear Algebra) 
notes as a refresher on that topic.
\end{document}
