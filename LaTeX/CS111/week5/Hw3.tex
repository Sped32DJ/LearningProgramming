% https://www.overleaf.com/learn/latex/Main_Page <-- latex resources
% <--- percent sign starts a comment line in Latex

%----------------------------------------------------------------------------------
%
% This is a sample assignment .tex file. Put your name, assignment number and the
% due date below, as shown.
% If you work with a partner, create another line with command \student{his/her name}
%
% Before you typeset your own assignment try to preview and print this one. If you
% have LaTeX installed on your machine, you need to:
%   1. Save this in a file, say hw.tex
%   2. Run "pdflatex hw"
%	3. LaTeX produces a pdf file that you can view with/print any pdf viewer
%
% Alternatively you can also compile and preview it in overleaf.com, the online
% LaTeX editor.
%

%----------------------------------------------------------------------------------

\documentclass[11pt]{article}

%----------------------------------------------------------------------------------
% this is a list of latex packages that may be useful in this document
%----------------------------------------------------------------------------------

\usepackage{fullpage}
\usepackage{graphicx}
\usepackage{amsfonts,amsmath,latexsym,amssymb,amsthm,enumitem}
\usepackage[nothing]{algorithm}
\usepackage{algorithmicx}
\usepackage[noend]{algpseudocode}
\usepackage{hyperref}
%----------------------------------------------------------------------------------
% some macros below. don't worry about these.
%----------------------------------------------------------------------------------


\newcommand{\student}[1]{{\noindent\Large\em {#1} \hfill}\vskip 0.1in}

\newcommand{\assignment}[1]{\centerline{\large\bf CS111 Homework {#1}}}

\newcommand{\duedate}[1]{{\centerline{due {#1}}}}

\newcommand{\assign}{\,\leftarrow\,}

\newcounter{prnum}
\setcounter{prnum}{0}
\newenvironment{problem}{{\vskip 0.2in\noindent\bf Problem
       \addtocounter{prnum}{1} \arabic{prnum}.}}{\vskip 0.1in}

%----------------------------------------------------------------------------------
% the actual source for the document starts here
%----------------------------------------------------------------------------------

\begin{document}

\student{Danny Topete} % <-- Replace with your name
\vskip 0.1in\noindent\hrule\vskip 0.2in
\assignment{2}                           % <-- ASSIGNMENT NUMBER ******
\duedate{Monday, Nov 4th}              % <-- DUE DATE ***************

%----------------------------------------------------------

\begin{problem}
\begin{enumerate}[label=\alph*)]
    \item Consider the following linear homogeneous recurrence relation: $R_n = 4R_{n-1} - 3R_{n-2}.$
      It is known that: $R_0 = 1, R_2 = 5$. Find $R_3$
      Characteristic Equation: 
      \begin{equation}
        x^2 - 4x + 3 = 0
      \end{equation}
      \begin{itemize}
        \item Roots: 1, 3
        \item Particular Solution: \\
         Given $R_0 = 1$ \Rightarrow $\alpha_1(1)^0 + \alpha_2(3)^0$ = 1\\
         Given $R_2 = 5$ \Rightarrow $\alpha_1(1)^2 + \alpha_2(3)^2$ = 5


      \end{itemize}

        $\begin{bmatrix} 1 & 1 \\ 1 & 9 \end{bmatrix}^{-1} \begin{bmatrix} 1 \\ 5 \end{bmatrix}$ = 
        $\begin{bmatrix} \alpha_1 = 0.5\\ \alpha_2 = 0.5 \end{bmatrix}$
      General Solution: 
      \begin{equation}
        R_n = 1/2 (1)^n + 1/2 (3)^n
      \end{equation}

    \item Determine the general solution of the recurrence equation if its characteristic
      equation has the following roots: 1,1,-2,-2,7.\\


        General Solution: 

      \begin{equation}
        \alpha_1(-2)^n+\alpha_2n(-2)^n+\alpha_3n^2(2)^n+\alpha_4(1)^n+\alpha_5n(1)^n+\alpha_6(7)^n
      \end{equation}

    \item Determine the general solution of the recurrence relation $A_n = 256A_{n-4}$
      \begin{itemize}
        \item $x^4 = 256$
        \item 256 = $4^4$
        \item \Rightarrow $\sqrt[4]{x} = \sqrt[4]{256}$
        \item x = \pm 4\\
         General Solution: 
         \begin{equation}
         \alpha_1(4)^n+\alpha_2(-4)^n
         \end{equation}
      \end{itemize}
    \item Solve the homogeneous recurrence question:
      \begin{itemize}
      \item $f_n = f_{n-1} + 4 f_{n-2} + 2 f_{n-3}$
        \item $f_0 = 0$
        \item $f_1$ = 1
        \item $f_2$ = 4
        \item Characteristic Equation: 
        \item \Rightarrow $x^3 - x^2 - 4x - 2$ = 0 
        \item \Rightarrow Roots: $-1, 1 - \sqrt{3}, 1 + \sqrt{3}$
          \begin{equation}
        \alpha_{1} (-1)^{n} + \alpha_{2} (1-\sqrt{3})^{n} + \alpha_{3} (1 + \sqrt{3})^n
          \end{equation}
      \end{itemize}
  \end{enumerate}
\end{problem} 


%----------------------------------------------------------

\begin{problem} \\
\end{problem}


%----------------------------------------------------------
\begin{problem}
\end{problem}

%----------------------------------------------------------

\vskip 0.2in
\paragraph{Academic integrity declaration.}
\end{document}
