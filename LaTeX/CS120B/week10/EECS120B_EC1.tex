% --------------------------------------------------------------
% This is all preamble stuff that you don't have to worry about.
% Head down to where it says "Start here"
% --------------------------------------------------------------
 
\documentclass[12pt]{article}
 
\usepackage[margin=1in]{geometry} 
\usepackage{setspace,fancyhdr,listings,xcolor,graphicx}
\lstset{
    language=C,
    numbers=left,
    numberstyle=\tiny\color{gray},
    stepnumber=1,
    numbersep=10pt,
    backgroundcolor=\color{white},
    showspaces=false,
    showstringspaces=false,
    showtabs=false,
    frame=single,
    rulecolor=\color{black},
    tabsize=2,
    captionpos=b,
    breaklines=true,
    breakatwhitespace=false,
    title=\lstname,
    keywordstyle=\color{blue},
    commentstyle=\color{green},
    stringstyle=\color{red},
    basicstyle=\ttfamily\footnotesize,
    escapeinside={\%*}{*)},
    morekeywords={*,...},
    lineskip=-1pt % Adjust this value to make lines more compact
}
\pagestyle{fancy}
\fancyhf{}

\setlength{\headheight}{15pt}
\fancyhead[R]{Topete \thepage}

\begin{document}
 
% --------------------------------------------------------------
%                         Start here
% --------------------------------------------------------------
 
\title{Final Project: Guess The Color}
\author{Danny Topete\\ %replace with your name
EE/CS 120B}

\maketitle

\doublespacing

% TODO Make sure to fix this bad boy up
I wrote a guess the color game. The idea originally came from a friend who I saw constantly play CSS battle,
I always aspired to be that good at him at guessing the hex values instantly and rebuilding a website like that.
So I decided to make a game that would help me get better at it!
\pagebreak
\section{Hardware:}
\begin{enumerate}
  \item ST7735 Display
  \item RGB LED
  \item Shift Register
  \item IR Sensor / NEC remove
  \item 8 switch DIP
\end{enumerate}

\section{Inputs:}

\begin{tabular}{ l l l }
   \textbf{PC0}: & IR & \quad (Serial Data) \\
   \textbf{PC1}: & Switch 8 & \quad (0 = High; 1 = Low) \\
\end{tabular}

\section{Outputs:}

\begin{tabular}{ l l l }

   \textbf{PB5}: & Display - SCK & \quad (0 = Off; 1 = On) \\
   \textbf{PB3}: & Display - SDK & \quad (0 = Off; 1 = On) \\
   \textbf{PB2}: & Display - CS Pin \\
   \textbf{PD7}: & Shift Register - SH & \quad (0 = Off; 1 = On) \\
   \textbf{PD6}: & RGB - Blue & \quad (0 = Off; 1 = On) \\
   \textbf{PD5}: & RGB - Green & \quad (0 = Off; 1 = On) \\
   \textbf{PD4}: & Shift Register - ST  & \quad (0 = Off; 1 = On) \\
   \textbf{PD3}: & RGB - Red & \quad (0 = Off; 1 = On) \\
   \textbf{PD2}: & Shift Register - DS & \quad (0 = Off; 1 = On) \\
\end{tabular}

\pagebreak
\section{Tasks:}

% FIX: Figure what is wrong with the enumeration
\begin{enumerate}
  \item \textbf{RGB_TICK:}
    This is the duty cycle for the RGB LEDs PWM tasks.
    It handles when it is appropriate to have the LEDs on or off.
  \item \textbf{DISPLAY_TICK:}
    Helps take care of the two states, the start menu and the active playing menu.
    When you beat the game, it will strobe rainbow through the time that it took you to solve the color.
    Along with the start menu, and the strobing rainbow. 
    To strobe the rainbow, I have an array with the colors of the rainbow,
    and I iterate through them.
  \item \textbf{IR_TICK:}
    This function is always reading the IR sensor. 
    The power button starts the game from the start screen, and from the play state, it rolls a new color.
    While it rolls a new color, it resets the time, currVal, and goes to RGB_Tick to ask for a new color. 
    Since it grabs the inputs, it updates the currVal when ever it is modified either up or down;
    along with handling overflow.
  \item \textbf{RED_TICK:}
    For context, I had issues with digital PWM, and this was my next best option that worked perfectly fine. 
    I was able to get this working within the first week and get this out of the way.
    The PWM tick function for the green part of the RGB LED.
  \item \textbf{GREEN_TICK:}
    The PWM tick function for the green part of the RGB LED.
  \item \textbf{BLUE_TICK:}
    The PWM tick function for the blue part of the RGB LED.
  \item \textbf{SHIFT_Tick:}
    This runs in a single state to put the value \textbf{progress} into 
    the shiftOut() function. The LEDs are only ever turned on when 
    currVal is equal to the target value in the respect hex bits of Red, Green, and Blue.
  \item \textbf{ELAPSED_Tick:}
    Tracks the elapsed time in a single state.
    It pauses the elapsed time when you beat the game. 
    Then it pauses the time, so display tick can show the time in chroma.
\end{enumerate}

\pagebreak
\section{Software Libraries Used:}

\begin{itemize}
  \item \textbf{avr/io.h}:
    This library is used for the I/O operations on the AVR microcontroller.
  \item \textbf{stdlib.h}:
    I used for the rand() function to generate a random color to guess.
    It keeps the game interesting.
  \item \textbf{avr/interrupt.h}:
    I used this library to enable the interrupts for the IR sensor.
\end{itemize} 
 
\pagebreak
\section{SM Chart:}


  I am aware that I excluded A6, the Timer State.
  That is due to the limitation that it would probably be its own state machine to have the desired behavior.
  I am also aware with the lack of concurrency in the state machine.
  \pagebreak

  \section{SM in RIMS}
  \begin{lstlisting}[language=C]
#include "rims.h"

enum states {
  INIT,
  IDLE,
  WATERING,
  NUTRIENT,
  LIGHTING_KONTROL,
  TEMPERATURE_KONTROL,
  MANUAL_OVERRIDE,
  STOP
} State;

// State machine function
void Tick() {
  // State Transitions
  switch (State) {
  case INIT:
    // Initialize all sensors and outputs to default values
    if (A & 0x40) {
      State = INIT;
    }
    State = IDLE;

  case IDLE:
    if (~A & 0x01)
      State = WATERING;
    if (~A & 0x02)
      State = NUTRIENT;
    if (~A & 0x04)
      State = LIGHTING_KONTROL;
    if (~A & 0x08)
      State = TEMPERATURE_KONTROL;
    if (A & 0x10)
      State = MANUAL_OVERRIDE;
    if (A & 0x20)
      State = STOP;
    if (A & 0x40)
      State = INIT;
    State = IDLE;
    break;

  case WATERING:
    if (A & 0x01) {
      B = B & ~0x01;
      State = IDLE;
    }
    State = WATERING;
    break;

  case NUTRIENT:
    if (A & 0x02) {
      B = B & ~0x02;
      State = IDLE;
    }
    State = NUTRIENT;
    break;

  case LIGHTING_KONTROL:
    if (A & 0x04) {
      B = B & ~0x04;
      State = IDLE;
    }
    State = LIGHTING_KONTROL;
    break;

  case TEMPERATURE_KONTROL:
    if (A & 0x08) {
      B = B & ~0x08;
      State = IDLE;
    }
    State = TEMPERATURE_KONTROL;
    break;

  case MANUAL_OVERRIDE:
    if (~A & 0x10) {
      State = IDLE;
    }
    State = MANUAL_OVERRIDE;
    break;

  case STOP:
    if (~A & 0x20) {
      State = IDLE;
    }
    State = STOP;
    break;

  default:
    State = INIT;
    break;
  }

  // State Actions
  switch (State) {
  case INIT:
    B = 0x00;
    break;

  case IDLE:
    break;

  case WATERING:
    // B0 = 1;
    B = B | 0x01;
    break;

  case NUTRIENT:
    // B1 = 1;
    B = B | 0x02;
    break;

  case LIGHTING_KONTROL:
    // B2 = 1;
    B = B | 0x04;
    break;

  case TEMPERATURE_KONTROL:
    // B3 = 1;
    B = B | 0x08;
    break;

  case MANUAL_OVERRIDE:
    break;

  case STOP:
    B = 0x00;
    break;

  case default:
    break;
  }
}

int main() {

  State = INIT;

  while (1) {
    Tick();
  }

  return 0;
}

  \end{lstlisting}
% --------------------------------------------------------------
%     You don't have to mess with anything below this line.
% --------------------------------------------------------------
 
\end{document}t
