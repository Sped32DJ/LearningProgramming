% --------------------------------------------------------------
% This is all preamble stuff that you don't have to worry about.
% Head down to where it says "Start here"
% --------------------------------------------------------------
 
\documentclass[12pt]{article}
 
\usepackage[margin=1in]{geometry} 
\usepackage{setspace,fancyhdr}

\pagestyle{fancy} 
\pagestyle{fancy}
\fancyhf{}

\fancyhead[R]{Topete \thepage}

\begin{document}
 
% --------------------------------------------------------------
%                         Start here
% --------------------------------------------------------------
 
\title{EC 1: HGK}
\author{Danny Topete\\ %replace with your name
EE/CS 120B}

\maketitle

\doublespacing

% TODO Make sure to fix this bad boy up
For my real-world embedded system, I will make an automated hydroponic controller.
I call it, Hydroponic Garten Kontroller \textbf{(HGK)} for short. 
I believe this would be a great and creative way to think about real world use cases for embedded systems.
I assume that this is what real hydroponic plants do to manage their garden.
So it would be very fitting to add this 

\pagebreak
\section{Inputs:}

\begin{tabular}{ l l l }
   \textbf{A0}: & Water Level Sensor & \quad (0 = Low; 1 = Normal) \\
   \textbf{A1}: & Nutrient Level Sensor & \quad (0 = Low; 1 = Normal) \\
   \textbf{A2}: & Light Sensor & \quad (0 = Low; 1 = Normal) \\
   \textbf{A3}: & Temperature Sensor & \quad (0 = Out Of Range; 1 = Normal) \\
   \textbf{A4}: & Manual Override Switch & \quad (0 = Auto; 1 = Manual) \\
   \textbf{A5}: & Emergency Stop Button & \quad (0 = Running; 1 = Stop) \\
   \textbf{A6}: & Timer Signal & \quad (scheduled watering and light) \\
   \textbf{A7}: & Reset Signal & \quad (0 = Normal; 1 = Reset) \\
\end{tabular}

\section{Outputs:}

\begin{tabular}{ l l l }

   \textbf{B0}: & Water Pump & \quad (0 = Off; 1 = On) \\
   \textbf{B1}: & Nutrient Pump & \quad (0 = Off; 1 = On) \\
   \textbf{B2}: & Grow Light & \quad (0 = Off; 1 = On) \\
   \textbf{B3}: & Heater/Cooler & \quad (0 = Off; 1 = On) \\
   \textbf{B4}: & Status LED & \quad (0 = Off; 1 = On) \\
   \textbf{B5}: & Alarm & \quad (0 = Off; 1 = On) \\
   \textbf{B6}: & Manual Override Signal & \quad (0 = Off; 1 = On) \\
   \textbf{B7}: & System Status & \quad (0 = Off; 1 = On) \\
\end{tabular}

\pagebreak
\section{States:}

\begin{enumerate}
  \item\textbf{Idle State:}\\
    Monitors inputs without making any changes
  \item\textbf{Watering State:}\\
    Activates water pumps based on Water Level Sensor \textbf{(A0)}.
  \item\textbf{Nutrient State:}\\
    Activates the nutrient pump based on Nutrient Level Sensor \textbf{(A1)}.
  \item\textbf{Lighting Control State:}\\
    Activates based on Light Sensor \textbf{(A2)}.
  \item\textbf{Temperature Kontrol State:}\\
    Activates based on Temperature Sensor \textbf{(A3)}.
  \item\textbf{Error State:}\\
    Activate the alarm if any critical sensor indicates an error.
    It should be able to send out a mass email and SMS to everyone with a log file attached.\\
    Examples would be if the water pump remains on despite water level being reached (potential flooding).
    A calculation can be done of how much water has probably spilled by now.\\
   The temperature is having trouble and remains out of range after a set amount of time, or overheating and the heater stays on. 
\end{enumerate}

\section{State Transitions:}

\begin{itemize}
  \item if\textbf{(A5)}, emergency stop transition to Error State
\end{itemize} 
 

% --------------------------------------------------------------
%     You don't have to mess with anything below this line.
% --------------------------------------------------------------
 
\end{document}t
