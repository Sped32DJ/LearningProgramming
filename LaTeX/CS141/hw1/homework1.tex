\documentclass{article}[12pt]
\usepackage{fullpage}
\usepackage{multicol}
\usepackage{algpseudocode}
\usepackage{amsmath}
\usepackage{amsfonts}
\usepackage{amsthm}
\usepackage{amssymb}
\usepackage{tikz}
\usepackage{algpseudocode}
\usepackage{multicol}
\usepackage{enumitem}
\usepackage{url}
\usepackage{graphicx}
\usepackage{listings}

\ifx\solutiontemplate\undefined
\newcommand{\solkeyword}[1]{}
\newcommand{\policy}[1]{#1}
\else
\newcommand{\solkeyword}[1]{#1}
\newcommand{\policy}[1]{}
\fi

\newcommand\encircle[1]{\raisebox{.5pt}{\textcircled{\raisebox{-.9pt} {\footnotesize #1}}} }


\newenvironment{solution}[0]{\vspace{.1in} \textbf{Solution.} \,}{}

\newcommand{\deadline}{11:59pm, Mon 1/21, 2025}

\newcommand{\assigntitle}[1]{{
  \noindent \large \bf
  CS141, Winter 2025,
  Assignment \##1 \hfill Due: {\deadline}\\
  Name:  %put in your name here
  \hspace{2.5in}
  NetID:  %put in your id here
  \\
  [-.05in]
  \mbox{}\hrulefill \mbox{}\\}}

\begin{document}

\assigntitle{1}{}
\policy{\textbf{Deadline.} The homework is due 11:59pm, 1/21/2025. You must submit your solutions (in pdf
format generated by LaTeX) via GradeScope.}\\
\date{}
\policy{\textbf{Assignment Policy.}
\begin{itemize}
    \item Unless mentioned otherwise, late submission allowed for 20\% penalty for each calendar day.
    \item Assignments should be in pdf format generated by LaTeX
    \item If you are using any external source, you must cite it, clarify what exactly got out of it, and if it is from an LLM - you must include a link to the conversation.
    \item You are expected to understand any source you use and solve problems on your own.
    \item All pages relevant to a question must be assigned on Gradescope. Unassigned pages will not be graded.
\end{itemize}


}

\section{A Complex Complexity Problem(20pts)}
Xianghu and Michael recently learned the asymptotical analysis. The key idea is to evaluate the growth of a function. For example, They now know that $n^2$ grows faster than n. They want to know whether they really understand the idea, so they have created a little task. First of all, They found a lot of functions here ($\log {n}$ means $\log_2 n$):
\begin{center}
\begin{multicols}{4}
\begin{enumerate}[label=\encircle{\arabic*}]
  \item $(\sqrt{2})^{\log n}$
  \item $3^n$
  \item $n!$
  \item $n^2$
  \item $n^3$
  \item $\log^2n$
  \item $\log(n!)$
  \item $2^{2^n}$
  \item $\ln\ln n$
  \item $1$
  \item $\ln n$
  \item $\log \log n$
  \item $(n+1)!$
  \item $\sqrt{\log n}$
  \item $n$
  \item $2^n$
  \item $n\log n$
  \item $2^{2^n+1}$
  \item $5n^2-13n+6$
  \item $n/\log n$
\end{enumerate}
\end{multicols}
\end{center}
Reading the long list, Xianghu and Michael realized that things were not as simple as they thought, and felt really upset. Could you help them with the following problems? (You do not need to explain your answer for this whole problem.)

\begin{enumerate}[label=(\arabic*)]
  \item (1pt) What does \emph{polynomial} mean (use $O(\cdot)$, $\Omega(\cdot)$, $\omega(\cdot)$, $\Theta(\cdot)$, $o(\cdot)$ to present your answer)? (2pts) Which of them are polynomial?
  
  \item (1pt) What does \emph{polylogarithmic} mean (use $O(\cdot)$, $\Omega(\cdot)$, $\omega(\cdot)$, $\Theta(\cdot)$, $o(\cdot)$ to present your answer)? (2pts) Which of them are polylogarithmic?
  
  \item (1pt) What does \emph{sublinear} mean (use $O(\cdot)$, $\Omega(\cdot)$, $\omega(\cdot)$, $\Theta(\cdot)$, $o(\cdot)$ to present your answer)? (2pts) Which of them are sublinear?
 
  \item (1pt) What does \emph{superlinear} mean (use $O(\cdot)$, $\Omega(\cdot)$, $\omega(\cdot)$, $\Theta(\cdot)$, $o(\cdot)$ to present your answer)? (2pts) Which of them are superlinear?

  \item (2pts) Which of the above 20 functions grow the fastest (if you think there are multiple answers, list \emph{all} of them)?

  \item (6pts) Finally, please rank all the functions by ascending order of growth; Group together those functions that are big-Theta of one another.
  
\end{enumerate}


\section{Calculate the Fibonacci numbers(20pts)}
The Fibonacci numbers are the numbers in the following integer sequence. \\ \\
0, 1, 1, 2, 3, 5, 8, 13, 21, 34, 55, 89, 144,... \\ \\
In mathematical terms, the sequence F n of Fibonacci numbers is defined by the recurrence relation: \\ \\
$F_n = F_{n-1} + F_{n-2}$ , where $F_0 = 0,F_1 = 1$ \\ \\
Discuss the correctness of the following algorithms that calculate Fibonacci numbers:
\begin{enumerate}
\item {Algorithm 1}
\begin{lstlisting}
fib1(n)
{
  integer array f[n+1];
  f[0] = 0;
  f[1] = 1;

  for (i = 2 to n)
    f[i] = f[i-1] + f[i-2]
  return f[n];
}
\end{lstlisting}
\item {Algorithm 2}
\begin{lstlisting}
fib2(n)
{
    if(n > 1)
        return fib2(n-1) + fib2(n-2)
    else if(n < 1)
        return fib2(n)
}
\end{lstlisting}
\item {Algorithm 3}
\begin{lstlisting}
fib3(n)
{
  a = 0 
  b = 1
  if(n = 0)
    return a
  for (i = 2 to n)
  {
	c = a + b
	b = c
	a = b
  }
  return b;
}
\end{lstlisting}
\end{enumerate}

\section{Normal Loop analysis (10pts)}

Give a tight bound (using the big-theta notation) on the time complexity of following method as a function of $n$. For simplicity, you can assume $n$ to be a power of two. Justify your answer.

\hfill \break

\begin{algorithmic}
\Function{NormalLoop}{n: \textbf{int}}
\State $i = n$

\While{$i \geq 1$}
    \For{$j = 1$ \textbf{to} $n$}
        \State $k = 1$
        \While{$k \leq n$}
            \State $k = 2 \times k$
        \EndWhile
    \EndFor
    \State $i = i - 1$
\EndWhile
\EndFunction
\end{algorithmic}

\section{Weird Loop analysis (20pts)}

Give a tight bound (using the big-theta notation) on the time complexity of following method as a function of $n$. For simplicity, you can assume $n$ to be a power of two. Justify your answer.

\hfill \break

\begin{algorithmic}
\Function{WeirdLoop}{n: \textbf{int}}
\State $i = n$

\While{$i \geq 1$}
    \For{$j = 1$ \textbf{to} $i$}
        \State $k = 1$
        \While{$k \leq n$}
            \State $k = 2 \times k$
        \EndWhile
    \EndFor
    \State $i = \lfloor i / 2 \rfloor$
\EndWhile
\EndFunction
\end{algorithmic}



\section{Design of water container (20pts)}
Given $n$ non-negative integers $a_1, a_2, ..., a_n$, where each represents a point at coordinate $(i, a_i)$. $n$ vertical lines are drawn such that the two endpoints of line i is at $(i, a_i)$ and $(i, 0)$. Find two lines, which together with x-axis forms a container, such that the container contains the most water. Here are some examples:

\begin{itemize}
\item Example 1:
\subitem Input: array = [1, 5, 4, 3]
\subitem Output: 6
\subitem Explanation : 5 and 3 are distance 2 apart. So the size of the base = 2. Height of container = min(5, 3) = 3. So total area = 3 * 2 = 6
\item Example 2:
\subitem Input: array = [3, 1, 2, 4, 5]
\subitem Output: 12
\subitem Explanation : 5 and 3 are distance 4 apart. So the size of the base = 4. Height of container = min(5, 3) = 3. So total area = 4 * 3 = 12
\end{itemize}

\begin{enumerate}
\item {Solve the problem with brute-force approach by providing the pseudo-code for this problem. What is the time complexity of this approach?}
\item {Is there any better solution? If so, describe your solution, including pseudo-code, with time complexity analysis.}
\end{enumerate}
{Note: You may not slant the container and $n$ is at least $2$.}



\section{Statement judgment (20pts)}
Is the following statement true or false?
\begin{center}
{$3^n = \Theta(2^n)$}
\end{center}
{Justify your answer using the basic definition of the $\Theta$-notation.}





\end{document}

