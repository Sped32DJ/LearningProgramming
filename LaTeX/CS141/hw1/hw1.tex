\documentclass{article}[12pt]
\usepackage{fullpage}
\usepackage{multicol}
\usepackage{amsmath}
\usepackage{amsfonts}
\usepackage{amsthm}
\usepackage{amssymb}
\usepackage{tikz}
\usepackage[ruled,lined,linesnumbered]{algorithm2e}
\usepackage{algpseudocode}
\usepackage{multicol}
\usepackage{enumitem}
\usepackage{url}
\usepackage{graphicx}
\usepackage{listings}

\ifx\solutiontemplate\undefined
\newcommand{\solkeyword}[1]{}
\newcommand{\policy}[1]{#1}
\else
\newcommand{\solkeyword}[1]{#1}
\newcommand{\policy}[1]{}
\fi

\newcommand\encircle[1]{\raisebox{.5pt}{\textcircled{\raisebox{-.9pt} {\footnotesize #1}}} }


\newenvironment{solution}[0]{\vspace{.1in} \textbf{Solution.} \,}{}

\newcommand{\deadline}{11:59pm, Tues 1/21, 2024}

\newcommand{\assigntitle}[1]{{
  \noindent \large \bf
  CS141, Winter 2024,
  Assignment \##1 \hfill Due: {\deadline}\\
  Name: Danny Topete %put in your name here
  \hspace{2.5in}
  Student ID: 862389624 %put in your id here
  \\
  [-.05in]
  \mbox{}\hrulefill \mbox{}\\}}

\begin{document}

\assigntitle{1}{Assignment Title}
\date{\today}

\section{A Complex Complexity Problem (20pts)}

\begin{center}
\begin{multicols}{4}
\begin{enumerate}[label=\encircle{\arabic*}]
  \item $(\sqrt{2})^{\log n}$
  \item $3^n$
  \item $n!$
  \item $n^2$
  \item $n^3$
  \item $\log^2n$
  \item $\log(n!)$
  \item $2^{2^n}$
  \item $\ln\ln n$
  \item $1$
  \item $\ln n$
  \item $\log \log n$
  \item $(n+1)!$
  \item $\sqrt{\log n}$
  \item $n$
  \item $2^n$
  \item $n\log n$
  \item $2^{2^n+1}$
  \item $5n^2-13n+6$
  \item $n/\log n$
\end{enumerate}
\end{multicols}
\end{center}

\begin{enumerate}[label=(\arabic*)]
  \item (1pt) What does \emph{polynomial} mean (use $O(\cdot)$, $\Omega(\cdot)$, $\omega(\cdot)$, $\Theta(\cdot)$, $o(\cdot)$ to present your answer)? (2pts) Which of them are polynomial?
    \begin{itemize}
      \item polynomial means anything that is f(n) \epsilon  \Theta($n^k$) k > 0.
        \begin{itemize}
          \item $f(n)$ \epsilon $O(n^k)$: $f(n)$ grows at most as fast as $n^k$
          \item $f(n)$ \epsilon $\Omega(n^k)$: $f(n)$ grows at least as fast as $n^k$
          \item $f(n)$ \epsilon $\Theta(n^k)$: $f(n)$ grows at the same rate as $n^k$
          \item $f(n)$ $\epsilon$  $o(n^k)$: $f(n)$ grows slower than $n^k$ as $n \rightarrow \infty$
          \item $f(n)$ \epsilon $\omega(n^k)$: $f(n)$ grows faster than $n^k$ as $n$ $\rightarrow \infty$
        \end{itemize}
      \item the equations that would be polynomial would be:
\begin{center}
\begin{multicols}{4}
\begin{enumerate}[label=\encircle{\arabic*}]
  \item $n$
  \item $n^2$
  \item $n^3$
  \item $5n^2-13n+6$
\end{enumerate}
\end{multicols}
\end{center}
    \end{itemize}

  \item (1pt) What does \emph{polylogarithmic} mean (use $O(\cdot)$, $\Omega(\cdot)$, $\omega(\cdot)$, $\Theta(\cdot)$, $o(\cdot)$ to present your answer)? (2pts) Which of them are polylogarithmic?
    \begin{itemize}
      \item polylogarithmic function is a function that grows 
        slower than any polynomial function with the logarithm as its input.
        Where $f(n) \epsilon \Theta(\log^k n)$, k > 0.
        \begin{itemize}
          \item $f(n)$ \epsilon $O(\log^k n)$: $f(n)$ grows at most as fast as $\log^k n$
          \item $f(n)$ \epsilon $\Omega(\log^k n)$: $f(n)$ grows at least as fast as $\log^k n$
          \item $f(n)$ $\epsilon$  $o(\log^k n)$: $f(n)$ grows slower than $\log^k n$ as $n \rightarrow \infty$
          \item $f(n)$ \epsilon $\omega(\log^k n)$: $f(n)$ grows faster than $\log^k n$ as $n$ $\rightarrow \infty$
          \item $f(n)$ \epsilon $\Theta(\log^k n)$: $f(n)$ grows at the same rate as $\log^k n$
        \end{itemize}
      \item Equations that follow:
\begin{center}
\begin{multicols}{4}
\begin{enumerate}[label=\encircle{\arabic*}]
  \item $\log^2n$
  \item $\log(n!)$
  \item $\ln\ln n$
  \item $\ln n$
  \item $\log \log n$
  \item $n\log n$
  \item $n/\log n$
\end{enumerate}
\end{multicols}
\end{center}

    \end{itemize}

\pagebreak

  \item (1pt) What does \emph{sublinear} mean (use $O(\cdot)$, $\Omega(\cdot)$, $\omega(\cdot)$, $\Theta(\cdot)$, $o(\cdot)$ to present your answer)? (2pts) Which of them are sublinear?
    \begin{itemize}
      \item Sublinear growth is $f(n)$ that grows slower than a linear function as n \rightarrow \infty
      \item Characterized by 
        \begin{equation}
          lim_{n \rightarrow \infty} \frac{f(n)}{n} = 0
        \end{equation}

        \begin{itemize}
          \item $f(n)$ \epsilon $O(n)$: $f(n)$ grows at most as fast as $n$
          \item $f(n)$ \epsilon $\Omega(n)$: $f(n)$ grows at least as fast as $n$
          \item $f(n)$ \epsilon $\Theta(n)$: $f(n)$ grows at the same rate as $n$
          \item $f(n)$ $\epsilon$  $o(n)$: $f(n)$ grows slower than $n$ as $n \rightarrow \infty$
          \item $f(n)$ \epsilon $\omega(n)$: $f(n)$ grows faster than $n$ as $n$ $\rightarrow \infty$
        \end{itemize}
      \item Equations that follow:
\begin{center}
\begin{multicols}{4}
\begin{enumerate}[label=\encircle{\arabic*}]
  \item $(\sqrt{2})^{\log n}$
  \item $\log^2n$
  \item $\log(n!)$
  \item $\ln\ln n$
  \item $\ln n$
  \item $\log \log n$
  \item $\sqrt{\log n}$
  \item $n/\log n$
\end{enumerate}
\end{multicols}
\end{center}

    \end{itemize}

  \item (1pt) What does \emph{superlinear} mean (use $O(\cdot)$, $\Omega(\cdot)$, $\omega(\cdot)$, $\Theta(\cdot)$, $o(\cdot)$ to present your answer)? (2pts) Which of them are superlinear?
    \begin{itemize}
      \item Superlinear growth is $f(n)$ that grows faster than a linear function as n \rightarrow \infty
      \item Characterized by
        \begin{equation}
          lim_{n \rightarrow \infty} \frac{f(n)}{n} = \infty
        \end{equation}
        \begin{itemize}
          \item $f(n)$ \epsilon $O(n)$: $f(n)$ grows at most as fast as $n$
          \item $f(n)$ \epsilon $\Omega(n)$: $f(n)$ grows at least as fast as $n$
          \item $f(n)$ \epsilon $\Theta(n)$: $f(n)$ grows at the same rate as $n$
          \item $f(n)$ $\epsilon$  $o(n)$: $f(n)$ grows slower than $n$ as $n \rightarrow \infty$
          \item $f(n)$ \epsilon $\omega(n)$: $f(n)$ grows faster than $n$ as $n$ $\rightarrow \infty$
        \end{itemize}
      \item equations that follow:
\begin{center}
\begin{multicols}{4}
\begin{enumerate}[label=\encircle{\arabic*}]
  \item $3^n$
  \item $n!$
  \item $n^2$
  \item $n^3$
  \item $\log(n!)$
  \item $2^{2^n}$
  \item $(n+1)!$
  \item $2^n$
  \item $n\log n$
  \item $2^{2^n+1}$
  \item $5n^2-13n+6$
\end{enumerate}
\end{multicols}
\end{center}
    \end{itemize}

    \pagebreak
  \item (2pts) Which of the above 20 functions grow the fastest (if you think there are multiple answers, list \emph{all} of them)?
    \begin{itemize}
      \item The functions that grow the fastest are:
        \begin{itemize}
          \item $2^{2^n}$
          \item $2^{2^n+1}$
        \end{itemize}
      \item these functions are double exponential
      \item Double exponential grows faster than factorial, logarithmic, polynomial, polylogarithmic, sublinear, and exponential functions
        (Which are the properties of the given functions)
      \item These two functions are the fastests growing functions and are both superlinear
    \end{itemize}


  \item (6pts) Finally, please rank all the functions by ascending order of growth; Group together those functions that are big-Theta of one another.
\begin{center}
\begin{multicols}{4}
\begin{enumerate}[label=\encircle{\arabic*}]
  \item $1$
  \item $\log \log n$
  \item $\ln\ln n$
  \item $\sqrt{\log n}$
  \item $n/\log n$
  \item $\ln n$
  \item $(\sqrt{2})^{\log n}$
  \item $\log^2n$
  \item $n$
  \item $\log(n!)$
  \item $n\log n$
  \item $n^2$
  \item $5n^2-13n+6$
  \item $n^3$
  \item $2^n$
  \item $3^n$
  \item $n!$
  \item $(n+1)!$
  \item $2^{2^n}$
  \item $2^{2^n+1}$
\end{enumerate}
\end{multicols}
\end{center}

\end{enumerate}

\newpage
\section{Calculate the Fibonacci numbers (20pts)}
In mathematical terms, the sequence F n of Fibonacci numbers is defined by the recurrence relation:


$F_n = F_{n-1} + F_{n-2}$ , where $F_0 = 0,F_1 = 1$ 


Discuss the correctness of the following algorithms that calculate Fibonacci numbers:
\begin{enumerate}
\item {Algorithm 1}
\begin{lstlisting}
fib1(n)
{
  integer array f[n+1];
  f[0] = 0;
  f[1] = 1;

  for (i = 2 to n)
    f[i] = f[i-1] + f[i-2]
  return f[n];
}
\end{lstlisting}
\begin{itemize}
  \item The first algorithm is valid
  \item It has a base case
  \item It uses memory to store previous values.
    Does not require recursion to keep the values in the stack.
  \item The array is a sufficient size and will not cause problems.
    You will need $n$ memory to store that entire array.
\end{itemize}
\item {Algorithm 2}
\begin{lstlisting}
fib2(n)
{
    if(n > 1)
        return fib2(n-1) + fib2(n-2)
    else if(n < 1)
        return fib2(n)
}
\end{lstlisting}
\begin{itemize}
  \item There would be no base case
  \item it will keep terminating since there is no terminating
    case
  \item no matter the value you insert, it will keep looping
\end{itemize}

\pagebreak
\item {Algorithm 3}
\begin{lstlisting}
fib3(n)
{
  a = 0 
  b = 1
  if(n = 0)
    return a
  for (i = 2 to n)
  {
	c = a + b
	b = c
	a = b
  }
  return b;
}
\end{lstlisting}
    \begin{itemize}
      \item This algorithm can not work
      \item in the for loop, $b$ and $a$ will be equal to $c$
        since a copy is never saved
    \end{itemize}

\end{enumerate}

\newpage
\section{Normal Loop analysis (10pts)}

\newpage
\section{Weird Loop analysis (20pts)}

\newpage
\section{Design of water container (20pts)}

\newpage
\section{Statement judgment (20pts)}
Is the following statement true or false?
\begin{center}
{$3^n = \Theta(2^n)$}
\end{center}
{Justify your answer using the basic definition of the $\Theta$-notation.}

\begin{itemize}
  \item This would not be true
\end{itemize}

\newpage
\begin{thebibliography}{99}
\bibitem{ref1} This definition is from...
\end{thebibliography}
\end{document}
