\documentclass{article}[12pt]
\usepackage{fullpage}
\usepackage{multicol}
\usepackage{amsmath}
\usepackage{amsfonts}
\usepackage{amsthm}
\usepackage{amssymb}
\usepackage{tikz}
\usepackage[ruled,lined,linesnumbered]{algorithm2e}
\usepackage{algpseudocode}
\usepackage{multicol}
\usepackage{enumitem}
\usepackage{url}
\usepackage{graphicx}
\usepackage{listings}

\ifx\solutiontemplate\undefined
\newcommand{\solkeyword}[1]{}
\newcommand{\policy}[1]{#1}
\else
\newcommand{\solkeyword}[1]{#1}
\newcommand{\policy}[1]{}
\fi

\newcommand\encircle[1]{\raisebox{.5pt}{\textcircled{\raisebox{-.9pt} {\footnotesize #1}}} }


\newenvironment{solution}[0]{\vspace{.1in} \textbf{Solution.} \,}{}

\newcommand{\deadline}{11:59 PM, 2/18/2025}

\newcommand{\assigntitle}[1]{{
  \noindent \large \bf
  CS141, Winter 2025,
  Assignment \##1 \hfill Due: {\deadline}\\
  Name: Danny Topete  %put in your name here
  \hspace{2.5in}
  Net ID: dtope004   %put in your id here
  \\
  [-.05in]
  \mbox{}\hrulefill \mbox{}\\}}

\begin{document}

\assigntitle{3}{}
\policy{\textbf{Deadline.} The homework is due 11:59 PM, 2/18/2025. You must submit your solutions (in pdf
format generated by LaTeX) via GradeScope. Canvas will not be accepted.}\\
\date{}
\policy{\textbf{Assignment Policy.}
\begin{itemize}
    \item Unless mentioned otherwise, late submission allowed for 20\% penalty for each calendar day.
    \item Assignments should be in pdf format generated by LaTeX
    \item If you are using any external source, you must cite it and clarify what exactly got out of it.
    \item You are expected to understand any source you use and solve problems in your own.
    \item All pages relevant to a question must be assigned on Gradescope. Unassigned pages will not be graded.
    \item Unless stated in the question, your designed algorithm should use the most optimized algorithm you have learned so far. A too-complex algorithm may result in penalties.
\end{itemize}
}

\section{Make A and B equal. (20 points)}

Given two strings A and B, of length $m$ and $n$, we can defined the distance between the two strings as the number of \textit{insert()}, \textit{delete()}, and \textit{replace()} operations that are performed on A to make it identical to B. 

\noindent\textbf{Your Tasks:}
\begin{enumerate}
    \item Design a dynamic programming algorithm that returns the minimum number of operations to me A identical to B.
    \item Describe the Size and Dimensionality of the dynamic programming table
    \item Write the recurrence relationship for algorithm in the form
    \[
    dp[i][j] = \begin{cases}
        ...
    \end{cases}
    \]
\end{enumerate}
\noindent\textbf{Technical Note:} Use "\textbackslash begin\{cases\}" to write your recurrence relationship.

\begin{algorithm}[H]
 \caption{Edit Distance}
\KwIn{Two strings \( A \) of length \( m \) and \( B \) of length \( n \)}
\KwOut{The minimum number of operations to convert \( A \) to \( B \)}

\SetKwFunction{FMain}{EditDistance}
\SetKwProg{Fn}{Function}{:}{}
\Fn{\FMain{$A$, $B$}}{
    Let \( m \) be the length of \( A \) \;
    Let \( n \) be the length of \( B \) \;
    Create a 2D array \( dp \) of size \( [m+1] [n+1] \) \;


    \Comment{Initialize the first row and first column}

    \For{$i = 0$ \textbf{to} $m$}{
        \( dp[i][0] = i \) \;
    }
    \For{$j = 0$ \textbf{to} $n$}{
        \( dp[0][j] = j \) \;
    }

    \Comment{Fill the rest of the table}

    \For{$i = 1$ \textbf{to} $m$}{
        \For{$j = 1$ \textbf{to} $n$}{
          \Comment{Nested loop to go through each character of A and B}

            \If{$A[i-1] = B[j-1]$}{
                \( dp[i][j] = dp[i-1][j-1] \) \;
            }
            \Else{
                \( dp[i][j] = \min \left( dp[i-1][j] + 1, dp[i][j-1] + 1, dp[i-1][j-1] + 1 \right) \) \;
            }
        }
    }
    \Return{$dp[m][n]$} \;
}
\end{algorithm}

\begin{itemize}
  \item The size of the dynamic programming table is \( (m+1) \times (n+1) \).
  \item The dimensionality of the table would be 2D
  \item Recurrence Relationship:
\end{itemize}

\begin{math}
  f(n) = \begin{cases}
    0 & \text{if } n = 0 \\
    1 & \text{if } n = 1 \\
    f(n-1) + f(n-2) & \text{if } n > 1
  \end{cases}
\end{math}


\newpage
\section{The Matrix Chain Multiplication Problem. (20 points)}
Design a \textbf{recursive version} of dynamic programming algorithm (top-down) to construct the actual solution of the matrix chain multiplication problem (i.e., the parentheses order). For this problem, please :
\noindent\textbf{Your Tasks:}
\begin{enumerate}
\item Write down the recursive function in pseudocode and write down the dynamic table and matrix output on the following examples
\item Show the dynamic programming table and matrix output for 
    \begin{enumerate}
        \item  Three matrices (A, B, and C) with dimensions $10 \times 50 \times 5 \times 100$, respectively.
        \item Four matrices (A, B, C, and D) with dimensions $20 \times 5 \times 10 \times 30 \times 10$, respectively.
    \end{enumerate}

\end{enumerate}

\begin{algorithm}[H]
\caption{Matrix Chain Order}
\KwIn{A matrix chain of length \( m \)}
\KwOut{The minimum number of operations to multiply the matrix chain}

\SetKwFunction{FMain}{MatrixChainOrder}
\SetKwProg{Fn}{Function}{:}{}
\Fn{\FMain{$p$}}{
  \Comment {Let $p$ be the array of dimensions of the matrices}

    $n$ = length of $p - 1$ \;
    Let $m$[1..n,1,..n] and $s$[1..n-1,2..n-1] be new tables \;

    \For{$i = 1$ \textbf{to} $n$}{
        \( m[i][i] = 0 \) \;
    }
    \For{$l = 2$ \textbf{to} $n$}{
      \For{$i = 1$ \textbf{to} $n - l + 1$}{
        $j = i + l - 1$ \;
        $m[i][j] = \infty$ \;
        \For{$k = i$ \textbf{to} $j - 1$}{
          $q = m[i][k] + m[k+1][j] + p_{i-1} * p_{k} * p_{j}$ \;
          \If{$q < m[i][j]$}{
            \Comment $m$ is the minimum number of operations to multiply the matrix chain

            $m[i][j] = q$ \;
            \Comment $s$ is the matrix output

            $s[i][j] = k$ \;
          }
        }
      }
    }
    \Return{$m$ and $s$} \;
}
\end{algorithm}


\pagebreak
\begin{enumerate}
  \item Three matrices (A, B, and C) with dimensions 
    $10 \times 50 \times 5 \times 100$, respectively.
    \begin{itemize}
      \item $p = [10, 50, 5, 100]$
      \item $n = 3$
      \item $m$ =
        $\begin{Bmatrix}
          0 & 2500 & 7500 \\
          0 & 0 & 25000 \\
          0 & 0 & 0 \\
        \end{Bmatrix}$
      \item $s$ =
        $\begin{Bmatrix}
          0 & 1 & 1 \\
          0 & 0 & 2 \\
          0 & 0 & 0 \\
        \end{Bmatrix}$
    \end{itemize}
  \item Four matrices (A, B, C, and D) with dimensions
    $20 \times 5 \times 10 \times 30 \times 10$, respectively.
    \begin{itemize}
      \item $p = [20, 5, 10, 30, 10]$
      \item $n = 4$
      \item $m$ =
        $\begin{Bmatrix}
          0 & 1000 & 2500 & 4500 \\
          0 & 0 & 1500 & 3500 \\
          0 & 0 & 0 & 3000 \\
          0 & 0 & 0 & 0
        \end{Bmatrix}$

      \item $s$ =
        $\begin{Bmatrix}
          0 & 1 & 2 & 3 \\
          0 & 0 & 2 & 3 \\
          0 & 0 & 0 & 3 \\
          0 & 0 & 0 & 0
        \end{Bmatrix}$

    \end{itemize}
\end{enumerate}

\pagebreak

\section{The Largest Square Block. (20 points)}
Let A be a $n$ x $m$ matrix of 0's and 1's. 
\textbf{Design a dynamic programming \textit{O(nm)} time algorithm} for 
finding the largest square block of A that contains 1's only. 
When designing the dynamic programming algorithm, please describe the size and dimensionality 
of the dynamic programming table, and explicitly formulate the recurrence relationship. 


\noindent\textbf{Hint:} Define the dynamic programming table l(i, j) be the length of the side of the largest square block of 1's whose bottom right corner is A[i, j].


\begin{algorithm}[H]
\caption{Largest Square Block}
\KwIn{$A$ be a matrix of 0's and 1's with dimensions $n \times m$}
\KwOut{The length of the side of the largest square block of 1's}

\SetKwFunction{FMain}{LargestSquareBlock}
\SetKwProg{Fn}{Function}{:}{}
\Fn{\FMain{$A$}}{
  Let $A$ be the matrix of 0's and 1's

  Let $l(i, j)$ be the length of the side of the largest square block of 1's whose bottom right corner is A[i, j]

  \Comment{Let $n$ be the number of rows in $A$}

  $n$ = A.length()
  \Comment{Let $m$ be the number of columns in $A$}

  $m$ = A[0].length()

  \Comment{Fill DP table}

  \For{$i = 0$ \textbf{to} $n$}{
    \For{$j = 0$ \textbf{to} $m$}{
      \If{$A[i][j] = 0$}{
        $l(i, j) = 0$
      }
      \ElseIf{$i = 0$ or $j = 0$}{
        $l(i, j) = 1$
      }
      \Else{
        \Comment{Recurrence Relationship:}

        $l(i, j) = \min(l(i-1, j), l(i, j-1), l(i-1, j-1)) + 1$
      }
    }
  }
  \Return{$l(n-1, m-1)$} \;

}
\end{algorithm}

\begin{itemize}
  \item The size of the dynamic programming table is $n \times m$.
  \item The dimensionality of the table would be 2D
  \item Recurrence Relationship:
  \item $l(i, j)$ = $\begin{cases}
    0 & \text{if } A[i, j] = 0 \\
    \min(l(i-1, j), l(i, j-1), l(i-1, j-1)) + 1 & \text{if } A[i, j] = 1
  \end{cases}$
\end{itemize}

\pagebreak

\section{RNG (20 points) }

Given $n$ enumerated random number generators each with m possible outputs, numbered from 1 to m, \textbf{design a dynamic programming algorithm} to find the number of ways our random number generators can reach a desired sum $ds$, where $ds$ is the sum of values each generator outputs.

When designing the dynamic programming algorithm, please describe the size and dimensionality of the dynamic programming table, and write the recurrence relationship for algorithm in the form
    \[
    dp[i][j] = \begin{cases}
        ...
    \end{cases}
    \]

\begin{algorithm}[H]
\caption{RNG}
\KwIn{$n$ enumerated random number generators 
each with m possible outputs, numbered from 1 to m, and a desired sum $ds$}
\KwOut{The number of ways our random number generators can reach a desired sum $ds$}

\SetKwFunction{FMain}{RNG}
\SetKwProg{Fn}{Function}{:}{}
\Fn{\FMain{$n, m, ds$}}{
  Let $dp$ be the matrix
  
  \Comment{Initialize DP table with 0s}

  \For{$i = 0$ \textbf{to} $n$}{
    \For{$j = 0$ \textbf{to} $ds$}{
      $dp$[$i$][$j$] = 0
    }
  }

  \Comment{base case; if sum is 0, there is 1 way to achieve it}

  $dp$[0][0] = 1

  \For{$i = 0$ \textbf{to} $n$}{
    \For{$j = 0$ \textbf{to} $m$}{
      \For{$k = 0$ \textbf{to} $ds$}{
        \If{$k \geq j$}{
          $dp$[$i$][$k$] += $dp$[$i-1$][$k-j$]
        }
      }
    }
  }
  \Comment{Return the number of ways to achieve the desired sum}

  \Return{$dp$[$n$][$ds$] \;}

}
\end{algorithm}

\begin{itemize}
  \item The size of the dynamic programming table is $n \times ds$.
  \item The dimensionality of the table would be 2D
  \item Recurrence Relationship:
  \item $dp$[$i$][$j$] = 
    $\begin{cases}
      1 & \text{if } j = 0 \text{ and } i = 0 \\
    0 & \text{if } i = 0 \text{ and } j > 0 \\
    dp$[$i-1$][$j$] + dp$[$i$][$j-1$] & \text{if } i > 0 \text{ and } j > 0
  \end{cases}$
\end{itemize}

\newpage
\section{Auctioning Server Space (20 points)}

The drug discovery startup is not working out, Michael and Xianghu are starting a business as server hosts -- however there is no seed funding so they're making clients bring their own hardware. They have a single $n$U server rack (a server rack with $n$ standard units of space). They plan to auction off space in the server space as a means of making money. Servers can be any size from 1U to $n$U, but there can only be $k$ servers per rack due to power limitations. Clients will pay based on the size of space they're allocated, which is stored in a a given array $P[1...n]$, $P_i$ denotes the price a client will pay for a $i$ units of server space.

\noindent\textbf{Your Tasks:}
\begin{enumerate}
    \item Design a dynamic programming algorithm to maximize the amount of money Michael and Xianghu will make.
    \item Give the time complexity of a brute-force solution that does not use dynamic programming or memoization.
    \item Give the time complexity of your dynamic programming implementation
    \item Describe how you could change the algorithm to give the actual divisions to maximize profit
\end{enumerate}

\begin{algorithm}[H]
\caption{Auctioning Server Space}
\KwIn{$n$ units of server space, $k$ servers per rack, and a given array $P[1...n]$}
\KwOut{Calculate the maximum amount of money Michael and Xianghu will make}

\SetKwFunction{FMain}{MaxProfit}
\SetKwProg{Fn}{Function}{:}{}
\Fn{\FMain{$n, k, P $}}{
  Let $dp$ \textbf{and} parent be the matrix
  
  \Comment{Initialize DP table with 0s}

  \For{$i = 0$ \textbf{to} $n+1$}{
    \For{$j = 0$ \textbf{to} $k+1$}{
      $dp$[$i$][$j$] = $- \infty$
    }
  }

  \Comment{Initialize parent table with 0s; backtracking table}

  \For{$i = 0$ \textbf{to} $n+1$}{
    \For{$j = 0$ \textbf{to} $k+1$}{
      $parent$[$i$][$j$] = $0$
    }
  }

  \Comment{base case}

  $dp$[0][0] = 0



  \For{$i = 0$ \textbf{to} $n$}{
    \For{$j = 0$ \textbf{to} $m$}{
      maxV = $-\infty$

      maxServer = 0
      \Comment{s can be from 1 to $(j - i + 1)$}

      \For {$s = 1$ \textbf{to} $(j - i + 1)$}{

        prevJ = $j-s$

        \If{$dp[i-1][prevJ] + P[s] $ > $maxV$ \textbf{and} $prevJ \geq 0$}{

          $maxV$ = $dp[i-1][prevJ] + P[s]$

          $maxServer$ = $s$

          dp[i][j] = $maxV$

          parent[i][j] = $prevJ$

          \Comment{Recurrence Relationship:}

          $maxV$ = max($maxV, dp[i-1][prevJ] + P[s]$)
        }

        \Comment{Recurrence Relationship:}

        $maxV$ = max($maxV, dp[i-1][j-s] + P[s]$)
      }
    }
  }
  \Comment{Return the maximun profit}

  \If{$dp$[$n$][$k$] == $-\infty$}{
    \Return{0}
  }

  let $servers$ be a matrix holding the servers

  currJ = $n$

  \For{$i = n$ \textbf{to} $1$}{
    $servers$[$i$] = $currJ - parent$[$i$][$currJ$]
    $currJ$ = $parent$[$i$][$currJ$]
  }
  \Comment{Return the maximun profit and the servers}

  \Return{$dp$[$k$][$n$], servers[] \;}

}
\end{algorithm}

\pagebreak

\begin{enumerate}
  \item The alogorithm is provided above
  \item The time complexity of a brute-force solution that does not use dynamic programming
    would be\\
    $O$
    $\begin{pmatrix}
      n-1 \\
      k-1 \end{pmatrix}$

    \begin{itemize}
      \item This is because all combinations of $k$ servers summing to $n$ must be checked.
        Which can be calculated using stars and bars formula.
      \item This would make the time complexity factorial once you break apart the combinations.
    \end{itemize}
  \item The time complexity of the dynamic programming implementation is $O(k\cdot n^{2})$
    \begin{itemize}
      \item There are $O$($k \cdot n$) states in the DP table
      \item Each state takes $O(n)$ time to compute
      \item Then reference to the previous state takes $O(1)$ time
    \end{itemize}
  \item Modifying the algorithm to output the necessary divisions to maximize profit,
    we can add a backtracking table to keep track of the previous state.
    Allowing us to backtrack from the final state to the initial state
    that were previously used to maximize profit.
    \begin{itemize}
      \item The backtracking table is initialized to 0s.
      \item The backtracking table is updated in the same loop as the dynamic programming table.
      \item The backtracking table is used to backtrack from the
        final state to the initial state to preserve the necessary divisions that will maximize profit.
    \end{itemize}
\end{enumerate}

\pagebreak
\section{Subset Sum (20 points)  }
Let $A = \left\{a_1 , a_2 ,. . . , a_n \right\}$ and be a set of n positive integer and let \textit{T} be another integer. \textbf{Design a dynamic programming algorithm} that determines whether there exists a \textit{subset of A} whose total sum is \textit{exactly T}. Analyze the time complexity
of your solution.

For instance, if A = \{4, 5, 17, 23, 11, 2\} and T = 35 the algorithm should return True because the subset \{5, 17, 11, 2\} sums to 35. For the same set of numbers if we choose T = 31 the problem has no solution, and the algorithm will return False.

When designing the dynamic programming algorithm, please describe the size and dimensionality of the dynamic programming table, and explicitly formulate the recurrence relationship. 



\end{document}

