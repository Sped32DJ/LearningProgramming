\documentclass{article}[12pt]
\usepackage{fullpage}
\usepackage{multicol}
\usepackage{amsmath}
\usepackage{amsfonts}
\usepackage{amssymb}
\usepackage{tikz}
\usepackage[ruled,lined,linesnumbered]{algorithm2e}
\usepackage{algpseudocode}
\usepackage{multicol}
\usepackage{enumitem}
\usepackage{url}
\usepackage{graphicx}
\usepackage{listings}
%\usepackage{mathtools}
\usepackage{forest}

\ifx\solutiontemplate\undefined
\newcommand{\solkeyword}[1]{}
\newcommand{\policy}[1]{#1}
\else
\newcommand{\solkeyword}[1]{#1}
\newcommand{\policy}[1]{}
\fi

\newcommand\encircle[1]{\raisebox{.5pt}{\textcircled{\raisebox{-.9pt} {\footnotesize #1}}} }


\newenvironment{solution}[0]{\vspace{.1in} \textbf{Solution.} \,}{}

\newcommand{\deadline}{11:59 PM, 3/15/2025}

\newcommand{\assigntitle}[1]{{
  \noindent \large \bf
  CS141, Winter 2025,
  Assignment \##1 \hfill Due: {\deadline}\\
  Name: Danny Topete %put in your name here
  \hspace{2.5in}
  Net ID: dtope004 %put in your id here
  \\
  [-.05in]
  \mbox{}\hrulefill \mbox{}\\}}

%%%%%%%%%%%%%%%%%%%%%%%%%%%%%%%
% Very Important: Promote subsection to section %%%
%%%%%%%%%%%%%%%%%%%%%%%%%%%%%%%
\let\subsection\section

\def\isinput{} % Exclude preamble in questions
\begin{document}

\assigntitle{5}{}
\policy{\textbf{Deadline.} The homework is due 11:59 PM, 3/4/2025. You must submit your solutions (in pdf
format generated by LaTeX) via GradeScope. Canvas will not be accepted.}\\
\date{}
\policy{\textbf{Assignment Policy.}
\begin{itemize}
    \item Unless mentioned otherwise, late submission allowed for 20\% penalty for each calendar day.
    \item Assignments should be in pdf format generated by LaTeX
    \item If you are using any external source, you must cite it and clarify what exactly got out of it.
    \item You are expected to understand any source you use and solve problems in your own.
    \item All pages relevant to a question must be assigned on Gradescope. Unassigned pages will not be graded.
    \item Unless stated in the question, your designed algorithm should use the most optimized algorithm you have learned so far. A too-complex algorithm may result in penalties.
\end{itemize}


}


\section{Reverse Delete (20 points)}
\begin{itemize}
  \item \textbf{Reverse Delete} is not the most efficient algorithm for computing the minimum spanning tree of a graph. 
    However, it does compute the optimal MST.
  \item An MST has the following characteristics, it is an acyclic graph with $n-1$ edges,
    it is a tree with minimal cost edges that is yet connected.
  \item The Reverse Delete algorithm is removing the more expensive edges first, as long as it doesn't disconnect the graph.
\end{itemize}

\newpage
\section{Water Transport (20 points)}
\begin{enumerate}
  \item Representing problem as a graph problem: 
    \begin{itemize}
      \item We can begin with 3 nodes, $A$, $B$, and $C$. They form a complete graph.
      \item We would initialize these nodes as $A$ = 0/10, $B$ = 4/4, and $C$ = 7/7.
        With an included capacity and current weight.
      \item Now we can transfer the weights to each other until they have reached their max capacity, or the source node is empty
    \end{itemize}
  \item An algorithm we can use to find the solution would be
    \item This problem does have a solution, there are two approaches: \\ 
      \textbf{1.} You transfer two weights from $B$ to $A$, resulting in $B$ = 2/4 and $A$ = 2/10.  \\
      \textbf{2.} You transfer five weights from $C$ to $A$, resulting in $C$ = 2/7 and  $A$ = 5/10.
\end{enumerate}

\newpage
\section{Huffman Encoding (20 points)}
\begin{itemize}
  \item We can begin the Huffman encoding by grabbing the character frequency
    of the string "abracadabra".
  \item Frequency of characters: a = 5, b = 2, r = 2, c = 1, d = 1
%  \item The length of the code would be $\ceil[\big]{log_2(n)}$, where $n$ is the number of characters in the string.
\end{itemize}

\begin{forest}
for tree={align=center, l sep=20pt, s sep=15pt, edge={->}, parent anchor=south, child anchor=north}
[11
  [a\\(5) \\ Code: 0]
  [6
    [2
      [c\\(1) \\ Code: 100]
      [d\\(1) \\ Code: 101]
    ]
    [4
      [b\\(2) \\ Code: 110]
      [r\\(2) \\ Code: 111]
    ]
  ]
]
\end{forest}
\begin{itemize}
  \item a = 5*1 = 5
  \item b = 2*3 = 6
  \item r = 2*3 = 6
  \item c = 1*3 = 3
  \item d = 1*3 = 3
  \item The total length of encoding "abracadabra" would be length of 23 bits.
\end{itemize}

\newpage
\section{Bipartite Graphs (20 points)}
% Bonus Qs
\newpage
\section{Infrastructure (10 point bonus)}

\newpage
\section{Boolean Matrix (10 point bonus)}

\newpage
\section{Two Parts (10 point bonus)}

\end{document}

