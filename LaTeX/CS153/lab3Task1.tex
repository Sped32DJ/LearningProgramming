\documentclass{article}

\title{Lab 3 Task 1}
\date{May 23, 2025}
\author{Danny Topete\\ CS 153, Prof. Song}

\begin{document}
\maketitle

\begin{itemize}
  \item Explain how $walkphdir$ defined in $vm.c$ works.
    Specifically, what are $PDX$, $PTX$, $PTE\_E$, $PTE\_ADDR$,
    $PTE\_W$, and $PTE\_U$?
    \begin{itemize}
      \item $walkphdir$ is a function that locates the Page Table Entry (PTE)
        for a given Virtual Address (va) in a given page
        directory (pgdir).
        If the relevant page is not found, it allocates the memory for a
        new page table.
      \item $PTE\_E$, Checks if the page table is present
      \item $PDX$, Extracts the page directory index
      \item $PTX$, extracts the page table index
      \item $PTE\_P$, Page Table Entry Present, indicates if the page is present in memory
      \item $PTE\_W$, PTE Writable, indicates if page is writable
      \item $PTE\_U$, PTE User flag, indicates if page is accessible from user mode
      \item $PTE\_ADDR$, extracts the physical address of the page table entry
    \end{itemize}

    \begin{enumerate}
      \item Find Page Directory Entry,

        \begin{verbatim}
        pde = &pgdir[PDX(va)];
        \end{verbatim}

      \item Check if Page Table Exists

        \begin{verbatim}
if(*pde & PTE_P){
  pgtab = (pte_t*)P2V(PTE_ADDR(*pde));
}
        \end{verbatim}

      \item Allocate a New Page Table if needed
        \begin{verbatim}
else {
  if(!alloc || (pgtab = (pte_t*)kalloc()) == 0)
    return 0;
  memset(pgtab, 0, PGSIZE);
  *pde = V2P(pgtab) | PTE_P | PTE_W | PTE_U;
}        \end{verbatim}

\item Return the Address of the PTE
  \begin{verbatim}
  return &pgtab[PTX(va)];
  \end{verbatim}

    \end{enumerate}

  \item Reason about $struct kmap$ and $setupkvm$,
    explain how $P2V$ and $V2P$ work?
    Specifically, why add and subtract $KERNBASE$ can convert physical address
    to/from kernel virtual address?
  \item Based on the above understanding, explain how $uva2ka$ maps user
    virtual address to kernel address.
\end{itemize}

\end{document}

