\documentclass[12pt]{article}

%
%Margin - 1 inch on all sides
%
\usepackage[letterpaper]{geometry}
\usepackage{times}
\usepackage{lmodern}
\geometry{top=1.0in, bottom=1.0in, left=1.0in, right=1.0in}

%
%Doublespacing
%
\usepackage{setspace}
\doublespacing

%
%Rotating tables (e.g. sideways when too long)
%
\usepackage{rotating}


%
%Fancy-header package to modify header/page numbering (insert last name)
%
\usepackage{fancyhdr}
\pagestyle{fancy}
\lhead{}
\chead{}
\rhead{Topete \thepage}
\lfoot{}
\cfoot{}
\rfoot{}
\renewcommand{\headrulewidth}{0pt}
\renewcommand{\footrulewidth}{0pt}
%To make sure we actually have header 0.5in away from top edge
%12pt is one-sixth of an inch. Subtract this from 0.5in to get headsep value
\setlength\headsep{0.333in}

%
%Works cited environment
%(to start, use \begin{workscited...}, each entry preceded by \bibent)
% - from Ryan Alcock's MLA style file
%
\newcommand{\bibent}{\noindent \hangindent 40pt}
\newenvironment{workscited}{\newpage \begin{center} Works Cited \end{center}}{\newpage }


%
%Begin document
%
\begin{document}
\begin{flushleft}

%%%%First page name, class, etc
Danny Topete\\
Professor Miyake\\
TEST 005\\
\today\\


%%%%Title
\begin{center}
Reading Journal 3
\end{center}


%%%%Changes paragraph indentation to 0.5in
\setlength{\parindent}{0.5in}
%%%%Begin body of paper here

For this Journal, I will focus on the week 8 readings with a focus on \textit{Brown} by Nitasha Tamar Sharma
and the reading \textit{Terrorism} by Rajini Srikanth. These readings highlight the connection between
the meaning of words and the historical contexts in which they are used.
In the case of \textit{Brown}, Sharma discusses how the term
to describe people of color has changed over time, particularly in the context of the United States.
Then the reading \textit{Terrorism} by Srikanth discusses the term has originated in the French Revolution
and how more recently it has shifted to a new meaning.

% Quote 1
"Since 9/11, both the Wars on Terror and the Arab Spring have shifted
yet again the always under-construction lines of "Brown," so that it now
refers to people from the Middle East and North Africa and more broadly to the religion of Islam...
Thus, "Brown" as a reference to a people's phenotype, like "Black," is not merely descriptive or U.S. based - it
is political and global." (Sharma, 19)

The Wars on Terror were tragic to everyone involved in it.
It led to many civilians being displaced and killed.
All the instances of bombing targeting exclusively innocent people
because there was a "slight chance" that their target was there.
The current political climate decides who will be included into the
evolving category of "Brown".
The word has only ever been used to describe people who are a minority in the US,
especially to describe perpetual foreigners.

% Quote 2
"Within the United States, the Ku Klux Klan terrorized
African Americans (Cunningham 2008) in the late
19th and first half of the 20th centuries...
 However, the word “terrorism” forcefully
entered public consciousness with the 1993 bombing of
the World Trade Center and the 1995 bombing of the
Oklahoma federal building. These two attacks spurred
Congress to pass the 1996 Anti-Terrorism and Effective
Death Penalty Act (AEDPA), which introduced sanctions
leading to deportation of noncitizens convicted
of felonies, regardless of whether they were committed
when the person was a minor and regardless of any time
served." (Srikanth, 230)

I find this quote interesting since the paragraph begins with discussing
America's first terrorist group, the KKK, and how they serve to oppress minority groups.
Then to discussing it wasn't until this act against the government,
that they started taking terrorism seriously.
The reading highlights how terrorism has always existed,
the US just never attached this label until the bombing and terrorist attacks of 9/11.

This act goes to show how the government decides what is categorized as terrorism.
Recently, as in, most 9/11, "terrorism" is a word that has been exclusively used by "brown" people.
Putting this label has been dangerous due to this act, given the AEDPA.
Recently, Trump has extended the war on drugs onto Cartels located in Latin America.
Cartels being labeled as "terrorist" by President Trump is scary, especially to people
living in Mexico. This gives the US freedom to enter and invade Mexico.
Historically, civilians have always been the victim of US invasion.
A great example can be the 20-year-long war in Afghanistan that came
after the 9/11 attacks.


% Scrap this shit below
The following quote is a reflection of how no one is immune to terrorism.
By the government and the way they label threats, we are led to believe
that terrorism only happens when brown people act out on "acts of terror".
Many times, these "acts of terror" are state sponsored, and they were put in place
to oppress minorities.
"There is a debate over whether the word "terrorism" can be applied
to violent actions by governments on citizens of their own or other nations...
in principle, anyone can commit terrorist acts; it is important, therefore, to focus
on the act itself and not the actor." (Srikanth, 229)


\newpage


%%%%Works cited
\begin{workscited}

\bibent
Allen, R.L. \textit{The American Farm Book; or Compend of Ameri can Agriculture; Being a Practical Treatise on Soils, Manures, Draining, Irrigation, Grasses, Grain, Roots, Fruits, Cotton, Tobacco, Sugar Cane, Rice, and Every Staple Product of the United States with the Best Methods of Planting, Cultivating, and Prep aration for Market.} New York: Saxton, 1849. Print.

\bibent
Baker, Gladys L., Wayne D. Rasmussen, Vivian Wiser, and Jane M. Porter. \textit{Century of Service: The First 100 Years of the United States Department of Agriculture.}[Federal Government], 1996. Print.

\bibent
Danhof, Clarence H. \textit{Change in Agriculture: The Northern United States, 1820-1870.} Cambridge: Harvard UP, 1969. Print.


\end{workscited}

\end{flushleft}
\end{document}
\}
