\documentclass{article}
\title{Glossary of Terms}
\author{Danny Topete}

\begin{document}
  
\maketitle

From Discussion 1
\begin{enumerate}
  \item Race:
    \begin{itemize}
      \item A social construct:
      \item Fluid  definition
      \item Not ethnicity
      \item Races:
        White/caucaian, Native, Asian, African/Black
      \item Not a race: (is an ethnicty)
        Latino, Hispanic
      \item Both race and ethnicity can easily be confused since we don't 
        really know the differences
      \item Scientific racism was used to catergorized
        to exclude people.
        Always used to benbefit white people while putting
        down other people
      \item Race changes based off what society deems,
        It is fluid and is socially constructed. 
        So it can change overtime.
      \item A way to catergorize people based off race
    \end{itemize}
    Race is a social construct that originated
    as an attempt at categorize people 
    based on their physical appearances and
    characteristics. It is a fluid definition.
    The race of a person can changed based
    upon current events and has changed
    to benefit or oppress a group of people.
    Then there are rules such as the one drop rule,
    which don't make sense and go to show
    that race was made in order to oppress people
    in order to bring up the white race.

  \item Ethnicity
    \begin{itemize}
      \item Cultural heritage and tradition
      \item It may even be with who people hang around with
    \end{itemize}
    Ethnicity is the the cultural heritage and traditions
    from which a person comes from.
    It can even be the group of people
    that a person hangs around with,
    or to be more exact, possibly
    grew up with. It is a label
    that we put upon people who
    originated from a geographical locations.
  \item Nationality
    \begin{itemize}
      \item The country where people would come from
    \end{itemize}
  \item Intersectionality
    \begin{itemize}
      \item Analyzing the different approaches
        of how people take on different identities
      \item The multiple things that describe
        out identity and taking the one that we have
        for our benefit
      \item We show that for example, white women had their
        suffrage movement first,
        then leaving out black women. Who
        got their right to vote decades later.
      \item It can birth oppression olympics of gay men comparing their 
        oppression to a black man. They are both oppressed in different
        way.
        Or saying how people are oppressed because they are
        black, woman, and gay. 
        Therefore they deserve more in the society.
      \item The idea is understanding that we are all as a people, all
        fucked up by the white men at the top of the 
        pyramid
      \item It is not about oppression olympics, but about
        understanding that we are all oppressed in different ways.
        And that we need to work together to fight against the 
        oppression.
      \item If one people are oppressed, then we are all oppressed,
        we must lift each other up.
        All to win in this system.
    \end{itemize}
    The original author, Kimberle Crenshaw, coined the term
    to describe the uneven complex and uneven ways that the law
    and social power operated operated to grant people
    different experiences of dominant legal and political discourse.
    This would be the combinations of social
    identities that we give to a person that decides
    the way society treats them. Along with the law
    may discriminate against them due to a combination
    of their identities; being race, ethnicity, nationality, sex, and sexual orientation.
\end{enumerate}


\end{document}

