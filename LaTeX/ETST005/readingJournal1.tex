\documentclass{article}
\title{Week 1b Reading}
\author{Danny Topete\\ ETST005 \\ Professor Miyake}
\date{April 3, 2025}

\begin{document}
\maketitle

\section{ch51 Race - Junaid Rana}
"Race is a social construction in which biology
and culture are often conflated as a rhetorical logic and
material practice in a system of domination"(202)

\subsection{Race as a social construct}

"In the barest scholarly definition, race is a social
construct."(203)

"Race is inextricably a concept of the
modern episteme, intertwined in systems of imperial-
ism, colonization, capitalism, and social structure that
emerged out of the European Enlightenment (Goldberg
1993; Mills 1997; Silva 2007; Winant 2001)" (203)

"The inequality at the center of racism and white
supremacy is based on the enduring power of race as a
flexible and shifting category."(203)

Racism is a social construct that was purely formed for the purpose
of bringing up the white race while bringing down all other
races that are not white. It is assigned to people
at birth in order to let them know their place in society.
Whether the were inferior or not along with
with the oppurtunities that they would have in life.

\subsection{Race as a form of oppression}

"Subsequently, the idea of race
became an explanatory dogma that combined notions
of physical difference, culture, and ancestry, leading to
the predominance of scientific racism..." (203)

"The ascension of Christianity through colonization
combined with the ideology of white supremacy devel-
oped in an epistemological and moral order in which
race and religion became precepts of social hierarchy." (203)
"... biological difference such as skin color and hair type, and justified
by the belief in a divine right ordained to Christian civilization 
and the notions of moral development embedded 
in this worldview." (203)

"Using the idioms of blood, skin color, and phenotypic
difference, scientific racism was used to enforce social
boundaries and regulations including legal statutes and
spatial segregation. The system of race and racializa-
tion was embedded in social structures and hierarchies
that depended on notions of culture and biology to fix
cultural essences as naturalized traits" (203)

"...racial performance is an important
aspect of interpreting structures organized in relation-
ship to whiteness, including social economies based in
notions of beauty, desire, and sexual preference." (204)



\section{Intersectionality}

\section{Racial Formation}

\section*{Reading Week2$_{a}$}

\section{Women of Color Feminism}
Judy Tzu-Chun Wu
\begin{itemize}
  \item Radicalism, Liberalism, and Invisibility
  \item Third World feminimsm came to create a movement 
    capable of spanning borders of nation and ethnicity.
  \item 
\end{itemize}

\section{Emergence of Yellow Power in America}
\begin{itemize}
  \item Asian Americans have also had their fights
    and are also victim. Although lesss visible scars.
  \item "yellow power" movement relevant to black power
    movement as part of Third Worl struggle of all colored people
\end{itemize}
\subsection{Mistaken Identity}
\begin{itemize}
  \item Japanese and Chinese americans have accepted
    the token of acceptance of white America.
    Meanwhile the other Asian Americans are excluded.
  \item Asian Americans can gain complete
    acceptance by denying their color.
    They have become white in every respect but color.
  \item Asian people have been taken the mistaken identity
    and realize they live in a "White democracy"
  \item Asian Americans have been developing into 
    white Americans
    \begin{itemize}
      \item Asians have tried to transform themselvess into white men-
        mentally and physically.
      \item They have given up their own language,
        customs, histories, and cultural values. Then
        adopted the "American way of life"
      \item Resulting in extreme self-hatred and self-denial.
      \item Just like African Americans, Asian Americans
        aspire to meet those same white beauty standards.
        Causing them to deny their natural beauty
    \end{itemize}

    \section*{Self Acceptance is the First Step}
    \begin{itemize}
      \item Asian Americans have become the stereotype
        of how they have been described.
        That being passsive, accomidating, 
        and unemotional.
      \item These are stereotypes that have been
        created by white America.
        It is a cause of the oppression of Asian Americans.
      \item The reading states examples of how in a Japanese American
        went to Japan and was surpised by open displays
        of love and affection.
        Then how in lectures the lecturor would ask
        "are their any questions", and rather than 
        silence, many students would ask questions.
    \end{itemize}

  \section*{Silent, Passive Image}
  \item Asian Americans live in a silent and passive world
    where they can not express themselves honestly.
  \item This comes from how Asian Americans have been historically treated.
    The first set of Chinese Americans came to the West 1850 to 1880s. 
    Faced extreme white racism: economic subordination, 
    to denial of rights of naturalization, to physical violence.
  \item There was a moment were stoning Chinese Americans in the streets
    and breaking their shops and laundries (their businesses).

  \section*{Asian Americans are still scared}
  \item WWII put 110,000 Japanese in detention camps
  \item They were immediately told to leave their homes and
    leaves posessions behind and had no voice opposition.
    "Evacuated without protest"
  \item Reading mentions how Asian Americans stay silent 
    in order to keep national attention on black people.

  \section*{False pride in own economic progress}
  \item Asian Americans have established their own image
    as the stereotypes that were formed due to oppression.
  \item Asian Americans were able to improve their economic conditions
    after the war.
    Black people became target group of West Coast discrimination.
    After WW2, huge influx of black migrated to the west,
    taking racism away from Asian Americans and onto Black people.
    (That is odd to say)
  \section*{Asian Americans perpetuate white racism}
\item 1st/2nd generation Japanese and Filipinos were hired as farm laborers 
  and gardeners. Then Chinese as laundries and restaurants.
  Meanwhile, Black people were widely unemployed. 
  There's a quote about how black migrant, unlike the immigrant
  found little oppurtunity in the West. They arrived too late
  and the unskilled labor they provded was not needed.
\item Reading states about how Asian Americans
  nurse their own feelings of inferiority and insecurity by holding
  themselves superior to black people. (?? This is insane)
\end{itemize}

\subsection{Relevance of Power for Asians in America}
\begin{itemize}
  \item Asian Amercans have two common assumptions,
    \begin{enumerate}
     \item Completely powerless in the US.
    \item Have already obtained "economic" equality.
    \end{enumerate}
  \item In 1960 Black people were \% 10+ the population meanwhile
    Asian Americans were less than half a percent.

  \section*{Potentionoal Political Power on West Coast}
  \item Despite being a small part of the total population
    Asian Americans made a good percentage of the West Coast.
    Specifically urban California areas.
  \item Giving them real political power
    under the representation of California
  \section*{"More of the money pie"}
  \item Althoughy Asian Americans leads all
    minorities in terms of economic progress.
    Fallacy that Asian Americans
    enjoy full economic opportunity. 
  \section*{Statistical Discrepancies}
\item 1960s Statistics show that High School and College for Japanese/Chinese
  males is higher than white males.
\item Unemployment rates were higher for whites rather than Japanese/Chinese
\item Despite that, Japanese/Chinese had a lower income than whites
\item Although it was higher than Blacks, Filipino Americans
  had a lower wage than Black Americans.

 \section*{Myths of Asian American Success}
\item Ghetto communities exist such as Little Tokyo
  in LA, and Chinatown in SF.
\item Elderly Japanese live in rundown hotels in social and 
  cultural isolation.
\item Chinese families suffer in poor living conditions
  and 2$^{nd}$ highest turberculosis rate.
\end{itemize}

\section{SF State College Strike (1968-1969)}
\begin{itemize}
  \item A strike began at San Francisco State College in 1968
    for the establishment of ethnic studies programs.
  \item This was during a time of MLK Jr's assassination and the Vietnam War.
  \item It helped with the representation of "Third World People"
  \item It started with a Black Studies program that
    was exclusive for San Francisco State College.
    It would later get adopted by several different American universities.
\end{itemize}

\pagebreak

\section*{Week2b}
\section{Movement}
42 - Daryl Joji Maeda
\begin{itemize}
  \item The word "movement" is a term that is used to describe
    the social movements that are happening in the world.
    Variety of meanings from Migration, transnationalism,
    and diaspora.
  \item Asian Americans participates many segments of the movement
    most notably students, civil rights, Black power,
    and antiwar components.
  \item Asians faced discrimination and exploitations
    in their communities. All Asians shared a common relationship
    to US Racism.
  \item "Third World" people came to oppose war in particular
    and US imperialism more generally.
  \item America discriminated against and explouted
    nonwhite people within its borders. Then genocide
    against Asians abroad. In the ideologies of the New Left.
  \item There were many single ethnic groups (AAPA \& AAA) then
    single ethnic but part of multiethnic Asian American Coalition. (ICSA, KDP, \& PACE)
  \item KDP would be Filipinos fighting against International Hotel in SF
    and opposing imposition of martial law in Phillipines.
  \item In Chinatown (SF and NYC), groups IWK and WMS and Little Tokyo-LA (LTPRO)
    would all be local groups that all aligned with general
    anticaist, anti-imperialism ideologies.
  \item Artists would bring culturally diverse array and writes to present
    to multiethnic audiences and promote arts and literature in various Asian
    ethnic communities. The community would bring these events together.
  \item Bringing ethnic studies to higher education would make Asian ghettos
    a better place to live, work by providing social services such as
    food, healthcare, and legal aid. Oppresed development would display
    poor residents, or Asian Americans working in sweatshops, restaurants,
    and service industries, and attempt to create distinct Asian American culture.
  \item 1970s, SF's Chinatown, AAC established "Server the People" programs
    to provide free food and healthcare for elderly, and operated a drop-in
    comminity center.
  \item Large scale immigration from Asia enabled immigration reforms of 1965
    making it more ethincally diverse, increasingly immigrant based,
    and more heterogeneous with regard to class than previous waves.
  \item Large influx of activists entering 
    comminity based nonprofits, unions,
    academia, or even electoral politics.
\end{itemize}

\section{Political Asian America}
Diane Fujino

Afro-Asian Solidarity, Third World internationalism, and the 
Origins of the Asian American Movement

\subsection{Introduction}
\begin{itemize}
  \item AAPA started in Berkeley and holds great historical significance.

    They began the 'Asian American' term to nationally
    mobiize people of Asian descent.
  \item AAPA embrances the ideas of pan-Asian and Third Worldist, local and global,
    and antiracist and anti-imperialist.
\end{itemize}

Essay covers three parts to the AAPA
\begin{enumerate}
  \item Concept of Political Asian America
  \item AAPA's intertwined goals of Asian American liberations
    and Third World radicalism,
    while raising questions about tensions in coalition work
  \item AAPA's internationalist politics that pivot away from domestic race
  \item The secondary aim:

    Explore brief history of growth of AAPA to show how themes
    of Political Asian America shaped the development of national AAM.

\end{enumerate}

\subsection{Emergence of "Political Asian American"}
\begin{itemize}
  \item There was a labor strike and grape boycott
    going on in Delnato, CA.
    AAPA met with Filipino and Chincano farmworkers
  \item AAPA started the earliest discussion about Japanese American concentration camps 
  \item Later with their skill, opposed McCarran Act detention
    of Black radicals and Leftists in moment of Black Power militancy.
  \item AAPA exposed and opposed all racism of all races.
    always intertwined with Capitalism, militarism, and imperialism in political formation.
    Impactig Asian American communities and "all oppressed people"
\end{itemize}

\section*{AAPA's Program and Principles}
\begin{itemize}
  \item Document called "AAPA Perspectives" functioned as guiding principles.
  \item This document would note how
    America is still racist and how Asians have accomidated themselves in order to surive

    Relating to white standard for gain acceptability

    The goal to support all non-white liberation movements nd all
    minorities in order to be truly liberated to get control over
    political, economical, and eduational institutions.

    Oppose imperialist policies pursued by American Government.
  \item AAPA wanted control over communities and institutions
    to be able to make decisions for themselves.
    Beinng economic, political, and education in the neighborhoods
  \item Imperialist politices 
    \begin{enumerate}
      \item would be US wars in Vietnam and Korea.
      \item bombing Hiroshinma and Nagasaki
      \item military precense in Okinawa
      \item US colonial takeover of Hawai'i, Perto Rico, Guam, and the Philippines
    \end{enumerate}

    The US had a perspective of domination, colonialism, and exploitation.

  \item 1969 document that replacing Capitalist to Socialist system
    can end oppression and inequality that exists in the nation.


\end{itemize}


\section*{The Racialization of Asian America}
\begin{itemize}
  \item Post WW2 caused a change 
    From: harsh anti-Asian exclusion, alien land laws, and yellow peril fears

    To: unexpected opening of jobs previously barred, access to
    suburban housing, shifting representations of Asian Americans in popular culture

  \item Worked to reject model minority impositions and advanced coalition politics with
    Black Liberation and Third World decolonization.
  \item AAPA was born after a Black Power mmovement and global
    anti-colonial movements.

    1968, month after AAPA formed, Dr. MLK was assassinated.Exhausted patience 
    with integration and gtaining equality through non-violence.

    2 days later, police killed Black Panther Bobby Hutton in Oakland.

    Rise of Black Power salutes in 1968 Mexico city olympics.
    Which Mexican police and military massacred hundreds of protestors.

  \item Through the common language, culture 
    (music, art, and literature),
    and contact (schools) would enable a connection with the younger 
    demographic.

\end{itemize}

\section*{Relational Race and Social Movements}
\begin{itemize}
  \item AAPA launched AAM emerged during Black Power, Chicano, and American Indians
    movements. Shared common antiracist and anti-imperialist goals. All
    from Third World decolonization movements.
  \item AAPA came early Cold War when Asian Americans greater access
    to professional jobs and suburban neighborhoods compared to Balck and Chicano communities
    but were still subject to racial inequality and wage gaps with White peers.
  \item Model Minority impacted Black, Chicano and white activists 
    who failed to see the oppression impacting Asian American Communities
\end{itemize}

\subsection{Self-determination and coalition practices: Afro-Asian solidarity}

\subsection{Third World Internationalism: Global in the local}

\subsection{AAPA expands Nationally: origins of the Asian American Movement}

  
\end{document}
