\documentclass{article}
\title{Week 1b Reading}
\author{Danny Topete\\ ETST005 \\ Professor Miyake}
\date{April 3, 2025}

\begin{document}
\maketitle

\section{ch51 Race - Junaid Rana}
"Race is a social construction in which biology
and culture are often conflated as a rhetorical logic and
material practice in a system of domination"(202)

\subsection{Race as a social construct}

"In the barest scholarly definition, race is a social
construct."(203)

"Race is inextricably a concept of the
modern episteme, intertwined in systems of imperial-
ism, colonization, capitalism, and social structure that
emerged out of the European Enlightenment (Goldberg
1993; Mills 1997; Silva 2007; Winant 2001)" (203)

"The inequality at the center of racism and white
supremacy is based on the enduring power of race as a
flexible and shifting category."(203)

Racism is a social construct that was purely formed for the purpose
of bringing up the white race while bringing down all other
races that are not white. It is assigned to people
at birth in order to let them know their place in society.
Whether the were inferior or not along with
with the oppurtunities that they would have in life.

\subsection{Race as a form of oppression}

"Subsequently, the idea of race
became an explanatory dogma that combined notions
of physical difference, culture, and ancestry, leading to
the predominance of scientific racism..." (203)

"The ascension of Christianity through colonization
combined with the ideology of white supremacy devel-
oped in an epistemological and moral order in which
race and religion became precepts of social hierarchy." (203)
"... biological difference such as skin color and hair type, and justified
by the belief in a divine right ordained to Christian civilization 
and the notions of moral development embedded 
in this worldview." (203)

"Using the idioms of blood, skin color, and phenotypic
difference, scientific racism was used to enforce social
boundaries and regulations including legal statutes and
spatial segregation. The system of race and racializa-
tion was embedded in social structures and hierarchies
that depended on notions of culture and biology to fix
cultural essences as naturalized traits" (203)

"...racial performance is an important
aspect of interpreting structures organized in relation-
ship to whiteness, including social economies based in
notions of beauty, desire, and sexual preference." (204)



\section{Intersectionality}

\section{Racial Formation}

\section*{Reading Week2$_{a}$}

\section{Women of Color Feminism}
Judy Tzu-Chun Wu
\begin{itemize}
  \item Radicalism, Liberalism, and Invisibility
  \item Third World feminimsm came to create a movement 
    capable of spanning borders of nation and ethnicity.
  \item 
\end{itemize}

\section{Emergence of Yellow Power in America}
\begin{itemize}
  \item Asian Americans have also had their fights
    and are also victim. Although lesss visible scars.
  \item "yellow power" movement relevant to black power
    movement as part of Third Worl struggle of all colored people
\end{itemize}
\subsection{Mistaken Identity}
\begin{itemize}
  \item Japanese and Chinese americans have accepted
    the token of acceptance of white America.
    Meanwhile the other Asian Americans are excluded.
  \item Asian Americans can gain complete
    acceptance by denying their color.
    They have become white in every respect but color.
  \item Asian people have been taken the mistaken identity
    and realize they live in a "White democracy"
  \item Asian Americans have been developing into 
    white Americans
    \begin{itemize}
      \item Asians have tried to transform themselvess into white men-
        mentally and physically.
      \item They have given up their own language,
        customs, histories, and cultural values. Then
        adopted the "American way of life"
      \item Resulting in extreme self-hatred and self-denial.
      \item Just like African Americans, Asian Americans
        aspire to meet those same white beauty standards.
        Causing them to deny their natural beauty
    \end{itemize}

    \section*{Self Acceptance is the First Step}
    \begin{itemize}
      \item Asian Americans have become the stereotype
        of how they have been described.
        That being passsive, accomidating, 
        and unemotional.
      \item These are stereotypes that have been
        created by white America.
        It is a cause of the oppression of Asian Americans.
      \item The reading states examples of how in a Japanese American
        went to Japan and was surpised by open displays
        of love and affection.
        Then how in lectures the lecturor would ask
        "are their any questions", and rather than 
        silence, many students would ask questions.
    \end{itemize}

  \section*{Silent, Passive Image}
  \item Asian Americans live in a silent and passive world
    where they can not express themselves honestly.
  \item This comes from how Asian Americans have been historically treated.
    The first set of Chinese Americans came to the West 1850 to 1880s. 
    Faced extreme white racism: economic subordination, 
    to denial of rights of naturalization, to physical violence.
  \item There was a moment were stoning Chinese Americans in the streets
    and breaking their shops and laundries (their businesses).

  \section*{Asian Americans are still scared}
  \item WWII put 110,000 Japanese in detention camps
  \item They were immediately told to leave their homes and
    leaves posessions behind and had no voice opposition.
    "Evacuated without protest"
  \item Reading mentions how Asian Americans stay silent 
    in order to keep national attention on black people.

  \section*{False pride in own economic progress}
  \item Asian Americans have established their own image
    as the stereotypes that were formed due to oppression.
  \item Asian Americans were able to improve their economic conditions
    after the war.
    Black people became target group of West Coast discrimination.
    After WW2, huge influx of black migrated to the west,
    taking racism away from Asian Americans and onto Black people.
    (That is odd to say)
  \section*{Asian Americans perpetuate white racism}
\item 1st/2nd generation Japanese and Filipinos were hired as farm laborers 
  and gardeners. Then Chinese as laundries and restaurants.
  Meanwhile, Black people were widely unemployed. 
  There's a quote about how black migrant, unlike the immigrant
  found little oppurtunity in the West. They arrived too late
  and the unskilled labor they provded was not needed.
\item Reading states about how Asian Americans
  nurse their own feelings of inferiority and insecurity by holding
  themselves superior to black people. (?? This is insane)
\end{itemize}

\subsection{Relevance of Power for Asians in America}
\begin{itemize}
  \item Asian Amercans have two common assumptions,
    \begin{enumerate}
     \item Completely powerless in the US.
    \item Have already obtained "economic" equality.
    \end{enumerate}
  \item In 1960 Black people were \% 10+ the population meanwhile
    Asian Americans were less than half a percent.

  \section*{Potentionoal Political Power on West Coast}
  \item Despite being a small part of the total population
    Asian Americans made a good percentage of the West Coast.
    Specifically urban California areas.
  \item Giving them real political power
    under the representation of California
  \section*{"More of the money pie"}
  \item Althoughy Asian Americans leads all
    minorities in terms of economic progress.
    Fallacy that Asian Americans
    enjoy full economic opportunity. 
  \section*{Statistical Discrepancies}
\item 1960s Statistics show that High School and College for Japanese/Chinese
  males is higher than white males.
\item Unemployment rates were higher for whites rather than Japanese/Chinese
\item Despite that, Japanese/Chinese had a lower income than whites
\item Although it was higher than Blacks, Filipino Americans
  had a lower wage than Black Americans.

 \section*{Myths of Asian American Success}
\item Ghetto communities exist such as Little Tokyo
  in LA, and Chinatown in SF.
\item Elderly Japanese live in rundown hotels in social and 
  cultural isolation.
\item Chinese families suffer in poor living conditions
  and 2$^{nd}$ highest turberculosis rate.
\end{itemize}

\section{SF State College Strike (1968-1969)}
\begin{itemize}
  \item A strike began at San Francisco State College in 1968
    for the establishment of ethnic studies programs.
  \item This was during a time of MLK Jr's assassination and the Vietnam War.
  \item It helped with the representation of "Third World People"
  \item It started with a Black Studies program that
    was exclusive for San Francisco State College.
    It would later get adopted by several different American universities.
\end{itemize}
  
\end{document}
