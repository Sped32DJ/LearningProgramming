\documentclass{article}
\title{Week 1b Reading}
\author{Danny Topete\\ ETST005 \\ Professor Miyake}
\date{April 3, 2025}

\begin{document}
\maketitle

\section{ch51 Race - Junaid Rana}
"Race is a social construction in which biology
and culture are often conflated as a rhetorical logic and
material practice in a system of domination"(202)

\subsection{Race as a social construct}

"In the barest scholarly definition, race is a social
construct."(203)

"Race is inextricably a concept of the
modern episteme, intertwined in systems of imperial-
ism, colonization, capitalism, and social structure that
emerged out of the European Enlightenment (Goldberg
1993; Mills 1997; Silva 2007; Winant 2001)" (203)

"The inequality at the center of racism and white
supremacy is based on the enduring power of race as a
flexible and shifting category."(203)

Racism is a social construct that was purely formed for the purpose
of bringing up the white race while bringing down all other
races that are not white. It is assigned to people
at birth in order to let them know their place in society.
Whether the were inferior or not along with
with the oppurtunities that they would have in life.

\subsection{Race as a form of oppression}

"Subsequently, the idea of race
became an explanatory dogma that combined notions
of physical difference, culture, and ancestry, leading to
the predominance of scientific racism..." (203)

"The ascension of Christianity through colonization
combined with the ideology of white supremacy devel-
oped in an epistemological and moral order in which
race and religion became precepts of social hierarchy." (203)
"... biological difference such as skin color and hair type, and justified
by the belief in a divine right ordained to Christian civilization 
and the notions of moral development embedded 
in this worldview." (203)

"Using the idioms of blood, skin color, and phenotypic
difference, scientific racism was used to enforce social
boundaries and regulations including legal statutes and
spatial segregation. The system of race and racializa-
tion was embedded in social structures and hierarchies
that depended on notions of culture and biology to fix
cultural essences as naturalized traits" (203)

"...racial performance is an important
aspect of interpreting structures organized in relation-
ship to whiteness, including social economies based in
notions of beauty, desire, and sexual preference." (204)



\section{Intersectionality}

\section{Racial Formation}
  
\end{document}
