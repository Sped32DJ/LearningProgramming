\documentclass{article}
\title{Journal 2 Readings}
\author{Danny Topete\\ ETST 005\\ Professor Miyake}
\date{April 21, 2025}

\begin{document}
\maketitle

\section*{Week 4a Readings}

\section{Citizenship}
Chapter 5
by Helen Heran Jun
\subsection{Foundations of Citizenship}
\begin{itemize}
  \item Citizenship is a legal status that confers rights and privileges.
  \item Citizenship is a social status that confers identity and belonging.
  \item Citizenship grants or denies such rights and duties to
    inhabitants of a state.
  \item All citizenss should have equal standing before
    the law and are formally equivalent to one another
    \begin{itemize}
      \item Rights such as voting, as they are all treated equally
    \end{itemize}
\end{itemize}

\subsection{Politics of Citizenship}
\begin{itemize}
  \item All citizens participate in an imaginary
    political sphere of equality and formal equivalence.
  \item Political emancipation via citizenhip replaces and displaces
    possibilities for actual emancipation.
  \item Abstract equality is the substance of citizenship
  \item Universalism is designed to offer merely abstract equality.
\end{itemize}

\subsection{Marxist Critques}
\begin{itemize}
  \item Critiques on Citizenship as abstract and illusionary
  \item There is a contradiction to citizenship
    that inhere to racialized citizenship.
  \item Racial difference institutionally merges in contradiction
    to the universally promised by political emancipation
    through citizenship.
  \item The classical Marxist critique has not accounted for modalities of
    differentiation as being crucial to the development of capitalism
  \item Asian Americans have never been abstract labor nor abstract citizens

    But, they have been historically formed in contradiction
    to both the economic and the political spheres.
\end{itemize}

\subsection{Exclusionist Citizenship for Asian Americans}

\begin{itemize}
  \item Asian American discources had an emphasis on both
    its denial and negation torwardsd Asians.
  \item Asian labor was expoited in the west while those
    same workers were systematically denied citizenship status
  \item Through 1790 Naturalization Act, deemed only "free white persons"
    eligible for naturalized citizenship. Leading to
    systematic exclusion of new Asian emigrants.
  \item Alien Land laws prohibited Asians already in the US
    from owning property. This included their home and businesses.
  \item Even to those born in the US from immigrants were refered to as
    "alien citizenship", their legal status was often disregarded,
    such as mass internment of Japanese American Citizens.
\end{itemize}

\subsection{History of Exclusionist Citizenships}
\begin{itemize}
  \item Critical to exclude people from citizenship to keep them
    vulnerable and keep them getting exploited for capital.
  \item Reaction to excluding Asian immigrant workers from citizenship
    was a reaction to the formal inclusion of African Americans
    into citizenship in 1868 following the abolition of slavery
\end{itemize}

\subsection{Post WWII \& Civil Rights}
\begin{itemize}
  \item There was a demand for US Labor markets
  \item Civil rights mobilization lead by African Americans
    and others brought an end to exclusionary legislation
    torwards Asian Americans
  \item Asian Americans incorperated into US Citizenship via
    liberalized immigration policies resulting from
    Immigration and Naturalization Act, or Hart-Celler Act of 1965.
\end{itemize}
\subsection{Post Civil Rights}
\begin{itemize}
  \item All racial groups finally experienced
    full equality in the realm of the law.
  \item Despite all the civil rights progress since 1965,
    the only real progress took place in the
    legal realm of abstract formal equivalence.
  \item legal equivality only enables partial
    socioeconomic mobility for US Racial minorities.
  \item Yet there remains uneven mortality rates,
    unequal access to functioning public education,
    a racialized, gendered labor market, and subject to
    state violence and detention.
  \item Even the middle class and elite Asian Americans
    can still be the basis for a critique of
    citizenship with different political horizons that would be
    transformative of the brutal conditions endured by the
    racialized poor in this country and beyond.
  \item Citizenship still dictates acccess to labor market
    and to host of state-regulated resources,
    including housing, healthcare, and education.
\end{itemize}


\pagebreak

\section{Loyalty to Empire}
Moon-Ho Jung (2023)
\subsection{Loyalty to the State}
\begin{itemize}
  \item "Speaing for myself, I'm no flag-waver, no patriot, and am fully
    aware that venality, brutalitym and hypocracy are imprinted on the leaden soul of every state"
  \item People are always scared to be labeled "Anti-American" and maybe terrorist of the
    state for opposing something that the state is actively doing.
    Even if its related to an active war.
  \item Roy saw no hope in teh America as it was an empire terrorizing the world.
  \item This nationalist view will make Asians finally into
    full-fledged Americans.
\end{itemize}

\subsection{Searching for a Textbook}
\begin{itemize}
  \item The right book helps students with the way they see and engage the world.
    So choosing the right one is important.
  \item The author notes about how reading about people whose
    backgrounds resonated with his own resulted in the Author
    abondoning law school to become a historian; Books can change lives.
  \item The history of America is essentially the story of immigrants.
  \item The "US is a nation of immigrants" is just a social construct.
  \item It then transformed into immigrants to remove colonialism and slavery from the
    nation's founding.
  \item Asian Americans subscribed to the notion
\end{itemize}


\section{Obnoxious to their very nature}
volpp - 2001

\section{Week 4b Readings}

\subsection{Settler Colonialism}
by Dean Itsuji Saranillo

\begin{itemize}
  \item Bigger fish will see a wave coming, and get ready to gobble up the
    smaller fish. This sort of bullying is done by the US to others.
  \item Settler colonialism even makes sense to genocide,
    juridical and military force and normilizing occupation. Making it seem
    as if it is the settler's self-right to be there.
    Replace the current production for your own they believe
    their method is lesser than the settler's.
  \item "justifies settler colonialism as an evolving form
    of colonial power that justifies settler hegemony through an
    antiprimitive logic akin to antiblackness" (284)
  \item Settlers achieve "whiteness" by introducing physical violence
    and representationoal erasure to indegenous communities.
  \item Maintaining white supremacy includes
    physical violence and (mis)representation "erasure" of indegenous people
    \item Indigenous people were forced into politics of recognition and authenticity
      in order to not be disqualified by the white democracy
    \item The settler colonialism has caused an army that is racist gendered violence
      for the sake of sustaining capital, resource extraction, and global hegemony.

\end{itemize}

\section*{A Future Wish}
\begin{itemize}
  \item
\end{itemize}

\subsection{Settlers of Color and "Immigrant" Hegemony}
by Haunani-Kai Trask

\section{Discussion Week 4}

\begin{itemize}
  \item Midterm due Friday May 2nd, opens April 25th

    Due to the lecture main canvas
  \item We will not be going over the questions, there will be extended office hours,
    You just can't show people the papers (only for reading journals).

    Only for clarifying questions
  \item Zoom OH this week
  \item Before next Friday, Reading journal grade will be out
  \item No discussion week 5, there will still be an assignment
  \item Midterm, 3 short answer questions, open book, open note
  \item Each notes 3 things, 9 total mini questions
  \item Answers, don't require any formatting, proper citations is the
    most important part + work cited.
  \item Readings are required for the midterm
  \item Reading notes + lectures will help with everything related to this
  \item Questions are very straight forward, based off reading and lecture
  \item Each question is worth 10 points, sum 30 questions.
  \item She is not allowed to go through the questions, this is new for her...
  \item When the questions starts with How?

    You will explain and describe
  \end{itemize}

  \subsection{Activity}

  \begin{enumerate}
    \item Inner Circle: "What we carry"
      \begin{itemize}
        \item San Bernardino is always facing troubles with immigration plans
          \item We have many people with varying levels of immigration status from not having a
          visa at all to finally being a US Citizen

      \end{itemize}

    \item "Middle Circle: Where we might be comlicit"
      \begin{itemize}
        \item
      \end{itemize}

    \item Outer Circle: "Paths Torward Solidarity \& Repair"
      \begin{itemize}
        \item
      \end{itemize}
    \item Homework: native-land.ca
  \end{enumerate}

  \section*{Week 5a Readings}
  \section{Minority}
  \section{The Misbegotten Crituque of Model Minority Myth}

  \section{Success Story minority group}
  \begin{itemize}
    \item Low crime rates in the China town in SF
    \item People were taught that their success depends on their own efforts
    \item There is the theme that other minorities should catch up
      to the sucess of the Chinese people of China town

      "Chinese-Americans are moving ahead on their own -- with no help from anyone else"
    \item Big role of kindship in this reading, how no one fails
      because everyone is on big family. They help each other; they are family.
    \item The kids are well disciplines and know how to act
    \item "The kids are so obedient and they don't even spank their kids."(This was even more meaningful in 1960s)
    \item Chinese American kids are the best behaved and most intelligent kids.
      If a teacher complains about student performance, there is immediate improvement.
    \item Then there is a mention about how African Americans are low on the socioeconomic scale
      due to their own performance. They are where they are now due to their own faults.
      (or something like that; basically, model minority and compare african american to them...)
    \item This article is very racist and only further pushing the narrative that Chinese Americans
      are the pristine model minority, they outperform all the other kids in schools and have
      the strongest culture in America. To the poiunt where Teachers prefer Chinese American students.
  \end{itemize}

  \section*{Week 5b Readings}
  \subsection{Strategic Orientalism}
  \subsection{Diversity}

  \section*{Week 6b Readings}

  \section{Queer}
  by Matrin F. Manlansan IV

  Notes on this sources, really, really long sentences and
  it is hard to read

  \subsection{Modern Usage of Queer}
  \begin{itemize}
    \item Queer is a ubuquitous term used from schoarly, mass media, and political discourse
    \item Used in scholarly settings to TV shows that poke fun at queer people
    \item Queer is an umbrella term for identities, behaviors, and bodies
      as nonconforming to specific notions of normal.
  \end{itemize}

  \subsection{Origin}
  \begin{itemize}
    \item Origins from 17th century
    \item Label things, people, and situations that were considered renegade,

      waywar, strange, counterfiet, and/or perverted
    \item 20th century, consolidation of sexual orientation as
      cultural identity marker
    \item Has become derogatory stand-in for "homosexual" and gender insobordination.
  \end{itemize}

  \subsection{Roots with Feminism}
  \begin{itemize}
    \item sexual orientation and identity categories such as gay and lesbian share
      parallel realms with feminist theory
      and Third World and Women-of-color feminism.
    \item The feminist movements were responsbile for
  \end{itemize}

  \subsection{1980s activism}
  \begin{itemize}
    \item Gay and Lesbian studies emerged 1980s
    \item Came from gay/lesbian and feminist activisms and
      early studies of homosexuality of 1970s.
    \item The AIDS pandemic brought a lot of scare
    \item 1980s influx of AIDS cases who were nonwhite,
      nonmainstream, and mostly from migrant and/or racialized communities.
    \item It was actually a myth that queer people
      were more likely to be HIV positive.
      But the myth lived strong in the media.
    \item Queer moved from questions of sex and gender
      to context of biological destiny and engagements
      with cultural and historical exigencies.
  \end{itemize}

  \subsection{Queer in Asian American}
  \begin{itemize}
    \item Asian American queer in the past have had issues
      with varied miscegenation, bachelor societies,
      failed masculinuty, orientalism, domesticity,
      and perpetual foreignness.
    \item Early chinatowns and filipino agricultural camps were
      predominantly male-dominated sites. These sites were
      rather a social stages for the performance of hegemonic
      understanding of race, gender, and sexuality.
    \item Example of Chinese and Filipino men as failed masculinities
      or always already criminals, sexual enuchs or sexual predetors.
    \item With the existing notion of Chiense
      ineligible for citizenship due to exclusion laws and filipinos not
      deemed or treated as citizens.
    \item Links between Asian American focused gay and lesbian or queer studies
      and connections to larger questions in field
  \end{itemize}



  \section{Open In Emergency}
  by Mimi Khuc


  \subsection{Queer Asian American}


  \section{Gay World Makeover}
  \section{Neoliberalism is stealing trans liberation}

\end{document}
