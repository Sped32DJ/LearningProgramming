\documentclass{article}
\title{Journal 3 Readings}
\author{Danny Topete\\ TEST 005\\ Professor Miyake}
\date{May 5th, 2025}

\begin{document}
\maketitle


  \section*{Week 7a Reading}
  \section{Militarism}
  by Vernadette Vicuna Gonzalez

  Published to NYU Press (2015)

  \begin{itemize}
    \item The land theft and genocide of indigenous people
      influence how US Militarism woulld be turned
      in Asia, Asians, and Asia Americans from mid 1800s to present.
    \item The US is in constant war post-WW2 and has the most bloated military budget
    \item Japan's highest officials confirmed
      effectiveness of military superiority and that caused a threat.
      Wanting to compete with European imperial powers, Asia's trade became a threat and
      established an outpost in the Kingdom of Hawaii called Pu'uloa that was
      later renamed Pearl Habor.
    \item Filipino island revolution against Spain, then they realized
      America's imperial instincts to pickup where Spain left off.
    \item American War used brutal techniques,
      Filipinos were racialized into savages, the water cure
      and concecntration camps were used as modes of individual torture and population control.
    \item They were set to kill everyone over the age of ten (real quote)
    \item Vietnam also had racialized violence and military strategy with
      regard of Asian life. "Oriental doesn't put the same high price on life as does a westerner"
    \item WW2 most cited watershed momeny for how race is deployed by militarism
      \begin{itemize}
        \item Japan attacks Pearl Habor and immediately incarcerates Japanese and Japanese Americans
          living in US and Latin America
        \item Then German and Italian Americans were exempt from US discrimination
      \end{itemize}
  \end{itemize}

  \section{Good Refugee}
  \section{Rony 2010}

  \newpage
  \section*{Week 7b Readings}
  \section{Deportation}
  \section{Deporting Cambodian Refugees}



  \newpage
  \section*{Week 8a Readings}

  \section{Brown}
  by Nitasha Tamar Sharma

  \subsection{Definition of Brown}
  \begin{itemize}
    \item Brown is in the dictionary to refer to people
      with brown or dusky color skin;
      almost like a racial characteristic.
      This includes many groups of people from Africa, Asia, Europe,
      the Pacific, and Latin America.
  \end{itemize}

  \subsection{To be Brown}
  \begin{itemize}
    \item Naming a group of people by a color refers to not
      a thing or persons as much as it means the process through
      which these are given meaning.
    \item To be brown means that you are neither whte nor black,
      an effect of British colonizations.
    \item Brown is the intermediate hierarchil position.
  \end{itemize}

  \subsection{Why it came to be}
  \begin{itemize}
    \item White colonialist distringuished themselves from Filipino and
      Indian subjects.
    \item Filipinos would be "Brown monkeys"
    \item South Asians were patrionized as they were British,
      as "little Brown brothers" and "Brown cousins"
    \item A 5th race, "Malay" that was prohibited to marrying Whites.
    \item Then Supreme Court case in 1923 categorizing South Asians from
      Caucasians, making them legally white, to non-White (thus Brown)
      upon which declaring them "aliens ineligible for citizenship"
  \end{itemize}

  \subsection{Modern Brown}
  \begin{itemize}
    \item Brown is such an ambiguous term that it is used to refer to
      people of color, but also to refer to people who are not
      white, but not black. Then people get confused and misrecognized
      as a result.
    \item Now Latinos propose "Mexican Brown" to refrane from racial
      lumping with other races.
    \item The definition of "Brown" is constantly shifting, especially after
      9/11, and Wars on Terror. Now it is used to refers to people
      frokm the Middle East and North Africa, more broadly, people
      from area commonly recognized to be practicing Islam.
      (despite places like East Asia practice Islam, but it is not recognized)
    \item Brown doesn't belong to any one particular race
    \item Being Brown has brought "Brown Pride", akin to Black Pride movements
    \item Post 9/11, "Brown" has operated as political and disporic identity
    \item Brownness as a political concept and identity has been shared
      discussing surveillance and oppression that links middle eastern countries
      with teir hmelands and across diasporas
    \item "Brown" expands beyond Asian America. Referring to Latinos,
      Filipinos, South and Southesst Asians, Arabs, "Mulim-looking" people,
      and others.
    \item The definition has gone across boundaries of racial categories
  \end{itemize}

  \section{The 21st Century Problem of Anti-Muslim Racist}
  by Nadine Naber and Junaid Rana

  \subsection{Trump and the Muslim Ban}
  \begin{itemize}
    \item A year aftert the Trump administration travel ban
      that is targeting Muslim-Majoirty countries.
    \item Objectifying Musliims as migrants and racial others
    \item Banned entry of immigrants from a number of these countries
    \item When it was questioned whether or not it was
      discriminatory, it was intent or motive to do harm based on hostility
      (what ever that means)
    \item US has a link between policies of Immigration being linked
      to war and empire-building in what described US Imperialism
  \end{itemize}

  \subsection{Neocolonialism}
  \begin{itemize}
    \item Despite having an American passport, who is allowed
      to freely travel across borders in the US depends on very
      definitions of Nationality and sovereignty.
      Both social constructs that is their choice to enforce.
      Concepts founded by settler societies.
    \item Colonialism never really ended in the US, it just evolved
    \item The constant expansion and the imperial power the US
      has become has caused problems such as Racism as a form of
      violence and oppression.
      A result of slavery and native genocide and conquest that resulted
      in lots of violence and killing produing global War on Terror
      as a product of political conflict.
  \end{itemize}


  \subsection{Racialization of Muslims}
  \begin{itemize}
    \item Stats from racism against incongrous conflation
      of nationality, race, and religion.
    \item US being against movements led by Arab, Muslim, and South Asian
      activists along with their immediate allies.
  \end{itemize}

  \subsection{Why do we call it Anti-Muslim Racism?}
  \begin{itemize}
    \item Anti-Muslim racism has influenced popular and state forms of
      violence, policing, and surveillance.
    \item Social movements have brought collective teaching,
      theorizing, and activism.
    \item There has also been a rise of social movements against imperial racism,
      that including resurganve of the combat of anti-black racism
      and state targeted of undocumented immigrants.
    \item This anti-muslim racism comes from the assumption of
      "Muslims" being enemies and pose a threat to the US nation.
    \item There is different racism, ex: for Black Muslims they are combining
      the anit-Black and anti-Muslim racism. Which is different compared to
      recent Arab and South Asian immigrants who are perceived to be Muslim
      then being potentially connected to terrorism.
    \item Labeling people are "potetial Muslim terrorist" has been used
      as a target against Arab Americans. Now more recently, against Palestinians
    \item Post 9/11, white supremacists have had hate crimes against 'Muslim-appearing' people
      that being Balbir Singh Sodhi in Arizona or an attack in Sikh Temple in 2012
  \end{itemize}

  \subsection{Extend to this racism}
  \begin{itemize}
    \item This oppression has faced anyone to be Muslim to be subject
      to government policies and legal exclusions from Airport profiling,
      surveillance, detection, deportation, employment, and housin discrimination, immigrant exclusion,
      political repression, and so on.
    \item This racism targets those who may have some common identifiers from wearing a hijab
      to having a "Muslim" or "Arabic" name, to being from a predominantly
      Muslim populations country.
  \end{itemize}

  \subsection{Anti-Muslim Racism in Policy and Law}
  \begin{itemize}
    \item This racism has gone mainstream to the US
    \item Anti-Mulsim racism isn't only against a single religion, it is
      also including many people, even if they are not associated
    \item History shows that US immigration and security policy
      has a discriminatory and exclusionary history as using
      Muslims as the 21$^{st}$ century scapegoat.
    \item Seeing who is eligible of travel is based upon categories set by
      Immigration and Nationality Act (INA) and the US Department of State.
      Thus inviting racism and discrimination against people to a flexible social
      construct.
  \end{itemize}

  \section{Terrifying Muslim racial panic islamic peril and terrorism}


  \newpage
\section*{Week 8b Reading}

\section{Terrorism}
by Rajini Srikanth

\subsection{Origins}
\begin{itemize}
\item The word "Terrorism" comes from the latin work terrorem
\item Started in France during the Revolution 1784-94 to describe
  group that intimidated the government and carried
\item Terrorism seeks the following outcomes:
  \begin{enumerate}
    \item regime change
    \item terrotorial controls
    \item territorial change
    \item policy change
    \item social control
    \item status quo maintenance
  \end{enumerate}
\item Types of terrorism:
  \begin{enumerate}
    \item state-sponsored terrorism
    \item religious
    \item suicide
    \item trasnational
    \item homegrown
  \end{enumerate}
\end{itemize}

\subsection{US state-sponsored Terrorism}
\begin{itemize}
\item Earliest Asian American victims to terrorism were Chinese immigrants
  targeted by anit-chinese groups between 1850s to early 1900s
\item They were being oppressed by white people who were in jobs that were later transferred over
  to chinese people. Such as railroads, gold, and Irish and Germenan immigrants
  that felt the Chinese were taking their jobs and undermining their opportunities of work.
\item 120k people of Japanese ancestry were put into internment camps;
  act of state-sponsored terrorism
\item Result of anti-Japanese racism that came post Pearl Harbor,
  then reperations did not come until 1988
  \item Many examples of revolutions that happeend from 1950s to 1960s.
    A lot of them were post-colonialism against colonial control and oppression.
    These being violence against oppressive regimes.

    Countries being Cuba, Algeria, Kenya, Sandisnista, Nicaragua, African National Congress, etc
  \item When the state is doing the terrorism, it normalizes oppression, presenting
    a narrative that ignores the claims and demands of the oppressors, challenging the state
\end{itemize}

\subsection{Modern Terrorism}
by Rajini Shrikanth
\begin{itemize}
  \item Groups such as the KKK haved existed in the US.
    There are plenty of radical left terror groups.
    Then Weather Underground Organization conducting antigovernment attacks
    and protesting bombing campaigns in Southeast Asia
  \item The word Terrorism didn't come into public consciousness until the 2003 bombing
    of the World Trade Center and 1995 bombing of Oklahoma federal building.
  \item These two attacks caused the 1996 Anti-Terrorism and Effective
    Death Penalty Act (AEDPA).
    Providing sanctions leading to deporting noncitizens with convicted felenoies,
    regardless of time served of if the crime was committed as a minor.
  \item This caused many young Cambodian descent men in their 20s to be deported
    to Cambodia due to felonies that were committed as adolescents growing up
    in tough urban neighborhoods.
  \item Then 9/11 attacks caused dramatic escalation and state's attention to terrorism.
    George W Bush declaring the "global war on terror".
    Which he would vowe to seek out any terrorist anywhere in the world and pursue them
  \item Political speaking, it kept going in 2009 with Obama since it gave them
    permission to violate internatioal human rights laws.

    Examples:
    Guantanamo Bay detection facility in Cuba
  \item Guantanamo Bay would be used to prison and torture 700 Muslim men who were
    picked up from Afghanistan which were charged on insubstantial evidence with
    involvement to al-Qaeda.
  \item Since the attacks, constant surveillance of 23 Muslim-Majority countries and
    North Korea. FBI is in constant watch of people there and coming from there.
    Leading to many noncitizens from these countries to be deported
    for minor immigration violantions.

\end{itemize}


\section{Terrorism, Orientalism, and Imperialism}
by Stephen Morton

\section{Bans}
by Palk

\end{document}
