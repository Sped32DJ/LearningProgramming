\documentclass{article}

\title{ETST005 - Week 10}
\author{Danny Topete}
\date{\today}

\begin{document}
\maketitle

\section*{Lecture w10a}

\textbf{Final stuff}
\begin{itemize}
  \item The final is out now
  \item Should be within the scope of the class
  \item Office hours with TA and Prof are available
\end{itemize}

\section{Solidarity}
\begin{itemize}
  \item What does solidarity look like?
  \item Workers being able to have their rights within their jobs and their rights
  \item Coming together to fight against poor working conditions
\end{itemize}

\subsection{Definition}
\begin{itemize}
  \item Class and racial based solidarity
  \item Coorperation through distinct but related struggles
  \item Self-determination
  \item Risks reinscribing power dynamics if not carefully considered
  \item Internationalist
\end{itemize}

\subsection{Asian American Solidarities}
\begin{itemize}
  \item Beyond self-interests
  \item Asian America as heterogeneus
  \item Inter-group dynamics
  \item Self-reflection
  \item There must be mutual accountability (not sure what this means)
\end{itemize}

\subsection{"Asian American" Solidarities}
\begin{itemize}
  \item Asia American has roots beginning from the first migrants here have
    faced quite a lot of discrimination and oppression
  \item Then some Asian Americans migranted as refugees of war and on top of that
    they come to the US and face systematic oppression
  \item Asian Americans have lived a lot of struggle in the past and
    their word is valid for the sharing the struggles they have lived
\end{itemize}

\subsection{Coalition and Alliance}
\begin{itemize}
  \item A time of solidarity
  \item A political formation in which different groups come together toward
    some shared goal
  \item Coalitions are more temporary and often issue-based
  \item Alliances are more long-term and often based on shared identity
  \item Various alliances and worker unions can come together
    as for solidarity to fight issues that affect them all
\end{itemize}

\subsection{Question of Coalition}
\begin{itemize}
  \item Radical vs conservative solidarity
  \item Racidal solidarity is more about the liberation of all people
  \item Conservative solidarity is more about the liberation of a specific group
\end{itemize}

\section*{Lecture w10b}
Ditched it lol...

\newpage

\section{Discussion w10}
\section*{Takeaways from the class; checkin}
\begin{itemize}
  \item Race is ever changing
  \item Orientalism and its effects to minimize people to
    characteristics to how everyone in a group of people act and are
  \item Voice and speak up, we are all minorities in one way or another.
    We need to speak up for each other since we all share
    the same struggles.
    \begin{itemize}
      \item These struggles are started from politics or big coorperations
      \item They will chase their own greed or benefit
        rather than the greater good
      \item Part of coalition and solidarity between all people
    \end{itemize}
\end{itemize}

\section*{Reading key lessons}
\begin{itemize}
  \item Resistance
    \begin{itemize}
      \item Daily acts of survival
      \item Seeing how resistance can mean more than just
        people actively resistance with arms
      \item People mowing forwarding and resisting oppression
      \item People that are in warzones or US capitalism, their
        level of resistance is every day
    \end{itemize}
  \item Palestine
    \begin{itemize}
      \item Current happening event regarding a genocide happening in genocide
      \item Goes to show that you have a voice and can speak up
    \end{itemize}
  \item Radical Love
    \begin{itemize}
      \item Love between the community and being able to love
        others along with yourself in order to leave these systems
      \item There is love with the community and loving people
        that not be the same as you, and showing that compassion and care
    \end{itemize}
  \item Abolition + Care
    \begin{itemize}
      \item Caring abpit abolition
    \end{itemize}
\end{itemize}

\section*{Final Practice}

\textbf{Question 3:}
\begin{itemize}
  \item Empire: US imperialism, militarism, colonialism
    \begin{itemize}
      \item Used to reference these key terms in the relation to the US
    \end{itemize}
  \item Self-determination: Struggles by oppressed groups to

    gain autonomy, sovereignty, liberation
    \begin{itemize}
      \item Oppressed groups want to define themselves rather than getting oppressed
        by others
    \end{itemize}
  \item Empire uses terrorism as a tool to control oppressed groups
\end{itemize}

\noindent \textbf{Question 4:}
\begin{itemize}
  \item Responses to anti-asian hate
  \item US decided to counter this with
    more policing
  \item Asians should lean to abolitionist since
    police don't usually know how to engage on conflicts.
    They are biased and just lead to violence.
  \item How love and care can reduce to anti-asian hate
    rather than having a police officer taking care of it.
    This is just a bandaid to the problem, but the
\end{itemize}

\end{document}
