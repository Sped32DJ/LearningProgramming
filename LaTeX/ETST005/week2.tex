\documentclass{article}

\title{Week2 ETST005}
\author{Danny Topete}
\date{April 7th, 2025}

\begin{document}
\maketitle

\section*{Week2a}

\section{Origins of Study of Race}
\begin{itemize}
  \item Anthropology and Naturalist
  \item Abolitionists and Black Radical Thinkers
  \item Chicago School
    \begin{itemize}
      \item Wanting to study ethnicity to study the human difference
      \item Studying the Urban scape and how 
        different ethnic groups and the power
        differentials.
      \item Then how people live with each other in ethnically 
        segregated neighborhoods.
      \item This was all a product of maintream white racism
    \end{itemize}
\end{itemize}
\subsection{Contemporary Ethnic Studies}
\begin{itemize}
  \item Origins in West Coast colleges and universities
  \item Thirs world curriculum
  \item The field is rooted in communities
  \item Originated from SF State
  \item To bring representation to the 3$^{rd}$ World People
  \item It was an extension of the civil right's movement
  \item There were so few students from this population. 

    ex: There was only 75 Filipino students
  \item The 3rd world people would ban together
    to help bring representation to each other.
  \item PASTE was of Philipino students who started this political movement
    It was to help bring representation to the Philipino students
  \item These movements would bring a lot of police activity to the SF State campus 
  \item Police came with experimental weapons such as mace and they had guns and tear gas
  \item This movement and the resistance would spread to many campuses
  \item Campus had to close down for a week
  \item It was kinda hard for POC to get an education
    since they weren't really allowed an education.
\end{itemize}

\subsection{Third World?}
\begin{itemize}
  \item 1st world countries were originally named that by the NATO
    countries and the big power enemies.
    Then all of our enemies would be part of developing countries.
  \item Historically oppressed by colonial forces would be the developing countries
    that were labeled as 3rd world.
  \item After the Treaty of Versailles, the carved up colonial powers revolted
    and got their own independence. Despite being carved up by different European
    powers after the Treaty.
  \item There was a big surge of decolonialism.
  \item Some of these rooted into forms of socialism since capitalism
    is what took over their countries and Greed was evil for those countries.
\end{itemize}

\subsection{Third World Feminism}
\begin{itemize}
  \item Looked into 3rd world movement for inspiration
  \item Racial identity as a point of connection
    and politiciazation
  \item It was respones to 2nd wave feminism to solve
    capitalism, emperialism, sexism, and racism.
  \item This movement was anti-white-racist-oppression

    Bringing the global struggles against colonialization, emperialism, and
\end{itemize}

\subsection{Outcome of Strikes}
\begin{itemize}
  \item SF State and UC Berkeley got their Ethnic Studies programs
    for Chicano, Black, Asian, and Native coming later.
\end{itemize}

\subsection{Liberal Asian American Feminism}
\begin{itemize}
  \item Pragmatic and reformist approach to social change
  \item Invested in electoral politics 
  \item Politics of inclusion
  \item If people gain political power, they can make change top down
  \item In Hawaii, the people have been able to achieve great feats.
    But there are still the effects of colonialism in affect and unhoused native communities
\end{itemize}

\subsection{Radicalism vs Liberalism}
\begin{itemize}
  \item Liberalism:
    Rooted in Capitalism,
    idividual rights,
    protection of private property,
    anti discrimination.

    Begs questions such as law and order and people
    who are more subject to violence.

    Everyone being equal under the law.
    Diversity = justice. 

    Critiques on it's basis of capitalism and racism.
    There is the racist celebration of colorblindness.
    Private property and class live in a liberal world.

    Views race as operating independantly from capitalism.
    Rather than racism and capitalism being linked.

  \item Radicalism:
\end{itemize}

\pagebreak
\section{Yellow Power Movement}
\begin{itemize}
  \item late 1960s
  \item Rejected the silent steoreotype 
  \item Reclaiming political and racial idenitity
\end{itemize}

\subsection{Pair Share}
\begin{itemize}
  \item Asian Americans were the model minority
  \item They seem to have got their stuff together,
    were the most advanced economically and knew silenced
    their own identities in order to be "prostituted" to the white society.
\end{itemize}
\subsection{Ethnic Studies at UCR}
\begin{itemize}
  \item Options of Africana, Asian American, Chicano, Native American, Ethinic Studies
  \item Broad when it comes to intro courses
  \item Upper division courses are intersectionality and specific
  \item Research/Capstone courses:
    Can be working in research or working with a community
  \item Careers:
    community and sociology. Working in unions, DEI, civil service,
\end{itemize}


\end{document}
  
