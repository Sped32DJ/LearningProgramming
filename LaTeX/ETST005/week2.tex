\documentclass{article}

\title{Week2 ETST005}
\author{Danny Topete}
\date{April 7th, 2025}

\begin{document}
\maketitle

\section*{Week2a}

\section{Origins of Study of Race}
\begin{itemize}
  \item Anthropology and Naturalist
  \item Abolitionists and Black Radical Thinkers
  \item Chicago School
    \begin{itemize}
      \item Wanting to study ethnicity to study the human difference
      \item Studying the Urban scape and how 
        different ethnic groups and the power
        differentials.
      \item Then how people live with each other in ethnically 
        segregated neighborhoods.
      \item This was all a product of maintream white racism
    \end{itemize}
\end{itemize}
\subsection{Contemporary Ethnic Studies}
\begin{itemize}
  \item Origins in West Coast colleges and universities
  \item Thirs world curriculum
  \item The field is rooted in communities
  \item Originated from SF State
  \item To bring representation to the 3$^{rd}$ World People
  \item It was an extension of the civil right's movement
  \item There were so few students from this population. 

    ex: There was only 75 Filipino students
  \item The 3rd world people would ban together
    to help bring representation to each other.
  \item PASTE was of Philipino students who started this political movement
    It was to help bring representation to the Philipino students
  \item These movements would bring a lot of police activity to the SF State campus 
  \item Police came with experimental weapons such as mace and they had guns and tear gas
  \item This movement and the resistance would spread to many campuses
  \item Campus had to close down for a week
  \item It was kinda hard for POC to get an education
    since they weren't really allowed an education.
\end{itemize}

\subsection{Third World?}
\begin{itemize}
  \item 1st world countries were originally named that by the NATO
    countries and the big power enemies.
    Then all of our enemies would be part of developing countries.
  \item Historically oppressed by colonial forces would be the developing countries
    that were labeled as 3rd world.
  \item After the Treaty of Versailles, the carved up colonial powers revolted
    and got their own independence. Despite being carved up by different European
    powers after the Treaty.
  \item There was a big surge of decolonialism.
  \item Some of these rooted into forms of socialism since capitalism
    is what took over their countries and Greed was evil for those countries.
\end{itemize}

\subsection{Third World Feminism}
\begin{itemize}
  \item Looked into 3rd world movement for inspiration
  \item Racial identity as a point of connection
    and politiciazation
  \item It was respones to 2nd wave feminism to solve
    capitalism, emperialism, sexism, and racism.
  \item This movement was anti-white-racist-oppression

    Bringing the global struggles against colonialization, emperialism, and
\end{itemize}

\subsection{Outcome of Strikes}
\begin{itemize}
  \item SF State and UC Berkeley got their Ethnic Studies programs
    for Chicano, Black, Asian, and Native coming later.
\end{itemize}

\subsection{Liberal Asian American Feminism}
\begin{itemize}
  \item Pragmatic and reformist approach to social change
  \item Invested in electoral politics 
  \item Politics of inclusion
  \item If people gain political power, they can make change top down
  \item In Hawaii, the people have been able to achieve great feats.
    But there are still the effects of colonialism in affect and unhoused native communities
\end{itemize}

\subsection{Radicalism vs Liberalism}
\begin{itemize}
  \item Liberalism:
    Rooted in Capitalism,
    idividual rights,
    protection of private property,
    anti discrimination.

    Begs questions such as law and order and people
    who are more subject to violence.

    Everyone being equal under the law.
    Diversity = justice. 

    Critiques on it's basis of capitalism and racism.
    There is the racist celebration of colorblindness.
    Private property and class live in a liberal world.

    Views race as operating independantly from capitalism.
    Rather than racism and capitalism being linked.

  \item Radicalism:
\end{itemize}

\pagebreak
\section{Yellow Power Movement}
\begin{itemize}
  \item late 1960s
  \item Rejected the silent steoreotype 
  \item Reclaiming political and racial idenitity
\end{itemize}

\subsection{Pair Share}
\begin{itemize}
  \item Asian Americans were the model minority
  \item They seem to have got their stuff together,
    were the most advanced economically and knew silenced
    their own identities in order to be "prostituted" to the white society.
\end{itemize}
\subsection{Ethnic Studies at UCR}
\begin{itemize}
  \item Options of Africana, Asian American, Chicano, Native American, Ethinic Studies
  \item Broad when it comes to intro courses
  \item Upper division courses are intersectionality and specific
  \item Research/Capstone courses:
    Can be working in research or working with a community
  \item Careers:
    community and sociology. Working in unions, DEI, civil service,
\end{itemize}

\pagebreak

\section*{Week2b}
\section{Movement}
42 - Daryl Joji Maeda
\begin{itemize}
  \item The word "movement" is a term that is used to describe
    the social movements that are happening in the world.
    Variety of meanings from Migration, transnationalism,
    and diaspora.
  \item Asian Americans participates many segments of the movement
    most notably students, civil rights, Black power,
    and antiwar components.
  \item Asians faced discrimination and exploitations
    in their communities. All Asians shared a common relationship
    to US Racism.
  \item "Third World" people came to oppose war in particular
    and US imperialism more generally.
  \item America discriminated against and explouted
    nonwhite people within its borders. Then genocide
    against Asians abroad. In the ideologies of the New Left.
  \item There were many single ethnic groups (AAPA \& AAA) then
    single ethnic but part of multiethnic Asian American Coalition. (ICSA, KDP, \& PACE)
  \item KDP would be Filipinos fighting against International Hotel in SF
    and opposing imposition of martial law in Phillipines.
  \item In Chinatown (SF and NYC), groups IWK and WMS and Little Tokyo-LA (LTPRO)
    would all be local groups that all aligned with general
    anticaist, anti-imperialism ideologies.
  \item Artists would bring culturally diverse array and writes to present
    to multiethnic audiences and promote arts and literature in various Asian
    ethnic communities. The community would bring these events together.
  \item Bringing ethnic studies to higher education would make Asian ghettos
    a better place to live, work by providing social services such as
    food, healthcare, and legal aid. Oppresed development would display
    poor residents, or Asian Americans working in sweatshops, restaurants,
    and service industries, and attempt to create distinct Asian American culture.
  \item 1970s, SF's Chinatown, AAC established "Server the People" programs
    to provide free food and healthcare for elderly, and operated a drop-in
    comminity center.
  \item Large scale immigration from Asia enabled immigration reforms of 1965
    making it more ethincally diverse, increasingly immigrant based,
    and more heterogeneous with regard to class than previous waves.
  \item Large influx of activists entering 
    comminity based nonprofits, unions,
    academia, or even electoral politics.
\end{itemize}

\section{Political Asian America}
Diane Fujino

Afro-Asian Solidarity, Third World internationalism, and the 
Origins of the Asian American Movement

\subsection{Introduction}
\begin{itemize}
  \item AAPA started in Berkeley and holds great historical significance.

    They began the 'Asian American' term to nationally
    mobiize people of Asian descent.
  \item AAPA embrances the ideas of pan-Asian and Third Worldist, local and global,
    and antiracist and anti-imperialist.
\end{itemize}

Essay covers three parts to the AAPA
\begin{enumerate}
  \item Concept of Political Asian America
  \item AAPA's intertwined goals of Asian American liberations
    and Third World radicalism,
    while raising questions about tensions in coalition work
  \item AAPA's internationalist politics that pivot away from domestic race
  \item The secondary aim:

    Explore brief history of growth of AAPA to show how themes
    of Political Asian America shaped the development of national AAM.

\end{enumerate}

\subsection{Emergence of "Political Asian American"}
\begin{itemize}
  \item There was a labor strike and grape boycott
    going on in Delnato, CA.
    AAPA met with Filipino and Chincano farmworkers
  \item AAPA started the earliest discussion about Japanese American concentration camps 
  \item Later with their skill, opposed McCarran Act detention
    of Black radicals and Leftists in moment of Black Power militancy.
  \item AAPA exposed and opposed all racism of all races.
    always intertwined with Capitalism, militarism, and imperialism in political formation.
    Impactig Asian American communities and "all oppressed people"
\end{itemize}

\section*{AAPA's Program and Principles}
\begin{itemize}
  \item Document called "AAPA Perspectives" functioned as guiding principles.
  \item This document would note how
    America is still racist and how Asians have accomidated themselves in order to surive

    Relating to white standard for gain acceptability

    The goal to support all non-white liberation movements nd all
    minorities in order to be truly liberated to get control over
    political, economical, and eduational institutions.

    Oppose imperialist policies pursued by American Government.
  \item AAPA wanted control over communities and institutions
    to be able to make decisions for themselves.
    Beinng economic, political, and education in the neighborhoods
  \item Imperialist politices 
    \begin{enumerate}
      \item would be US wars in Vietnam and Korea.
      \item bombing Hiroshinma and Nagasaki
      \item military precense in Okinawa
      \item US colonial takeover of Hawai'i, Perto Rico, Guam, and the Philippines
    \end{enumerate}

    The US had a perspective of domination, colonialism, and exploitation.

  \item 1969 document that replacing Capitalist to Socialist system
    can end oppression and inequality that exists in the nation.


\end{itemize}


\section*{The Racialization of Asian America}
\begin{itemize}
  \item Post WW2 caused a change 
    From: harsh anti-Asian exclusion, alien land laws, and yellow peril fears

    To: unexpected opening of jobs previously barred, access to
    suburban housing, shifting representations of Asian Americans in popular culture

  \item Worked to reject model minority impositions and advanced coalition politics with
    Black Liberation and Third World decolonization.
  \item AAPA was born after a Black Power mmovement and global
    anti-colonial movements.

    1968, month after AAPA formed, Dr. MLK was assassinated.Exhausted patience 
    with integration and gtaining equality through non-violence.

    2 days later, police killed Black Panther Bobby Hutton in Oakland.

    Rise of Black Power salutes in 1968 Mexico city olympics.
    Which Mexican police and military massacred hundreds of protestors.

  \item Through the common language, culture 
    (music, art, and literature),
    and contact (schools) would enable a connection with the younger 
    demographic.

\end{itemize}

\section*{Relational Race and Social Movements}
\begin{itemize}
  \item AAPA launched AAM emerged during Black Power, Chicano, and American Indians
    movements. Shared common antiracist and anti-imperialist goals. All
    from Third World decolonization movements.
  \item AAPA came early Cold War when Asian Americans greater access
    to professional jobs and suburban neighborhoods compared to Balck and Chicano communities
    but were still subject to racial inequality and wage gaps with White peers.
  \item Model Minority impacted Black, Chicano and white activists 
    who failed to see the oppression impacting Asian American Communities
\end{itemize}

\subsection{Self-determination and coalition practices: Afro-Asian solidarity}

\subsection{Third World Internationalism: Global in the local}

\subsection{AAPA expands Nationally: origins of the Asian American Movement}

\end{document}
  
