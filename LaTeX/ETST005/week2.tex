\documentclass{article}

\title{Week2 ETST005}
\author{Danny Topete}
\date{April 7th, 2025}

\begin{document}
\maketitle

\section*{Week2a}

\section{Origins of Study of Race}
\begin{itemize}
  \item Anthropology and Naturalist
  \item Abolitionists and Black Radical Thinkers
  \item Chicago School
    \begin{itemize}
      \item Wanting to study ethnicity to study the human difference
      \item Studying the Urban scape and how 
        different ethnic groups and the power
        differentials.
      \item Then how people live with each other in ethnically 
        segregated neighborhoods.
      \item This was all a product of maintream white racism
    \end{itemize}
\end{itemize}
\subsection{Contemporary Ethnic Studies}
\begin{itemize}
  \item Origins in West Coast colleges and universities
  \item Thirs world curriculum
  \item The field is rooted in communities
  \item Originated from SF State
  \item To bring representation to the 3$^{rd}$ World People
  \item It was an extension of the civil right's movement
  \item There were so few students from this population. 

    ex: There was only 75 Filipino students
  \item The 3rd world people would ban together
    to help bring representation to each other.
  \item PASTE was of Philipino students who started this political movement
    It was to help bring representation to the Philipino students
  \item These movements would bring a lot of police activity to the SF State campus 
  \item Police came with experimental weapons such as mace and they had guns and tear gas
  \item This movement and the resistance would spread to many campuses
  \item Campus had to close down for a week
  \item It was kinda hard for POC to get an education
    since they weren't really allowed an education.
\end{itemize}

\subsection{Third World?}
\begin{itemize}
  \item 1st world countries were originally named that by the NATO
    countries and the big power enemies.
    Then all of our enemies would be part of developing countries.
  \item Historically oppressed by colonial forces would be the developing countries
    that were labeled as 3rd world.
  \item After the Treaty of Versailles, the carved up colonial powers revolted
    and got their own independence. Despite being carved up by different European
    powers after the Treaty.
  \item There was a big surge of decolonialism.
  \item Some of these rooted into forms of socialism since capitalism
    is what took over their countries and Greed was evil for those countries.
\end{itemize}

\subsection{Third World Feminism}
\begin{itemize}
  \item Looked into 3rd world movement for inspiration
  \item Racial identity as a point of connection
    and politiciazation
  \item It was respones to 2nd wave feminism to solve
    capitalism, emperialism, sexism, and racism.
  \item This movement was anti-white-racist-oppression

    Bringing the global struggles against colonialization, emperialism, and
\end{itemize}

\subsection{Outcome of Strikes}
\begin{itemize}
  \item SF State and UC Berkeley got their Ethnic Studies programs
    for Chicano, Black, Asian, and Native coming later.
\end{itemize}

\subsection{Liberal Asian American Feminism}
\begin{itemize}
  \item Pragmatic and reformist approach to social change
  \item Invested in electoral politics 
  \item Politics of inclusion
  \item If people gain political power, they can make change top down
  \item In Hawaii, the people have been able to achieve great feats.
    But there are still the effects of colonialism in affect and unhoused native communities
\end{itemize}

\subsection{Radicalism vs Liberalism}
\begin{itemize}
  \item Liberalism:
    Rooted in Capitalism,
    idividual rights,
    protection of private property,
    anti discrimination.

    Begs questions such as law and order and people
    who are more subject to violence.

    Everyone being equal under the law.
    Diversity = justice. 

    Critiques on it's basis of capitalism and racism.
    There is the racist celebration of colorblindness.
    Private property and class live in a liberal world.

    Views race as operating independantly from capitalism.
    Rather than racism and capitalism being linked.

  \item Radicalism:
\end{itemize}

\pagebreak
\section{Yellow Power Movement}
\begin{itemize}
  \item late 1960s
  \item Rejected the silent steoreotype 
  \item Reclaiming political and racial idenitity
\end{itemize}

\subsection{Pair Share}
\begin{itemize}
  \item Asian Americans were the model minority
  \item They seem to have got their stuff together,
    were the most advanced economically and knew silenced
    their own identities in order to be "prostituted" to the white society.
\end{itemize}
\subsection{Ethnic Studies at UCR}
\begin{itemize}
  \item Options of Africana, Asian American, Chicano, Native American, Ethinic Studies
  \item Broad when it comes to intro courses
  \item Upper division courses are intersectionality and specific
  \item Research/Capstone courses:
    Can be working in research or working with a community
  \item Careers:
    community and sociology. Working in unions, DEI, civil service,
\end{itemize}

\pagebreak

\section*{Lecture Week 2b}
April 9th, 2025

\section{Birth of Term Asian American}
\begin{itemize}
  \item Formation of AAPA was a milestone for the history of Asian Americans. It 
    brought individuals together under pan-Asian racial and organizational identity
  \item This was not exclusive to Asian ethnic groups
    rather it would include all people of Asia, broadly.
    Originally it included Chinese, Japanese, and Philipino people.
  \item There are class difference between Asian american populations and 
    national groups. Making it quite heterogeneous.
\end{itemize}

\subsection{Identity based politics}
\begin{itemize}
  \item Our politics shapes our identity and vice versa
\end{itemize}

\subsection{Imperialism}
\begin{itemize}
  \item Techniques of power and control of larger powers  
    onto smaller
  \item Financial
  \item Military
  \item Cultural/educational
  \item Colonialism?
  \item This causes organizations such as the UN it order to make sure every
    country has their voice heard.
    Then how there is control over global decision making
    and global finance.
  \item The act of global survalleance and warfare
  \item There are justifications for imperialism by
    talking about the white man's burden
    and helping out the developing powers.
\end{itemize}

\subsection{Free Write: Capitalism}
\begin{itemize}
  \item Capitalism works by having the working class make money and pay
    taxes for the government. Keeping the government working while 
    the working class is paying their own goods.
  \item Everyone must work for their own goods
    and some will be unjustfully wealthier than others.
    Those business men and born into wealth will not require to work
    meanwhile they can enjoy their wealth and spend it on anything.
  \item The role of capitalism is to encourage those to work
    hard since they are rightfully rewarded. Although there
    are somes who had it very easy. Then some
    who were born with little wealth and began
    in generational poverty.
  \item Privately own land and means of production.
    There is private individuals and coorperations that handle and 
    figure out what they do with that land.
    Build a factor, use it to farm, or to live in that land.
    Private property means the indegenous and wildlife 
    is now displaced.
  \item Focused on wage, labor, and profit. It is meant to get away from slave labor,
    now everyone is compensated for their work.
    Working for a company should be voluntary labor that is correctly paid.
  \item Use labor vs exchange value.
    Exchange value is the profit gained from selling a good that
    had the cost of material and cost of labor in order to produce and sell this good.
  \item Driven by the desire to maximize profit and demnad. Rather 
    than the good for the people. 
    Would minimize labor cost as much as possible. You need enough for those
    people to sustain their own lives and hoard wealth. Rather than to
    evenly distribute funds within the coorperation.
  \item Exploitation of labor requires people to be seen as not equal, that
    being race and sexual identity. Then the fear of having workers unionize.
  \item Requires a power differential between workers and capitalist.
    Then workers must not complain or ask for more and stay submissive
    to the capitalist.
  \item Racial capitalism works by politicing and militarism against
    those oppressed groups of people.
  \item Racism provides the structure for capitalism's ideal environment.
    Exploitation of workers.
  \item You can't get rid of racism without capitalism. They are ever fueling
    each other.
\end{itemize}

\subsection{Maximization of Profits}
\begin{itemize}
  \item There are fixed costs and variable costs
  \item Labor and labor time are variable costs. This is what
    you can change to maximize profit.
  \item This is important since labor is typically outsourced in 
    other countries where labor is cheaper
  \item Then try to make more product during the day since you can increase production
  \item decreasing wages is another method, but typically done by finding cheaper
    labor markets.
  \item Advances in technology to automize and reduce human cost, since
    computers are cheaper than humans.
  \item Skirting labor and environmental laws. such as decreasing safety
\end{itemize}

\subsection{Class}
\begin{itemize}
  \item Owners of enterprises/bosses, they hire and fire
  \item Waged workers
  \item un/underemployed surplus population
  \item People can produce goods without managers, but this
    only matters for the production of profits.
  \item Landlords are in the topic of owners, they are part of 
    primitie accumulation.
    They don't contribute to society, they just profit from the leases.
  \item Profits are the unpaid wages to the workers who worked for it.
    In an egalitaion society, the workers would be paid for their labor.
    Rather, it goes to their shareholders as an incentive.
  \item Surplus workers are those not contributing to the economy
    and are not working. They are not producing goods and are not
    being paid for their labor. If they work, they can be threatened to be
    fired and their wages are strategically low.
  \item Since there is a surplus of workers, and your current workers are threatening to unionize,
    you could easily fire all your workers and find new ones (since a surplus exists)
  \item Surplus can include those taking government benefits. Removing these would put
    people at risk and make people work for less.
  \item The state pays to incarcerate people, prisons are privately owned
    and they are profiting from the labor of those incarcerated.
  \item Public funds goes into private hands, all to pay to incarcerate people.
    Surplus populations are funneled into prisons.
    Then taking people their communities makes them not able to work
    and make money for their communities. Along with
    capitalism profiting from the labor of those incarcerated.
\end{itemize}


\subsection{Cold War Domestic Racial Politics}
\begin{itemize}
  \item Black Civil Rights Movement
  \item Displacement of Black resistance and "deviancy"
  \item Birth Asian Americans being the Model Minority Myth and assimilating
    with white society.
  \item If people of color, such as Asian Americans are able to succed, then that means
    other people of color are dysfunctional and not able to succeed.
  \item Criticism to Black people about keeping themselves in "Cultures of Poverty"
\end{itemize}

\subsection{AAPA and Self-Determination}
\begin{itemize}
  \item Self-Determination
  \item AAPA focused on self-determination and the right to self-determination
  \item Self Governnance
  \item This thought is quite nationalist
  \item Political ideology of self Governnance
\end{itemize}

\subsection{Anti-Assimilationist Politics}
\begin{itemize}
  \item Become more like the majority

    Invested in the capitalist system and be like the rest of America.
    Japanese wanted to show they were very American right after the war.
  \item Power?
    \item Respectability Points
\end{itemize}

\subsection{Solidarity and Incommensurability}
\begin{itemize}
  \item Contradictions of solidarity and Incommensurability
  \item Anti-Black and Anti-Asian racism
\end{itemize}

             g
\end{document}
             e
\section*{Reading Journals Guide}
\begin{itemize}
  \item 400 minumun and 600 max words
    Will not get graded if you go under or over
  \item Submission as a .PDF -> Canvas
  \item You can choose any of the assigned readings

    Options for Journal 1: Week 1, 2, 3

    You can mix and match reading between the different weeks
  \item Journals due: week 3, 6, 9
  \item 2 meaningful quotes that are properly cited 

    Quotes can be from the same reading, preferably from two different readings
  \item Get internal context: anything that was discussed inside of the class
  \item External context: A connection to yourself or the world.
  \item No work cited, rather In Text Citation (Author and page number)

    That doesn't count torwards word count
  \item The quote should not take more than two lines
  \item paragraph style, no bulletpoints :kekw:
\end{itemize}
  
