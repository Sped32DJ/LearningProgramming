\documentclass{article}
\title{Week3 Reading}
\author{Danny Topete}

\begin{document}
\maketitle

\section{Week 3a - April 14th}
\subsection{Free write - Immigration Timeline}
What are some trends that I notice in 
https://immigrationhistory.org/timeline/
\begin{itemize}
  \item I see that there's a lot of acts exclusionary acts
    then court cases that are trying to overturn those acts
  \item Immigration is something that we made illegal by our own choice.
    It was America's choice to make it illegal.
    Then discriminate based on the nationality or legal
    status of the people. Along
    with all of this being quite recent as in
    start in the beginning of the 1800s until now.
  \item The timeline includes times where people were excluded 
    as many times as there were acts to help people from being
    excluded. So an equal amount of fair and unfair acts.
    Every step forwards had three steps back for immigrants.
  \item There are plenty of supreme court cases and challenging
    the US constitution.
  \item This shows the immigraiton policy over time
\end{itemize}

\subsection{Transatlantic Slave Trade}
\begin{itemize}
  \item The slave ships brought many people into the Americas
  \item Over 10m people came to the Americas
  \item 400k to North America
  \item Meaning that the people were very tightly packed in these
    slave ships
  \item Millions more killed or perished, meaning slaves were very
    replaceable and didn't mind killing them.
    They were so cheap and abundant
  \item Abolished in 1808
  \item Meaning that now these slaves must reproduce in order to make
    new slaves
\end{itemize}
  
\subsection{}
\begin{itemize}
  \item Good christians couldn't enslave other christians. But they could
    morally enslave those who are not christian, meaning all those slaves.
  \item These slaves were treated like commodities.
  \item People started as Africans and became black through the 
    Transatlantic Trade
\end{itemize}

\subsection{Chinese Labor}
\begin{itemize}
  \item California Gold Rush
  \item Post-Emancipation
  \item Labor competition
  \item They were seen as lesser and were justified low wages.
    There were racial logic against chinese laborers
  \item Later this would grow as white resistance
    then there would be jobs specifically for white people.
  \item Racialization produced laws and containment of 
    Chinese populations. Making many China towns
  \item Then people stoning Chinese due to being a threat
    to the white man's jobs.
\end{itemize}

\subsection{Why do people migrate?}
\begin{itemize}
  \item Choice (prefer the weather or education), 
  \item cyclical (nomatic and seasonal people),
  \item chain (people who follow thier family),
  \item economic (better oppurtunitys, those not provided back home),
  \item political or refugees,
    fleeing warzones, coming to a safer place.
  \item Internally Displaced Persons
    Forced to move within a country due to famine, war, political
  \item Do migrants take jobs? 
    no, they typically work in the lowest wage jobs.
    Typically take on multiple jobs and get exploited by their
    boss.
    There are jobs not being fulfilled since there are not enough 
    low skilled workers.
\end{itemize}

\subsection{Migrant vs Immigrant}
\begin{itemize}
  \item Immigrant typically means that a nation decides
    to exclude foreigners from entering their nation.
  \item Migrant does not discriminate against people
    who move by choice or coersion.
\end{itemize}

\subsection{Illegality}
\begin{itemize}
  \item Migrants come and some take advantage of the cheap
    and exploitable labor.
  \item Laws are the state's attempts at racialized and
    gendered to control these "illegal" aliens
  \item Gives nations permission to discriminate against
    those who enter the nation from another.
  \item They typically don't have the same protections as those
    who would be a citizen in that nation.
\end{itemize}

\subsection{Pair Share - Illegality}
\begin{itemize}
  \item Illegal is typically a word that is used for someone
    who is a criminal and has broken the law.
    Althogh, we are freely using this against people who migrate into
    the US as a form to discriminate.
  \item Illegal has quite the negative connotation and can easily
    paint innocent people escaping a warzone as a criminal for
    daring to step foot across the wrong man-made line.
  \item Illegality helps shape racial capitalism since it
    helps justify the exploitation of the cheap labor.
    Especially to those that can easily be threatened and will work
    such a low wage. Especially since they don't have many options
    and they may be scared of the government.
  \item Being illegal means that people can be detained, policed,
    can't go on strike, have no rights,
    can be deported, can be taken out of their country
    and away from the family they have in the current country.
    They can get their tips taken away and work hard manual labor
    for so cheap.

\end{itemize}

\subsection{No one is Criminal}
\begin{itemize}
  \item No one is illegal?
  \item No one is a criminal
  \item This then paints the question between
    "good" and "bad" immigrants and who deserves it and who 
    doesn't. Such as those who have criminal records and come to the
    US since the police, an opposing gang, or selling drugs
    in a new market.
\end{itemize}

\subsection{What are borders?}
\begin{itemize}
  \item Territorial boundaries, lines
      in the sand. An interface between two different places.
    \item Interfaces that connect places between each other.
    \item They are porous boundaries, so it lets some things
      pass through while other things are blocked. This goes
      for people and goods. Inclusion/Exclusion
    \item Beyond territory, so it goes for all the detention facilities,
      military zones in other countries, etc.
    \item Historically been used to discriminate against people
\end{itemize}

\subsection{Dividing Native Lands}
\begin{itemize}
  \item Indigenous dispossession
    In the way of making border, it completely ignores the previous
    native people who once lived there, thus dividing natives.
  \item Tohono O'Odham, a community that is between US and Mexico.
    Between Arizona and Sonora.
    Goes to show how this is just sovereignty and how unnatural it is.
  \item Divided communities
  \item Customs and Border Protections.
    Now natural resources of private property goes under ownership
    of he who owns it.
\end{itemize}

\subsection{Who or what crosses borders "legally"?}
\begin{itemize}
  \item People,
  \item Goods, but this is regulated to the desired goods and
    excludes the ones the government doesn't really want.
  \item Capital, used to dictate the flow of capital between two
    nations. We also criminalize where money goes, such as
    "terrorist organizations", such as those who help out
    Palestine civilians, then the expected Cartels.
\end{itemize}

\subsection{Who or what transgresses borders?}
\begin{itemize}
  \item People
  \item Goods, those legal or labelled illegal. There's a reason
    some are retricted and others are allowed.
    Either since they are harmful to the nation state for any
    of their reasons.
  \item Animals and environment, they are seperated by the walls.  
    Meaning their ecosystems are divided up.
\end{itemize}

\subsection{No Borders}
\begin{itemize}
  \item No one is criminal
  \item Freedom of movement
  \item Border reform doesn't work, just leads to more
    pain to innocent people.
  \item Ablish the nation-state and citizenship
  \item Based on anarchist principles and undoing racial capitalism
\end{itemize}

\subsection{Border Abolition}
\begin{itemize}
  \item Builds on the call for no borders
  \item Building up and simltaneosly tearing down
  \item Decolonial
\end{itemize}

\pagebreak

\section{Lecture - Week 3b}

\subsection{Orientalism}
\begin{itemize}
  \item Protrayals of the Orient
  \item Authority over the Orient
  \item Edward Said, his take on Orientalism

    \begin{itemize}
  \item Why when we think of the middle east, we immediately have
    plenty of misconceptions about these people.

  \item This is how we are able to make so many assumptions about a person
    based off the color of their skin and how they look
  \item This is how we are able to make people of the 
    middle east threatening and have
    all this media about how these people are.
  \item Media portrayals of the middle east as
    they are enemies.
    Especially in movies where plenty of muslim terrorist
    are killed and these people are always the bad guys.
  \item The conceptions that arabs are the enemies of the west.
    \end{itemize}

  \item Colonization, imperialism, Zionis
\end{itemize}

\subsection{Varieties of Orientalism}
\begin{itemize}
  \item Ancient Greek
    \begin{itemize}
      \item Said that they look like Asians
        suffer from Lavandis (look yellow)
    \end{itemize}
  \item French and British
    \begin{itemize}
      \item Used it as justification to justify
        Zionism around WWII.
      \item Viewed Palestine as unhealthy and dirty
        and the people as uncivilized. Waiting to get
        habited and taken over by 
        the rightful ruler
    \end{itemize}
  \item American
    \begin{itemize}
      \item Portrayals of Asians as foreigners
      \item People that can easily be excluded from society
        and can be dominated in politics.
    \end{itemize}
\end{itemize}

\subsection{Orientalism as Racial Formations}
\begin{itemize}
  \item Exoticism
    \begin{itemize}
      \item Roamnticizing the orient
      \item Making the women as sexual objects
      \item Then demasculating the men
    \end{itemize}
  \item Stereotypes and racist trope
  \item Derogatory as an assertion of power
\end{itemize}

\subsection{Consolidation of a white racial Formation}
\begin{itemize}
  \item Orientatalism poses the other against the collective
    white racial formation
  \item Law, policy, media, culture
  \item A series of court cases where Asians,
    were repeatedly denied citizenship (since they were non white)

    Notably Takao Ozawa
    \begin{itemize}
      \item He attended UC, born in Japan, spoke English.
      \item Asked court to be a citizen
      \item Saying that at heart he was a true american
      \item Said, If he had stayed out of the sun, he would be whiter than those white
    \end{itemize}
  \item Bhagat Singh Thind
    \begin{itemize}
      \item Applied to citizenship after serving WWI
      \item He was a "high case indian of aryan descent"
        in fact, caucas mountains was in India. Caucasian people
        originated in India, then moved to Europe.
        Then the court dissagreed because that's not 
        European.
    \end{itemize}
\end{itemize}

\subsection{Orientation in Pop Culture}
\begin{itemize}
  \item In what way does Orientalism in popular culture function as
    a political tool or have political ramifications?
    \begin{itemize}
      \item Orientalism enforces ideas that Americans
        already have about groups of people.
        Such as Iron Man being captured by
        rich Arabs and he gets stranded into a cave on the
        side of the mountain and he rebuilds his suit and
        kills a bunch of brown people. Especially when this movie released
        a decade after 9/11 and during the Afghanistan War.
      \item It also helps bring these thoughts to new people such 
        as children that don't know about all the politics going
        on. Instead they just see that who are always the bad people
        during movies and shows.
      \item Then how movies were happening such as Rambo during the Vietnam war.
    \end{itemize}
\end{itemize}
\section*{Yellowface and Brownface}
\begin{itemize}
  \item Actors of different races are looking like another character
    that is playing upon Stereotypes
  \item This is when they serve as the punchline for something
    regarding the race they are supposed to be portrayed as.
  \item Most of the time, this is just the roles they were offered by 
    the writers or producers
\end{itemize}

\section*{Pair Share - What about the impact yellow/brown face has on society? }
\begin{itemize}
  \item The impact it has on society is that this is typically done
    just to act upon Stereotypes.
  \item I believe that the impact on soceity is that it just
    stunts our progress of how far we have come to gain our rights.
  \item I believe this practice is still persistent because it isn't
    blackface. I believe that we already have came to a concencus that blackface
    is offensive and this is not unanomously offensive yet.
  \item Then they just play upon "compliments" torward Asian
    people such as making them the smartest person in the show.
\end{itemize}

\subsection{Politics of Representation}
\begin{itemize}
  \item Representation is important but insufficient
  \item Control over stories
  \item Liberal diversity and inclusion politics
  \item Liberalism is justice without the state inflicting violence
    and giving people their freedoms. 
    Especially without government power leaning over them (imperialism, etc).
\end{itemize}

\subsection{Yellow Peril}
\begin{itemize}
  \item Racial construction of Asians as a threat
  \item Uncivilized, lower intelligence, dangerous
  \item Perceived labor competition
  \item Anti-chinese and Japanese organizing, riots, propaganda, and laws.
  \item Contemporary yellow Peril
  \item White men stopped working the automated manual labor jobs since
    the migrants would work for a lower wage.
  \item Many Asian migrants were laundary workers along
    with working on the railroads.
  \item Response to new wave of Asian migrants,
    they were demonized by white settlers.
  \item Men were just seen as opium and gambling addicts. Then
    the women as prostitutes that were tempting white america.
  \item Japan then became a new threat due to new manufacturing strength.
    Then they were seen as a threat due to the US auto industry going down.
\end{itemize}

\end{document}
