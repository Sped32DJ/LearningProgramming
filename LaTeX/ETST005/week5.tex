\documentclass{article}

\title{Week5 ETST005}
\author{Danny Topete}
\date{April 28th, 2025}

\begin{document}
\maketitle

\section*{Lecture - Week 5a}
\section{Model Minority}

\section*{Pair Share}
Reactions to the story?

The story from 1966 about in the Chinatown, the children are
raised so well. They strive for a good education.
Then the safest part of town is the Chinatown. The city around
it is more dangerous.
The cutlure is different and everyone sees each other like
as part of their family.

\subsection{Discourse}
\begin{itemize}
  \item Certain racial groups can 'make it' in American society
  \item Cultural disposition for sucess
  \item Liberal multiculturalism can incorporate racial others
  \item 1965 Hard Celler Act
    \begin{itemize}
      \item People with professional degrees have class advantages over
        other citizens
      \item Black citizens were never able to get an education under Jim Crow
      \item Asian Americans came with a level of class leverage.
        Skewed the image of Asian Migrants.
    \end{itemize}

  \item Anyone can make it into society, as long as they are hard working
  \item Anyone that didn't make it, they are not working hard enough
  \item They are not educated and are failing at being a good citizen.
    Asian Americans have perfectly fulfilled this model.
  \item This justifies the incarceration and deportation of other groups.
    Since they are not trying hard enough and they are culturally
    handicapped.
\end{itemize}

\subsection{Anti-Blackness}
\begin{itemize}
  \item It is more justification as to why Black people deserve what they got
    since they are not doing as well as Asian Americans.
  \item The sucess of Asian Americans is used against Black Americans
  \item Assumes that all Asian Americans have better education, economic, cultural values,
  \item Asian Americans typically live in Major US cities and major states, which makes
    their statistics for their race make sense. Since majority of Asian Americans live in
    California, Washington, Hawaii, New York, etc.
  \item Then it excludes the South East Asians that don't have the same statistics of
      educations as Taiwanese, which have a the highest statistic for prevelance of bachelor degrees.
    \item Model Minority myth ignores all the laws and how we are structurally built to make
      black Americans fail.
\end{itemize}

\subsection{Operations}
\begin{itemize}
  \item Political Domain
    \begin{itemize}
      \item To show that Asians did such a good job at suceeding, why
        can't other monirity groups achieve the same heights?
    \end{itemize}

  \item Cultural domain
    \begin{itemize}
      \item Asians are efficient and aspire great heights
    \end{itemize}

  \item Economic Domain
    \begin{itemize}
      \item Abstract labor, interchangeable under capitalist orders
      \item One asian can be replaced for another,
        they are not the same as white workers
      \item They are an economic threat as yellow peril
      \item They are a threat to white professionals in the workplace
    \end{itemize}


  \item Transnational Domain
    \begin{itemize}
      \item Produces Asia as a threat to empire
    \end{itemize}
\end{itemize}

\subsection{As an Anti-Asian Phenomenon}

\begin{itemize}
  \item Homogenizes and racializes
  \item Places an expectation on Asian Americans to overperform
    \begin{itemize}
      \item Anyone who can not meet these expecations are considered "failures"
    \end{itemize}
  \item Reverse racism and affirmative action?
    \begin{itemize}
      \item They are discriminated against in job and university applications
        since they are poor competition agianst their peers.
      \item Rather than looking at their general intelligence, they are
        comparing it to other Asians
      \item There is programs to promote diversity are now seen to be as racist,
        then as DEI inside jobs (or what ever)
    \end{itemize}
\end{itemize}

\section{Anti-Black Racism pt2}

\begin{itemize}
  \item The upward preassure of the Model Minority is a function of anti-black racism.
  \item Model Minority as a response to black power
  \item False equivalencies of different forms of racism
  \item Model monirity myths says all different forms of racism people face are similar.
    If Asians can survive with racism, then so can black americans. 
    When in reality, the real reason they succeed is since some can only
    migrate from their home country with professional degrees, or with existing capital.
\end{itemize}

\subsection{Anti-Blackness \& Anti-Black Racism}

The following may be incoherent since I couldn't not comprehend what
prof was saying...

\begin{itemize}
  \item Anti-blackness is a social structure,
    it is a world view. White is the superior and Black is the anthesesis of whiteness.
  \item Anti-blackness is unique as it holds people against themselves for being a race.
  \item Anti-black racism is distinct since it goes against black people that started with slavery.
  \item Structured by racism that is formed by racial social and legal structures against
    black people. All strucutred by white racism.
  \item Anti-blackness there is no such thing as solidarity as long as it exists (??)
  \item Racial regimes are not fixed or permanent but "incredibly fragile 
    and must be constantly remade".
\end{itemize}

\subsection{Murder of Akai Gurley and Support for Peter Liang}
Free-write: How do you think anti-asian racism masquerades anti-black
racism is the murder of Akai Gurley and prosecution.

I believe this is a mix of racism against Asian and Black people.
There are several cases where white officers kill black people for no reason
at all, or exert excessive force which ends up killing the person, leading
to the officer murdering someone.
I do believe that Liang should've received first degree manslaughter,
but it is true that several cases have happened to where white officers face
no repercassion because they accidentally killed an innocent black man.
I believe that the court is asking the wrong questions, such as why officer Liang
had his gun out, why he was at a housing project, and why he shot in the dark.
No one goes to a home with their gun out unless they believe their life is threatened.

I believe this is anti-asian because white officers are well known for facing
no repercaussions at all for killing innocent black men. They historically
have never been held accountable for situations like these.

Then this is anti-black since there was only one reason why

\subsection{Different Racisms}
\begin{itemize}
  \item Anti-Asian Racism is real
  \item Anti black racism not equal anti-asian racism
  \item anti blackness valorizes asian-ness
\end{itemize}

\subsection{Asian Americans and Black Minstrelsy}
nvm, this one was never covered.


\section*{Lecture Week 5b}

\section{Diversity}
Free write, how were you educated about diversity before this course?
\begin{itemize}
  \item Diversity is the act of getting everyone along and not exclusing people from race
  \item Growing up in a majority hispanic city, the students who were other races would be excluded,
    (mostly since students of similar backgrounds would friend each other)
    so diversity would be including that students from different backgrounds and hearing them out.
  \item Making sure that everyone's voice is heard out along with their 
    particular issues that they might be facing in their community,
    issues unknown to other people from other backgrounds.
  \item People have different perspectives on topics because everyone has lived
    different experiences along with different cultural experiences.

\end{itemize}

\subsection{Capitalism Needs Racism}
\begin{itemize}
  \item Capitalism is based upon stealing things
  \item Primitive accumulation, capitalist steal land and labor
  \item Race was invented in order to justify slavery
  \item Then putting two nations and thining that they are the same
    because "they look alike"
  \item Racism is used to show that other people are lower in hierarchy, therefore
    they should not be treated equally 
  \item Racism came from capitalism and the only way to get rid of racism
    is to get rid of capitalism
\end{itemize}

Why is important to recognize the racial roots of capitalism?

This is important since America has historically exploited workers. Non white Americans
have always histtorically been exploited meanwhile white Americans have always been the ones exploiting
those migrants workers and not really feeling any sympathy. Capitalism has no sympathy and just operates on
reducing labor costs as much as possible to increase profit.

Why do we care about capitalism in Ethnic Studies?

We care about capitalism since this is the basis of how racism was born in the US.
Getting peopel from other countries, putting them on ships to do labor for us.
Then justifying that white america is helping them out or as to why they are allowed to do that.
Capitalism was born from racial logics. 

\subsection{Racialism}
\begin{itemize}
  \item Permeated feudal Europe prior to capitalism
  \item First proletarians were racial subjects
    \begin{itemize}
      \item Irish, Slavs, Jews, Roma
    \end{itemize}
  \item Depended on dospossession, slavery, colonialism
  \item Development of racial logics and tactics
  \item Racialism predates capitalism
  \item The slave ships created black people
\end{itemize}

\subsection{Racial Capitalism}
\begin{itemize}
  \item South African anti-apartheid movement
  \item Cedric Robindson's Black Marxism
  \item All capitalism is racial capitalism
    \begin{itemize}
      \item You can't remove apartheid without removing capitalism 
      \item There is still capitalism in south africa and 
        what remains from apartheid still lives
    \end{itemize}
  \item Robinson: Black Radical Tradition is revolutionary consciousness 
  \item Race and capitalism are mutually constituting,
    it calls for new means of racial differentiation
\end{itemize}

\subsection{Black Radical Tradition}
\begin{itemize}
  \item Throughout history, Black intellectals
    developed sharp analyses of race and capitalism
  \item Radical consciousness rooted in struggles against white supremacy
  \item Diasporic
  \item Black Lives Matter
  \item White people did not take race seriously, but Black people did need to
    since it is the structure that decides their fate.
\end{itemize}

\subsection{Geographies of Racial Capitalism}
Pair Share: What do you think is meant by "capitalism requires
inequality and race enshrines it?"

capitalism requires inequalities. Global ideas of race are required to have 
this unanomous agreement and acceptance of what capitalist will pay their
laborers.

\subsection{Exploitations of Labor}
\begin{itemize}
  \item Slavery and Coolies
  \item Squeeze labor costs through racial exploitation
  \item Pit workers against each other
  \item Importation of cheaper labor
  \item "The racist logic is blaming the migrants for being poor and accepting the low wages" 
    (I guess this makes sense)
  \item Workers must stay in line and keep being Victims of wage theft. Especially if your job
    makes it seem as if you are repleaceable or are threatened of having the law
    sent torwards you if you are considered "illegal"
\end{itemize}

\subsection{Other Realms of Racial Capitalism}
\begin{itemize}
  \item Legal or de facto discrimination
  \item Imperialism and labor offshoring
  \item Rights, freedom, citizenship, imprisonment, abandonment
  \item Flint Michigan has seen all the coorperations move their offices,
    along with abondoning the city. Leaving the people who lived there unemployed,
    then the city can't repair the issues with the water since the coorperate taxes are gone.
  \item Rights, freedom, citizenship, imprisonment, abandonment
  \item Race evolves to accomodate capitalism
  \item The state
  \item Red lining, which decides what area banks will not approve a loan for.
    Making housing unaccessible for certain people in certain areas.
\end{itemize}

\subsection{Diversity and Racial Capitalism}
\begin{itemize}
  \item How might DEI initiatives be implicated in racial capitalism?
    Do you think DEI is a solution to the inequalities associated with racial capitalism?
  \item The way that capitalism exists is just going through new means of diversity
  \item They are in the way of repairing past harms
\end{itemize}

\subsection{multiculturalism, diversity, equity, and inclusion}
\begin{itemize}
  \item Particular vision of antiracism
  \item Different races and cultures exist simultanously within a society,
    we are a salad bowl rather than a melting pot
  \item Relatinoships between different communities
  \item Celebrates diversity and representation
  \item Rejects forced cultural
\end{itemize}
  
\subsection{Racial Liberalism}
\begin{itemize}
  \item Rooted in philantropy and equal oppurtunity
  \item Individual education
  \item Regulatiev function
  \item Global capitalist
  \item Just because you are aware of racial liberalism,
    it doesn't mean that racial reperations have been done and racism has
    been solved
  \item To Kill a Moking Bird is not very good in the way there are
    topics of white man's burden and white people helping out black people and "white saviorism"(that's what I caught)
\end{itemize}

\subsection{Liberal multiculturalism}
\begin{itemize}
  \item Civil Rights movement overcame racism
  \item White liberals absolved of racism through diversity
  \item Co-opted antiracist movements
  \item There was a level of white saviorism, when it came to We Are The World.
    Micheal Jackson co-produced a song with many other people. Then you can donate directly
    to people in Africa who need the money
\end{itemize}

\subsection{Ethnic Studies Requirement at UCR}
\begin{itemize}
  \item The university's answer to addressing systemic inequalities
  \item Is this a race radical initiative or diversity management?
  \item Ethnic studies is a site of potentially liberatory education
  \item Institutionalized ethnic studies can also build complacency
\end{itemize}

\subsection{DEI as a Threat}
What do you think the attacks on DEI mean for Asian Americans?

\begin{itemize}
  \item DEI - Diversity, equity, Inclusion
  \item Came from anti-exclusion laws
  \item Helps foster a more inclusive campus
  \item Some say that it prioritizes some groups over others, then how there are
    costly efforts
  \item It is in College Campuses help out with diversity and systemic racism.
    But it is ineffective.
\end{itemize}

\end{document}
