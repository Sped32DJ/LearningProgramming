\documentclass{article}
\title{Week 6 ETST005}

\author{Danny Topete}
\date{May 5th, 2025}

\begin{document}

\maketitle

\section*{Week 6a lecture}

\section{Feminism}

\subsection{Black Feminism}
\begin{itemize}
  \item Comabhe River Collective:
    interlocking oppression
  \item Barba Smithc: neither white feminism nor Black nationalism
  \item Collins: Reclaming and constructing Black women's knowledges
  \item Identity politics, these women have a right to show off their
    identity and work for their people of their identity
  \item Taking the stand point of a black woman, we can raise consciousness
    for helping build a public consciousness (Not sure how to interpret this)
  \item Black Feminism thoughts and knowedlges already exist in the work that
    black women do. We need to create a public consciousness to set up
    a political project
\end{itemize}

\subsection{Different Women of Color}
\begin{itemize}
  
  \item Grace Hong: Women of color feminism as coalition through difference
  \item Chandra Mohanty: third world feminism is necessary by the globalization of capital
  \item Chicana feminsim
    \begin{itemize}
      \item Not something that someone is born with
      \item This is recognizing the importance of how ones is identified
      \item Coming to this identity is to bring others who also idenitify
        to this
    \end{itemize}
  \item indigenous feminism
\end{itemize}

\subsection{This Bridge Called my Back}
\begin{itemize}
  \item Love letters by women to women
  \item Listening and learning to see each other's "ways of seeing and being"
  \item Striving toward a Third World Feminist political formation across racial divides,
    but specific to different racial circumstances.
  \item Politics to third world feminism is not universal
  \item What unites them is the uniqueness, but each have their own
    specific problems and political projects
  \item They are brought together by similar ways of oppression
  \item Third World writting is form of self presevation and survival
  \item "A bridge gets walked over, over and over again"
\end{itemize}

\subsection{On the Origins of "Women of Color"}
\begin{itemize}
  \item What does it mean to say "women of color"
  \item Began with the "Black Women's Agenda", then 
    all the minority women wanted to be included. Then
    eventually it was called "Women of Color"
  \item This began as a collaboration for all oppressed women of color
  \item It is a set back to call it a single race problem, when getting
    together that is also oppressed equally, that is setting us back
  \item We made that saying, not white people
  \item This is deliberate self-naming to get everyone together
\end{itemize}

\subsection{Asian American Feminisms}
  How do we counter racial constructions of the
    "model minority" that sustain racial capitalism and settler colonialism

    \textbf{Free write}: What does it mean to you for Asian Americans
    to counter their "racial construction as the 'model minority' "
\begin{itemize}
  \item The racial construction of Asian Americans being model minorities came from
    when Asian Americans were constructing their public image
  \item It means that Asian Americans are trying to counter these false bold claims
    towards them about how they should not be struggling and they are supposed
    to be well adjusted.
  \item To really go against the model minority, they must be against it.
    Not abiding by the social preassures and not letting yourself be defined
    by a stereotype
  \item Gives Asian Americans high standards for the way that they should be living.
  \item To really counter what society expects of Asian Americans, there must be
    movements and representation of more Asian Americans where they are not
    playing into the stereotype
  \item This is about racializing Asian people, but also racializing all people
    around Asian People
\end{itemize}

\subsection{Yuri Kochiyama \& Grace Lee Boggs}
\begin{itemize}
  \item Yuri Kochiyama is the one who cradled Malcom X in her arms
    after he got shot
  \item She was involved in reperations movements for Japanese people
  \item Anti imperialism and pan asian work
  \item There were movements for reperations have caused \$10K of families that were incarcerated
    and were in inturnment camps
  \item Grade Lee Boggs and James Boggs
  \item Kochiyama and Boggs were two leads that
  \item They never thought about Grace as being Chinese American female,
    she was just Grace. Very influencial for Blacks power activist.
  \item She still kept talking about revolution even into her 90s
    in Detroit circles
\end{itemize}

\subsection{Feminist Refugee Epistemology}

\textbf{Refugee:} Someone who has fled or been displaced from their place
of origin for various reasons

\textbf{Epistemology:} a way of knowing or coming to understand something, e.g.,
how we understand the world we are living in, care, or survival.

Refugee activism goes on to the refugees who came to a new place
to restart their life and rather have came to face new of the same problems.
Epistemology is about how these people have already been displaced,
how do we go about this and being accepted.

\section{Some Gendered Forms of Labor}
\begin{itemize}
  \item Reproductive Labor
    \begin{itemize}
      \item The biological reproduction
      \item Along with taking care of the child, that including
        the kids that you are helping to care for
    \end{itemize}
  \item Care work
    \begin{itemize}
      \item The caretaker workers
      \item Often times this is undocumented women
    \end{itemize}
  \item Feminized labor
    \begin{itemize}
      \item Teachers, healtcare workers
    \end{itemize}
  \item Support capitalism
  \item Affective labor
  \item Racial capitalism wouldn't exist withut gendered forms of labor
  \item Many of the time this type of labor is paid, unpaid, or underpaid.
\end{itemize}

\subsection{Care as Feminist Praxis}
\begin{itemize}
  \item "a language of care that focuses on refugee wellbeing as a 
    politics of hope and survival"
  \item social and kinship systems
  \item kindship, not just blood relations, but respect and mutuality between people
\end{itemize}

\subsection{Feminist Democracy}
\begin{itemize}
  \item "feminist democracy suggests a different order of relationsamong people.
    It suggests understanding socioeconomic, ideological, cultural, and psychic
    hierarchies of rule (like those of class, gender, race, sexuality, and nation), their"
    interconnectedness, and their effects on disenfranchised peoples within
    the context of transformative collective or organizational practice.
    Thus, the transformation of relationships, selves, communities, and the practices

    MAY SHOW UP IN AN EXAM
    Pair Share:What does Asian American feminism mean to you in both your daily lives and in your
    boarder political commitments
\end{itemize}

\section*{Week 6B Lecture}

\section{Queer Theory}
\begin{itemize}
  \item Saying queer theory is not part of Black Studies is like
    saying that feminist movements are not part of civil rights activism
    Or black rights movements.
  \item Queer theory is included in ethnic studies is part of 
  \item This is threatening because saying that Queer theory is just agenda
    and minimzing that people actually exists with real claims about queer theory.
  \item Queer theory is important when it comes to black studies due to the
    intersectionality of how people of color that are also queer will additionally
    add to how people are oppressed.
  \item An example can be how black gay men were seen as dirty in the 80s with the AIDS epidemic
    since being queer makes it acceptable to discriminate against people again
    after accepting that it is not acceptable to discrimnate against people of color.
    A new way to discriminate agianst people.
\end{itemize}

\subsection{Why is queer theory part of Ethnic studies?}
\begin{itemize}
  \item People of color don't just experience their life through their race,
    now there is the intersectional approach of the way queer people of color exist
  \item Queer studies helps us allow to think about social structures in a different
    way.
  \item The multiple layers of oppression that people face even without mentioning their sexuality
  \item Queer indexes the multiple layers of oppression
    POC face regardless of their sexuality
  \item Heteronormativity structures our lives, creates gendered forms of labor
\end{itemize}

\subsection{Queer}
\begin{itemize}
  \item Sexuality is socially constructed
  \item Destabilizes identity categories
  \item Not just an adjective, but also a verb
  \item These categories of people that have been constructed by society
  \item It can be used a verb against people
\end{itemize}

\subsection{Queer of Color Critiques}
\begin{itemize}
  \item Reclaims queer politics
  \item Beyond identity politics
  \item Political economic analysis
  \item Stonewall was a police riot
    \begin{itemize}
      \item The criminalization against queer folks
      \item Police were harassing "gay crossdressers" (how they were identified in the 60s)
      \item Then the police started beating them
      \item Saying it was a police riot is like saying that police were starting
        a riot against queer people
    \end{itemize}
  \item Homonationalism, such as Trump banning queer people from the Army
  \item The concept of LGBT is constantly shifting, therefore
    it is more useful to add queer critique as
  \item Marxist and non-marxist takes on political economy
\end{itemize}

\subsection{Queer "Progress" Narratives}
\begin{itemize}
  \item The myth that "gay folks have it better
    now than ever"
  \item Rather than a progress narrative, there hasn't been linear
    growth and it hasn't been really getting better
  \item There are messages telling young people that "it gets better",
    but it doesn't get better
  \item THere are laws applied unevenly when it comes to queer folks
\end{itemize}

\section{Heteronormativity}
\begin{itemize}
  \item The normalizing practice and power that has been the 
  \item Power comes from sexual identity
  \item Assumes relationships should be monogamous and opposite sex
    for the purpose to produce offspring.
  \item Can be oppressive to anyone who deviates from this
\end{itemize}

\subsection{Homonormativity/Transnormality}
\begin{itemize}
  \item Homonationativity: normalizes homosexual and its adoption of heterosexual ideals
  \item Transnormativity: adhering to dominant binaristic gender norms
\end{itemize}

\subsection{Asian American X Queer Studies}
\begin{itemize}
  \item Asian Americanist critiques within queer studies
  \item Queer of color critiques within Asian American studies
  \item It is an important part of identity politics
\end{itemize}

\section{HIV/AIDS}
\begin{itemize}
  \item HIV/AIDS as queer issue
  \item Asian Americans typically have a low rates
    of HIV/AIDS
  \item Low testing rates
  \item Barriers to testing and treatment
  \item Notions that Asian Americans are cleaner and safer
    than other communities as to why they have lower rates
\end{itemize}

\subsection{Surviving Voices}
\begin{itemize}
  \item it was too taboo to talk about homosexuality, especially for Asian Americans
  \item Everyone was scared to talk about being homosexual and their voices were silenced
    since there was an association with HIV/AIDS
\end{itemize}

\section{Neoliberalism}
\begin{itemize}
  \item Free markets
  \item Self-government and regulation
    \begin{itemize}
      \item There should be politicing communities themselves
    \end{itemize}
  \item Economic rationalities
  \item Normative benchmarks are bad (standardized testing, etc)
  \item Cost-benefit analysis, supply and demand logics that get naturalized
    and maximization of profit/time/money
\end{itemize}

\subsection{Neoliberal Multiculturalism}
\begin{itemize}
  \item mutliculturalism is "the spirit of neoliberalism"
  \item individual characteristics, choices
  \item Differentiated citizenship
  \item Postracial discourse/racism without racists.
  \item Everyone has equal oppurtunity to oppress each other
  \item People can say that they don't see race, have a queer friend, or a black friend.
\end{itemize}

\subsection{Homonationalism and Pinkwashing}
\begin{itemize}
  \item Liberal lesbian and gay (BTQ+?) politics uphold the settler capitalist state
  \item Diverts attention away from multiple oppressions
  \item Masquerading dominance through neoliberalism multicultural politics
  \item "inclusivity" and citizenship based on racialized exclusion
  \item "asking someone for their pronouns before bombing them"
  \item Chevron funding queer folks is their way of pinkwashing their
    extreme environmental degredation
\end{itemize}

\subsection{Disability and "Normalcy"}
"Conceiving wellness as default, as the standard mode, is that
it invents that illness as temporary"
Then people that are sickness is temporary and care is normal


\subsection{Open in Emergency}
\begin{itemize}
  \item The Student
  \item The Mother
  \item The Village
  \item it is a card deck where the reading is to read the experiences, trauma
    and care.
  \item It was created to help shed light on what it means to 
    shed light to each character.
    Then what it means to go against it
  \item They are cards with the perspective from Asian Americans
\end{itemize}

\section*{Discussion week 6}

\begin{itemize}
  \item Work cited page for Reading Journal \#2
\end{itemize}


  
\end{document}
