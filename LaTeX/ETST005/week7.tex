\documentclass{article}
\title{Week 7 ETST005}

\author{Danny Topete}
\date{May 12th, 2025}

\begin{document}

\maketitle

\section*{Week 7a lecture}
Guest lecture

Proessor Emily Hue


\section{Refugee Project}
\begin{itemize}
  \item There has been a significant amount of people that are
    escaping countries after conflicts.
  \item Then these regfugees seek a new home to many countries away
    from their home country. Then many are not able to escape
\end{itemize}

\section{Good Refugee}

\begin{itemize}
  \item These are the people who are "good" and not
    part of the opposing force.
  \item 1980 Refugee Act, only certain people are allowed
    to come to seek asylum in the US.
  \item Only the people that were anti-communist, then originally helped
    the US.
  \item This act was created by the US government, originally in response to the
    Vietnam War, to provide a premanent and systematic procedure for the admission
    to the US of refugees of special humanitarian concern to the US,
    and to provide comprehensive and uniform provisions for the 
    effective resettlement of those refugees who are admitted.
  \item They leaned from previous histories of exclusion and containment, where
    people create ethnic enclaves. (Which is bad for the US, they prefer not having them)
    They learned that after the Chinese and Japanese exclusion that created
    China towns and Japan towns.
\end{itemize}

\section*{Pair Share}

Trihn Mai, Quiet (2015)

\begin{itemize}
  \item It is a long scrolls related to the Vietnam War. They are all
    photos of children and mothers. I assume they are all victims of the war.
  \item The portraits of the people are very respectively portraying them
  \item They are haging from the ceiling, which is them above
  \item It is a way of a memorial for these people, their portraits are
    burned into the scrolls
  \item It is mourning people that went missing, the only thing we can mourn
    are their faces, but we could never find their bodies.
  \item In east asia, morning is white, rather than black. These are people
    who have not been mourned.
\end{itemize}

Ding Q. Le, untitled, 2003

\begin{itemize}
  \item This is an art of "Napalm Girl", in a colorful and artistic fasion
  \item There are many different branded objects there, they are all American brands
  \item It is possible that they funded the war or even supported the war
  \item Many of these candies are from Nestle, which are quite controversial,
    especially with their involvement in exploiting developing countries
  \item But all the candy does go to show that she was just a girl and it is children
    that are being affected.
\end{itemize}

\subsection{Immigration Reform and Immigrant Responsibility Act of 1996}
\begin{itemize}
  \item Thousands of legal immigrants to detention and deportation every year for old
    or significant offenses.
  \item Deportation after people have served time causes the double detention
  \item Mistimeaner to felony makes it a deportable offense
  \item Immigration law and criminal justice come together
\end{itemize}

\section*{Week 7b lecture}
Topics for today
\begin{itemize}
  \item Deportation
  \item SEAA Deportations
  \item Crimmigration
\end{itemize}

\section{Deportation}
\begin{itemize}
  \item THe removal of people from a country for any reason
  \item Reasons can vary
  \item Deportation Process
  \item Deportation Logicos
  \item There is current legislation that is looking to find a way to deport
    US Citizens
  \item There is a loop hole they found about
  \item Paper marriages
  \item Then there are ways where people can be deported without a judge order or
    court hearing
  \item Trump is working to have more expedited deportations,
    claiming that courts don't have enough time to hear all the cases.
  \item This only invites further oppression and people have calmed down and gone silent as result.
    Prmotes work place exploitation and people that are not attending strikes since they may get
    their visa revoked then deported.
  \item DACA helps children of immigrants since they did not migrate illegally and they
    should not suffer from their parent's actions (prof said crimes, kinda brutal)
\end{itemize}

\subsection{Shut Down Adelanto}
What are the arguments for and against closing Adelanto Detection Centers?
How do you think closing it would affect the local community?
\begin{itemize}
  \item It would help for tax dollars to not go to this waste of tax dollars
  \item Plenty of ethical problems and inhumane actions going on in the prisons
  \item Private prisons are just taking money away from the people
    and exploiting tax dollars
\end{itemize}

\subsection{History of Deportation}
\begin{itemize}
  \item 1798, Alien and Sedition acts, for anyone that is the enemy of the state.
    Anyone that can be our politcal rivals.
  \item Chinese Exclusion Act of 1882, Chinese people must have their citizenship
    with them at all times or else risk deportation
  \item Immigration Act of 1891
  \item Palmer Raids of 1919-20
  \item Mexican Repatriation of the 1930s, many people were involuntarily deported
    to Mexico, even if they were US citizens. Even Filipinos that were not even
    associated to Mexico at all. Many people with no ties to the US. US just deported
    everyone they didn't want to Mexico.
  \item Operation Weback in 1954
  \item IIRIRA and Antiterrorism and Effective Death Penalty Act of 1996,
    created more reasons to deport people
\end{itemize}

\section*{Deportation of Lawful Residents}
\begin{itemize}
  \item Trump administration is targeting students in particular for many reasons
  \item Very easy to pick on, therefore very easy targets
  \item They are here under a student visa or maybe not even a visa. So it is really
    easy to get them.
  \item You can set an example out of them
  \item College students are more naive on their own rights and may
    be acting more carelessly
  \item Students have always been hosting the majority of the protest.
    Many protests start with college students and eventually spread to
    the public.
  \item Student visas are probably week compared to someone with a work visa
    to where a company can defend you and stop you from being deported.
  \item Students are also the next generation, so he can get rid of 
    people who will eventually become US citizens and become the next generation
    of Americans.
  \item Students are also broke and don't have money to get represented
  \item Also in a way, bleeding ethnic studies programs and create fear for those
    who want to be involved
\end{itemize}

\section{SEAA}

\subsection{SEAA Deportation Context}
\begin{itemize}
  \item Enter as a refugee, leave as 'criminals'
  \item IIRIRA
  \item Retraumatization
    \begin{itemize}
      \item It is being a refugee one again
      \item The kids may not have trauma from war, they
        are being put to a place that you don't know
        or don't even speak the language.
    \end{itemize}
\end{itemize}

\subsection{\#ReleaseMN8}
\begin{itemize}
  \item 8 Men detained by ICE in 2016
    \begin{itemize}
      \item They were not even born in Cambodia, they
        would be deported
      \item They didn't speak or comprehend the language
      \item They fought back to get out of this situation
    \end{itemize}
  \item Born in Refugee Camps
  \item Families organized around their detention
  \item Developed understandings of the detention and deportation process
  \item People who were raised in the US and then were deported
    after having family and kids in the US
\end{itemize}

\section{Crimmigration}
\begin{itemize}
  \item Intersection between criminal law and immigration law
  \item 'Criminal aliens'
  \item There are pety crimes that are grouped up here
    that are specially strict to immigrants
  \item Having detention camps and those that are equivalent to prisons
    for people who had crossed the border.
\end{itemize}

\section*{Discussion Week 7}
\begin{itemize}
  \item 
\end{itemize}

\end{document}
