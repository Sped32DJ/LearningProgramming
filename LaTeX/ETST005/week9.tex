\documentclass{article}

\usepackage{cancel}
\title{Week 9}
\author{Danny Topete\\ Ethnic Studies 005}
\date{\today}


\begin{document}
\maketitle

\section{Policing}
\begin{itemize}
  \item Polciing is not just about responding 911
  \item Over policing, hostpots, and broken window policing
  \item Policing is racist, classist, sexist, queer and trans antagonistic, and ableist
  \item Policing is a structure, not a Bad Apple
  \item Hostpots are areas where police are more likely to be present, due
    to their "likeliness of criminal activity"
    \begin{itemize}
      \item This is likely to lead to more arrests, death, and fines for no reason
      \item Broken Window policing is the idea where
        areas that have more "disorder" are more likely to have crime
      \item Typically just targeting areas with more poverty and people of color
    \end{itemize}
  \item Police trained to shoot first and ask questions later
  \item Police do not meaningly contribute to the public and rather
    just terrorize the local community
  \item Black people are more likely to be stopped, searched, and arrested.
  \item Policing justifiets itself by saying it is protecting the public.
    Along with having statistics of people who were arrested even
    if it wasn't really justified.
\end{itemize}

\subsection{Asian Americanist Critique of Policing}
\begin{itemize}
  \item Policing is not isolated to Black people
  \item Policing domestically is connected to imperialism and wars abroad
    and wars abroad
  \item Asian Americans offer a corollary to anti-Black racism
  \item Policing is originally to surveil and control the population, specifiaclly
    black people. It has gone off to every minority group.
  \item Policing serves to hold dominance
\end{itemize}

\subsection{Incarceration}
\begin{itemize}
  \item The confinements of people in cages
  \item Prisons vs Jails
    \begin{itemize}
      \item Prisons hold people who have
        not even been convicted of a crime yet.
        just people awaing trial or
        can't afford bail.
    \end{itemize}
  \item Detention facilities
  \item Public vs private prisons
  \item They typically profit off every bed that is filled,
    incentivised to keep people in prisons as much as possible
\end{itemize}

\subsection{Peculiar Institution}
\begin{itemize}
  \item Slavery
    \begin{itemize}
      \item People thought slavery would never go away...
        but it didn't...
      \item Abolitionist were the reason for the killing down slavery
      \item Lincoln was not who freed slaves, but Black People fought for their
        own freedom.
      \item He used this to attract black workers to the north.
    \end{itemize}
  \item Prisons
  \item Criminal Punishment Reforms
  \item Prisons punish but do not reduce harms
  \item Imagining a world before prisons seems unnatural. but they are not
    natural and were not always there.
    They are the typical form of punishment
  \item Reformers developed the form of Prisons to rehabilitate people.
    Rather than just serving death sentence or torture. So instead of death,
    the idea was to "heal" people.
    It only turns people into harder criminals.
  \item Prisons exist solely to take people out of their communities
\end{itemize}

\subsection{Who is incarcerated?}
Why should we care about people impacted by the criminal justice system?
\begin{itemize}
  \item Most vulnerable populations
    \begin{itemize}
      \item People facing poverty
      \item drug addicts
      \item People who need help, rather than to be put into prisons.
    \end{itemize}
  \item Poor white people
    \begin{itemize}
      \item Significant population of prisons
    \end{itemize}
  \item Migrants
  \item Groups based on race or religion
  \item Society has failed these people, and they
    end up in prisons
\end{itemize}

\subsection{Why are People Incarcerated?}
\begin{itemize}
  \item How is "Crime" defined?
  \item "Crime" vs Harm
  \item National security and fear
\end{itemize}

\subsection{Your Relationship to Police and Prisons}
\begin{itemize}
  \item I believe that Prisons are so quick to contain people. Even
    if they have not even done anything.
  \item There are instances where they could be holding the wrong person
    in containment and while they are being held,
    the prison is reeking monetary the benefits of having yet another bed filled.
  \item Prisons definitely shape the community because people come back from prison
 and it is almost as if they didn't serve time. It is a minor inconvenience to them.
 It is almost as if it is not effective at doing what it was meant to do.
\item I have seen in the community that people who go to a prison or jail
  it is not rare that they would be recurring.
\item They are definitely not the solution to taking care of problems, they just
  temporarily take away people from the community in order to reduce potential violence
  in the area. But that is just a bandaid to the problem of gang violence, if that
  is a concern in the community.
\item Having police in the area only increases the anxiety of others.
  People don't feel safe because there is a police in the community.
  Accompanies the saying "Driving while black". Since they \textbf{WILL}
  find any reason to give you a ticket. Even if that is tinted windows.
\item Getting pulled over by the police is never anything good. Something that
  will always cause fear onto anyone since it is never a good reason.
\end{itemize}

\subsection{The Prison Industrial Complex (PIC)}
\begin{itemize}
  \item A mass complex
  \item $\cancel{Mass Incarcerations}$
    \begin{itemize}
      \item Focusing our problem as incarceration
      \item But the problem really runs much deeper, such
        as needing more half way houses and perol. Along with mental health support.
    \end{itemize}
  \item Rise of the PIC
  \item Decline?
  \item Labeling Crack Cocaine vs Powder Cocaine. Since Black people used the cheaper
    crack cocaine and powder cocaine was more expensive and they would be
    regulated differently.
  \item There is a saying "if you build it, they will fill it", makes sense for
    prisons and lanes in highways.
  \item It started as a good idea and then slowly became as a way to punish
    people for their identity and their differences.
  \item Encourages to criminalize people for their identity
  \item Prisons are just a natural part of our world that it is hard
    to imagine a world without them.

    They are not a good solution to reduce crime, but rather produce more
\end{itemize}

\subsection{Why should we care?}
\begin{itemize}
  \item The PIC Exacerbates Social Problems
  \item Racism, capitalism, heteropatriarchy, etc.
  \item Segregate, destroy communities rather than address underlying problems
  \item Take away resource from life-affirming institutions
  \item After the three strikes law, after the third felony, you are sentenced to life in prison.
    This is a way to keep people in prison for longer and longer.
  \item Gender is being policed since gender affirming prisons don't exists,
    but if they did... they are still prisons
  \item Really expensive to run prisons and it is taking people away from their communities
    \begin{itemize}
  \item These are funds that could've been used to help the community
    and help people get back on their feet.
    \end{itemize}
\end{itemize}

\section*{Discussion Week 9}

What are some positives with the model minority myth?

\begin{itemize}
  \item Discussing model minority myth; it will popup in the final
  \item It is an optimistic approach to minorities even if they are
    coming to the US with nothing, not even a previous education due to
    a country in a bad socieconomic status.
  \item Increases the confidence of knowing what is possible to achieve
\end{itemize}


\noindent
\textbf{For Reading Journal \#3}
\begin{itemize}
  \item Don't miss internal and external context
  \item Minimum 400 words
  \item Do a paragraph of each, internal and external
  \item Needs citations, doesn't need to be perfect, just an attempt
    + the work cited
\end{itemize}

\noindent \textbf{Regarding Midterm}
\begin{itemize}
  \item Directions were a bit confusing at first,
    will be more clear next time
  \item Amount of examples required will be listed this time.
  \item Summary vs Analysis, you can't have a summary, rather the prof
    wants a critical analysis.
    Analysis is your own interpretation on the subject
  \item Final Due 6/9, Jenni has her office hours open for this.
    Open Monday to Monday
\end{itemize}

\noindent \textbf{Prisoner Abolishion}
\begin{itemize}
  \item We can abolish the police, but what is the other option?
    Who will "keep us safe", what are the alternatives?
  \item Transformative Justice
  \item Restorative Justice
\end{itemize}

\noindent \textbf{Prep for Final}
Structure:
\begin{itemize}
  \item Will be 4 questions instead of 3 this time
\end{itemize}

Part 1: Model Minority Myth
\begin{itemize}
  \item Emerged in 1960s (civil rights movement)
  \item Framed as am as quietly successful
  \item Proof that racism doesn't exist
    \begin{itemize}
      \item People are only as successful as they want to be
      \item There is totally no other reasons or structural
        racism
    \end{itemize}
  \item Black \& Brown com. are to blame for conditions
  \item Isolates Asians
    \begin{enumerate}
      \item Mental Health
        \begin{itemize}
          \item Immense preassure to perform perfectly
            in roles or work
          \item You are not allowed to talk about your
            struggles or issues
          \item Lots of unacknowledged suffering,
            since they are not allowed to talk about struggles
          \item Suicide rates are leading cause of death
            for young asian adults
          \item Stigma around Mental Health
        \end{itemize}
      \item Invisibility of struggle
        \begin{itemize}
          \item Ignores class, ethinc diversity, etc
          \item Asian Americans are not allowed
            to be poor or gang members
        \end{itemize}
      \item Division of community
        \begin{itemize}
          \item Racial wedge between other minorities
          \item Affirms racial hierarchy
          \item Black and Brown people can never relate to Asian Americans
          \item If you can't relate, then they are growing up just like
            white folks

            Creating a division in the community. Since people
            can not connect to each other.
        \end{itemize}
      \item Weaponized citizenship
        \begin{itemize}
          \item The model status is only conditional
          \item Despite how great they are,
            they will always be perpetual foreigners.
            They will always be told "go back to China"
            despite their actual country of origin or how many
            generations have lived in the US.
        \end{itemize}
    \end{enumerate}
\end{itemize}

Part 2: Immigration
\begin{itemize}
  \item Must content to at least one of these acts,
    know them. You may need to connect one from the past to
    a present one
  \item Page Act of 1875
  \item Chinese Exclusion Act 1882
  \item Wong Kim Ark vs US (1898)
    \begin{itemize}
      \item Birth right citizenship
    \end{itemize}

  \item IIRIRA
  \item Mention how all of these are racist and how these
    racist thoughts live to this day
\end{itemize}

\end{document}
