\documentclass{article}
\title{Reading Journal 1}
\author{Danny Topete}
\date{June 23, 2025}

\begin{document}
\maketitle

\noindent \textbf{Monday Required Readings}

\section{The Structure of Thinking in Technology}
by Henryk Skolimowski

\subsection{What is Technology?}
\begin{itemize}
  \item Technology in the early days was seen as demigodern, a bridge between the divine and the mundane.
  \item The role of machines in the human role is discussed

    Turning human being into a technological compoenent. (probably in the context of the Industrial Revolution; factory workers)
  \item A distinction is made between technology and science, with technology being more practical and applied.
  \item Philosophy of technology is introduced as a field that examines the implications and ethics of technology.
  \item Philisophy of technology and a technolofical philosophy have different focuses:
    \begin{itemize}
      \item Philosophy of technology: examines the implications and ethics of technology.
      \item Technological philosophy: focuses on the nature and essence of technology itself.
    \end{itemize}
\end{itemize}

\subsection{Moloch of Technology}
\begin{itemize}
  \item Some prophesy that our civilization will be devoured by the Moloch of technology,
    a metaphor for the destructive potential of unchecked technological advancement.
  \item These people that believe in the Moloch of technology view the world througgh a new kind of monism, technological monism,
    which sees all things as interconnected through technology.
  \item This perspective is a prophecy; but can not serve as a substitute
    for a philosophy of technology. It is but a branch of human learning
  \item "monolithic technical world" is but a graphhic and perhaps fearsome expepression but not reality.

    It is what people fear, but it is not what they know.
  \item Given the easy availability of technology, or even simple
    things like paper back books are due to advances in technology,
    serve the cause of highbrow culture, not technological culture.
  \item in spite of the semiautonomous development of technology,
    a substatial part of \textbf{modern tech is moved by nontechnological forces}.

    ex: \textbf{Politics, economics, and culture} play significant roles in shaping technology,
    such as the development of the \textbf{atomic bomb}.

    ex: The need of \textbf{transportation} has lead to development of the \textbf{automobile} and even
    the airplane.
  \item A lot of our technology is not driven by technological forces, but rather by
    \textbf{human needs and desires}.

    ex: The development of the \textbf{television} was driven by the need for entertainment and information.
  \item Tech being event driven

\end{itemize}

\newpage{}
\subsection{Arguments of this paper}
\begin{itemize}
  \item It is erronous to consider tech as being an applied science
  \item Tech is not a science
  \item difference between science and technology:
    \begin{itemize}
      \item Science is about understanding the world, while technology is about changing it.
      \item Science is theoretical, while technology is practical.
      \item Science seeks knowledge for its own sake, while technology seeks to solve problems.
    \end{itemize}
\end{itemize}

\subsection{Technology and Science}
\begin{itemize}
  \item No matter how sophisticated technology becomes, they are still crafts by humans
    (it will never be able to replace the human mind).
  \item The scientific part of technology can be decomposed into a particular science,
    such as physics or chemistry, but the technological part is not a science.
  \item Technological progress is key to understanding technology.
  \item You must comprehend tech progress in order to understand technology,
    and this is not a scientific process.
  \item Tech progresses based upon problems and situations, that is
    the perspective that reducing tech to an applied science is wrong.

    It can be seen as a pure science, but it misses the point of technology.
  \item There are some problems that are discovered during the advancements of technology.

    Many properties of physical objects are discovered through advancements in technology, that being a pure science.

    Ex: Computers resulted in replacing vacum tubes for transistors, which were discovered through technological advancements.
    Now we understand the properties and laws of semiconductors, which is a pure science.
    Now we know about metal fatigue, which is a pure science. Along wit other phenomena from
    scaling down the size of transistors such as quantum tunneling and heat dissipation.
\end{itemize}

\end{document}
