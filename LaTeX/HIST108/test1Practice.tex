\documentclass{article}
\title{HIST108 Test 1 Practice}
\author{Danny Topete}
\date{June 30, 2025}

\begin{document}
\maketitle

\section{Pulled from study guide}

\begin{itemize}
  \item To cover Power Points 1-3
  \item 25 MCQs, and two definitions / short answer questions
  \item Sample questions:
\end{itemize}

\subsection{Skolimovski}
\begin{itemize}
  \item Skalimovski's beliefs, about science and technology
  \item Technology is applied science, this is true, but Skalimovski
    believes that technology is not just applied science, but also a way of
    investigating nature and discovering natural laws

    Skolimowski believed that technology is not just applied science
  \item Technology absolutely cannot develop without science
  \item One of the aims of technology is to investigate nature and discover natural laws
  \item Unlike science, technology is not based on trial and error
  \item Basic methodological factors that account for the growth of technology is
    quite different from the factors that account for the growth of science
\end{itemize}

\newpage
\noindent \textbf{Employment of practical effects of one scientific field in those of another, such
as using the electriconic devices and approaches to detect
events in the nervous system is called:}
\begin{itemize}
  \item Substantive theory

    This is the theory of the science behind these electronic devices.
    The theory that has to go behind designing these devices.
  \item Operative theory

    This is a theory on teaching people how to use these electronic devices
  \item Cross-field application

    Using the application of one field of science into another.
    Such as using electrical engineering to detect events in the nervous system.
  \item None of these
\end{itemize}

\noindent \textbf{The development of trade and transporation in the second millenium
BC in Egypt and Mesopotamia indicates that:}
\begin{itemize}
  \item They made roads to conquer neighbor states

    No, quite the contrary, if they wanted to fight each other,
    they would prefer to not have roads as that would be
    a wasteful resources.
  \item They did not have the resources to enable them to develop
    their civilization on a basis of self-sufficiency

    Civilizaations that are self-sufficient do not need to trade.
    Meanwhile, these civilizations required to trade with each other
    so suffice each other's needs.
  \item Traveling to discover new lands was one of the characteristic features of
    the Mesopotamian and Egyptian civilizations

    Traveling was not a characteristic feature of these civilizations.
    They were not explorers or nomads, they were traders.
    They were quite sedentary and were not nomaidc.
  \item None of these
\end{itemize}

\section{Sample Definition Short Answer Questions}
\noindent \textbf{Give a list of technologies developed due to the domestivation of animals}
\begin{itemize}
  \item The loom
  \item The wheel
  \item Wheat and grain grounder using Donkey
\end{itemize}

\noindent \textbf{What is Praxeology}
\begin{itemize}
  \item Analyzes actions from the point
    of view of efficiency
  \item Establishes values, practical values, and asses our actions
    in terms of these values
\end{itemize}

\section{studying the Quizlet}
\begin{itemize}
  \item How do we construct and accomplish proress
    in technology.

    Using praxeological models, models if efficiency was practiced.
    If something got better at doing the job they were made to do.
    Like a car that gets made with better gas mileage or better safety.
  \item
\end{itemize}

\end{document}
