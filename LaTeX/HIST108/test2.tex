\documentclass{article}

\title{HIST108 Test 2}
\date{July 10, 2025}
\author{Danny Topete}


\begin{document}
\maketitle
\section*{Instructions}
\begin{itemize}
  \item Everything in PPT 4-8
  \item PPT 4: Early Sources of Power
  \item PPT 5: Science before Greeks
  \item PPT 6: On Pyramid Construction
  \item PPT 7: Ancient Asian Architecture
  \item PPT 8: Africa and Central America - Mayan Water Management
\end{itemize}

\section{PPT 4: Early Sources of Power}
\subsection{Prehistoric to Historic}
\begin{itemize}
  \item Paleolithic age is the earliest period of human history,
    within that age we discover fire, tools, and language
  \item Neolithic age begins when we begin agriculture
  \item Our evolution between Neolithic and ancient age began when we
    started writing down, then that goes from
    prehistoric to historic
  \item Ancient Age ages from 3000 BC to 500 AD, begining
    with the Fall of Western Roman Empire
  \item Medieval ages is between Fall of Western Roman Empire
    and the Fall of Constantinople and Columbus Discovering America, 500 AD to 1500 AD
  \item Modern Age is between Columbus Discovering America and the
    French Revolution, 1500 AD to 1789 AD
  \item Contemporary Age is from the French Revolution to today, 1789 AD to present
\end{itemize}

\subsection{How do we survey the past?}
\textbf{Survey:}
\begin{itemize}
  \item General conditions of period under discussion
  \item Dominant materials and sources of power of thea period
  \item Their application to food production
  \item Manufacturing industry
  \item Building construction
  \item Transport and communication
  \item Military technology
  \item Medical technology
  \item Sociocultural consequences of technological change
\end{itemize}

\textbf{Social Involvement in Tech Advances:}
\begin{itemize}
  \item The three requirements of social involvement in technological advances:
  \item Our tech advances based on social needs
  \item Our social resources
  \item Sympathetic social ethos,

    That being the spirit of the culture, era, or community
    as it is manifested in its beliefs and aspirations
\end{itemize}

\subsection{Babylonian Math and Astronomy}
\begin{itemize}
  \item First fruits of relationship between science and technology
\item \textbf{Improved abilities to answer}
\item Land, Weight, Time
\item Practical techniques, essential to any complex society
\item Inconceivable without literacy and beginnings of scientific observation
\item You require a complex and good langauge that can be written down
  in order to observe and record complex things.
  Along with being able to communicate complex ideas to others.
\end{itemize}

\section*{Metal Age}
\begin{itemize}
  \item Stone age was quite long, then came copper + tin + zinc, all soft metals
  \item Then came Iron, a stronger metal, which was used for tools and weapons
  \item then finally Diamond and gold minning which
  \item The discovery of alloying and really hot furnaces and kilns
    required for metalworking and pottery, and new artisans such as glassworkers
\end{itemize}

\textbf{Timeline}
\begin{itemize}
  \item Gold 6000 BC
  \item Copper 4200
  \item Silver, 4000 BC
  \item Lead 3500 BC
  \item Tin 1750 BC
  \item Iroin 1500 BC
  \item Mercury 750 BC
\end{itemize}

\subsection{Transportation}
\begin{itemize}
  \item Development of sailing ships in Egypt
  \item Sailing shihps would begin quite small and eventually evolve
    into large ships that could carry large amounts of cargo
  \item They would be used for trade, military, and exploration
\end{itemize}

\subsection{Irrigation Systems}
\begin{itemize}
  \item Need for a high degree of social control
  \item When there are larger populations, you need to
    control the water supply and distribution
  \item Need a way to distribute water to our needs (agriculture, drinking, etc)
  \item Canals and aqueducts were built to transport water across lengthy distances
  \item Along with technology such as water-raising devices such as shadoof
\end{itemize}

\section*{Persian Aquaducts}
\begin{itemize}
  \item Persia had an elaborate tunnel system called
    Quanats for extracting groundwater in the dry mountain.
\end{itemize}

\subsection{Urban Revolution 3000-500 BCE}
\begin{itemize}
  \item Invention of City
  \item Better agricultural skills >> growth in population
  \item Larger populations >> need for more products
  \item Need for more products >> more specialized craftsmen
\end{itemize}

\section*{Urban Manufacturing}
\begin{itemize}
  \item Urban manufacturing was the first time we had
    specialized craftsmen, such as potters, weavers, and metalworkers
  \item They would produce goods for the local market and for export
  \item This would lead to the development of trade networks
  \item Trade networks would lead to the development of money
  \item Manufacturing industry would begin with pottery, wine,
    oils, and cosmetics
  \item Trade with neighboring societies would lead to the
    development of new technologies and ideas. Beneficial
    relationships with other societies
  \item Military technology would advances tech and bring metal plates for armor
  \item Where there is surplus there is a requirement for storage
    and storage requires security from neighbors
  \item Development of press to product wine and oils
    more efficienctly (manufacturing)
\end{itemize}

\section*{Potter's Wheel}
\begin{itemize}
  \item Speciliazed tool between 6000 and 4000 BCE
    invented in Mesopotamia that revolutionized pottery production
  \item Ancient Egyptian mythology, deity Khnum said to have
    formed the first humans on a potter's wheel
  \item Allowed for more uniform and efficient production of pottery
\end{itemize}

\section*{Grind Stones \rightarrow{} Mills}
\begin{itemize}
  \item Ground stones from Neolithic era would grind up grains by swaying
  \item This would be done by hand
  \item This would then be replaced by the millstone
\end{itemize}

\subsection{Politics of Power}
\begin{itemize}
  \item All the loyalty would be at the top of the
    social hierarchy and then at the bottom would be the slaves
  \item The slaves would be the driving force of the economy
  \item Making slaves the most important part of the economy,
    making their labor the most valuable
  \item Machines, power, and the ancient economy were all
    based on the labor of slaves
  \item Population > need for more production
    > natural restrictions/availability of resources > problem of self
    sufficiency > Management
  \item Eventually when a process is not very self-sufficient,
    it requires management, and that management is the
    government, which is the state, which has lots of surplus
    of funds to find a more efficient way to do things (production of technology)
\end{itemize}

\section*{Fulfilling social needs}
\begin{itemize}
  \item Population growth \rightarrow{}
    food/shelter/clothing/jobs
  \item Need for more food \rightarrow{} agriculture,
    land, irrigation, transportation, energy, processing
  \item Natural restrictions \rightarrow{}
    land, water, fgertility, climate, energy, appropriate products, other natural
    resources, available technologies
  \item Problems of self-sufficiency \rightarrow{}
    administrating agriculture, irrigation, transportation,
    distribution, processing, storage, security, marketing,
    defense, trade
    need for more production, need for more energy, need for more
    transportation, need for more processing, need for more
    storage, need for more security
\end{itemize}

\subsection{Early Sources of Mechanical Power}
\begin{itemize}
  \item Mills were the holy grail of mechanical power
  \item Would evolve to human powered, then to more
    efficient methods, to animal, to water and wind powered
    (depending on the location; what is more abundant)

    Make sense since energy $does$ not come from nowhere,
    animals need to eat, and water and wind are "Free" sources of energy
\end{itemize}

\section*{POWER}
\begin{itemize}
  \item Until humankind learned to domesticate and harness
    animals were the sole source of power of its own muscles
  \item In difficult cases: using more man power or simple decvices
  \item If there are no tools, then more man power is thrown at the problem
  \item Rome had 200,000 slaves. Mostly prisoners of war,
    They would take care of public services >> maintaining the roads, mining,
    acqueducts, and construction, etc
  \item Management of gang systems (large scale use of gang-labor)
\end{itemize}

\subsection{Labor-Saving Machines}
\begin{itemize}
  \item Early form of labor-saving machiens were the
    lever, pulley, and wheel
  \item Archimedes: Give me a place to stand and I will move the Earth
    with a lever (given a large enough lever and a fulcrum)
  \item The lever understood the principle of having a small force acting through a long distance
    can move a large weight through a small distance
  \item There were more applcaitions to lever such as
    wedge, screw, compound pulley, wheel and axle, and inclined plane
  \item Multiplying the forces of the lever throughout multiple
    pulleys, you can lift a large weight with a small force.

    EX: Having gang of slaves pulling on a rope or rowing a boat
  \item Then using screws and levers in olive oil and wine presses
  \item The compound pulley was quite important since that birthed the spring.
    That would allow for crossbows, catapults, and other
  \item Catapults were strong, they used a
    spring, lever, pulley, gears, and winch
  \item simple bows were limited to strength of the archer,
    meanwhile the winch or leve would combine man power with tools
\end{itemize}

\subsection{Natural Energy Sources}
\begin{itemize}
  \item Water mils were widely used in medival Europe
  \item Other machines would be watermill or windmill
    depending on the region and what was more abundant
  \item There were 10-12k mills in England by 1300.
    mass adoption!
  \item Horizontal Windmills probably originated in Persia, 500-900 AD
  \item European Horizontal windmills were independent and around 1100s
\end{itemize}

\section*{Water vs Wind mills}
\begin{itemize}
  \item Watermills were always next to rivers, streams, or lakes
  \item Watermills were efficient and predictable
    since water is always flowing, and the flow of water
    is predictable, and can be controlled with dams
  \item Wind does not always blow. When it does, wind veolocity and direction
    was not very predictableis not constant
  \item Windmills had no efficient method to control the strength of the wind
  \item Not all regions were suitable watermills, no sufficient water sources
  \item Meanwhile other regions were too flat and their regions did not have enough flow,
    these regions would have the evolve to something else
    since they could not get comfortable with their current technology
  \item Some regions had rivers that froze during the winter
  \item Windmills appeared to spread fast for those regions that
    had issues with watermills (frozen water, rivers with no flow due to flat terrain, no bodies of water, etc)
\end{itemize}

\section*{Natural engines}
\begin{itemize}
  \item They were free
  \item producing more than animals and at a lower cost
  \item Their limitation was transmission of energy
  \item Energy should be very close to the mills (only produced mechanical energy;
    and a place to transfer that energy to where it is needed)
\end{itemize}

\subsection{Buildings}
\begin{itemize}
  \item Late stone age brought communities of Mesopotamia to build extensively in
    sun-dried brick
  \item Urban revolution kept the same building materials,
    but a larger scale of opeartions
  \item Building massive square temples, ziggurats, and palaces
  \item One of these massive structures made of rock can be seen in Egypt, the Great Pyramid of Giza
  \item In Egypt, clay was scarce, but building stones were abundant
  \item Stones were pulled on rollers and raised up to successive stages of structure
    by ramps and by balanced levers. (very complex building techniques)
  \item The building of Egypt were overseen by prist-architects who were mathematicians and astronomers,
    as evident by precise astronomical alignment of the pyramids
  \item Then the building of obelisks and massive staight stone pillars.
  \item One of these rocks would be 1167 tons = 17 M1 Abrams tanks
\end{itemize}

\subsection{Conclusion}
\begin{itemize}
  \item Degree of development in any society was set by the size of its surplus
  \item Surplus = Extent to which generation of goods and resources exceeds their consumption
  \item Size of surplus were determined by 4 factors:
    \begin{enumerate}
      \item Amount of energy available
      \item society's technology
      \item mix of economy (private and state monopolies)
      \item size of population
    \end{enumerate}
  \item \textbf{Urban Revoltion} 3000-500 BCE
    \begin{itemize}
      \item Energy Sources: Human/Animal muscles \rightarrow{} water and wind powered
      \item Tools/Machines: Lever, pulley, wheel \rightarrow{}
        compound pulley, spring, gears, winch
      \item 5 Mecanical devices related to lever:

        Wedge, Screw, compound pulley, wheel and axle, inclined plane
      \item Building Materials: Clay / Mud / Sun-dried brick (Mesopotamia) / Stone in Egypt;
        stone, wood, and metal
      \item Irrigation:
        Canals, water raising devices (shadoof), aqueducts, qanats
      \item Military: copper alloys
      \item Invention of City
      \item Better agricultural skills >> growth in population
      \item Larger populations >> need for more products
      \item Need for more products >> more specialized craftsmen
    \end{itemize}
\end{itemize}

\section{PPT 5: Science before Greeks}
A survey of Mathematical Sciences in Early Civilization - I
\begin{itemize}
  \item \textbf{By 3500 BC}
  \item The wheel was invented
  \item Connection between seasons and some astronomical phenomena
  \item Metal smelting, pottery, tool making, employment of simple machines,
    writing, basic mathematics were developed
  \item Theraputic herbs were discovered
\end{itemize}


\section*{Theoretical or science?}
\begin{itemize}
  \item Theoretical knowledge or science?
  \item Definitions of science and technology were still not clear
  \item Presentism: Employment of present-day ideas and perspectives
    in interpretation of the past (Whiggish history)
  \item It is one thing to know $how$ to do things,
    another to know $why$ they work

    (trial and error vs scientific method \rightarrow{} theoretical)

    You can know that the planet revolves around the sun, but another
    to understand the Newtonian mechanics behind it
\end{itemize}

\section*{Prehistoric being Oral}
\begin{itemize}
  \item Prehistoric culture was oral,
    there was an absense of writing, verbal communication was their only form of communication
  \item The only archive is the human memory
  \item Information and facts are aggregative rather than analytic
  \item Thoughts are closer to reference to human world (reduction to the familiar)
  \item Lack of conception of "Laws of Nature" in preliterate tradition
  \item Lack deterministic causal mechanisms to explain natural phenomena (never asked $why$)
  \item Projection of human and biological traits onto natural phenomena
  \item Personalization and individualization of causes to natural phenomena
\end{itemize}

\section*{Origin Story in Egypt}
\begin{itemize}
  \item Such example of conceptualization of natural phenomena
    is the Egyptian creation myth
  \item Air and Moisture (Shu and Tefnut) were the first
    gods to emerge from the primordial waters of Nun
  \item They gave birth to Geb (Earth) and Nut (Sky)
  \item Then turn Geb and Nut (earth and Sky) mated and produced Osiris, Isis, Seth, and Nephthys
  \item These represent creatures of this world, whether human, divine, or cosmic
\end{itemize}

\section*{Back to the point}
\begin{itemize}
  \item Oral tradition universe would consist of sky, earth, the underworld and deity, as
    well as invisible forces that ghostss of dead and
    spirits that can be controlled via magical rituals
  \item Medical practice nad healing arts withing oral cultures are inseparable from religion and magic
  \item 'wise woman' or 'medicine man' was valued both for their medical knowledge and their knowledge
    of the devine (they were old enough to hear all the previous generation stories and knowledge)
\end{itemize}

\subsection{Writing system}
\begin{itemize}
  \item Around 3000 BC, the first writing systems were developed
    in Mesopotamia and Egypt, which allowed for the recording of knowledge
  \item Syllabic System appeared 1500 BC
  \item  Fully alphabetic appeared 800 BC with phonecians
\item \textbf{Hierarchy of information:}
    Information >> archives >> analysis >> Criticism
  \item Invention of writinng was a pre-req for development of phoilosophy and science
  \item Good writing system allows for recording complex ideas (good for science and philosophy)
  \item Sophisticated concepts and abstractions can develop within an efficient langauge and writing system
  \item Earliest roots of Western Science are found in Mesopotamia:
    Development of mathematics, geometry, astronomy, medicine, number system, and calendar.
\end{itemize}

\subsection{Early Science and Mathematics}

\section*{Egyptian Math}
\begin{itemize}
  \item Greeks believed that math originated in Egypt and Mesopotamia
  \item Herodotus reports that Pythagoras traveled to Egypt and Babylon and learned math (570-496 BC)
  \item 3000 BC Egyptians had a number system that was decimal in character
  \item Later developed to writing seven |s, they used symbols that would change every 10$^n$, n times, (base 10)
    then get repeated for how ever many times
  \item Egyptian arithmetic was good for addition and substraction, but multiplication and division
    were done by repeated addition and substraction and difficult
  \item Egyptians geometrical knowledge was practical in surveying and building
  \item Calculation of circumference, surface area, and volume of geometric figures
    such as circles, triangles, and pyramids
  \item They had the concept of \pi{} as 3.1602
\end{itemize}

\section*{Egyptian Calendar}
\begin{itemize}
  \item Nut, goddes of sky supported Shu, god of air, who held up the sky
  \item They have a lunar calendar for religious festivals and rituals
  \item Then a solar calendar was for daily life and civil purposes (tax collection, agriculture, etc)
  \item Solar calendar was 360 + 5 days per year (intercalary month was 5 days long)
  \item 1 week = 10 days
  \item 3 weeks = 1 month
  \item four months = 1 season
  \item Three seasona and five holy days one year
\end{itemize}

\section*{Babylonian Math and Astronomy}
\begin{itemize}
  \item Superior to Egyptian math
  \item Number system fully developed by 2000 BC
  \item Decimal number system, base 10
  \item Sexagesimal number system, base 60
  \item These numbers would be found on stones
  \item Sexagesimal system was good for approximating square roots
  \item Had arithmetic operations to solve quadratic equations,
    could linear equations and solve for unknowns
  \item Development of Observational and Mathematical Astronomy
  \item Lunar and solar calendars
  \item Development of Horoscopic Astrology
  \item The mapped the constellations, and the times where you would see them (calendar)
  \item Bablonain tablet of 133/32 BC calculate wqhere in the Zodiac the moon will begin of consecutive months
\end{itemize}

\section*{Chinese Mathematics}
\begin{itemize}
  \item Decimal place value system
  \item Very much problem based, motivated by problems in calendar, trade, land maesurement, and government
    records and taxes
  \item 4th Century BC, counting boards were used for calculating, which
    effectively meant that the decimal system value number system was used.
  \item 500AD, Chinese Mathematicians calculated \pi{} to 3.141592(6/7)
  \item Chinese mathematicians influenced by translated Indian Mathematical texts
  \item 13 C, Chinese would solve equations up to a degree of 10
  \item Birth of Chinese remainder theorem in 100AD
  \item Chinese remainder theorem states that if you have a system of linear congruences,
    then there is a unique solution modulo the product of the moduli
  \item 6th century AD, until the Telescope in Europe (early 17th century),
    Chinese astronomers were the most advanced in the world on records
    on eclipses, comets, novae, sunspots, and aurora borealis
\end{itemize}

\section*{Indian Mathematics}
\begin{itemize}
  \item Developed independently from Chinese and Babylonian math
  \item Mantras from early Vedic period (before 1000 BCE) invoke powers of ten
    from hundred all teh way up to trillion
  \item They conceptualized the number zero, and the decimal place value system
  \item 400 CE had roots of trigonometry, including real uses of
    sine, cosine, tangent, secant, and inverse sine functions
  \item 12th c. AD, Indian astronomers used trigonometry to calc relative distance of planets.
  \item Along with laws of trigonometry, such as sine law and cosine law
  \item Indian numerals arrived at Baghdad in 770 AD, and then
    were translated into Arabic, and then spread to Europe
  \item Leonardo Fibonacci (1170-1250) introduced the Hindu-Arabic numeral system to Europe.
    Studied Arabic numeral system in Europe with book Liber Abaci (1202)
\end{itemize}

\section*{Mayan Mathematics}
\begin{itemize}
  \item Developed independently from other civilizations
  \item Used vigesimal number system of base 20
  \item Concept of zero by at least 36 BCE,
    by pre-classical Maya and their neighbors
\end{itemize}

\section*{Mayan Calendar}
\begin{itemize}
  \item Mayan Calendar was more accurate than Ptolemaic calendar at the time
  \item Had a more accurate synodic period of Mars, Venus, Solar year, lunar month, etc
\end{itemize}

\section{PPT 6: On Pyramid Construction}

\subsection{Monuments to Power}
\begin{itemize}
  \item Ancient civilizations built huge monuments
  \item They would demonstrate religious/ideological beliefs and
    perform rituals
  \item Reassert the power of the society to its ruler
  \item Protect the society and its resources (??)
  \item Megaprotects were only possible through
    collaboration of majority of people at the beginning of a society or an era
  \item Collaboration of the majority of people to florify a deity
    or to commemorate an important person or event
  \item Collaboration of the majority to protect the society
  \item often, this majority would be slaeves, the bottom of social hierarchy
  \item \textbf{Technical Issues:}
    Probably cutting was harder than carrying the stones
  \item Needs to construct long lasting buildings
  \item Sometimes big stones just for cladding
  \item Smaller building constructions have been ruined by now
\end{itemize}

\subsection{Ancient Egypt}
\begin{itemize}
  \item It survived from 2650 BC to 30 BC
  \item Goes from Old Kingdom (Pyramid Age)
  \item To Ptolemaic Period (Greek Rule)
  \item Egypt went through many different periods where
    near the end, the rulers would be foreigners
  \item Most pyramids were build during Old and Middle Kingdom eras,
    very early in the life of the civilization
\end{itemize}

\subsection{Pyramids - Great Pyramid of Giza}
\begin{itemize}
  \item Base and diagnol shafts are astronomically aligned
  \item Total mass is 5.9 million tons
  \item Precession: Every 26,000 years, the Earth's axis
    of rotation shifts by 23.5 degree, which means that the stars
    will appear to shift in the sky. Shifting from Polaris to Vega as the North Star
  \item For 3800 years ago, they were the largest structure in the world
  \item Tombs aligned north-south to an accuracy of 0.05 degrees
  \item Casing stones were quite important since about 144,000 casing stones
    were used to cover the pyramid, and they were cut to fit perfectly.
    They were highly polished and flat to 0.5mm accuracy. Each weightin 15 tons with nearly
    perfect right angles for all sides
  \item Pyramid made of 2.3 billion stones, weighing 2 to 30 tons each. With some at 70 tons
\end{itemize}

\subsection{General Pyramid Building}
\begin{itemize}
  \item Not unique to Egypt
  \item Also appeared as Ziggurats in Mesopotamia and
    Mexan Pyramids in Mesoamerica
  \item What makes Ehyptian pyramids unique is the
    orientation,
    size, precision, and magnitude of construction project
  \item Dimension used in stones is based on Golden Ration: Length divided by width
    is 1.6180339887, which is the Golden Ratio
  \item The pyramid is so precise that they meet the same point in the sky:
    making them true pyramid
  \item Pyramids were built along a lost
    stream of the Nile
\end{itemize}

\subsection{Main Questions}
\begin{itemize}
  \item Methods of Construction
  \item Methods of delivering stones to higher courses,
    (ramp theory)
  \item Methods of orientation correction
  \item Methods of placing the casing stones
  \item Logistics of such a large project
  \item Quarrying and transporting the stones
  \item Ancient engieers had two main problems :
    \begin{itemize}
      \item Raise the stones to great heights
      \item Place stone in position with accuracy
      \item Solution to one of these problems
        would cancel out solution to other
    \end{itemize}
\end{itemize}

\subsection{Theories}
\begin{itemize}
  \item Perpendicular ramps, there was a ramp that would
    go up to the pyramid, and then a ramp that would go around the pyramid
    to get to the next level
  \item External Ramp and Crane theories,
    ramp was built on one side,
    as pyramid grew, ramp was raised,
    then blocks could be moved right up to the top
\end{itemize}

\section*{Exterior Ramp Theory}
\begin{itemize}
  \item To build perpendicular ramps, huge amounts of
    earth would be required, and the ramps would be 5-10 degrees,
    requiring really big ramps
  \item There would require there to be 4 ramps, one for each side of the pyramid
  \item Wood? Was good for small works, but given teh great weights, wood was not sufficient
  \item Burnt Brick? Not employed by Egypt until Roman times
  \item Sand? lol
  \item Sun-dried mud brick? What about rain? And after 380 feet, brick would crush itself
  \item Stone masonry? Very complex, and would require
    a lot of manpower to build the ramps, and then
    the ramps would be destroyed after the pyramid was built.
    This was built in 12 years, so it was quite efficient what they used
  \item Building with stone is time consuming, and
    requires a lot of manpower, and the ramps would be destroyed after the pyramid was built.
    A ramp of 12 degrees would mean a ramp of one mile long
\end{itemize}

\section*{Interior Spiral Ramp Theory}
\begin{itemize}
  \item Dificult to manuver large blocks of stone around corners
  \item Measuring and controlling shape of pyramid also difficult since ramp covers most of pyramid
  \item Not possible
  \item Second theory centers on Herodotus Machines, Egyptian farmers used wooden cranelike devices
    called shaduf to raise water from the Nile for irrigation
  \item Problems: Involve tremendous amount of wood that Egypt didn't have
  \item Nowhere to place cranes
  \item Aliens? Nah, that would just mean more labor for them, and highly unlikely
\end{itemize}

\subsection{Frank Muller-Romer}
\begin{itemize}
  \item Construction of pyramids took place in several phases
  \item We can see step pyramid and Sneferu's Bent Pyramid,
    as earlier and failed attempts to build the pyramids
  \item Earlier attempts show off the evolution of pyramid construction
  \item Goes to show that the Great Pyramid of Giza started off as a step pyramid
  \item Given dates are accurate, the Pyramids of Giza were built seperately
  \item The pyramid must have been built in phases
    \begin{enumerate}
      \item Laying out the base
      \item Control form of pyramid
      \item Movement of stones
    \end{enumerate}
  \item Meidum was built in three successive phases (step pyramid),
    there is a visible part, which is the tower, and
    most of it is buried under the sand
  \item Meidum, was seen as teh formula for all future pyramids.
    The core was constructed, but future pyramids would be "smooth"
  \item Meidum would be built in multiple levels, then "smooth" pyramids
    would have been smoothed out
  \item Building a pyramid would require stepped core, ramps would be removed and casing, then
    auxilary platform and tangential ramps are used
  \item Inner core, very rough, with many internal ramps within the levels
  \item Casing and placing pyramidion
  \item Smoothing of casing from top to bottom
  \item Egyptian construction used tehcnique to pull rahter
    than lift these large stones
  \item The seven courses of the step would be visible in the inner core
    thus proving Frank Muller-Romer's theory
  \item Then some sort of sewing cranes to raise stones
\end{itemize}

\subsection{Internal Ramp Theory}
Presented by Jean-Pierre Houdin
\begin{enumerate}
  \item Ramp used to raise blocks to the top
  \item Ramp was internal and still exists inside pyramid
  \item Bottom third of pyramid was built using external ramp
  \item As bottom of pyramid was built, second ramp was built inside pyramid, and another
    top 2/3rds of pyramid would be hauled
  \item There is density imaging that supports this theory
\end{enumerate}

\subsection{More Egypt}
\begin{itemize}
  \item 31 Egyptian pyramids were built on a 64km
    long-lost branch of the Nile, supporting why they were built there
  \item Genome of Old Kingdom Egyptians shows a link
    with Mesopotamia
  \item Obelisks were not very documented, but the one that
    went to the Vatian was. The whole journey was documented
\end{itemize}

\section{PPT 7: Ancient Asian Architecture}
\subsection{Part 1: China}
\begin{itemize}
  \item Ancient Asian Architecture was unique in their
    building types, statues, gardens, shrines, and temples
  \item 4 great inventions: compass, gunpowder, papermaing, and printing
  \item Chinese astronomy, They were the earliest to record a supernova in 1400 BC
\end{itemize}

\section*{Astronomy}
\begin{itemize}
  \item Lunar and Solar calendars
  \item Astrology
  \item Chinese constellations
  \item Prediction of astronomical phenomena
  \item Recording comets
  \item Chinese astronomers recorded every instance of Halley's comet from 3000 years ago,
    the only civilization in the world to do so
  \item Astronomical orientation of ancient chinese buildings
  \item Taosi Archelogical Site, 2300-1900 BC, oldest in East Asia

    This would be an astronomical observatory
\end{itemize}

\section*{Shang Dynasty 1600-1050 BC}
\begin{itemize}
  \item Bronze age for tools, weapons, and religious objects
  \item Development of writing
  \item Construction of religious sites
  \item Kings were considered to have divine nature
  \item Docorated underground tombs
  \item Developed ceramic
  \item Structures had symmetry
\end{itemize}

\section*{Important features of Chiense Architecture}
\begin{itemize}
  \item Emphasize articulation and bilatery symmetry
  \item Enclosure
  \item Hierarchy of space
  \item Horizontal Emphasis
  \item Cosmological concepts
  \item Gardens
\end{itemize}

\section*{Chinese Gardens}
\begin{itemize}
  \item complex, decorated rocks and rock garden
  \item Beautiful plants, trees, flowers, and water element always present
  \item Most Chinese gardens are enclosed by wall and have winding paths
  \item You can't see the entire garden all at once. Built up my small scenes
  \item Rocks are important since they were concentreated amount of natural energy
\end{itemize}

\section*{Early Magnetism}
\begin{itemize}
  \item Qin Dynasty 221-206 BC
  \item Magnetism in loadstone
  \item Used to make simple compass
  \item Used for geomancy and fortune telling by the Chinese
  \item Compass later used for nivation in Song Dynasty 960-1279 AD
  \item Predates European mentio of compass by 150 years
\end{itemize}

\section*{Great Wall of China}
\begin{itemize}
  \item 208BC - 1630 AD
  \item Series of fortifications made of stone, brick, tamped earth, wood, and other materials
  \item Longest structure in the world, nearly 4k miles long, Mexico-US border is only 1,954 miles long
  \item Construction began Quin Dynasty (208 BC)
  \item 5 seperate walls constructed in different phases of construction
  \item 5th phase is the most popular tourist attraction, the Ming Dynasty wall
\end{itemize}

\section*{Age of Asian Technology AD 700-1100}
\begin{itemize}
  \item History of Science and Technology in non-western civilization is relatively a new
    field of study
  \item (Not many documented the science and techonlogy of China until very recently)
  \item History of science and technology in China:
    Joseph Needham, 1900-1995
  \item Interactions between population and technology
  \item Increases in poulation stur for technological innovation,
    especially when more food and necessities are produced from a
    fixed area or land
  \item Survival Technology >> New cropping pattersn in China, West Area, and Europe from AD 700 >>
    techn change
\end{itemize}


\section{PPT 8: Africa and Central America - Mayan Water Management}
\end{document}

