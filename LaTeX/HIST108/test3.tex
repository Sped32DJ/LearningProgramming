\documentclass{article}
\title{Study for TEST 3}
\author{Danny Topete}
\date{July 20, 2025}

\begin{document}
  \maketitle

  \section*{Study Guide}
  \begin{itemize}
    \item Exam 3: Tuesday, July 22nd
    \item Everything from PPT 9 to 14
  \end{itemize}

\section{PPT 9 - Alexandria and After}

\subsection{Ancient Greek Civilizations}
\begin{itemize}
  \item Ancient Greek civiliztions had two phases:
    \begin{enumerate}
      \item Hellenic:
        Independent city-states in Ionia and on the
        Greek peninsula, no Greek king.
      \item Hellenistic:
        4th century BCE, marked successively
        by confederation, imperialism, and conquest.

        Alexander the Great conquers Persia, Egypt,
        and parts of India. Greek culture spreads.
    \end{enumerate}
  \item The result was a vast expansion of Greek culture and learning
  \item Hellenic world was Greece and the Balkans
  \item Hellenistic World included Egypt, Mesopotamia, Persia, and parts of India
    (Alexander the Great's Expansion)
\end{itemize}

\subsection{Greek and Hellenic Science}
\begin{itemize}
  \item First Ionian / Greek philosophers

    Thales/Anaximander/Anaximenes/Pythagoras/Parmenides/Heraclitus/Empedocles/Anaxagoras/Democritus/Leucippus..
  \item Speculations on nature, investigations of fundamental princiles of nature of what exists,

    Then "the essence of things."
  \item Greek science took influence on Egyptian and Babylonian knowledge,
    Greeks were first to look for general principles beyond observations
  \item These philosophers were pre-Socrates, since he was a milestone philosopher
  \item Thales: Water is the source of all things
  \item Anaximander: Apeiron is the source of all things (apeiron = infinite))
  \item Anaximenes: aer is the source of all things
  \item Pythagoras: numbers are the source of all things
  \item Heracleitus: What is change?
  \item Parmenides: Earliest surviving example of deductive reasoning
  \item Emppedocles: First pholosopher to suggest that there are 4 fundamental elements
    (earth, air, fire, water)
  \item Democritus: Tiny uncuttable particles (atomos) make up everything,
    which are uniform and move through empty space (void)
  \item Epicurus: Developed Democritus' ideas into a full system which survived into
    early modern times
\end{itemize}

\section*{Aristotle (384-322 BCE)}
\begin{itemize}
  \item Established a comprehensive system of philosophy and science
  \item Coherent theory of cosmos and categorized objects of universe are arranged in
    distinct configurations
  \item Wrote systematically about physics, metaphysics, biology, zoology, logic,
    ethics, politics, rhetoric, poetics, and even theater (Everything)
  \item Aristotle's Physics:
    \begin{itemize}
      \item Four causes: material, formal, efficient, final
      \item Natural place of elements (earth, water, air, fire)
      \item Natural motion vs violent motion
      \item Celestial spheres and uniform circular motion
      \item Geocentric universe
    \end{itemize}
  \item Shaped medieval scholarship and influences extended into 16th century
\item \textbf{Based on our experiences}:
  \begin{itemize}
    \item Heavenly bodies revolve around Earth
    \item Heavens don't fall or recede us
    \item Earth is at the center of the universe,
      and heavens move around the center
    \item Earth, motions are either downward or upward:
      stone falls down, but smoke rises
    \item Earth rotational motion of body is not natural;
      it is forced motion
    \item heavens motion is circular and uniform
    \item Terrestial regions are the place of change,
      generation, and corruption
    \item Basic elements in terrestrial region are earth, water, air, fire;
      they are arranged based on their property of gravity (or levity)
    \item Arrangement shows their natural location or places
    \item Since nature and motion of heavens is different,
      they are made out a 5th element, aether
    \item Earth has two regions, celestial and terrestrial
    \item Two different substances
    \item two different concepts of motion, concepts of change,
      physical constitutions, physical behaviors, etc
    \item The uniforminty of everything existing as two
  \end{itemize}
\end{itemize}

\section*{Epistemology / Logic}
\begin{itemize}
  \item Syllogism: a deductive scheme of a formal argument consisting of a major and a manor premise
    and a conclusion
  \item Celestial motnios are uniform and circular (major premise; already known)
  \item Comets are not moving uniformly (minor premise; observation))
  \item Comets are not celestial bodies (conclusion)
  \item This philosophy scheme was used to understand the most
    fundamental ways things $were$ and $why$ they behaved the way they did
\end{itemize}

\section*{Alexander}
\begin{itemize}
  \item Forged most extensive empire in ancient world,
    connected Greece, Egypt, Persia, and parts of India
  \item His death led to a split of multiple nations, India returned to its previous state,
    then 3 kingdoms emerged: Macedonia, Egypt, and Seleucid Empire
  \item Roots of scientific culture were planted in Egypt,
    governed by Ptolemaic Egypt
  \item Ptolemaic Kingdom started with Ptolemy I (323 BCE),
    and ended with Cleopatra VII and Roman Conquest in (30 BCE)
\end{itemize}

\section*{Thinkers of Alexandrian Era}
\begin{itemize}
  \item Archimedes: Archimedes Principle, lever, pulley, buoyancy
  \item Appollonius: Cone sections, ellipses, parabolas, hyperbolas
  \item Euclid: Book Elements in math, systematic organization of geometry
  \item Aratosthenes: Measured Earth's circumference and tilt of eclitic.
  \item Ptolemy: The $Almagest$, geocentric model of universe, math bible of astronomy
    until 17th century
  \item Galen: Foundation of medicine until 17th century
  \item Tech developments:
    \begin{itemize}
      \item Water clocks
      \item Planetaria
      \item Catapults
      \item Archimedes screw
      \item Odometer
      \item cogged gears
      \item screw press
      \item Glassblowing
      \item hollow bronze casting
      \item Surveying instruments
      \item piston pumps
    \end{itemize}
  \item Antikythera mechanism, Hellenistic analog comuter designed
    to calculate astronomical positions
\end{itemize}

\section*{Galen \& Human Physiology}
\begin{itemize}
  \item There are four elements to human physiology:
    \begin{itemize}
      \item Earth: cold and dry
      \item Water: cold and wet
      \item Air: hot and wet
      \item Fire: hot and dry
    \end{itemize}
  \item Each four elements correspond to four humors:
    \begin{itemize}
      \item Earth: black bile
      \item Water: phlegm
      \item Air: blood
      \item Fire: yellow bile
    \end{itemize}
  \item Brain/nevers distribute "Animal spiprits"
  \item Heart/arteries distributes "Vital spirits"
  \item Liver/veins distribute "Natural spirits"
\end{itemize}

\section*{Medieval Astronomy / Celestial Anomalies}
\begin{itemize}
  \item Length of seasons were not equal?
  \item What causes retrograde motion of planets?
  \item Eudoxos of Knidos and his student proposed a concentric model,
    where revolution of spheres on different axes in different directions
    with different speeds produced the observed motion of planets
  \item Aristarchus of Samos,
    proposed a heliocentric model of the universe, where sun is at the center

    Aristarchus also estimated size of sun and moon, as well as their
    distances from the Earth in Earth radii

    Heliocentric model was rejected, not for anti-intellectual bias, but rather
    essential implausibility of the model
  \item The problem with Aristarchus system was stellar parallax,
    (Turns out its ecliptic tilt)
  \item Size of the the universe in Ptolemaic system: 20,000 Earth radii
  \item Epicyle/Defenrent Model,
    2nd century AD, Claudius Ptolemy developed planetary model in which planets
    were revolving uniformly around a small orb (epicycle), whose center
    was moving uniformly on a bigger circle, called deferent.

    (Planets reolved around Earth of a path, but they followed that path
    following a smaller circle, which caused retrograde motion)
  \item Ptolemy postulates that center of Epicycle (smaller sub-orbit) moves
    on the Defenrent (larger orbic)

    Center of deferent does not coincide with the Earth, but is slightly offset (foreshading eliptical)

    Planet describes equal angles around another center, hypthetical point called Equant
    (Adding more complexity to this model to explpain discrepancies)
    The equant, being the point where it appears that the planet moves uniformly (not Earth),
    perfectly explains the eliptical nature of orbits
\end{itemize}

\section*{Hellenistic Science:}
\begin{itemize}
  \item Represent historical melding or hybridization of tradition
    of Hellenic natural philosophy with pattersn of state-supported sciences
  \item Kings and emperors patronized bireaucratic science that leaned to useful appplications
  \item Hellenistic science combined with ancient Near East combined those disparate traditions
  \item Ptolemaios Soter, began traditional royal patronage of science
  \item His successor, Prolemaios Philadelphus, estabnlished
    Museum at Alexandria, which existed for 700 years into
    5th century CE
  \item Museum was a research institute, library, and
    place of worship for the Muses
  \item Hellenistic Civilizations (300BC - 200 AD)
    represent fusion of Ancient Greek world with Near East, Middle East and Southwest Asia
    \begin{enumerate}
      \item Massive inter-penetration of Greek and non-greek ideas
      \item Increasing specialization of sciences
      \item Developp new research centers
      \item Increase kingly patronage of sciences
    \end{enumerate}
  \item Algament and Euclid's Elements were the math bibles of the era
\end{itemize}

\subsection{Roman Empire (27 BCE - 476 CE)}
\begin{itemize}
  \item Takes over Britain, Gaul, Spain, North Africa, Greece, Asia Minor,
    Egypt, and parts of the Middle East
  \item Compared to Hellenic and Hellenistic civilizations, no major
    developments in science (except medicine) and philosophy
  \item Romans were greatest technologist and engineers of ancient world
  \item Contributed to art, architecture, urban developpment, warfare and medicine
  \item Pont du Gard, in France, aqueduct, 19BC, still stands

    Had indoor lead plumbing

    Had baths and Pompeii Public Fountain
  \item Ancient roads that still exist today
  \item A grand Roman Navy that conquests the Mediterranean

    Roman warships were 35 meters long around 5 meters wide.
    Developed to be lightweight and provide max maneuverability in battle
  \item Roman concrete in late 3rd century BC

    Volcanic dust called Pozzoloana, mortar made a mixture of lime and volcanic ash,
    bricks or rock pieces and water made into a concrete.

    This concrete was very durable and could set underwater,
    it would harden over time and buildings still stand today
  \item Colloseum 80 AD, could hold 50,000 spectators, still stands today

    Along with Amphitheaters in Pompeii and Verona, Pantheon (126 AD)
  \item Roman science is Greek science (no grand advancements)
  \item Roman great ability for architecture, engineering, law, urban management, military skill,
    and ruling a complex, multiracial, multicultural empire, science is
    adapted from Greek natural philosophers
\end{itemize}

\subsection{Decline and Fall: Conclusion}
\begin{itemize}
  \item Lack of clear social role of science
  \item No empoloyment for scientists or philosophers
  \item Seperation of science and technology
  \item Role of Religion was great (churches and state)
  \item Wars and decline of Roman Empire
  \item West loss of ancient science occured in two stages:
    \begin{enumerate}
      \item Decline of quality and equantity of scientific activity

        Commentaries, encylopedias, and general texts
      \item Disappearance of traditional learning,

        Europeans deprived of ancient learned traidtions and documents
    \end{enumerate}
  \item More Reasons for decline:
    \begin{enumerate}
      \item Invasion from Barbarians
      \item Economic troubles, overreliance of slave labor
      \item Rise of Eastern Empire
      \item Overexpansion and military overspending
      \item Government corruption and political instability
      \item Arrival of Huns and migration of Barbarian tribes
      \item Spread of Christianity and change of social/educational setting
      \item Decline of urban life
      \item Decline of long-distance trade
      \item Decline of literacy
      \item Decline of education
      \item Decline of patronage
      \item Decline of political stability
    \end{enumerate}
\end{itemize}

\section{PPT 10: Catapults \& Greek Fire\\
Ancient and Medival War Technology}

\begin{itemize}
  \item Technological progress
  \item Tech based upon needs and resources
  \item Bottlenecks and endpoints that limited progresws
  \item Grand innovations that promoted success
  \item Along with revolutions of technology
\end{itemize}

\subsection{Astronomy}
\begin{itemize}
  \item Invention of the telescope in 1610,
    there was a single lense therefore chromatic aberration
  \item This would keep evolving and eventually make a Newtonian Telescope in the late 17th century
    with high quality lenses and mirrors and mechanisms
\end{itemize}

\subsection{War Technology}
\begin{itemize}
  \item Espionage cameras, shifting cameras to become even smaller,

    From using the same film technology and making it smaller to other innovations
    along with compensating for the small size of the camera.
  \item War technology brought personal issues to state/national level problems
    of importance with greater budgets and resources
  \item Survival technology that would be directed connected to state budgets and resources
  \item Connected to nationa interest and would compete against other nations
  \item Catapults were reliable information of mechanical characteristics,
    and were used to hurl stones, fire, and other projectiles
  \item Marcus Vitruvius Pollio, Roamn writer, architect, and engineer,
    wrote book on the basis of Roman architecture and engineering
  \item Philo of Byzantium, author of Mechanike Syntaxis,
    it had eight sections on his book of artillery
  \item Catapults were very successful and widely adopted.
  \item Played a great role in Greek scinece to play the developments of catapults
  \item The consequence of highly developed catapult technology
    let entire cities to be destroyed and warfare to be more destructive
  \item Advancements lead to the bow and arrow, crossbow, and ballista
  \item Along with advancements of using Mount of Energy, of how much energy can be stored in bows
    being determined by string length, thickness, and material, then ultimately
    human strength (100lbs being the near limit)
  \item bowman's arm would be the bottleneck/end point. Technology would not advance
    beyond this point, and would be limited by human strength
  \item Evolutions would permit to use more energy, to
    go from arm muscles and back muscles, since they were stronger.
\end{itemize}

\subsection{Level based devices}
\begin{itemize}
  \item The greatest labor saving technology was the lever
  \item Brought the wedge, screw, compound pulley, wheel and axle, and inclined plane
  \item Larger and more powerful bow-catapults would be made, that
    were stand-mounted to do more damage upon the same principle
  \item elasticity of the bow was used to store energy,
    the limitation of bow-catapults
\end{itemize}

\section*{Catapults and their shortcomings}
\begin{itemize}
  \item Bow-Catauplts were suicceeded during course of fourth century by Torsion catapults
  \item Catapult physics, physics of stored energy and hurling a projectile without
    the use of an explosive. Three primary energy storage mechanisms:
    Tension, torsion, and gravity
  \item To counter catapults, there were advancements in armor and fortifications (making buildings last longer)
  \item Aeneas Tacticus, devoted only nine short chapters of his Poliorketika
    to assault techniques. Bulk of work was a manual for commanders to besiege cities, dealt with
    methods to prevent a city's betrayal. Mentioned missile engines only once.
  \item Torsion catapults had important shortcomings. They were slow and cumbersome,
    disadvantages grew as catapult grew more powerful.
  \item Problems came from twisted skeins were tensions of skeins varied with changes
    of atmospheric humidity, temperature, and other factors. (end of the road for catapults)
\end{itemize}

\section*{Counterweight Trebuchet}
\begin{itemize}
  \item Revolutionary invention, came when catapult were at their peak
    and stopped developing innovations
  \item Use of gravity and angular momentum to hurl projectiles
  \item More powerful than earlier missile-thrower and far simpler design
\end{itemize}

\subsection{Technology and Secrecy}

\section*{Greek Fire}
\begin{itemize}
  \item Constantinople, 678AD, under attack by Arab Naval power.
    This was their secret weapon, it would even burn on water
  \item It was kept very secret since they wouldn't want the enemy with this technology
  \item Kallinikos, A greek architect or egnineer brought new and decisive invention to the Byzantines
  \item Bosphorus had mightly fortifications
  \item Greek fire was a napalm-like substance that burned in water and
    projected great distances from bow of ships
  \item Relying on his weapons, the Byzantine defeated enemies. Three Arab invasions defeneded the Byzantine Empire
  \item Henry Pirenne (Belgian historian) argued Arab at Constantinople
    blocked Muslim expansion and left European civilization to find an independent course into the modern world

    (Blocking arab expansion into Europe allowed to Europe to develop independently)
  \item Greek Fire had great characteristics:
    \begin{itemize}
      \item Burned on water
      \item portrayed as liquid
      \item Always shot out of tubes or siphons
        located in bows
      \item Appearance of smoke and loud discharge or booming noise when flaming liquid
        left the tube or syphon
      \item Could be projected from a tube or siphon
      \item Could be used in naval warfare
      \item Could be used in land warfare
      \item Could be used to set fire to ships, buildings, and fortifications
    \end{itemize}
  \item The most unique characteristics was the tubes
  \item The real secret behind the weapon was preheating and preassurifzing the liquid below the
    decks before discharging it from siphons
  \item Greek Fire was a weapon system not a single forumal. An array of knowledge
    associated with several components of the system.
  \item Reconstructed, came from preheated oil, and a pump to preassurize. Then
    the syphons would unleash the valve and ignite it with a match or lamp
  \item The story of Greek fire is as an ultamite weapon that was a state secret
    that was kept for 500 years until it was lost by the keepers
\end{itemize}

\section{Science vs Technology secrecy}
\begin{itemize}
  \item Science is papyrophillic, it loves paper.
    The only mechanism for scientific advancement is the written word,
    for advancement of knowledge and sharing of ideas
  \item Meanwhile technology is papyrophobic, it does not like paper.

    Technology is a secretive process, it is a process of trial and error,
    and the only way to advance is to keep the secrets of the technology
    to yourself, and not share it with others.
  \item Craft knowledge is also secret for many industries and traditional
    means of penetrating secrets of technology
  \item Industrial espionage, apprenticeship, migration of craftsmen, and
    the use of secret codes and ciphers were used to keep secrets
  \item In the modern world, craft secrets came embedded in patents
  \item Secrecy is more important for military technology
  \item Leonardo da Vinci, protected all his inventions by his unique reverse writing.
  \item Science and technology of atomic bomb was kept secret by
    compartmentalization. Only a few people knew the whole picture,
  \item Personnel constructed and operated various equipment, (such as centrifuges to isolate uranium-235),
    they did not know what was the full picture of the project.

    Those who were doing it, did not know why they were doing it.
  \item Greek fire provides us rare and illuminating window into
    several features of tecnology in general and military technology
  \item Byzantines managed to keep their secret
  \item Coca-Cola as their secret formula, is a modern example of a secret
    that has been kept for over 100 years, all with various key holders,
    and they never travel together, and the formula is never written down

    This recipe is responsible for 1.9 billion products every day in 200 countries.
  \item Guarding secret requires appreciation of components of technology
  \item Technology may be defined as purposeful human manipulation of the material world.
  \item Consists of four components: matter, power, a tool/machine, and technique.
  \item Greek fire was a formula that provides only matter
  \item power-fire was ovvious but the machines and technique were not
  \item Techniques itself would have been secret of almost as much sophistication as the formula
  \item Componenets of weapon system come into enemy hands,
    no guarantee that it would be used effectively
    (Bulgars captures ships with Greek fire, but did not know how to use it)
  \item Byzantines guaarded their secrets by two ways,
    \begin{enumerate}
      \item limited its use to defense of the capital
        or other fleet engagements, employing trusted commanders
      \item Compartmentalized knowedlge of weapon system. Make sure it doesn't fall in the wrong hands.
    \end{enumerate}
  \item Kallinikos fire was immediately recognized as a weapon system of crutial importance
  \item Secret of its preperation and use was compartmentalized and restricted to a few
    trusted commanders and their crews
  \item The chaos of succession disrupted the transmission of knowledge
    and the secret was lost
\end{itemize}

\subsection{Internationalism in Science}
\begin{itemize}
  \item Theoretical studies / laws of nature,

    good cooperation, through discoveries assumed national pride
  \item Technical / industrial know-how

    Good as long as employed in scientific studies or by allies
  \item Military science

    Always considered secrete
\end{itemize}

\subsection{Lost Tecnologyy know-how}
\begin{itemize}
  \item Damascus Steel
  \item Stradivari Violins
  \item Chinese tower clocks
  \item Roman concrete
\end{itemize}


\section{PPT 11: Technology in Islamic Period}

\subsection{Translation Movement}
\begin{itemize}
  \item Civil engiuneering was quite interesting from their air conditioning systems,
    to their water supply systems, to their sewage systems
  \item Translation Movement, from 8th to 10th century,
    House of Wisdom in Baghadad,
    all Greek and Hellnistic texts were translated into Arabic
  \item Muslim scholars not only translated Greek and Hellenistic works, but work numerous
    commentaries to explain or critize them
  \item They made new observation, solved difficult math equations, developed trigonometry,
    built observatories and hospitals
  \item Scientific activities continued among Muslims while new intellectual
    activities, and the Scientific Revoltion, emerged in Europe
\end{itemize}

\subsection{Patronage: Science and Technology}
\begin{itemize}
  \item Islamic fine technology
  \item Concept of prestige technology
  \item Techonology connected with gardens and astronomy
  \item automata
\end{itemize}

\section*{Astronomy}
\begin{itemize}
  \item There were sundials and whole buildings that were constructed
    to show time of day
  \item Islamic civilization combined three advantages which were very important in history
    of science and technology:
    \begin{enumerate}
      \item Providing a direct contact with the far east
      \item Preserving Greek intellectual materials
      \item Persuing scientific inquiry
    \end{enumerate}
  \item Muslim architecture is a combination of
    Hellenistic, Persian, and local traditions presented in Islamic style
  \item Had an impact on religion, local traditions, and geography
\end{itemize}

\section*{Impact of religion}

\begin{itemize}
  \item Orientation of mosquest and shrines / prayer niche
  \item Aniconism (no statues or paintings),
    limitation
  \item Geometric meotifs / Vegetal Patterns,
    a display of mathematical knowledge
  \item Calligraphy,
    the art of beautiful writing
  \item Minarats (Towers)
  \item Fountains and water
  \item light
  \item Orientation of mosuqes and shrines were a great in math
    and astronomy, since they had to be oriented towards Mecca
  \item aniconism, you couldn't have statues or paintaings, therefore
    geometric motifs, vegetal patterns, and calligraphy
    were the best way to display art.

    Intensely complex geometricasl patterns in display of religious places
  \item Minerats (towers),

    mosquests had many towers of great heights and beauitful designs,
  \item Light was shown through stained glass windows
  \item These Muslim temples had domes just like the Roman Pantheon,
    and were made with great precision and engineering
\end{itemize}

\subsection{Infrastructure}

\begin{itemize}
  \item They used many of same materials as the Romans,
    used Stone, Rubble, Baked/unbaked bricks, clay, timber, mortar and plaster, and Tile
  \item Muhtasib, a person appointed to police the enforcement
    of Islamic law in particular area.

    Exercised a close supervision over building construction.

    Muhtasib had templets and codes to check buildings
\end{itemize}

\section*{Roads and Bridges}
\begin{itemize}
  \item Roads and bridges were important for trade,
    warfare, pilgramage routes to Mecca,
    connection to Silk Road,
    and state postal service (barid)
  \item River Crossing: Pontoon Bridges, Masonry Bridges,
    then great bridges to cross rivers
\end{itemize}

\section*{Irrigation and Water Supply}
\begin{itemize}
  \item Irrigation Systems:
    Water problem in dry lands, canalds, dams, qunats required
  \item Great canal network to connect Baghdad area using
    rivers Euphrates and Tigris
\end{itemize}

\section*{Surveying}
\begin{itemize}
  \item Astronomical observation / calculation of latitudes and logitudes
  \item Accurate latitude was easy
  \item Accurate longitude was difficult and required a reliable chronometer
  \item Observation of lunar eclips contact from two longitudes
  \item Basic requirement for public works, as to setting-out large buildings,
    excavation of canals, etc. Leveling and alignment
\end{itemize}

\section*{Air conditioning}
\begin{itemize}
  \item Hot air enters, goes through a qanat, enterse cooled basement,
    then exist through a wind tower (negative preassure to circulate air)
\end{itemize}

\subsection{Astronomy}
\begin{itemize}
  \item Ptolemy postulated that orbits follow an equant, that is the real center of motion
  \item This then evolved into an elliptical orbit, which was
    a major breakthrough in astronomy,
    Then how there is no real epicycle (inner circle)
  \item 30 degree in 30 days of 1 degree per day,
    360 degrees in 360 days, then 365.25 days in a year
  \item Ptolemy's model was mathematically sound and accurate, but it
    had a physics problem
  \item Violated principle of concentric uniform circulat motion
  \item Planetry motions must be uniform respect to center of universe, Earth
    has not respect this point, the equant did
  \item Solving problems in Ptolemaic model came from several Muslim astronomers

    These writings would go down in Arabic
  \item Nasir al-Din al-Tusi (1201-1274),
    developed a new model of planetary motion, which was
    a combination of Ptolemaic and Copernican models

    This solution eliminated the equant
  \item Tusi introduced arrangement of circular motion of two spheres
    produced linear motion displacing the point,
    closer or further to central point
  \item Arrangement was able to produce non-uniform motion
  \item This model then became to where there is an outer big gear, and inside
    is a smaller gear, which would rotate inside the big gear,
  \item This being called Tusi's couple, which was a
    mathematical model of planetary motion that was used in
    the Copernican model of the universe

    two spheres create linear motion
  \item Tusi's couple was used in the Copernican model of the universe,
    which was a major breakthrough in astronomys

    There were many similarities between them, meaning this work was likely translated
  \item Copernicus' model included Tusi's Couple, Urdi's lemma,
    Ibn al-Shatir's Moon model and his Mercury model,

    Historians question whether he invented these indepedently or borrowed them
\end{itemize}

\newpage

\section{PPT 12: Agricultural \& Military Revolutions in Europe (800s - 1500s)}

\subsection{Migration Period (Dark Ages)}
\begin{itemize}
  \item Dark Ages or Early Middle Ages,
    early medieval period of Western European history
  \item There was no Roman emperor in the West
    in the period between about 500 to 1000 CE,
  \item This marked frequent warfare and virtual dissappearance of urban life
  \item Name of the period refers to barbarian peoples,
    Huns, Goths, Vandals, Bulgars, Alani, Seubi, and Franks into what had
    been the Western Roman Empire
  \item Dark Ages was between the end of Roman period, then Islamic Era, then came
    the Renaissance period
\end{itemize}

\subsection{Medieval Europe}
\begin{itemize}
  \item series of interlocked technical innnovations
    shpaing the history of medieval Europe
  \item Agricultural Revolution:
    \begin{itemize}
      \item New crops, new tools, new techniques
      \item New land, new methods of land use
      \item New social organization
    \end{itemize}
  \item New military technology:
    \begin{itemize}
      \item New weapons, new tactics, new strategies
      \item New social organization
      \item New political organization
    \end{itemize}
  \item Depdendence on wind and water for generation of power and transportation (water)
  \item After 9th century, Europe transformed itself from economy scarcely more advanced
    from traditional Neolithic economy to vibrant and unique civilization to lead
    development of science and industry
  \item Population of Europe grew 38\% between 600 and 1000 CE.
  \item Agriculture revolition was response to population growth and land shortage
  \item Medieval Europe land was used for agriculture and food production,

    Pasture ddairy animals for food, cattle and horses for traction (work), and sheep for wool (clothing)
  \item Expanding cities reduced acreage available for agriculture,
    and increased demand (outcome of trade)
  \item Forests provide timber for construction and shipbuilding,
    and fuel for heating and cooking
  \item Wood was used as fuel in making iron
\end{itemize}

\section*{800: Charlemange \rightarrow{} heavy plow}
\begin{itemize}
  \item Crowned emperro fo Rome and established the Carolingian Empire
  \item No water or irrigation problem
  \item problem: Plowing heavy soils,
    unique ecological conditions in Northern Europe, led to technological innovations
    and European Agriculture Revolution
  \item \textbf{First innovation}: Heavy plow, with the use of cattle

    Huge tool made of wood and iron mounted on wheels and armed with an iron cutter.
    Tore soil and turned it over forming a groove for seeds.
    (Farmers and military were the two greatest users of iron)
  \item Heavy plow provided more friction, needed more traction,
    somtimes pulled by 8 oxen
  \item Mediterrranean scratch plow was adapted to light soils, then heavy plow for heavy soils
  \item Traction problem (limitation): oxen transfered their power
    with yoke beam by pushing with the top of their neck,

    Europeans adapted the horse collar from Chinese who employed it centuries before
  \item \textbf{Second Innovation:} Horse collar transfered power from preassure points from windpipe to shoulders,
    increasing the traction of horses 4-5 fold
  \item \textbf{Third Innovation}: Three-field rotation system,
    two-field farming system farming one field while lewaving anoter fallow
  \item Three-field pattern: arable land was divided into three fields, planting rotated over
    three year cycle: two seasonal plantings with third field left fallow
\end{itemize}

\section*{Social Consequences of Agricultural Revolution}
\begin{itemize}
  \item Deep plow made it possible to farm new lands,
    particular rich and alluvial soils
  \item Northward shift of European agriculture
  \item Heavy plow and and oxen were expensive
  \item Development of Collective ownership and patterns of communal
    agriculture and animal husbandry
  \item Solidifying the medieval village and manorial system as bedrock of European society
  \item Three-field rotation System:
    \begin{itemize}
      \item Spring crop of vegetables and oats
      \item Improved diets of common people
      \item Increase productive capabilities of agriculture
        from 33 to 50 percent
      \item More food, more meat, more surplus \rightarrow{} Population increase
    \end{itemize}
  \item By 1300s, population of Europe trebled to 79 million from low fo 26 million in 600s
  \item Paris increase population 10x (228k in 1300 to 280k in 1400s)
  \item Increase of Cathedrals and Universities
\end{itemize}

\subsection{Military Revolution}
\begin{itemize}
  \item Tech innovations in military affairs
  \item Role of military in feudalism
  \item Military revolution and Europes eventual global
    dominance
  \item Feudalism was a system of land ownership and duties,
    where land was owned by a lord, who would give it to vassals in exchange for military service.
    The kngihts would protect the kings and the lower class of workers
  \item prior to 8th century warrior fought by foot
  \item Stirrupt would provide stability to most skilled horsemen for true cavalry
    and can swing swords and stretch bows wtihout losing mount
  \item Chinese invented stirrups in 5th century,
    but it was not adopted in Europe until 8th century
  \item European knight in horse would be equivalent to the \textbf{tank} of the medieval period
  \item Knights replaced the peasant-soldier common in early middle ages, being
    kjnight was full-time job
  \item Knights were important for land protection, and were
    the backbone of feudalism, they were the elite warriors of the time
  \item no strong central governmkent required to manage agriculture economy
  \item Manorial system was well adapted to European ecology
  \item Knight-village relation became characteristic to European feudalism
  \item prior 15th century, Europeans were behind Muslim, Chinese, and other major civilizations
    in their ability to wage war
  \item Tranamtic transformation led to Military Revolution
  \item Military Revoluition shifted power from local feudal authorities
    to centralized kingdoms and nation-states
  \item Gunpower and early firearms orginiated in Chine,
    large cannons seem to originate in Europe in 1310-20
  \item Technology soon came back to Middle East (1330s) and Asia (1356 for China)
  \item Early cannons and bombards were so large they were casted on the site and could not move
  \item Bornze cannons were common, until 1541 English mastered casting iron cannnon under
    King Henry VIII
  \item Smaller cannons were now mobile and board ships
  \item "gunpowder revolution" was a major turning point in military history,
    undermined military roles of feudal knight and feudal lord, and replaced
    with gunpowder armies and navies fianced by central governments
  \item New weaponry in Europe required large budgets of European governments
  \item Need for more surplus and tax systems
  \item Cannons reduced medieval castles and old-style city walls (require harder reinforcements)
  \item Requirement for new and stronger, more expensive countermeasures
  \item Larger political entities, notably centralized natio-states
    with greater taxing power and wealth to afford weaponry and necessary fortifications
\end{itemize}

\subsection{Fortifications}
\begin{itemize}
  \item Castle fell to cannon it replaced by complex system of fortifications
  \item Castles with projections known as $trace italienne$
  \item Effective as defense system and proved to be extraordinarily expensive
\end{itemize}

\subsection{Military Revolution - Conclusion}
\begin{enumerate}
  \item Military technology led European societies to \textbf{centralized authority}
  \item Military Revolution initiated wave of European colonialism and beginning of European global conquest
  \item  Military Revolution introduced competition between states and dynamic social mechanisms that relentlessly favored technical
    development
\end{enumerate}

\section*{Viewing map and European expansion (From 1550 to 1754)}
\begin{itemize}
  \item Russia went from small north western area to modern Russia and USSR states
  \item Then Spain conquesring more of the Americas such as Florida, and more of South America
  \item then more European powers in Africa, Indian coast, and South East Asia
\end{itemize}

\subsection{Three Elements of the Military Revoltion}
\begin{enumerate}
  \item Replacing heavility armored cavalry
    for infrantry for more effective warfare
  \item Introduced gunpowder weapons, most importantly, artillery (replacing catapilts)
  \item Rise of massive armies
\end{enumerate}

\section*{Further developments}
\begin{itemize}
  \item Developments of new machines and new sources of power
  \item water/wind mills
  \item Time of the compilation of the Domesday Book (1086) shows an estimate 6500 watermills
    in England Alone
  \item Watermills in England were responsible for more production,
    demand for raw materials, trade, and more surplus
  \item Europeans began to use wind and water
    power on unprecendented scale.
\end{itemize}

\newpage

\section{PPT 13: Books and Gowns\\ Education in Medieval Europe}
\begin{itemize}
  \item European north of Alphs never been on teh scene of much
    higher learning prior to 12th century
  \item 789 Charlemagne established Palace School at Aachen
    and invited scholars to teach his children
  \item "Cathedral schools" in each bishopric in order to guarantee a supply
    of literate priests for an otherwise illitarate society
\end{itemize}

\subsection{Education Curriculum}
\begin{itemize}
  \item Early Middle Ages
  \item Seven Liberal Arts
    \begin{itemize}
      \item Trivium: Grammar, Rhetoric, Logic
      \item Quadrivium: Arithmetic, Geometry, Music, Astronomy
    \end{itemize}
  \item Some knowledge of astronomy for astrological and calendrical purposes,
    especially for Easter's date
  \item Intellectual emphasis remained on theology and religious affairs rather than science
  \item \textbf{Almost no original scientific research took place}

    (It was mostly bringing back knowledge from the past along
    with catching up with the Muslim and Asian world)
  \item Gerbert of Aurillac (later: Pope Sylvester II) (946-1003)
    was the first to introduce Arabic numerals and the abacus to Europe

    He studied Arab and Greco Roman Arithmetic, mathematics, and astronomy in Spain.

    Reintroduced to Europe the abacus and armillary sphere
  \item Muslim Spain and Chirstian kingdoms
\end{itemize}

\section*{European Universities}
\begin{itemize}
  \item Weakly organized learning in early Middle Ages
  \item Eruption of European universities in 12th century,
    spread of higher education across Europe
  \item By 1200s, Europeans recovered much of ancient science along with
    several centuries of scientific and pholosophical accomplishments within Islamic World

    (rediscovery of ancient science and catching up with Islamic science)
  \item University of Bologna (1088) was the first university in Europe,
    Paris (1096), Oxford (1096), Padua (1222), then 80 more by 1500s
  \item Agricultural Revolution:
    \begin{itemize}
      \item Flourishing cities
      \item Growing economy
      \item Increase wealth
      \item Need for urban institutions to produced learned people
      \item Universities depend on studnets to pay, learn, and have job prospects
    \end{itemize}
  \item European univesity unique institutions, modeled after craft guilds of medieval Europe
  \item Universities evolved as nomianally secular (uninvolved with religion) communities of students and master teachers
  \item Universities did not depend on state or individual patronage
    like scribal schools of antiquity or Islamic madrasa
  \item They were not state organs, but remained independent, typically deufal institutions
  \item European universities and Standardization of higher education, curriculum, and licensing
  \item European Universities and Natural Sciences
  \item Unlike universities today, mediaval universities were not research institutions,
    but were teaching institutions, with no research component,
    science was not pursued primarily as an end in itself
\end{itemize}

\subsection{Source of Knowledge - Translation Movement}
\begin{itemize}
  \item Toledo: Major center of Islamic culture - fell to Chrisitans 1085
  \item Toledo becmae center of translation acitivty
  \item Translation classical scientific and philosophical texts
    from Arabic (and sometimes Greek) into Latin
  \item Role of Bilingual Jewish scholars
  \item Translation in souther Italy and in Sicily
  \item By 1200, European recoeverd ancient science along with several scientiic philosophical accomplishments produced
    in Islamic world
  \item Adelard of Bath translated Euclid's Element (From Arabic)
  \item Gerard of Cremona (traveled to Spain to locate copy of Ptolemy's Algamest, stayed 40 years to translate
    Algamest, but 78 oter worlds from Arabic)
  \item Archimedes, Galen, HIppocrates, Aristotle, and Islamic commentaries on Aristotle were translated
  \item Renaissance better translation were made of original sources
  \item 12th century: Period of Translation
  \item 13th century: Period of Assimilation, absorptioon of scientific and philosophical knowledge
    into European intellectual life
  \item Problem: reconcile traditional Chrisitian worldview with Aristotle and other pagan Greek Traditions
\end{itemize}

\subsection{Problems with Aristotle - Conflict with Christianity }

\begin{itemize}
  \item Problem of eternity of universe, it was made by God
  \item Problem of the soul,
  \item problem of cause and effect, it is already destined to happen
  \item Problem of unmoved mover,
  \item Thomas Aquinas (1225-1274) reconciled Aristotle with Christianity
    and made it compatible with Christian theology

    provided intellectual system of rational thought about God, man, and nature

    Aristotle's logic, explanatory philisophy, and intellectual amalgam of Christian thoelogy and Aristotelian science
    produced coherent and unified vision of world and humanity's place in it

  \item Birth of Scholastic Philosophy
  \item Institutional conflicts caused intellectual problems assimilating Aristotle into standard theology
  \item \textbf{The condemnation of 1277} by Bishop of Paris, Etienne Tempier,
    condemned 219 Aristotelian propositions that were incompatible with Christian theology.
    Anyone caugt holding or teaching them were subject to excommunication
  \item Some reserachers argue freeing medieval thnkers from Aristotle liberated them to conceive
    new alternatives in solving long-standing problems in Aristotelian science and natural philosophy

    Not being able to teach these 219 proposition made them think outside the box
  \item Jean Buridan and Nicole Oresme, two of the most important thinkers of the 14th century,
    were not bound by Aristotelian dogma and were able to develop new ideas.

    Ideas of earth's daily rotation on its axis and how motion occurs in nature
  \item First chancellor of Oxford, Robert Frosseteste, argued active investigation of nature on account
    is hailed as fatehr of experimental methods of sciences
  \item Jean Buridan: projectiles and Nicole Oresme: Graphs, represent quality and qualitative change geometrically
  \item Oresme depitected uniformly accelerated motion, motion that is "uniformly difform" that we see from
    falling bodies (foundation of $g$ gravitational acceleration constant being $9.8\frac{m}{s^2}$))
  \item Cosmos illustrated in The Divine Comedy by Dante Alighieri (1265-1321)
    was a medieval synthesis of Aristotelian and Ptolemaic cosmology

    Proposed hierarchty of setting of the universe and divine
    laws are all compatible with Aristotle's physics
  \item Medieval scholars interpreted world primarily from thoelogical POV
  \item Reason can play role in human understanding of divine
  \item Humans can learn existence and nature of Godd from his world as well as his word (study of Bible)
  \item Context of medieval worldview secular natural science took second place
    whenever Aristotle's natural philosophy clased with traditional Christian theology
  \item Despite universities having secular worldviews, natural sciences were second, and first
    was theology...
\end{itemize}

\subsection{Selections from condemnation of 1277 (reference, not on exam)}
\begin{itemize}
  \item 1. That there is no more excellent state than to study philosophy
  \item 2. That the only wise men in the world are philosophers
  \item 4. One should not hold anything unless it is self-evident or can
    be manifestived from self-evident principles
  \item 17. That wat is impossible absolitely speaking cannot be brought
    about by God or by another agent. Tihs is erroneus if we mean when is impossible according to nature
  \item 66. God could not move the heaven in a straight line, reason being that he would leave a vacuum
  \item 73. Heavenly bodies are moved by an intrinsic principle, and not by an extrinsic one.
    This intrisic principle is the soul, and tat they are moved by a soul and appetitive power, like an animal
\end{itemize}


\section{Technology and the natural sciences at the Threshold of the Scientific Revolution}

\end{document}

