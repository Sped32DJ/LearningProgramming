\documentclass{article}
\title{Week 1}
\author{Danny Topete}
\date{\today}

\begin{document}
\maketitle

\section{Lecture 1: Syllabus day}
\begin{itemize}
  \item Exams
    \begin{itemize}
      \item Requires lockdown browser
      \item Requires camera to be enabled during exam
      \item Occur during lecture time
      \item AI isn't even worth using, sometimes
        there will be sources about his own PhD dissertation
    \end{itemize}
  \item Get the book that is required for this class
\end{itemize}

\section*{Why History of Science and Technology?}
\begin{itemize}
  \item Science and technology are social products
    \begin{itemize}
      \item Products from people who live in a society and make
        advancements
      \item Ways that people solve problems with their advancements
      \item If we never had our world war 2, and all the
        money backing up research and labs.
        We would not have nuclear technology
      \item The impact of politics and technology that
        cause technology to advance
      \item We discovered the ntron in 1932, but
        it wasn't until the 1940s that we were able to
        use it for nuclear technology (very fast!)
    \end{itemize}
  \item The first printed book came in 1778,
    the Muslims knew about the printing press.
    Then there was a 300 year delay, between mass adoptation
  \item Muslims believe that writting the Quran
    is a form of worship, so they didn't want to
    print it.
  \item Meanwhile the Bible was being printed
    in the 1400s, since there were many errors in early bibles.
    So the printing press was used to print the Bible.
    We also now made owning a bible cheap and
    accessible to the masses.
\end{itemize}

\section*{Periodization of History}
\begin{itemize}
  \item Scientific Revolution (1500s to 1700s)
    \begin{itemize}
      \item The period of time when science was born
      \item The period of time when the scientific method was born
      \item The period of time when the scientific method was used
        to solve problems
      \item New approach to understand nature
      \item Mathematization of nature
      \item Explaining nature with numbers.

        Making laws of nature with matematical experimenting.
        Calling natural and universal laws from seeing consistent
        patterns in natures of numbers.
      \item Natural laws
        \begin{itemize}
          \item We know that an electron here on earth
            is equal to an electron on the moon.
        \end{itemize}
      \item Experimental philosophy
      \item Instrumentation
        \begin{itemize}
          \item Need to make instruments in order to measure temperature
            and force in order to make experiments.
          \item How can we make a device to measure natural phenomena?
          \item Along with parameters in the field of science
        \end{itemize}
      \item New sciences
        \begin{itemize}
          \item We created new sciences, antropology, geophysics,
            electromagnetism, thermodynamics, etc.
          \item This is part of a scientific revolution.

        \end{itemize}
      \item Wasn't very easy to make everyone accept
        the new scientific method. Along with the
        world revolving around the sun.
      \item Bringing science to the masses
        and sciencitific explanations to previously
        magical explanations or that had not been explained.
        Or if it had been explained with religion.
      \item Creating data by researching
        and experimenting, and then using that data
        to create a theory.
      \item Interfering with nature to understand it
        and then using that understanding to create
        new technology.
    \end{itemize}
  \item \textbf{The Industrial Reolution (Late 18th to mid 19th century)}
    \begin{itemize}
      \item Introduces new source of power (steam engine)
      \item Major shjift in agriculture, manufacturing,
        mining, transportation, and technology
      \item Mechanization of production, and elboration of the factory system
      \item Developments in the global market system to support indistrual production
      \item Iron, coal, and steam become the emblamatic resources of the society
    \end{itemize}
\end{itemize}

\section*{Definitions}
\begin{itemize}
  \item What is science?
  \item What is Applied Science?
  \item What is technology?
  \item What is the relation between pure science
\end{itemize}


Pre-historic people can be regarded as the founders
of science because they learned how to melt metals or develop agriculture.
\begin{itemize}
  \item Ishango bones, believed to be 40-20,000 years old held prime numbers.
   They were holding tally numbers
\end{itemize}


\noindent \textbf{Second View:}
\begin{itemize}
  \item Sciece is a body of theoreical knowledge, and systmatic way to discovery natural laws
  \item This definition, distinguishes science from other forms of knowledge
\end{itemize}

\newpage
\section{Lecture 2}

\subsection{Pure science related to technology}
\begin{itemize}
  \item The natural laws that are discovered during the tech advancements.
    The pure science that is discovered such as aerodynamics, thermodynamics, etc.
  \item We created tools and technology in order to ensure our advancements.
  \item Tools are an outcome of science and technology advancements. We wouldn't
    need instrumentation unless we are making experiments or advancements in science.
  \item Then we think about how we define energy or mass. What are these constants and how
    do we measure them?
  \item in terms of equations, what is the relationship between energy and mass?
    \begin{itemize}
      \item $E = mc^2$
      \item $E = hf$
      \item $E = kT$
    \end{itemize}
  \item Application of theoretical science with engineering.
    We are seeing how to make the most effiecient engine
    given the laws and constants of thermodynamics (nature).
  \item It is the advancements in technology that allow us to
    make advancements in science; pure science to be exact.
  \item There was an advancement about how subatomic particles
    and how they wobble, we discovered that they wobble a long time ago.
    Later we discover that they wobble in a way to make an MRI (a real application).
\end{itemize}

\subsection{Methodology and Epistemology of science}

\begin{itemize}
  \item What is the difference between methodology and epistemology?
    \begin{itemize}
      \item Methodology is the way we do science, the methods we use to
        make experiments and collect data.


      \item Epistemology is the study of knowledge, how we know what we know.
        It is the philosophy of science.
    \end{itemize}
  \item When it comes to the sun rotating, we don't know
    who is actually moving, the sun or the earth.

    From pure perspective, our obsercations are what we want to believe.
  \item Scientific method is a way to test our theories
    and hypotheses, and to collect data.

    We must hypethesize, make predictions, and then test those predictions

  \item From the scientific method, we need to have a starting
    point, decide where to start measuring based off your observations and questions.
  \item After doing experiemntations to various people, we can
    then make a theory based off the data we collected.

    But we still can't make an absolute statement, just what works for the
    people that were tested.
  \item Meanwhile when it comes to religion, there is only one edition of sacred book meanwhile
    science is always changing and evolving.

    There are thousands of editions of physics books but only one edition of the Bible.
  \item We have many data points about various sciences,
    but we must connect them all and how they are related.
    Making a theory that connects all the data points
  \item Or talk about the creation of the sun,
    what happens if we have trillions of tons of
    hydrogen and helium, and we compress it.

    We take knowledge from chemistry, physics, and astronomy
    in order to make a theory about how the sun was created.
    Based off the behavior of the particles and how they interact,
    especially at high temperatures and pressures. Then we can theorize
    the exact temperature and pressure required to create a star.
    Then how it was possible with the Big Bang.
    Then figure out more about the Big Bang and how it created
    everything we know today.
  \item It is all about connecting the dots, it is what Charles Darwin did
    with the theory of evolution. He connected the dots between
    the various species and how they evolved from a common ancestor.
  \item Then Big Bang theory is the same thing, it is a theory that connects
    all the dots between the various sciences and how they are related.
    It is a theory that explains how the universe was created and how it evolved
    over time.

    The universe is always expanding, and we can see that due to various
    laws of pure science.
\end{itemize}

\subsection{What about Astrology, Alchemy, and other "unscientific" approaches to discover nature?}
\begin{itemize}
  \item What about Astrology, Alchemy, and other "unscientific" approaches
    to discover nature?

  \item Alchemy was not real since it was pure trial and error,
    there was not scientific method or methodology.
    They had no knowledge of chemistry or physics,
  \item They just keep experimenting with various chemicals
    and trying to turn lead into gold.
    They had no idea what they were doing, they were just
    trying to find the philosopher's stone.
  \item They believed that everything below the moon was composed of four elements:
    earth, water, air, and fire.
    They believed that everything above the moon was composed of aether.
  \item It was not a science, but rather an approach toi know nature
  \item Although alchemy did make some advancements in chemistry,
    it was not a science since it did not follow the scientific method.

    Someone peed in a jar and boiled it, then they discovered phosphorus.

    They found the difference between elements and compounds.

    They disocvered about acids and bases, and how they react with each other.

    They made many tools and instruments to measure various chemicals, to distill
    chemicals, and to make various experiments.
  \item They had some knowledge, but had no real pure science to understand the concepts
    of universal laws of nature.
  \item Pure sciece coming from fundamental laws of nature
    and how they interact with each other.
\end{itemize}

\subsection{Relationship between pure science and applied science}
\begin{itemize}
  \item Johannas Kepler was an alchemist, but he was also a mathematician
    and astronomer. He used his knowledge of mathematics and astronomy
    to make predictions about the motion of planets and stars.
    He was able to predict the motion of planets with great accuracy,
    and he was able to explain why they moved the way they did.

    He didn't know why planets had to revolve around the sun in an elliptical orbit,
    but Newton was able to explain why they moved the way they did

  \item In 1820s we dioscovered electromagnetism, but it wasn't until the 1860s
    that we were able to use it for technology.
    We had to wait for the right materials and technology to be invented
    in order to make it work. There was no application for that knowledge
    until the 1860s when we were able to use it for telegraphy and telephony.

    The move from pure science to applied science
  \item First we discoeverd Lightning, then we discovered electricity,
    then we discovered electromagnetism, and then we discovered how to use it
    for technology. It was a long process, but it was worth it.
  \item Faraday discovered electromagnetic induction in 1831,
    but it wasn't until 1867 that we were able to use it for technology.
    It was a long process, but it was worth it.
  \item Famously, there was a prime minister in the UK
    that said that we should not invest in electricity since it was a waste of money.
    He said that it was just a fad and that it would never amount to anything.
    He was wrong, and we are now using electricity for everything.

    Micheal Faraday famously said that "I don't know what the application is but I know that it is important"
    (Or something like that).
  \item We discovered the electron in 1897, but it wasn't until the 1920s
    that we were able to use it for technology. We had to wait for the right materials
    and technology to be invented in order to make it work.
  \item After Kepler figured out that planets reolved in elliptical orbits,
    Newton was able to explain why they moved the way they did.
    He was able to explain the laws of motion and gravity, and how they
    interacted with each other. He was able to explain why planets
    revolved around the sun in an elliptical orbit, and why they moved
    the way they did.
  \item Kepler figured planets move in ellipsticla orbits,
    he knew no idea whty, but he had to use pure science and mathematics
    to help further figure out why they moved the way they did.
\end{itemize}

\subsection{Non-Eclecticism}
\begin{itemize}
  \item Riemman Hypothesis is a conjecture about the distribution of prime numbers.
    It is a pure science problem, but it has no application for technology.
    It is a pure science problem that has no application for technology,
    but it is still an important problem to solve.
  \item Riemann non-Eucledean geometry was a pure science problem
    that has no application for technology, but it is still an important problem to solve.
    It is a pure science problem that has application for technology
  \item Riemann's hypothesis was just a conjecture until Einstein
    was able to figure out that time space warps around a large mass
    and was able to apply it to the theory of relativity.
\end{itemize}

\subsection{Theory to Practice}
\begin{itemize}
  \item Many scientist discovered new things but never had applications
  \item Now as engineers, we take those theories and apply them to technology
    and make new inventions.
  \item Nazi Germany discovered that there was a chain reaction with Uranium, and very fast.

    We can split an atom and it releases a lot of energy. But we didn't know what to do with it.

    Uranium can be split and realase two neutrons, and those two neutrons can split, to hit others and
    cause a chain reaction. Which leads to the atomic bomb.
\end{itemize}

\subsection{Technological before pure science}
\begin{itemize}
  \item Before, tech was developed to solve problems.
  \item Such as parts of the world that had water problems discovered
    aqueducts meanwhile parts of Europe had no water problems and
    never had the need for aqueducts.
  \item In China there was always a mass population of people,
    so they had to develop technology to solve problems.
    Such as the Great Wall of China, which was built to protect
    against invasions from the north.

    Then had to figure out how to feed all the people.
  \item Tech comes from the real world and needs a basis of pure science
    in order to be able to solve problems.
  \item If there is no pure science to back the technology,
    You must need trial and error to figure out how to make it work.

    Morse was a painter, he wanted to travel from New York to Washington DC,
    to paint a portrait of the president.
    That is when a man on a horse came to him and said that his wife was dying.
    By the time he got to Washington DC, his wife had already died.
    He then thought about how to send messages faster,

    He was using existing technology about pure science and electricity/electromagnetism
  \item If you don't have a problem to solve, you will never think of
    the pure science that is required to solve it.
    You \textbf{need} a problem to solve in order to think about advancements in applied science
    using pure science.
  \item Morse materialized the telegraph, but it was not a pure science problem.
    It was a technology problem that he solved using existing technology
    and pure science.
\end{itemize}

\section*{Pre-science technology}
\begin{itemize}
  \item There was discoveries such as pottery since clay hardens after being heated.
    Then we discovered that we can make bricks and then buildings.
    We discovered that we can make glass by heating sand, and then
    we discovered that we can make windows.
  \item Then heating various things up, they discoeverd new things,
    such as milk going bad slower after being heated up.
    Then we discovered that we can pasteurize milk and make it last longer.
    Then how they discovered cheese and yogurt.
  \item Then how wine was discoeverd by accident, when grapes were left out
    and fermented (to be stored). Then we discovered that we can make alcohol by fermenting
    various things.

    Alcohol from fermenting fruits were the earliest ways to make alcohol that
    predates history. But people never knew germ theory or the science behind it.

    \item Some people died along with way in discoevering what we can eat and what we can't eat.
    But we discoeverd plants that can be pain relievers.

  \item We also learned from problems that when they feel sick,
    they chew specific leafs.
\end{itemize}

\section*{Trial and Error}
\begin{itemize}
  \item We discovered gambling from trial and error
  \item Karnough discovered heat engines from his experiemnts and discoevered
    theoretical thermodynamics.
  \item
\end{itemize}
\subsection{Crossfield Technology}
\begin{itemize}
  \item The relationships between crossfield, we discover something
    in electricity, then see how to use it in medical industry.
  \item Radio wave science/Tech and astrology helped up figure out
    long distance communication.
  \item Analyzing large plans and how they pick up their wings in order to fly

    now all plans have their wings curved up at the ends. Discovery
    by Richard T Whitcomb with studying big birds and their curled wings
\end{itemize}
\end{document}
