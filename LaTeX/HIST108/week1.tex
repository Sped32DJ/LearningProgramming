\documentclass{article}
\title{Week 1}
\author{Danny Topete}
\date{\today}

\begin{document}
\maketitle

\section{Lecture 1: Syllabus day}
\begin{itemize}
  \item Exams
    \begin{itemize}
      \item Requires lockdown browser
      \item Requires camera to be enabled during exam
      \item Occur during lecture time
      \item AI isn't even worth using, sometimes
        there will be sources about his own PhD dissertation
    \end{itemize}
  \item Get the book that is required for this class
\end{itemize}

\section*{Why History of Science and Technology?}
\begin{itemize}
  \item Science and technology are social products
    \begin{itemize}
      \item Products from people who live in a society and make
        advancements
      \item Ways that people solve problems with their advancements
      \item If we never had our world war 2, and all the
        money backing up research and labs.
        We would not have nuclear technology
      \item The impact of politics and technology that
        cause technology to advance
      \item We discovered the ntron in 1932, but
        it wasn't until the 1940s that we were able to
        use it for nuclear technology (very fast!)
    \end{itemize}
  \item The first printed book came in 1778,
    the Muslims knew about the printing press.
    Then there was a 300 year delay, between mass adoptation
  \item Muslims believe that writing the Quran
    is a form of worship, so they didn't want to
    print it.
  \item Meanwhile the Bible was being printed
    in the 1400s, since there were many errors in early bibles.
    So the printing press was used to print the Bible.
    We also now made owning a bible cheap and
    accessible to the masses.
\end{itemize}

\section*{Periodization of History}
\begin{itemize}
  \item Scientific Revolution (1500s to 1700s)
    \begin{itemize}
      \item The period of time when science was born
      \item The period of time when the scientific method was born
      \item The period of time when the scientific method was used
        to solve problems
      \item New approach to understand nature
      \item Mathematization of nature
      \item Explaining nature with numbers.

        Making laws of nature with matematical experimenting.
        Calling natural and universal laws from seeing consistent
        patterns in natures of numbers.
      \item Natural laws
        \begin{itemize}
          \item We know that an electron here on earth
            is equal to an electron on the moon.
        \end{itemize}
      \item Experimental philosophy
      \item Instrumentation
        \begin{itemize}
          \item Need to make instruments in order to measure temperature
            and force in order to make experiments.
          \item How can we make a device to measure natural phenomena?
          \item Along with parameters in the field of science
        \end{itemize}
      \item New sciences
        \begin{itemize}
          \item We created new sciences, antropology, geophysics,
            electromagnetism, thermodynamics, etc.
          \item This is part of a scientific revolution.

        \end{itemize}
      \item Wasn't very easy to make everyone accept
        the new scientific method. Along with the
        world revolving around the sun.
      \item Bringing science to the masses
        and sciencitific explanations to previously
        magical explanations or that had not been explained.
        Or if it had been explained with religion.
      \item Creating data by researching
        and experimenting, and then using that data
        to create a theory.
      \item Interfering with nature to understand it
        and then using that understanding to create
        new technology.
    \end{itemize}
  \item \textbf{The Industrial Reolution (Late 18th to mid 19th century)}
    \begin{itemize}
      \item Introduces new source of power (steam engine)
      \item Major shjift in agriculture, manufacturing,
        mining, transportation, and technology
      \item Mechanization of production, and elboration of the factory system
      \item Developments in the global market system to support indistrual production
      \item Iron, coal, and steam become the emblematic resources of the society
    \end{itemize}
\end{itemize}

\section*{Definitions}
\begin{itemize}
  \item What is science?
  \item What is Applied Science?
  \item What is technology?
  \item What is the relation between pure science
\end{itemize}


Pre-historic people can be regarded as the founders
of science because they learned how to melt metals or develop agriculture.
\begin{itemize}
  \item Ishango bones, believed to be 40-20,000 years old held prime numbers.
   They were holding tally numbers
\end{itemize}


\noindent \textbf{Second View:}
\begin{itemize}
  \item Sciece is a body of theoreical knowledge, and systmatic way to discovery natural laws
  \item This definition, distinguishes science from other forms of knowledge
\end{itemize}

\newpage
\section{Lecture 2}

\subsection{Pure science related to technology}
\begin{itemize}
  \item The natural laws that are discovered during the tech advancements.
    The pure science that is discovered such as aerodynamics, thermodynamics, etc.
  \item We created tools and technology in order to ensure our advancements.
  \item Tools are an outcome of science and technology advancements. We wouldn't
    need instrumentation unless we are making experiments or advancements in science.
  \item Then we think about how we define energy or mass. What are these constants and how
    do we measure them?
  \item in terms of equations, what is the relationship between energy and mass?
    \begin{itemize}
      \item $E = mc^2$
      \item $E = hf$
      \item $E = kT$
    \end{itemize}
  \item Application of theoretical science with engineering.
    We are seeing how to make the most effiecient engine
    given the laws and constants of thermodynamics (nature).
  \item It is the advancements in technology that allow us to
    make advancements in science; pure science to be exact.
  \item There was an advancement about how subatomic particles
    and how they wobble, we discovered that they wobble a long time ago.
    Later we discover that they wobble in a way to make an MRI (a real application).
\end{itemize}

\subsection{Methodology and Epistemology of science}

\begin{itemize}
  \item What is the difference between methodology and epistemology?
    \begin{itemize}
      \item Methodology is the way we do science, the methods we use to
        make experiments and collect data.


      \item Epistemology is the study of knowledge, how we know what we know.
        It is the philosophy of science.
    \end{itemize}
  \item When it comes to the sun rotating, we don't know
    who is actually moving, the sun or the earth.

    From pure perspective, our obsercations are what we want to believe.
  \item Scientific method is a way to test our theories
    and hypotheses, and to collect data.

    We must hypethesize, make predictions, and then test those predictions

  \item From the scientific method, we need to have a starting
    point, decide where to start measuring based off your observations and questions.
  \item After doing experiemntations to various people, we can
    then make a theory based off the data we collected.

    But we still can't make an absolute statement, just what works for the
    people that were tested.
  \item Meanwhile when it comes to religion, there is only one edition of sacred book meanwhile
    science is always changing and evolving.

    There are thousands of editions of physics books but only one edition of the Bible.
  \item We have many data points about various sciences,
    but we must connect them all and how they are related.
    Making a theory that connects all the data points
  \item Or talk about the creation of the sun,
    what happens if we have trillions of tons of
    hydrogen and helium, and we compress it.

    We take knowledge from chemistry, physics, and astronomy
    in order to make a theory about how the sun was created.
    Based off the behavior of the particles and how they interact,
    especially at high temperatures and pressures. Then we can theorize
    the exact temperature and pressure required to create a star.
    Then how it was possible with the Big Bang.
    Then figure out more about the Big Bang and how it created
    everything we know today.
  \item It is all about connecting the dots, it is what Charles Darwin did
    with the theory of evolution. He connected the dots between
    the various species and how they evolved from a common ancestor.
  \item Then Big Bang theory is the same thing, it is a theory that connects
    all the dots between the various sciences and how they are related.
    It is a theory that explains how the universe was created and how it evolved
    over time.

    The universe is always expanding, and we can see that due to various
    laws of pure science.
\end{itemize}

\subsection{What about Astrology, Alchemy, and other "unscientific" approaches to discover nature?}
\begin{itemize}
  \item What about Astrology, Alchemy, and other "unscientific" approaches
    to discover nature?

  \item Alchemy was not real since it was pure trial and error,
    there was not scientific method or methodology.
    They had no knowledge of chemistry or physics,
  \item They just keep experimenting with various chemicals
    and trying to turn lead into gold.
    They had no idea what they were doing, they were just
    trying to find the philosopher's stone.
  \item They believed that everything below the moon was composed of four elements:
    earth, water, air, and fire.
    They believed that everything above the moon was composed of aether.
  \item It was not a science, but rather an approach toi know nature
  \item Although alchemy did make some advancements in chemistry,
    it was not a science since it did not follow the scientific method.

    Someone peed in a jar and boiled it, then they discovered phosphorus.

    They found the difference between elements and compounds.

    They disocvered about acids and bases, and how they react with each other.

    They made many tools and instruments to measure various chemicals, to distill
    chemicals, and to make various experiments.
  \item They had some knowledge, but had no real pure science to understand the concepts
    of universal laws of nature.
  \item Pure sciece coming from fundamental laws of nature
    and how they interact with each other.
\end{itemize}

\subsection{Relationship between pure science and applied science}
\begin{itemize}
  \item Johannas Kepler was an alchemist, but he was also a mathematician
    and astronomer. He used his knowledge of mathematics and astronomy
    to make predictions about the motion of planets and stars.
    He was able to predict the motion of planets with great accuracy,
    and he was able to explain why they moved the way they did.

    He didn't know why planets had to revolve around the sun in an elliptical orbit,
    but Newton was able to explain why they moved the way they did

  \item In 1820s we dioscovered electromagnetism, but it wasn't until the 1860s
    that we were able to use it for technology.
    We had to wait for the right materials and technology to be invented
    in order to make it work. There was no application for that knowledge
    until the 1860s when we were able to use it for telegraphy and telephony.

    The move from pure science to applied science
  \item First we discoeverd Lightning, then we discovered electricity,
    then we discovered electromagnetism, and then we discovered how to use it
    for technology. It was a long process, but it was worth it.
  \item Faraday discovered electromagnetic induction in 1831,
    but it wasn't until 1867 that we were able to use it for technology.
    It was a long process, but it was worth it.
  \item Famously, there was a prime minister in the UK
    that said that we should not invest in electricity since it was a waste of money.
    He said that it was just a fad and that it would never amount to anything.
    He was wrong, and we are now using electricity for everything.

    Micheal Faraday famously said that "I don't know what the application is but I know that it is important"
    (Or something like that).
  \item We discovered the electron in 1897, but it wasn't until the 1920s
    that we were able to use it for technology. We had to wait for the right materials
    and technology to be invented in order to make it work.
  \item After Kepler figured out that planets reolved in elliptical orbits,
    Newton was able to explain why they moved the way they did.
    He was able to explain the laws of motion and gravity, and how they
    interacted with each other. He was able to explain why planets
    revolved around the sun in an elliptical orbit, and why they moved
    the way they did.
  \item Kepler figured planets move in ellipsticla orbits,
    he knew no idea whty, but he had to use pure science and mathematics
    to help further figure out why they moved the way they did.
\end{itemize}

\subsection{Non-Eclecticism}
\begin{itemize}
  \item Riemman Hypothesis is a conjecture about the distribution of prime numbers.
    It is a pure science problem, but it has no application for technology.
    It is a pure science problem that has no application for technology,
    but it is still an important problem to solve.
  \item Riemann non-Eucledean geometry was a pure science problem
    that has no application for technology, but it is still an important problem to solve.
    It is a pure science problem that has application for technology
  \item Riemann's hypothesis was just a conjecture until Einstein
    was able to figure out that time space warps around a large mass
    and was able to apply it to the theory of relativity.
\end{itemize}

\subsection{Theory to Practice}
\begin{itemize}
  \item Many scientist discovered new things but never had applications
  \item Now as engineers, we take those theories and apply them to technology
    and make new inventions.
  \item Nazi Germany discovered that there was a chain reaction with Uranium, and very fast.

    We can split an atom and it releases a lot of energy. But we didn't know what to do with it.

    Uranium can be split and realase two neutrons, and those two neutrons can split, to hit others and
    cause a chain reaction. Which leads to the atomic bomb.
\end{itemize}

\subsection{Technological before pure science}
\begin{itemize}
  \item Before, tech was developed to solve problems.
  \item Such as parts of the world that had water problems discovered
    aqueducts meanwhile parts of Europe had no water problems and
    never had the need for aqueducts.
  \item In China there was always a mass population of people,
    so they had to develop technology to solve problems.
    Such as the Great Wall of China, which was built to protect
    against invasions from the north.

    Then had to figure out how to feed all the people.
  \item Tech comes from the real world and needs a basis of pure science
    in order to be able to solve problems.
  \item If there is no pure science to back the technology,
    You must need trial and error to figure out how to make it work.

    Morse was a painter, he wanted to travel from New York to Washington DC,
    to paint a portrait of the president.
    That is when a man on a horse came to him and said that his wife was dying.
    By the time he got to Washington DC, his wife had already died.
    He then thought about how to send messages faster,

    He was using existing technology about pure science and electricity/electromagnetism
  \item If you don't have a problem to solve, you will never think of
    the pure science that is required to solve it.
    You \textbf{need} a problem to solve in order to think about advancements in applied science
    using pure science.
  \item Morse materialized the telegraph, but it was not a pure science problem.
    It was a technology problem that he solved using existing technology
    and pure science.
\end{itemize}

\section*{Pre-science technology}
\begin{itemize}
  \item There was discoveries such as pottery since clay hardens after being heated.
    Then we discovered that we can make bricks and then buildings.
    We discovered that we can make glass by heating sand, and then
    we discovered that we can make windows.
  \item Then heating various things up, they discoeverd new things,
    such as milk going bad slower after being heated up.
    Then we discovered that we can pasteurize milk and make it last longer.
    Then how they discovered cheese and yogurt.
  \item Then how wine was discoeverd by accident, when grapes were left out
    and fermented (to be stored). Then we discovered that we can make alcohol by fermenting
    various things.

    Alcohol from fermenting fruits were the earliest ways to make alcohol that
    predates history. But people never knew germ theory or the science behind it.

    \item Some people died along with way in discoevering what we can eat and what we can't eat.
    But we discoeverd plants that can be pain relievers.

  \item We also learned from problems that when they feel sick,
    they chew specific leafs.
\end{itemize}

\section*{Trial and Error}
\begin{itemize}
  \item We discovered gambling from trial and error
  \item Karnough discovered heat engines from his experiemnts and discoevered
    theoretical thermodynamics.
  \item
\end{itemize}
\subsection{Crossfield Technology}
\begin{itemize}
  \item The relationships between crossfield, we discover something
    in electricity, then see how to use it in medical industry.
  \item Radio wave science/Tech and astrology helped up figure out
    long distance communication.
  \item Analyzing large plans and how they pick up their wings in order to fly

    now all plans have their wings curved up at the ends. Discovery
    by Richard T Whitcomb with studying big birds and their curled wings
\end{itemize}

\section{Lecture 3: Skolimowski}
June 25, 2025

\subsection{What is science and technology?}
\begin{itemize}
  \item Finishing the discussion about
    science and technology today
  \item Covering the difference between
    science and technology
  \item How they are defined differently
  \item they are just a philosophical approach
\end{itemize}

\section*{Historically}
\begin{itemize}
  \item Achievements of technology develop without using
    theoretical sciences
  \item They are those empirically either by accident or as a matter of
    trial and error and common experiences.
  \item Some inventions came from trial and error and
    revolutionized great ideas.
  \item Examples
    \begin{itemize}
      \item Pacemaker
      \item The wheel
      \item The lever
      \item The pulley
      \item The screw
    \end{itemize}
\end{itemize}

\subsection{Skolimowski}
Main arguments of Skolimowski:
\begin{enumerate}
  \item It is erreneous to consider technology as an applied science
  \item Technology is not a branch of science
  \item The difference between science and technology is
    that science is a theoretical discipline, while technology is a practical
    discipline.
\end{enumerate}


\begin{itemize}
  \item All techological pieces you see are combinations of
    science and technologies of past
  \item Many of these inventions are cross-disciplinary
    such as a film camera that requires
    chemistry, physics, and engineering.
\end{itemize}

\subsection{Methodology}
\begin{itemize}
  \item Basic methodological factors that account for
    growth of technology are different from those
    that account for growth in science.
  \item Ex: Is matter particles or waves?
    \begin{itemize}
      \item We can either use the definition that
        best benefits us.
      \item Engieners can treat it as particles
        meanwhile science can treat it like a wave
        since there is more complex theorems behind it.
    \end{itemize}
\end{itemize}

\subsection{What is technology?}
\begin{itemize}
  \item In technology we create based on design

    based upon the needs that we have
  \item Important difference between science and technology
    since science is just research to discover more
    about the natural world.

    Technology is revolutionizing based upon the problems of the
    current world.
  \item Science concerns itself with what \textbf{is}
    \begin{itemize}
      \item We didn't add anything new to the world,
        but now we know how black holes work.
    \end{itemize}
  \item Technology concerns itself with what is \textbf{to be}.
    \begin{itemize}
      \item We invented pens and pencils to help
        with our technical problems
    \end{itemize}
\end{itemize}

\section*{Technological Progress:}
\begin{itemize}
  \item Producing \textbf{NEW} objects
  \item producing \textbf{BETTER} objects of the same kind.
  \item The change of technology is that we make
    improvements to the objects that already exist.
    Along with adding new additions.
  \item Then there are improvements in making things smaller
    or more practical for other purposes
    and contexts
\end{itemize}

\section*{"betterness" and Praxeology}
\begin{itemize}
  \item Praxeology or praxiology analyzes action from the point of view
    of efficiency
  \item To make a tool more durable and/or more
    reliable
  \item There are various examples, from the first wheel
    to the latest wheels. Especially wheels
    in various applications from wooden wheels
    to race car, airplane, to bicicyle wheels.
  \item Or various ways that we made transportation
    faster, efficient, and more accessible (cheaper).
  \item Making tools more sensitive to measure even smaller
    changes of the unit they are measuring
  \item Along with tools that are faster at performing their
    function
  \item Example where the combination of all of these
    can be see are in computer technology
    \begin{itemize}
      \item We used to have 5MB hard drives that are forklift
        sized
      \item Now we have 512GB Flash Memory SD Drives that are
        a lot faster.
      \item They are also very accessible and cost a lot less
        than they did before
    \end{itemize}
  \item How to find problems in technology, in order to
    improve it!
  \item Make various designs or see what improvements were
    made in various fields of science and engineering
  \item There are improvements such as shortening
    the time of production, making more of the same product.
  \item Reducing the cost of the product and material cost
    if it gets more efficient.
  \item Ex: Henry Ford
    \begin{itemize}
      \item Henry Ford sold 2 million cars in 1924
      \item You don't just sell that many cars without a new
        infrastructure.
      \item You need infrastructure such as roads
        and people to sell fuel
      \item This demand brings renovations in the
        production of cars and the infrastructure
        that is needed to support it.
      \item Such as making law enforcement
        to make sure that people follow the rules
        of the road.
      \item People who product the glass
        get more efficient at making it to sell
        more to Ford (printing money for them).
      \item Then revolutions in traffic lights and
        ways to manage traffic
      \item Brings new field of engineering that
        is traffic engineering. Then designing
        cities to accompany cars
    \end{itemize}
  \item Technology now revolves faster, back then
    sending cars from Japan took 4 months so they had to be
    fast.
  \item Writing letters and sending them took a long time too

    The postal service was not very fast. Then
    you had to wait for them to response and for it
    to get back to them

    Now today, we have email, instant physical mail
  \item Now we have machines take care of many of our difficult
    tasks.
\end{itemize}

\subsection{Tech Progress:}
\begin{itemize}
  \item Creating new product
    \begin{itemize}
      \item Producing better objects of the same kind
    \end{itemize}
  \item Improves our ability to make an action
  \item Improves our accuracy in that field
  \item Our theory of electron was completely different from
    1890s, now we have a new theory.
    From particle to wave (part of string theory)
  \item Technology isn't harmful, but is when
    mixed with human nature.
  \item Humans create technology to use it
  \item Criteria of tech progress can not
    replace or can be meanningfully translated into the
    criteria of scientific progress
  \item Is there any attempt to find "better" or more
    "reliable" version of Newton's second law?
  \item Progress in bridge making vs. progress in Newtonian
    physics
  \item Newtonian mechanics vs. relativistic mechanics
  \item There are appliances that exists, that only fix
    some tools efficiently, but can't be used to fix other things.
    such as a tool great at fixing watches
    can't fix my sink
  \item In surveing technology arises an important topic,
    the accuracy and usefulness is the most important part.
    It doesn't matter whether or not it looks good,
    utility is the most important part.
  \item Civil engineering prioritizes durability,
    Mechanical Engineering prioritizes efficiency, etc
  \item We can discover new laws, but we can't
    really make changes to laws we already knew.
    They are set in stone until more information
    comes about it, but these are very well established
    tools.
  \item We are worries about tools very specific jobs,
    such as NASA when they were figuring out what
    type of wheel to use, or how to engineer it
\end{itemize}


\subsection{Science vs technology (conclusion)}

\section*{Science}
\begin{itemize}
  \item Deals with natural world
  \item Very concerned with what is {exists}
    in the natural world
  \item Concerned with processes that seek
    out of meaning of natural world by inquiry,
    discoverying, exploring, and using
    Scientific Method.
\end{itemize}

\section*{Technology}
\begin{itemize}
  \item Deals with how humans modify, change, alter,
    or control the natural world
  \item Is very concerned with what can or should be
    designed, made, or developed to satisfy human needs
  \item Ways to improve technique such as Mongolians,
    they need to find the best way to fight from
    horse back.
    Adopting the best technology
  \item Egyptions had to figure out pulleys
    to pull up heavy stones.
    Evolutions they learned from nearby
    civilizations, then they improved
    upon it and scaled it up.
  \item Cocerned with processes that we use to alter/change
    the natural world such as invention,
    innovation, practical problem solving, and design
  \item When we have a problem, we have a design process,
    what we gained from trial and error
  \item ex: We want to make faster communication,
    we can use maxwell equations to figure out our limitations
    and using these fundamental laws of physics to help us design
  \item Thinking of ways to create waves, then collect these waves
  \item Technology has a lot to do with debugging
  \item teaching innovation from history of technology
\end{itemize}

Next lecture is about Mario Bunge (1966)
This one was about Skolimowski. Along with paradimg changes in
science and technology.

\newpage

\section{Lecture 4}
Date: June 26, 2025

\subsection{Paradign changes in science and technology}
\begin{itemize}
  \item First we believed that the the universe was
    Earth centered
  \item After a paradigm change, we figured out that the
    universe was sun centered

    We eventually figured out the math and physics behind it (Newton's Universal Law of Gravity)
  \item Another paradigm change was discovering atomic particles,
    then applying that upon Maxwell's equations along with Neton's laws
  \item A revolution in physics that led to quantum mechanics
\end{itemize}

\begin{itemize}
  \item Humans used to have a relationship with animals
    to survive
  \item Nomatic tribes move with their animals and even
    move to get food for their animals
\end{itemize}

\subsection{Mario Bunge (1966)}
\begin{itemize}
  \item Technology is applied science
  \item Tecnology is about action, but an action heavily
    underpinned by theory-that
    is wyhat distinguishes technology from the arts and crafts and puts
    it on par with science.
  \item Skalimovski says that technology is not science or applied science,
    but rather a practical discipline that is concerned with what can or should be
    designed, made, or developed to satisfy human needs.
  \item Mario Bunge says that technology is applied science,
    and that it is about action, but an action heavily underpinned by theory.
  \item Bunge's view theory is based off the science whether
    you know about chemistry behind fermination of milk,
    you just need to know that it works.
  \item Bunge believes that technology is applying a technology
    based on trial and error from observation of natures.
    Applying the observed nature is technology.
  \item Such as not knowing the math behind wheels and
    how logs roll and they can be used as wheels.
  \item Or bending a branch of a tree and you release it,
    it exerts a force. Rather than knowing the vocabulary

\end{itemize}

\section*{Bungee's two types of technology}
Bunge distinguishes two theories in technology:
\begin{enumerate}
  \item Substantive Theories: Provide knowledge
    about the object of action
    \begin{itemize}
      \item You only need the necessary knowledge
        to make the action work.
      \item You don't need to know everything about the
        object of action, just enough to make it work.
      \item Classical mechanics and statics enter
        into the negineering of machinery
      \item Substantive theories are verifiable
        and scientific based theories.
    \end{itemize}
  \item Operative theories: Theories
    concerned with action iteself.
    \begin{itemize}
      \item Theories that are concerned with the action itself,
        such as how to make a wheel, or how to make a lever.
      \item How to make a lever, and how to use it.
      \item How to make a wheel, and how to use it.
      \item Concerned with how to use this device
      \item Operative theories are not verifiable
        and are not scientific based theories.
        They just solve your problems, end of story.
      \item As long as someone designs a circuit and it functions,
        it doesn't matter how they designed it.
        As long as it works, it is a good design.
      \item They follow different rules than substantive theories.
        They are focused on material resistance under different circumstances,
        product line design, machine design theory, etc.
%      \item The pilot must only know how to fly a plane,
%        they are do not really need to be aware of the functionality.
%        The pilot could not fix the jet engine, but must be able to assess it.
%        They focus on operative theories and the abstractions.
      \item Pilot primarily focus on operative theories,
        they probably do not know how to fix the jet engines.
        But, they are able to diagnose the problem or at
        least know where to look for the problem.

        They are focused on the abstractions and may have enough knowledge
        about how it works in the background, but their focus is operative theory.
      \item People use programs, programmers make the programs. People
        don't need to program the software to use it.

        Electrical Engineers design the hardware to let programmers
        program it. Proggrammers don't need to know
        how the hardware works to program it.

        But for both cases, it helps to know the lower levels of the stack
        to help debug and understand the system. Or maybe use
        be more efficient with the hardware usage.
      \item To make the glaze on a mug, you just need to know
        that it bakes at 3000F. Nothing more about
        material science or chemistry is needed.
      \item Operative theories are theories that are used
        especially when designing and building an interface
        that is easily accessible to the user.
      \item To make something, you must know material science,
        ways to make it and how to use it.
      \item To wire up a house, you must use substantive theories
        to know how to wire it up. Whether or not you need series or
        parallel circuits, or how to wire it up and in what order.

        Then use operative theories
        to know how to use the tools to wire it up to the given circuit.
      \item Using the circuit breaking is pure operative theory
    \end{itemize}
\end{enumerate}

\section*{Pottery}
\begin{itemize}
  \item Originated in multiple centers around the world
  \item \textbf{Pottery} formed key part of Neolithic Revolution
\end{itemize}

\subsection{Difference between Skalimovski and Bungee}
\begin{itemize}
  \item Both of them are philosophies to define science and technology
  \item Both ideas are different, but they just philosophical approaches
    to define science and technology.
\end{itemize}

\subsection{Back to Bungee}
\begin{itemize}
  \item Substantive theorioes of technology are indeedd
    lagely application of scientific theories.

    Since they are related to the subtance
  \item This is the part that comes from designing the engine
  \item Operative theories, in contrast, are not
    precedded by scientific theories, but rather
    born in applied research itself.
  \item  this is the part that comes
    from making an efficient assembly line to make
    the engine
  \item Not the part that is related to any research
  \item Technology is practice focused on the
    creation of artifacts and, of increasing importance,
    artifact-based services.
  \item The design process is structured process
    leading torward that goal, forms
    the core of practice of technology.
\end{itemize}

\section{Brief History of Science and Technology in Pre-Modern World}

\begin{itemize}
  \item The earliest forms of homosapeans, predates
    homosapeans.
  \item We always used tool to help us survive, but
    the human being was the first one to deliberately made
    tools to help us survive.
  \item Interesting that humans are the only species
    that can't survive wtihout tools.
  \item There is a mutual relationship between
    tool making and development of the brain.
  \item There is a saying that tool made the human being
    and hhuman being made the tool.
    Shows the mutual relationship between them.
  \item 150-100,000 years ago in Africa, one species,
    possibly more than one.
    They had capabilities to think and create tools.
  \item 70,000 years agom they had an explision of innovations.
    They had to make tools to survive and even made arts.
    They spread everywhere in the world except for the Poles.

\end{itemize}

\subsection{Pre-Historic Life:}
\begin{itemize}
  \item How was prehistoric lifestyle?
  \item Hunter-gatherers to survive.
  \item Hunting big animals and using their flesh for food,
    their skin for clothing, and their bones for tools.
  \item The number of these animals were decreasing and
    food became scarce. You can not make big communities.
  \item There were small communities of hunter-gatherers
    that were nomadic and moved around to find food.
    Along with being in competition with each other
    so there was violence.
    There are old skeletons that show signs of violence
  \item Eventually, they maded edible plants grown and raising captured animals (domestication).
  \item Mimicing nature to grow favorable trees, plants, and herbs that would be useful for them.
  \item Learning from nature how to collect seeds and plant them to mimic nature.
\end{itemize}

\subsection{Neolithic Revolution}
\begin{itemize}
  \item This was starting a culture and entering a new
    phase of life
  \item In different civilizations, it started in different times.
    In the Middle East, it started around 10,000 years ago.
  \item The neolithic revolition was the transition
    from nomadic hunter-gatherers to settled agricultural societies.

  \item A cultural revolution
  \item The change of lifestyle, economy, and society finally moving
    away from hunter-gatherers.
  \item There was a new culture where you had to settle
  \item You need shelters
  \item For agriculture, you need tools
  \item To plant seedsd, you must dig the dirt
  \item You need water, as in, a way to transport water
    or finding an area that is full of resources
  \item You can not eat barley and wheat like that, you must
    process it.
  \item Parallel to this revolution, came the domestication of animals.
    We went befriended wolves and they would always follow us.
    We then turned them into dogs.
    Some breeds of wolves were domesticated into dogs.
  \item A whole branch of technologies was then
    developed on the domestication of animals.
\end{itemize}

\section*{General overview of Neolithic Revolution}
\begin{itemize}
  \item End of the last Ice Age, around 12,000 years ago
  \item Result: A socioeconomic and technological revolution
  \item Shift from food-fathering to food-producing
  \item Domestication of wheat, rice, corn, and potatoes in different regions
  \item Domestication of animals
  \item We call this socieoconomic since it was not based upon a science,
    rather on surviving.
  \item They were learning from nature and mimicing nature
    to make their own food and grow it.
  \item Two Alternative paths to food production lef out of Paleolithic
    \begin{enumerate}
      \item From gathering to cereal horticulture (gardening),
        and then to plow agriculture
      \item From hunting to herding and pastoral nomadism
    \end{enumerate}
  \item Roles of Geography:
    \begin{itemize}
      \item Sufficient atmospheric or surface water:

        Horticulture and settled villages arose
      \item In grasslands too arid for farming:

        Nomadic people and herbs of animals retained a nomadic way of life
      \item Goes to show that people only evolved if they needed to,
        if it made surviving easier.
      \item Some had to revolutionize since they had it bad,
        meanwhile others never felt a need to revolutionize.
    \end{itemize}
  \item Nomaidic people were settled temporaly,
    They had to invent tents, tools, and ways to travel.
  \item These nomaidc people suich as Mongols and Bedouins
    had to invent ways to travel and survive in the desert or their perspective grasslands.
\end{itemize}

\subsection{Animal Domestication}

\begin{itemize}
  \item People started new technologies such as using domesticatedd cattle,
    goats, sheep, pigs, chickens, and later horses.
  \item They would use cattle for milk, meat, and leather.
  \item Then chickens for eggs and meat.
  \item Pigs for meat and leather.
  \item Learning to sheer the wool of sheep to
    use for people. Later made tools to turn this into yarn
    and make tools to make clothes.
  \item Animal droppings as fertilizer for crops
  \item They would use animal power for labor as a labor-saving technology.

    such as transporattion of goods or their force to grind wheat or barley
  \item Shifting the labor onto an animal with more strength then a person
\end{itemize}


\section*{Jericho}
\begin{itemize}
  \item Jericho had 9 towers
  \item Required labor in order to build these great
    buildings
\end{itemize}

\section*{Neolithic Revolution}
\begin{itemize}
  \item Techno-economy,
    occurred without aid or input of any independent "science"
  \item Pure revolution in technology with no advancements
    in natural, just trial and error
  \item \textbf{Neolithic Astronomy}, systemaitically observing the heavens,
    particularly the patterns of motion of the sun and moon and
    that they regularly create astronomically aligned monuments
    that served as seasonal calendars.
  \item People create astronomy over systematic observations
    of the heavens, particularly the patterns of motion of the sun and moon.
  \item The most important observation/discovery during this time was
    observations of the sun.

    They understood the sunrise location and sunset location perspective
    to the calendar month

    The rising point of the sun is not fixed. Then how the height of the sun is lower
    and you can observe that we have varying length days
  \item They discovered the planets since the stars were fixed in the sky,
    they kept their distance to each other, but the planets.

    There were 5 planets that moved in the sky
\end{itemize}

\subsection{Stonehenge}
\begin{itemize}
  \item It was built over many generations
  \item They dragged these stones from a quarry
    200 miles away
  \item Circle aligns with midsummer sunrise, midwinter sunset, and most southernly
    rising and notherly setting of the moon
  \item Estimated 30 million hours of labor
  \item The same pattern that is seen in other
    civilizatinos such as Mayans
\end{itemize}

\section*{Watching Lecture 4 again}
\begin{itemize}
  \item Paradign changes in science and technology
\end{itemize}

\end{document}
