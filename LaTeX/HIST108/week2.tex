\documentclass{article}
\title{HIST 108 Week 2}
\author{Danny Topete}
\date{June 30, 2025}

\begin{document}

\section*{Monday Lecture}
June 30, 2025

\begin{itemize}
  \item The exam is July 1st (tomorrow), at 12:45 after
    some lecture, then the last part of the class
    is the exam.
    It is a short exam
\end{itemize}

\section{Monuments}

\subsection{Stonehedge}

\begin{itemize}
  \item Ancient rocks that were in formations to see
    astrological events
\end{itemize}

\subsection{Easter Island}
\begin{itemize}
  \item Ancient rocks that were in formations to see
    astrological events
\end{itemize}


\subsection{How they were made}
\begin{itemize}
  \item The key transformation from Neolithin villages
    to civilization was the \textbf{surplus} of food.
  \item The surplus of food gave people the time to
    do other things, like build monuments and discover the sky.
  \item Agriculture became more intensive, the surplus increased
  \item Question: Who takes care of surplus? Who is in
    charge of distributing it?
  \item When there is surplus, that means everyone is not
    focused on survival, so that leaves time for other tasks.
  \item This permitted specialization of labor.
  \item Specialization of labor such as some people
    that were not involved in essential needs like food.
    Rather, they were involved in things like pottery,
    water management, etc
\end{itemize}

\section{Early Civilizations}
\begin{itemize}
  \item More people in villages and cities
  \item Construction of big monuments
  \item Invention of writing, mathematics, and calendars
  \item Development of astronomy and medicine
  \item Developments of religions, literatures, arts, philosophies,
    and other forms of culture
  \item Development of metal work
  \item Instead of drawing 10 cows to show someone has 10 cows,
    they discovered numbers, then drew the item that the person has.
    Using numbers with shapes (mathematics)
  \item Government had to take care of math,
    to collect surplus and figure out how to tax people
  \item Government also cared about taking care of organization
    of calendar. They needed this to be precise with solar movements.
\end{itemize}

\subsection{Mesopotamia}
\begin{itemize}
  \item Civilizations were based on irrigation agriculture
  \item \textbf{Review through this}
\end{itemize}

\subsection{Middle Elamite: \~1500 BCE}
\begin{itemize}
  \item In China and Mesoamerica they had the same trends
  \item Advancements in astronomy, mathematics, astronomy, and writing
  \item There was looms, domestication of animals, and
    metallurgy
  \item Presenting gifts to the king, there was an
    organization of labor and social ranks
  \item Sign of political authority
  \item Also made temples to worship gods
  \item Creation of artifacts, stone sculptures, and pottery
  \item Someone was clearly supporting them to make these
    seemly useless [to survival] things
  \item These artifact served a role for prestige
\end{itemize}

\subsection{Mesopotamian Ziggurats}
\begin{itemize}
  \item Grand temples that were made to worship gods
  \item There were sundials that they would use to
    tell the time of day
\end{itemize}

\subsection{Babynian Astronomy}
\begin{itemize}
  \item Cuform writing was put into tablets that would recor times of Venus
  \item They would worship the planet Venus and there was thousands tablets
    about Venus
  \item Babylonians were the first to discover that
    Venus was a planet, not a star
  \item They also knew the time an length of the day
  \item They made a solar and lunar calendar
  \item Great discoveries in mathematics
  \item They were all used for practical purposes they had
\end{itemize}

\subsection{Developmeny of Art and Prestige Technology}
\begin{itemize}
  \item They had artifact of Gold, they knew
    it was valuable and rare
  \item They also knew how to melt it down and make precise objects with
    wonderful designs and detail
  \item They had to find a source of heat and hammers, tools,
    and make these very precise tools
  \item These are objects that only kings and rulers
    would have. A peasant would not have these
  \item There were elephant scultpure that were full of jewels,
    they were able to polish and shape these objects
  \item There was this fine and prestige technology
  \item These items show developments in society
  \item They even had house plans by Sumerians 5000 years ago.
    They had plans to where to have the kitchen, bathroom, and rooms
  \item There were also Necklace beads that were silver and gold, they also
    had holes and string through them. Meaning they had small drill bits to make
    these holes. 2600-2500 BCE
  \item This shows a clear sign of surplus
  \item These societies were not self-sufficient, they were
    interdependent on each other
  \item In order to learn how to smelt Iron, they had to
    trade with those who knew how to
  \item To trade, they had to figure out the value of items,
    how much gold is the worth of a cow, etc

\end{itemize}

\subsection{Practice/survival technology}
\begin{itemize}
  \item Address problems of survival
  \item They were based on surplus, without surplus we would not havbe these advancements
\end{itemize}


\section{History of Technology}
\textbf{New Powerpoint}, not part of the test

\begin{itemize}
  \item Prehistoric: Paleolithic, Mesolithic, Neolithic
  \item We only really discover it through artifacts,
    since there is no written record
  \item History: Ancient, Medieval, modern, contemporary
  \item There is a written record, we can read
    about it, and we have artifacts
\end{itemize}

\subsection{How to survey history of technology}
\begin{itemize}
  \item We had to figure out how they maded their food
  \item Southeaster asia, their economy was based on fishing
  \item There are parts of agriculture you don't need to develop
    if you are self-sufficient in certain ways
  \item Agriculture is not easy, and some don't make it to that step,
    there are steps such as harvesting, processing, water management,
    wait for next seaosn, and repeat.
  \item When there is plenty of food, no need to change lifestyle
  \item There were revolutions that came from scarce of food
  \item Technology came to solve practical problems (given Skalomovski)
  \item Manufacturing industry
  \item Building industry,
    such as if they make everything out of clay or wood (depending on their region)
  \item Transportation and communication
  \item Military tech, this is always paid by the government,
    they are typically the wealthiest. They can
    force advancements to happen with big budgets.
  \item Medical tech, this was a survival tech,
    how much do these people know about medicine or the body.
    Or herbs nearby,
    or knowing how to fix a broken bone.

    The saying: Civiliztion started when human was able to cure a broken bone,
    since that goes to show that there is something wrong with the body,
    and knowing how to cure it. Outcome of hundreds of years of knowledge and application.
  \item Sociocultural consequences of tech changes

    We knew about the achievements of Babylonians and Egyptians,
    now the achievements of Greece.

    What happened to them? Why do we have all of these changes in technology?
    That by Islamic communities had the largest hospitals, universities, and libraries.
    They were very advanced in medicine, mathematics, and astronomy.

    Then Europeans create a different science and technology, these sociocultural
    parts that we need to study.
  \item Japan by 1860s, they had no major contributions to science and technology,
    closed their borders, and did not allow foreigners to enter.

    They had no idea what was happening in the greater world,
    they had no contributions to the revolutions that were going on in Europe
  \item Did these people know how to navigate with the sea, such
    as using the stars
\end{itemize}

\subsection{Social Involvement in technologicval advancements}
\begin{itemize}
  \item 3 requirements for social involvement in technological advancements:
  \item Social needs

   Irrigation systems,

  Are you able to build a dam?
  \item Social resources

    Are there people that know how to build a dam?
  \item Sympathetic social ethos,
    ethos: characteristics spirit of a culture, era, or community as manifested in beliefs and aspirations

    The first book being published was the Bible.
    You can see one of the first printed bibles

    You can print a standard book (one without errors at a low price).
    Before printing press, all books had errors. Then the errors
    would be the source of other errors. Then not everyone was able to afford and own a bible

    Printing press came with cheap and standardized books.

    Meanwhile the first written books in Islamic culture was until the 1700s since
    writing the quaran was a religious duty, and they did not want to cheat it with printing press.
    As it was sacred. Therefore there was a religious and cultural barrier (ethos)

    It requires ethos for a technology to be adopted and accepted into a society.
\end{itemize}

\subsection{Development of Babylonaiin Math and Astronomy}
\begin{itemize}
  \item First fruits of relationship between science and technology
  \item Improved abiolities to measure:
  \item Land
  \item Weight
  \item Time
  \item All practical techniques, essential to any complex society,
    and inconceivable without literacy and the beginning of scientific observation
\end{itemize}

\subsection{Metal Age}
\begin{itemize}
  \item When different societies discobered metal at different times
  \item First started with soft metals like gold and silver,
    then they discovered how to smelt copper, then bronze, then iron

    Pure Gold Nuggets being quite abundant meaning it was the first one
  \item Then later, harder metals like steel and iron
  \item Metals of antiquity
    \begin{enumerate}
      \item Gold 6000 BC
      \item Copper
      \item Silver
      \item Bronze (copper and tin)
      \item Iron
    \end{enumerate}
    \item Birth of alloys
\end{itemize}

\subsection{Transportation}
\begin{itemize}
  \item Development of sailing ships in Egypt,

    Using the power of wind to move ships of trade items
\end{itemize}

\subsection{Irrigation system}
\begin{itemize}
  \item Development of irrigation since agriculture came from rivers
  \item Need for a high degree of social control
  \item Justice in distribution of precise water
  \item Making canals and aqueducts and the use of water-raising devices such as shadoofs
  \item Requirement of geometry and the angle with the horizon in order to make
    water run without creating much errosion
\end{itemize}

\section{Urban Revolution}
\begin{itemize}
  \item Invention of the City
  \item Cities different from Neolithic villages in two principle ways
  \item Cities were center for everything, power, production, trader, military, religion, culture, etc
  \item Better agricultural skill >> growht in population
  \item Larger population >> need more products
  \item Need for more products >> more specialized craftsmen
\end{itemize}

\subsection{Urban Manufacturing}
\begin{itemize}
  \item Societies became more complex
  \item Manufactuiring concentrated in pottery, wines, oils, and cosmetics
  \item Trade with neuighboringh societies
  \item Potter's wheel was invented in Mesopotamia between 6000-4000 BC and
    revolutionized pottery production
  \item Ancient Egyptian mythology, deity Khnum was said to have formed the first humans on a potter's wheel
  \item They were able to product more durable pots and use them for storage
  \item Ground stones from neolithic period were used to grind grain by swaying
  \item Old Egyptian hieroglyphs show an early instance of domesticated animal (cows being milked)
\end{itemize}

\subsection{Hierarchy}
\begin{itemize}
  \item There were ranks in the society
  \item Went from Pharaoh, to government officials,
    soldies, scribes, merchants, artifasn, farmers, slaves
  \item The most important part are the countries that produce more power,
    slaves were the source of power, therefore they were so important
\end{itemize}

\section{Tuesday Lecture}
July 1st, 2025

\subsection{Hierarchy and Power}
\begin{itemize}
  \item The people who had power were at the top of the hierarchy
  \item They would decide the distribution of surplus
\end{itemize}

\begin{itemize}
  \item When big projects came, came the need of mass human power
  \item Human power would sometimes be in the form of slaves
    or be replaced by using animals
  \item Slaves were the main power for all the greatest empires, they
    would take care of civil sources, and build monuments
  \item They would take care of public works, roads,  irrigation systems, etc
\end{itemize}

\subsection{labor saving machines}
\begin{itemize}
  \item Lever: unknown origin, but Archimedes' law of the lever
  \item Archimedes' said if you are given a lever and a place to stand,
    you can move the world. (Given a long enough lever; no requirement of physics knowledge)
  \item Related devices to lever:
    \begin{enumerate}
      \item Inclined plane
      \item Wedge
      \item Screw
      \item Compound pulley
      \item Wheel and axle
    \end{enumerate}
  \item The silk road conneccted east to west and it did more than just trade,
    rather books, ideas, and religions were also traded
  \item Helped the east and west have the same technologies
  \item Different civilizations had the different lever related devices at different times,
    but they eventually all had them through trading ideas.
  \item Using levers to move large ships (paddles), then having
    slaves row the ships
  \item Screws and lever in olive oil and wine presses.

    Gear systems that let people basically use a stair master to keep these
    gears turning to press the olives and grapes. People only need to walk up stairs,
    an everyday activity, to power these presses.
  \item The Spring
    \begin{itemize}
      \item The bow and arrow is a spring
      \item The catapult is a spring
      \item The crossbow is a spring
      \item The windlass is a spring
      \end{itemize}
  \item There was obviously no Newtonian or hookeian physics
    to explain how these things worked, but they still worked.
    They just knew trial and error and knew which variable to modify
    to make it work better. (like a longer lever, or a stiffer spring)

  Then modifying the angle in order to get the best range. \textbf{Trial and error}
\item Windmills were widely used in Europe in medieval times,
  you had to know how to harvest the power from water and wind. This came to
  make human labor obsolete since water and wind mill were "free".
\item They would write textbooks about these different windmill/watermill machines
  and how to build them. This was a way to spread knowledge.
\item We know that there were between 10,000-12000 mills in England in 1300
\item These are great machines since they didn't use animal or human power,
  you need to feed animals, but wind and water are free. Making animals not completely "free"
\item Persia invented windmills since they had abundant wind but no water.
  Meanwhile England had abundant water.
\item Making use with the natural resources around you
\item European vertical windmill
\item Water vs. Wind, which one was more common, well, depends on geography and which
  ever was more abundant. But technology was built around what was available
\item Oblelist as a sun dial, this stone being 1168 tones
\item All of the neolithinc advancements were possible with surplus
\item The pyramids were not built by slaves, but by people who were paid in surplus
  to build them. It was a national project that everyone was involved in.
\item Ziggurats were built by exclusively by Mesopotamia and modern parts of Iraq and Persia
\end{itemize}

\newpage
\section*{Wednesday Lecture}

\section{A survey of Mathematical Science}
in Early civilizations - I

\begin{itemize}
  \item Neolithic revolution took over 10,000 years ago
    and took a while to have people's knowledge be made semantic.
  \item The technologies developed in the Neolithic
    revolution were not written down, but were passed down orally
    Or through artifacts.
  \item There was shunga bones that had notches in them
    for prime numbers since they did find them interesting.

    Then the same bones had counted the animals they had hunted.
  \item Then there was the addition of drawings and numbers. Such as
\end{itemize}

\subsection{By 3500 BC}
\begin{itemize}
  \item Wheel was invented
  \item Connection between seasons and some astronomical phenomena
    was discovered
  \item Metal smelting, pottery, tool making, employment of simple machines, writing, basic
    mathematics, and astronomy
  \item Did they have any theoretical knowledge?
    Not really, they didn't have the modern information of science
  \item They knew that there was an object moving around in the sky,
    whether that is the earth or the sun, but they could not measure the
    distance between them.
  \item They had no clue about plate tectonics, or the earth being a sphere.
    But they knew that it happened
  \item The same with discovering that heat rises and they knew they could
    probably do something based upon this observation.
  \item Presentism: Emplioyment of present-day iodeas
    and perspective in interpretations of the past (Whiggish history).
  \item It is one thing to know $how$ to do things, another to know $why$ they work.
  \item These early people only had to know that certain things worked,
    along with a technology perspective on how to take advantage of these natural phenomena.
  \item There was an early discovery of cancer, but back then there was only
    theories on how it would happen.

    Now we have deeper discovery as to where it originates and how it forms.
    Then try to solve this "how" problem by observing it.
  \item There is a difference between "how" questions and "why" questions.
    "How" questions are more practical, while "why" questions are more theoretical.
  \item You need enough information in order to answer the "why" question.
    You need to have a theory, and then you can answer the "why" question.
  \item Then when we figure out that radiation are also particles and
    why certain particles decay, then we can answer the "why" question.
\end{itemize}

\subsection{Prehistoric Culture}
\begin{itemize}
  \item Culture was oral
  \item The only archive is the human memory
  \item Information and facts are aggregative rather than analytic
  \item History would be written down by who ever survived to tell
    it and was able to spread it.
    This being by who ever had the most power and influence at the
    time.
  \item Thoughts are expresssed with relatively close reference to the \textbf{human life} world
    (reduction to the familiar)
  \item Lack of any conception of "Law of Nature"
    in preliterate tradition
  \item Lack of deterministic casual mechanisms to explain natural phenomena
  \item Projection of human or biological traits onto natural phenomena
  \item Personalization and inidividualization of natural phenomena
    (e.g. the sun is a god, the moon is a god, etc)
  \item When ever there are problems that happen when for example, the sun,
    we must praise them and talk to them to make them happy.
    Especially since the sun is the source of life, so it is
    appreciated and worshiped.
  \item We had to wait until the 17th century with newtonian physics
    to have a deterministic casual mechanism to explain natural phenomena.

    A very long time under we understood most basic physical laws.
\end{itemize}

\subsection{Egyptian Account of Creation}
\begin{itemize}
  \item They believed that Shu and Tefnut, air and moisture,
    gave birth to the Earth and Sky, Geb and Nut.
\end{itemize}

\subsection{Oral Traditions}
\begin{itemize}
  \item When it comes to these oral traditions, there
    is always forms of sky, earth, the underworld and deity.
    As invisible powers of the dead and spirits that can be controlled
    through magical rituals.
  \item Medical practices and healing arts within oral cultures are
    inseparable and indistinguishable from religion and magic.
  \item The 'wise woman' or the 'medicine man' was valued
    for their medical knowledge and knowledge of the supernatural.
  \item The eldest were always seen as the wisest, and they were the ones
    that would pass down knowledge to the next generation. Along with
    the ones that wouldd know all the knowledge that would be passed down.

    Given that this is knowledge that was passed down to them.
    They would be the living library of knowledge.
\end{itemize}

\subsection{Pictograms}
\begin{itemize}
  \item Around 3000BC, a system of word signs appeared in Egypt
    and Mesopotamia, which was the first form of writing.
  \item Syllabic systems appeared around 1500BC
  \item A fully alphabetic system appeared around 800BC (The phoenicians)
  \item Information >> Archives >> Analysis >> Criticism
\end{itemize}

\subsection{Writing}
\begin{itemize}
  \item Writing was a prereq for development of philosophy, science, and mathematics
  \item Development of philosophy, science, and mathematics
    was a function of efficiency of system of writing.
  \item Sophisticated concepts and abstractions can develop within
    an eficient language and writing system.
  \item Then some talk about how words create images in our minds
    that we completely understand. Like try describing
    "love" or "hate" without the use of a well developed language
    and writing system.

    Try doing that with sign language without establishing standards.
  \item Having an efficient type of writing was a prereq to understanding
    the abstract concepts
  \item It took long until the very first pieces of literature and
    story. Or even define a concept of a God
  \item Then with letting people create documents and write
    down their thoughts or taking notes

    Meanwhile people used to just keep their thoughts in their hearts
\end{itemize}

\section{Earliest roots of Western Science}
\begin{itemize}
  \item Earliest roots of Western Science were found in ancient Mesopotamia
  \item These ancient civilizations were found within the green crecent
    since that was a land that allowed people to settle down
    and develop agriculture.
  \item Development of astronomy, mathematics, Geometry, number system, and calendar
  \item The Greek's science and mathematics originated from Egypt and Mesopotamia
  \item Herodotus reports that Pythagoras traveled to Egypt and Babylon and learned Egyptian and Babyloanian mathematics
  \item Pyhtagoras of Samos (570-495 BC)
  \item Greeks didn't develop their own mathematics, but rather
    they took the mathematics from Egypt and Mesopotamia
    and they archived it and made it more systematic.
  \item Showing that history depends on who ever writes it down
    and spreads it the best.
\end{itemize}

\subsection{Egyptian Mathematics}
\begin{itemize}
  \item Around 3000BC Egyptians developed a number system that
    was decimal in character
  \item IT was very similar to roman numerals how there was
    symbols for 1, 10, 100, 1000, 10000, 100000, and 1000000

    that in combination would make the number.
  \item They also developed the system of a fraction,
    along with the first algebraic equations that
    find the unknown (solve for $x$)
  \item They had also invented symbols for fractions,
    but first they had to invent the concept of fractions.
  \item They had figured out the concept of dividing
    things into equal parts, such as dividing
  \item They had to discover geometry in order
    to figure out who owned what land since the lines on the
    sand would wash away with the Nile river flooding.
  \item In Egyptian arithmetic, addition and subtraction was easy, meanhwhile
    multiplication and division was more difficult.
  \item Egyptian geometrical knowledge developed to solve practical problems
    such as land measurement, construction, and astronomy.
  \item Calculation of circumference, SA, and volume of shapes,
    They had even figured out \pi{} to be 3.16 (quite close to 3.141)
\end{itemize}

\subsection{Egyptian Calendar}
\begin{itemize}
  \item Nut, goddes of sky supported Shu, god of air,
    who held up the sky, and Geb, god of earth.
  \item This would explain the day and night cycle, as well as the seasons.
  \item Lunar calendar was used for religious festivals and rituals
  \item Solar celendar was used for daily life and civil purposes
  \item Solar calendar contained 360 + 5 days per year
  \item one week was 10 days
  \item 3 weeks was one month
  \item 4 months was one season
  \item Three seasons and five holy dasy was one year
\end{itemize}

\subsection{Sexagisimal numbers System}
\begin{itemize}
  \item There was an interesting number
    system that required lots of math in order
    to even understand it.
  \item It would be based upon base 60, and it was
    used by the Babylonians.
\end{itemize}

\begin{itemize}
  \item Babylonian table YBC 7289
  \item Showing the sexagesimal number system
    and how they would write numbers in base 60.
  \item It would make it easy to calculate square root of 2 and
    it would be quite accurate.
  \item Babylonains did complex calculations such as
    square roots, cube roots, solving for $x$, and even solving by extension quadratic equations.
\end{itemize}

\subsection{Babylonian Astronomy}
\begin{itemize}
  \item Babylonins Development of Observational and Mathematical Astronomy,
    and were the father of this field
  \item They had a map and a coordinate system
    that would allow them to map the sky and the stars.
  \item They would map and track the path of the moon, planets, and sun.
  \item Their maps were quite accuarate
  \item Development of Horoscopic Astrology
  \item They knew exactly where Mars would be based upon the time of the year
    and day.
  \item They observed retrograde motion of planets,

    Mars would appear to move forward, then backward, then forward again.
  \item They would be able to map the loop Mars took, along when it would move forward and backward.
  \item This would be the birth of HHoroscopic Astrology,
    which is the basis of modern astrology.
  \item They would create lunar, solar, and lunar-solar calendars
    that would be used to predict the seasons and the time of the year.
  \item They analyzed lunar year and solar years.
    They discovered that 12 solar days is 354 days,
  \item Then solar year was 365 days, and they would
    have to add 11 days every 3 years to make up for the difference.
  \item Chinese and Jewish calendars were luinar-solar calendars
    that were based on the Babylonian calendar.

    Meanwhile the Islamic calendar was a lunar calendar.

    Making it hard to align both calendars together.
  \item Babylonains had distinguished 12 constellations
    that were used to predict the seasons and the time of the year.
\end{itemize}

\section*{Thursday Lecture}
July 3rd, 2025

Where previous lecture left off
\begin{itemize}
  \item Babylonian tablet would figure out the speed of the moon around the earth
  \item This would be additions about the calendar system
\end{itemize}

\subsection{Greek Science}
\begin{itemize}
  \item Ptolemy's Almagest (2nd century AD)
  \item This was a book that would be the basis of
    astronomy for the next 1500 years.
\end{itemize}

\subsection{Chinese Mathematics}
\begin{itemize}
  \item Now in another part of the world
  \item Comes to show Math developed everywhere,
    not just Babylon and Egypt
  \item Their approach also had a number system, they
    used these symbols for tax purposes
  \item Around 500AD, They were able to calculate \pi{} to 3.14024 (actual: 3.14159)
  \item 3rd century CE: Liu Hui's method for calculations
    would make area of circle given its diameter and circumference implicity takes \pi{}
  \item Chinese math influenced by Indian mathematics,
    that would be translated to Chinese
  \item 13C, Chinese were able to solve equations up to degree ten
  \item Math was developed based upon their needs, they
    had lots of people and had to keep track of them and their taxes.
  \item Lo Shu magic square around 100AD
    came from a practical problem

    Have multiple apples, put them in a box.
    This would be the birth of the Chinese
    Remainder Theorem
\end{itemize}

\subsection{Indian Mathematics}
\begin{itemize}
  \item This would be our he we got our modern number system
  \item It is a decimal place value system
  \item Along with the concept of zero along with a symbol for it.
    The symbol comes from Hindu when you are thinking of nothing.

    The babylonians were aware of it, just didn't have a symbol for it.
  \item 400CE, They were also able to figure out trigonometry
    and the sine,cosine, invernse sine, tangent, secant.
  \item 12th c. AD, Indian astronomers used trigonometry to calculate the
    distance between the earth and the moon and earth and the sun.
  \item Indian numerals probably arrived Baghdad in 773 AD with
    diplomatic mission from Sind to court of Caliph al-Mansur
  \item Leonardo Fibonacci (1170-1250) introduced Indian numerals to Europe
    in his book Liber Abaci (1202)
\end{itemize}

\section{Mayans}
\subsection{Mayan Mathematics}
\begin{itemize}
  \item Mayan and Mesoamerican cultures used vigesimal number system (base 20)
  \item They also had a concept of zero
\end{itemize}

\subsection{Mayan Calendar}
\begin{itemize}
  \item They had the most accurate calendar in the world
  \item Same goes with Syndodic period of Venus and Mars, along
    with the Solar year. Mayans were the most accurate

\end{itemize}

\begin{center}
\begin{tabular}{lccc}
\textbf{Standard} & \textbf{Modern} & \textbf{Maya} & \textbf{Ptolemy} \\
\hline
Lunar Month (days) & 29.53059 & 29.53086 & 29.53337 \\
Solar Year & 365.24219 & 365.242 & 365.24667
\end{tabular}
\end{center}

\section{New PPT: Ancient Egypt Pyramids Pt:1}
\begin{itemize}
  \item Egyptians were quite advanced in their engineering
\end{itemize}

\subsection{Technical Issues}
\begin{itemize}
  \item Probably, cutting was harder than carrying the stones
  \item Need to construct long-lasting buildings (construction)
  \item Sometimes they used big stones just for cladding
  \item Then the obvious of how smaller constructions
    have been ruined by now
\end{itemize}

\subsection{Dates of Pyramids}
\begin{center}
\begin{tabular}{lccc}
\textbf{Building} & \textbf{Year} \\
\hline
 Old Kingdom & 2649-2150 \\
 First Intermediate Period & 2150-2050 \\
 Middle Kingdom & 2030-1640 \\
\end{tabular}
\end{center}

\begin{itemize}
  \item Cutting stones into small pieces to make it easier to carry
  \item Calculating the volume of pyramid that had to be
    cut goes to show how much surface area they had to cut.
  \item The pyramid was 5.9 million ton
  \item They are also astronomically correct, they are aligned
    with the cardinal points of the compass.
  \item The axis of the earth has a wobble, so the pyramids
    were aligned with the cardinal points of the compass
    at the time they were built.
  \item Goes to show that the North used to be another different compared
    to our north of today
  \item The tombs were aligned north-south with an accuracy of 0.05 degrees
\end{itemize}

\subsection{Casing Stones}
\begin{itemize}
  \item 2,300,000 stones weighting from 2-30 tons each
    some weighing as much as 70 tons
  \item About 144,000 casing stones, all highly polished and flat to an accuracy of 0.5mm,
    weighing about 15 tons each with
    nearly perfect right angles
  \item These stones were quite previous, they were made of
    limestone and granite, and they were polished to a mirror finish.
\end{itemize}

\subsection{Pyramid Building}
\begin{itemize}
  \item Pyramid sstructures were not unique to Egypt,
    they were also seen in Mesoamerica and
    Ziggurats in Mesopotamia
  \item Ziggirats were made of clay bricks
  \item Mesoamerican pyramids were made of stone
    and they were used for religious purposes.
  \item The Egyptian pyramids were unique in several respects:
    \begin{itemize}
      \item They were made of limestone and granite
      \item They were built as tombs for the pharaohs
      \item They were aligned with the cardinal points of the compass
      \item They were very precise
      \item The four points meet all at one point int he sky,
        making them a true pyramid
      \item To make a pyramid this precise, you need to measure
        slopes of all faces as the structure rises, with utmost accuracy
      \end{itemize}
    \item When it comes to Egyptian pyramids, they were
      North Western bank of the Nile
\end{itemize}

\newpage
\textbf{Main questions when it comes to pyramids:}
\begin{itemize}
  \item Methods of construction
  \item Methods of delivering stonhes to higher courses
    \begin{itemize}
      \item Ramp theory
    \end{itemize}
  \item Methods of orientation correction
  \item Methods of placing the casing stones
  \item Logistics
    \begin{itemize}
      \item We know they made these pyramids in 10-12 years,
        meaning that there was a good level of organization
    \end{itemize}
  \item Quarrying
\end{itemize}

\section{Methods of Construction}

\subsection{Ramp Theory}
\begin{enumerate}
  \item Perpendicular Ramps
    \begin{itemize}
      \item A ramp was built on one side of the pyramid
      \item You need to make this ramp longer and longer as
        the pyramid gets higher and higher
      \item This method of construction could move
        right up to the top
    \end{itemize}
\end{enumerate}

\subsection{Herodotus Machines}
\begin{itemize}
  \item Second theory centers on herodotus Machines
  \item Until recently, Egyptian farmers used wooden, cranelike
  \item Not a very acceptable theory
\end{itemize}

\subsection{Aliens}
\begin{itemize}
  \item Not a very good theory
  \item People believe there were hidden technologies,
    but that is not true.
    It was just technology of lifting
  \item They had cutting and lifting technology,
    all that is required to make a scaled up project like this
  \item Aliens is a funny conclusion since
    space travel is very difficult
  \item It is even harder to find planets nearby stars, then even
    see that the star has humans in it
  \item You can't see the planets in other solar systems
    since the stars are that bright
  \item

\end{itemize}

\end{document}
