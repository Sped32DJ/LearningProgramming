\documentclass{article}
\title{HIST108: Week 3}

\author{Danny Topete}
\date{July 7, 2025}

\begin{document}
\maketitle

\section*{Monday lecture}
July 7, 2025

\section{Continuing from Egyptian Civilization}
\begin{itemize}
  \item There are problems and they must be solved with technology
  \item Frank Muller-Romer: Construction of pyramids took place in several phases,
    therefore we must look at the evolution of the pyramids
  \item The earlier pyramids were step pyramids, then a bent pyramid (failed attempt), and finally the true pyramid of Gyza
  \item The earlier methods were clearly fails and they learned from their mistakes
  \item You can also see the methods used in the construction of the pyramids
  \item Then the Meidum pyramid (collapsed pyramid) was built, which was a step pyramid
  \item He then proposed that pyaramids were made in three phases:
    \begin{enumerate}
      \item Laying out the base

        Requires many ramps to be made within the pyramid,
        very step like
      \item Control of the Form of the Pyramid

        Then fill in the slopes or remove the ramps
      \item Movement of the Stones

        Then moving the stones to the top of the pyramid through the ramps
    \end{enumerate}
  \item The bent pyramid was proof of this theory
  \item Alternative Theory: (fill this out)
  \item There was approaches like levers to lift up stones
\end{itemize}

\subsection{Internal Ramp Thoery}
\begin{itemize}
  \item Radical idea from Jean-Pierre Houdin (French architect)
    \begin{enumerate}
      \item A ramp was used to raise the blocks to the top
      \item The ramp was internal and still exists inside the pyramid
      \item The bottom third of the pyramid was built with an external ramp
      \item As the bottom of the pramid was build, a second ramp
        was being built, inside the pyramid, on which the blocks for the top two-thirds
        of the pyramid would be hauled up.
    \end{enumerate}
  \item AI Response below (NOTE: It did autofill his name):
    \begin{enumerate}
      \item He proposed that the pyramids were built from the inside out
      \item He proposed that the ramps were built inside the pyramid
      \item He proposed that the ramps were used to move the stones up
      \item He proposed that the ramps were used to move the stones up to the top of the pyramid
    \end{enumerate}
  \item Proposed there is an internal ramp that snakes around the pyramid
  \item The theory falls with the density of the pyramid at certain points
  \item Jean-Pierre was just a French architect, not an Egyptologist.

    They were all theories without visiting them. The findings came from
    books and other resources
  \item The main problem with the theory is that the internal ramp would have to be
    very steep, which would make it difficult to move the stones up. There
    would also be no lighting inside these tunnels.
\end{itemize}

\section*{ Egyptian Pyramids cont}
\begin{itemize}
  \item There was a long-lost branch of the Nile that was going along the pyramids
  \item Makes sense that the pyramids were built in that area
  \item There was DNA showing a link between Mesopotamia and Egypt
  \item A theory as to how they made Obelisks:
    \begin{itemize}
      \item They used a ramp to move the stones up
      \item They used a lever to lift the stones up
      \item They used a sled to move the stones
      \item They used water to lubricate the sled
      \item Then from the edge of the cliff, they would drop the obelisk down
        to the ground. Standing straight into a hole
    \end{itemize}
  \item Other drawings show that they were using scaffold to lift the stones up
\end{itemize}

\section*{Vatican Obelisk}
\begin{itemize}
  \item In 1658, Pope Sixtus V ordered the obelisk to be moved to the Vatican
  \item We have documentation of the move
\end{itemize}

\section{Ancient Asian Engineering}
\begin{itemize}
  \item Ancient Asian Architecture is rich and unique
  \item Unique
    \begin{itemize}
      \item Building types
      \item Statues
      \item Gardens and shrines
      \item Tower clocks
    \end{itemize}
  \item The Four Great Inventions: the compass, gunpowder, papermaking, and printing
\end{itemize}

\subsection{Chinese Astronomy}
\begin{itemize}
  \item They were able to observe a star explode (1400 BC)

    Earliest record of supernova explosion
  \item The heavier the star, the heavier the elements it produces
  \item Such as how our star can join protons to create nitrogen, carbon, and oxygen
  \item Some other stars can create heavier elements like iron, nickel, and cobalt.
    At the most, they can make Uranium (heaviest natural element)
  \item Lunar and Solar Calendars
  \item They made the first solar observatory in 2300-1900 BC
  \item The most history we know from science comes from Shang Dynasty (1600-1050 BC)
\end{itemize}

\subsection{Shang Dynasty 1600-1050 BC}
\begin{itemize}
  \item Tools/weapons/ religious objects
  \item IN their architecture, symmetry was very important
  \item Feng Shui: An ancient Chinese prpactice to harmonize people
    with their surrounding of home, office, and land
  \item Chinese architecture has many great example of symmetry such as the typical
    houses of the red chinese temples
  \item Chinese Gardens contain:
    \begin{itemize}
      \item Architecture, like building pavilions, bridges, and walls
      \item There are decorative rocks and rock garden
      \item Plants
      \item Trees, flowers, and water element like ponds
      \item Most chinese gardens are enclosed by wall and some have
        winding paths
      \item Chinese gardens you can't see the entire garden at once,
        small scenes are set up so that you can walk through the garden
      \item Chinese Garden The Hunting Library Pasadena, California
        \begin{itemize}
          \item The garden is divided into 5 sections
          \item Each section represents a different element of nature
          \item The garden is designed to be a place of contemplation and relaxation
          \item Every Chinese garden as some type of rock element
          \item Rocks are chosen based on their shape, texture, substance, and color
          \item Miniature mountains are common in Chinese gardens.
            Mountain have symbolic meaning: Stability, and belief of philosophy
          \item Rocks were believed the concentrated amount of natural energy
        \end{itemize}
    \end{itemize}
\end{itemize}

\subsection{Early Magnetism}
\begin{itemize}
  \item Qin Dynasty (221-206 BC) was the first to use a compass
  \item Discovery of Magnetism in loadstone
  \item They discovered magnetism before Greeks
  \item Discovery of simple compass
  \item Magnetic compass was used for a long time
\end{itemize}

\subsection{Great Wall of China 208BC - 1640 AD}
\begin{itemize}
  \item Series of fortifications that were built to protect the Chinese states
  \item Longest built structure in the world
  \item Stretches nearly 4000 miles
  \item Construction began in 208 BC (Qin Dynasty)
\end{itemize}

\subsection{Age of Asian Tech, AD 700-1100}
\begin{itemize}
  \item Science of Traditional China - Joseph Needham
  \item
\end{itemize}

\subsection{Great Silk Road}
\begin{itemize}
  \item Trade and Tech changes, there is mass trade
    network and people connect ideas
  \item Trade of luxury goods, Chine to the West
  \item Export of Chinese silk was stimulus of textile industry in the West
  \item Renaissance was stimulated by the trade of ideas
  \item Giving us more items like paper, printing, and gunpowder
  \item Then this caused people to mine more to afford these items
  \item The rush to mine caused the discovery of new technologies
\end{itemize}

\subsection{William Hardy McNeill}
\begin{itemize}
  \item Raipd expansion of the Iron industry in China
  \item Hebei and Henan provinces were main Iron or and Coal
  \item Wood or charcoal fuel the furnances was becoming scarce because of deforestation,
    and the expansion of the iron industry
    depended on an increase of use of coal and coke
  \item Uses to make really hot fires in order to smelt iron
  \item Coal needs to be charred due to the impurities, this is coke, it is almost pure
    carbon. Coal without impurities.
  \item Just like how charcoal is made of pure carbon, coke is made of pure carbon
\end{itemize}

\newpage
\section*{Tuesday Lecture:}
July 8, 2025

\begin{itemize}
  \item The test is on Thursday up until what we cover tomorrow
  \item There will be an announcement on what parts will be on the test
  \item Midterm on Thursday
\end{itemize}


\section*{Cont from last Lecture}
\begin{itemize}
  \item They used coke to smelt iron
  \item Coke was charred coal
  \item Charcoal is from carbonized (charred) wood
\end{itemize}

\subsection{Chinese Iron Industry}
\begin{itemize}
  \item Iron was very heavy, so transportation of iron
    was through rivers and canals
  \item They made canals through rivers to transport goods
  \item This would join two rivers together
  \item Making canals is a big engineering feat,
    you need the elevation to be the same
  \item \textbf{Chinese Hydralic Engineering:} They used pound locks and spillways,
    the same we have in Panama Canal
  \item Tax officials report the output of the iron industry

  \item 90,400 tons in 1064
  \item 125,000 tons in 1078
  \item Liao Threat (and Mongol warriors)
  \item Song China built up enormous army, exceeding 1 million soldiers
  \item Much iron went to manufacture military equipment
  \item Suits of armor
  \item Arrow-heads, 16 million iron arrow-heads per year for crossbow men
  \item Other uses:
  \item Buddhist temples, bells, statues, bridges
  \item Farmers: Ploughs and farm tools
  \item Government Interference:
  \item Introduction to a new variety of rice from Vietnam
  \item Champa rice > quick-growing rice
  \item Could be planted early, leaving time after it was harvested for a second crop
    to be grown and harvested in the same year
  \item It could grow on land where there waws inssufficient water for other types of rice
\end{itemize}

\subsection{Chinese Ships}
\begin{itemize}
  \item Chienese were very good at making ships
  \item They made the first multi-masted ships, similar
    to the one Columbus used
  \item Chinese on the seas, they always traveled across the coasts to south east Asia,
    to india, to middle east, and to Africa
  \item They traveled to Africa and described giraffes
  \item Water-powered mechanical clocks, 11th century tower clocks

    They would be in city centers. None survived,
    we are only aware of the drawings.
    They have technology state secrets. Making calendars
    and clocks were state secrets since they are related
    to government and taxation
  \item They lost this technology since they never wrote it down
\end{itemize}

\section{Ancient Asian Engineering: Japan}
\begin{itemize}
  \item Japan's enigneering was interesting since
    they were isolated
  \item They became a super power in science and technology by themselves
  \item Written history began in 6th century AD
  \item They were under heavy influence from China
  \item Traditional Japanese science, tech, and medicine are highly influenced by
    Chinese knowledge
  \item They had no universities, but the knowledge was passed down
    from family to family. Such as being a doctor or mathematician
\end{itemize}

\subsection{Japanese Architecture}
\begin{itemize}
  \item Architecture was based on geography
  \item They had to use bambu since it was abundant
  \item Screens and sliding doors
  \item Tatami: Flooring material traditionally made from rice straw
  \item Verandas: roofed, open-air galleries or porches
  \item Genkan: traditional Japanese entryway area for removing shoes for a house
  \item Relationship with nature
  \item This is why we don't have very old remaining structures (they were quite weak)
  \item They burn quick and were very susceptible to earthquakes
\end{itemize}

\subsection{Shinto}
\begin{itemize}
  \item The Japanese traditional religion
  \item Workship of kami (spirits associated with natural forces and human ancestors)
  \item Kami is inanimate things and influences from Buddhism and Confucianism
  \item Kami are sacred spirits or phenopmena that live everywhere, especially in nature
  \item Humans are able to live by receiving blessings from the kami
\end{itemize}

\subsection{Weapons in Japan}
\begin{itemize}
  \item Japanese Swords: symbol of the samurai class
  \item First use of Katana as a word described a long sword occurs as early
    as the Kamakura Period (1185-1333)
  \item The making of a Katana was a ritual
  \item They had to keep this furnace burning for 3 days
  \item They would sandwich different types of steel together
  \item From Hard steel, medium steel, and soft steel, being
    sandwiched together in layers.

    Today, Japanese knives are made the same way and also expensive
\end{itemize}

\subsection{Japanese Heredity}
\begin{itemize}
  \item No social role for scientist/researchers in Tokugawa
    period also called Edo period (1603-1867)
  \item There were no universities or higher education
  \item Scientific knowledge was passed down from family to family
\item Nishimura Tosato (astronomer) 1761: "Mathematics was a childish subject
  which people wishing seek fame by constructing impractical theories... will indulge"
\item Yamakawa Kenjiro (physicist) 1877: Unable to study mathematics at school
  because "mathematics was despised by samurai as something
  only the merchant should study... The same situation existed in every other region"
\item They closed themselves off from the world and technology stagnated
\item Japan made their first airplane in 1920, then made airplane carriers
  in 1922, and then made a jet engine in 1943
\item They were were very fast at catching up, and joined
  the Axis Powers
\end{itemize}

\section{A glance at African Science and Technology}
\begin{itemize}
  \item Europe's Exploration of Africa disrupted Africa's economy, social, and
    political structures
  \item Story of Africa has been systemcally denied for reasons of slavery and colonialism
  \item Hunting was always available and a part of the culture,
    meanwhile parts of Europe settled with being fishers
  \item Pro-colonial kingdoms of Africa had great trade networks and
    massive wealth
  \item 16th century, brought Colonialism and slave trade
\end{itemize}

\subsection{Kingdom of Kush}
\begin{itemize}
  \item Established in 1070 BC lasting until 350 AD
  \item Ancient Nubian empire reached its peak in the 2nd millennium BC
  \item They had metallurgy (had mirros), pottery, and weaving
\end{itemize}

\subsection{Mali Empire}
\begin{itemize}
  \item Flourished from 1230 to 1670 AD
  \item Very good miners, produced lots of gold
  \item Mansa Musa would make a pilgrimage to Mecca in 1324,
    and dished out so much gold that it created a massive
    10-year recession in all the cities he passed through
  \item Wealthy from trade in gold, salt, and slaves
  \item They had their own accounting system, number theory, etc
  \item Timbuktu was a center of learning and culture
\end{itemize}

\subsection{Kingdom of Zimbabwe}
\begin{itemize}
  \item Medieval kingdom
\end{itemize}

\subsection{Writing System of Africa}
\begin{itemize}
  \item Ancient Meroitic script of Nubia (Kush)
  \item Nsibidi, and old Nibuan used pictograms
  \item Metallurgy
\end{itemize}

\subsection{Metallurgy}
\begin{itemize}
  \item Between 1500-2000 years ago, AFricans living in modern Tanzania
  \item Produced carbon steel
  \item Had furnaces of greater heat than Europe
  \item They had charcoal (wood was abundant)
  \item They used a furnace,
    the base was made of termite mounds,
    surrounded by Red latoric earth
  \item Termite mound was excellent since it didn't absorb moisture
  \item They would have a way to blow air into the furnace to keep burning
  \item Based on their needs and trial and error, they would
    adjust the furnace height and diameter.
    To make them better than European furnaces
\end{itemize}

\subsection{Astronomy}
\begin{itemize}
  \item Began in Africa
  \item Namoratunga: Megalithic site in Kenya
  \item Alignment of 19 Basalt pillars
  \item Making this the oldest observatory in Africa;
    older than stonehenge
  \item 7000 year old stone circle tracked the summer solstice and arrival of annual monsoon season
  \item They were able to track the tilt of the moon, even though it was only 5 degrees
\end{itemize}


\subsection{Mathematics}
\begin{itemize}
  \item Counting and numeration systems
  \item Game and puzzles
  \item Geometry
  \item Graphs
  \item Record-keeping books
  \item Ishango bones are possibly the earliest form of mathematical activity

  They are dated to Upper Paleolithic, around 20,000 years ago.

  The lines on the Ishango bones are prime numbers.
  Some suggest that it represents a six-month lunar calendar
\item Berber numeral, it is vigesimal (base 20)

  West Africa used simple numerical notation among Berbers still use in the 19th century
\item Mathematics hidden in
  \begin{itemize}
    \item Architectural design
    \item Abstract patterms
    \item Geometrical patterns
  \end{itemize}
\end{itemize}

\subsection{Colonial Era}
\begin{itemize}
  \item Colonialism distorted pattern of economic development in many ways
  \item These languages and dialects are now extinct
\end{itemize}

\section{Central and South American}
\begin{itemize}
  \item We know people from Siberia moved to America
  \item This was during the last ice age
  \item Possibly highly developed civilizations from
    Mayans, incas, and aztecs
  \item They had their own astronomy, mathematics, and engineering
  \item They had very advanced irrigation systems
  \item They had many times where water was available, then scarce
  \item Water management system evolved a lot in these communities
\end{itemize}

\subsection{Mayan Civilization}
\begin{itemize}
  \item Had a great calendar
  \item Very good observers of the sky
  \item Astonomy was more accurate than the Europeans at the time
  \item They were not aware of metals, they knew gold and silver
  \item Invented a Technology Complex, they made
    pumps that would raise water

    These pumps would be used for plantation farming
  \item Arrival of Europeans, 1513
  \item Military invasion killed many through violence and brough diseases (smallpox and Measles)
  \item Mexico had 25-30 million people in 1500s and 1567, only 3 million remained
  \item They lost their confidence in their own culture and institutions
\end{itemize}

\subsection{Incan Empires}
\begin{itemize}
  \item They had suspeded bridges
  \item Paper from mulberry barks
  \item There are technologies from China, that could have traveled from Island to island
    until it reached south america
\end{itemize}

\subsection{Tikal}
\begin{itemize}
  \item There was a very interesting water management systems in this temple
  \item They had carved stones that would channel water
  \item They would collect rainwater into these stones
    and have many reservoirs,
    that would be connected by canals.
    These resevoirs would be connected to each other.
  \item They would use it for daily drinking water and plants
  \item They had a filter system
  \item Maya traveled many miles to collect sand for water filtration
    (sand that came many miles from the coast)
  \item Filtration boxes at entrances of some resevoirs
  \item These cities would be varying elevations to make the irrigation system work
  \item Tikal used quartz and zeolite to remove both heavy metals and biological contaminants
  \item Living in a tropical environment and lack of metals, they had to think about
    these things
  \item They had to be good at figuring out slopes and elevations of various parts
    of the city.
\end{itemize}

\section*{Wednesday Lecture}

\section{Greek}
\begin{itemize}
  \item Split between periods, Hellinistic and Classical
  \item Hellinistic science started the study of philosophy and mathematics
  \item People were debating about important questions about what is philosophy
  \item There were questions about what is the building block of the universe
  \item Speculation of nature and investigation
    of the fundamental nature of what exists, questions about
    the "essence" of things
  \item General principles beyond observations
  \item Makes people think about what is so special about the Earth and why they take a smooth
    path around the $Earth$
  \item They always asked for the building block of everything, how far you can break down
    everything to the smallest unit
\end{itemize}

\textbf{Pre-Socratic philosophy}
    \begin{itemize}
      \item  Thales - Water is source of all thinsg
      \item Anaximenes - Air is source of all things
      \item Anaximander
      \item etc,
      \item Many philosophers that thought about the source

        of all the things
      \item What is change - Heracleitus
      \item Empedocles, air, earth, fire, water, 4 fundamental elements
      \item Tiny uncutable particles - Democritus
      \item Developed Democritus' ideas - Epicurus
      \item \textbf{Pythagoras} - Numbesr were the ultimate reality

        Found mathematical relationship in musical notes (its just frequencies),
        how the length of the string, tension, and thickness of the string can help you
        mathematically predict what note it will play (or something like that)
      \item Interpreting natural phenomena from natural
        things without explanations of the supernatural
      \item Scientist using science to prove/disprove the existence of a God

       But scientist enter to use science to explain it, it is quite wrong
       since it is more empirical rather than philosophical. Along with
       the mathematical relationship.
    \end{itemize}

  \textbf{Aristotle 384-322BC}
  \begin{itemize}
    \item Established a comprehensive system of philosophy
    \item Established a coherent theory of cosomso wherein cqantities and processes
      were explained in terms of matter and form
    \item Wrote systematically about physics, metaphysics,
      poetry, zoology, music, logic, ethics, politics, rhetoric, government,
      and even theoter
    \item Aristotle's pholosophy and his views on physical science
      shaped medieval scholarship and philosophy into the 16th century
    \item When Aristotle and religions mixed, they were not able to put
    \item He categorized the entire universe and how heavenly bodies
      revolve around the Earth
    \item heavens don't fall or recede from us
    \item When we see the sun, we see a number of stars, the sky changes
      every day, and the moon changes every day
    \item The earth's rotational motion of a body
      is not natural; it is forced motion (quite accurate)
    \item Then how constollations are the same geometry all the time meanwhile
      the planets are always moving
    \item We are able to see how many stars and celestial figures are in the sky
    \item Then how these celestial bodies never change their shapes,
      the Moon just changes phases
    \item Artistotle's theory of the universe was geocentric, with the Earth at the center
      and the Sun, Moon, and planets revolving around it
    \item They believed everything around us is made of four elements:
      water (liquid), earth (solid), air (gas), fire (Greeks call this heat)
    \item You can not prove the motion of the Earth, it was
      impossible to prove until Newtonian physics
    \item Thinking about the nature of substances, everything in the
      sky is uniformly moving as a circle
    \item On earth, smoke goes up and objects fall down,
      so light objects rise meanwhile heavy objects fall down
    \item Two different regions, substances, concepts of motion,
      concepts of change, physical constitutions,
      physical behaviors
    \item Based off Aristotle, these thoughts eventually merged into
      religious thoughts. They would be the present of cosmology
      the way Aristotle thought about the universe.
      They would still believe that Earth is where God built his replica,
      and everything moves around his replica
    \item Aristotle's cosmos, Earth is at the center,
      Water is around is, then Air, then Fire, then at the ring outside of fire
      is the Moon
    \item Then past the celestial region is a sphere of fixed stars,
    \item When we prove that we are at the center, the universe must be finite
    \item Wasn't until 1928 that we postulated the Universe was expanding
  \end{itemize}

  \subsection{Basic Phisophical assumptions of Aristotle Cosmology}
  \begin{itemize}
    \item Four celestial elements and the Ether
    \item Aristotle was the first to philosophize the cosmos
    \item Universe has two regions, Terrestrial and Celestial
    \item Earth is at the center of the universe (by its nature)
    \item The only motion in celestial region is uniform circular motion
    \item This became the dominant cosmology until
      capernicus and Galileo
  \end{itemize}

  \section*{Epistemology / Logic of Aristotle}
  \begin{itemize}
    \item Syllogism: deductive scheme of formal argument
      consisting of a major and minor premise and a conclusion.
      It is formal and circular reasoning

    \item All men are mortal \textbf{(major premise)}, socrates is man \textbf{(minor premise)},
    socrates is mortal \textbf{(conclusion)}
  \item Comets don't only move uniformly, but they also change their
    brightness, and they are not stars, so they are not celestial bodies
  \item This reasoning helped us understand the
    way things were and why they behave as they did
  \end{itemize}

\section{Alexander Reign 336-323 BC - Birth of Hellenistic Period}
\begin{itemize}
  \item Forged the most extensive empire in the ancient world
  \item The begin of the hellenistic period
  \item After his death, it was split into four
    kingdoms, the Ptolemaic Kingdom of Egypt, the Seleucid Empire,
    the Kingdom of Pergamon, and the Antigonid dynasty in Macedon
\end{itemize}

\subsection{Hellenistic Egypt}
\begin{itemize}
  \item Roots of new scientific culture were planned
    in Egypt, governed by Greek Ruling class
    Ptolemic Egypt
  \item Ptolemaic Kingdom started with Ptolemy I Soter
    (one of Alexander's generals) and ended with Cleopatra and
    the Roman conquest in 30 BC
  \item They hired many astronomers, mathematicians, and philosophers
    to work in the Library of Alexandria.
    To generate knowledge and to make the library.
  \item Under Alexandria, they made catapults, then
    made a mathematician to calculate it to be more efficient
  \item This helped us find many basic laws of math and physics
\end{itemize}


\section*{Philosophers}
\begin{itemize}
  \item Archimedes, archmedes principle
  \item Apollonius of Perga, conic section
  \item Euclid, the book of elements of math
  \item Aratosthenes measured the circumference of the Earth and
    tilt of ecliptic
  \item Aristarchus proposed heliocentric model of cosmos
  \item Ptolemy: The Algemist, math bible of astronomy until 17th century
  \item Galen, foundations of medicine until 17th century
  \item Technological developments: cogged gears, pulleys, the screw,
    archimedes screw, water clocks, steam engines, and catapults
\end{itemize}

\section*{Galen physiology}
\begin{itemize}
  \item Galen's theory was based on ideas from Plato, Aristotle, and
    Hippocrates, and it was influential for many centuries
  \item Body's system, made from brain and nerves to heart and arties and the liver and veins
  \item There were 4 different types of fluids in the body,
    and the relationship between them
  \item Blood production, liver produces blood, carries it through the body
    through veins
  \item Four Elements of Human Phisiology, Water (Phlogm), Air (blood), Fire (yellow bile), Earth (black bile).
    Then in a square and the relationship with hot, wet, cold, dry
  \item When you get cold, the heat must bring hot material to bring it back
  \item They tried to cure people by bloodletting and breaking bones, etc.
    Then lots of trial and error to see what causes disease and what cures people
\end{itemize}

\subsection{Medival Astronomy / Celestial Anomalies}
\begin{itemize}
  \item Why are the lengths of seasons not equal?
  \item What causes the retrograde motion of some planets?
  \item We knew there were 2000 stars, but 7 of which not being outside
    our range and would have odd, circular, and retrograde motion
  \item They tried to theorize upon retrograde motion
  \item When the Earth is moving closer to the sun, it speeds up,
    then farther away it slows down.
    Then imagine if you believed if Earth was in the center of the Universe
    and you believed this. They also knew the sun was moving away and farther from the sun.
    The same goes with the planets.

    They had trouble figuring out this phenomenon given the Earth was at the center.
    They believed this was just nature doing its thing

\end{itemize}

\section*{Thursday lecture}
\begin{itemize}
  \item Const of last lecture
  \item They didn't care where the center was, but we knew summer was longer than winter
  \item For philosphers, knowing where the center of the universe was was quite
    important
  \item Plato's notion of Save the Phenomena
  \item Aristarchus of Samos - If the Earth rotates, that means we have very harsh storms.
    Meaning that if the Earth rotates that fast, then it would be impossible to live on Earth.
  \item stellar parallax: Size of universe in Ptolemaic system: 20,000 Earth radii.
    Since we don't see this parallax, the Earth must be at the center of the universe.
    (In reality, The angle of parallax is too small to be seen with the naked eye)
  \item Epicycle / Deferent Model of Ptolemy
    \begin{itemize}
      \item The planets move in small circles (epicycles) that move along larger circles (deferents)
      \item This was used to explain the retrograde motion of planets
      \item The Earth is at the center of the universe, and the planets move around it
      \item This model was used for over 1400 years until Copernicus proposed a heliocentric model
      \item There were two circles, one for general orbit, and the other of the orbit around the
        orbit line
    \end{itemize}
  \item Great discussions about where the real center of the Universe
    was until Copernicus proposed a heliocentric model and moved
    everything around the Sun
  \item Algamest was the most important book of astronomy
    until the 17th century, it was written by Ptolemy
\end{itemize}

\section*{Tuesday Lecture}
July 15, 2025
\begin{itemize}
  \item Quick recap
  \item Hellinistic period is where greek and non-greek ideas mix together
  \item There was new technology and secrecy
  \item There was a great focus on religion
  \item There was the rise of democracy and a philosophy to discussing these ideas
  \item There was new great ideas in religion
  \item Great pursuit for christianity, how to prove trinity,
    basic ideas of christianity, creation theory, etc
  \item Islam then appears in 610,
    they immediately make a big empire and expand from Southern Spain
    to China
\end{itemize}

\section{Islamic Revolution}

\begin{itemize}
  \item Islam appeared in 610
  \item They were able to conquer two main empires in East and West Mecca
  \item They expnaded their territory to borders of China
  \item They had to create new sciences with their folk medicine, astrology,
    these were people of trade and agriculture
  \item When they moved west, entered Egypt \rightarrow{} Morocco \rightarrow{} Spain, then eventually
    China. They became very wealthy
\end{itemize}

\subsection{Islamic Science}
\begin{itemize}
  \item Translatino Movement, 8th and 10th centuries
    they established the House of Wisdom in Baghdad
  \item They made this city and hired many bilingual people.
    Interesting enough, many were non-muslim. They were CHristian, Jewish, and Zoroastrian (Persian Empire)
  \item Almost all Greek and Hellenistic texts were translated into Arabic
    \begin{itemize}
      \item That being Aristotle, Plato, Euclid, Apollonius, Ptolemy, and Galen
      \item This required to invent new words for arabic. Having new terms
        for astronomical concepts
      \item Algamest was translated at least 5 times due to this issue
    \end{itemize}
  \item After a generation, there was the birth of new scientists
  \item Interestingly enough, we don't have the original of these works,
    but we have the translated pieces that still exists
\end{itemize}

\subsection{Patronage: Science and Technology}
\begin{itemize}
  \item Not only did they preserve science, but
    they added onto it
  \item Islamic fine (prestige) technology:
    \begin{itemize}
      \item Technology connected to gardens
      \item connefcted to astronomy
      \item autonomata
    \end{itemize}
  \item It is said that they paid translated the book's weight in gold
\end{itemize}

\subsection{Islamic Importance}
\begin{itemize}
  \item Islamic civilization had 3 advantages
  \item Connected East to West (in terms of ideas and part of the same civilization; same visa)
  \item Providing direct connect with far east
  \item Preserving Greek materials
\end{itemize}


\subsection{Islamic Architecture}
\begin{itemize}
\item Architecture was a mix of
  Hellenistic, Persian, and loca traditions
\item Impact of Religion, local traditions, geography
\item Expansion happened up to western parts of China that
  are still muslim to this day
\item In China, they learned to make paper
\item Before they were using bark, and more expensive,
  less convenient forms
\end{itemize}

\subsection{Impact of Religion}
\begin{itemize}
  \item All orientation of mosques and shrines would point toward Mecca
  \item Aniconism, no statues or paintings of their idols
  \item Geometric Motifs / Vegetal patterns
  \item Calligraphy
  \item Minerats (Towers)
  \item Fountains / Water \rightarrow{} they would need to wash their hands and
    feet before prayer
  \item Light \rightarrow{} they are a sign of God
\end{itemize}

\subsection{Orientation of mosques}
\begin{itemize}
  \item To find Mecca, it requires mathematics
  \item To find the exact location of Mecca and the angle
    to pray to that direction
  \item Then they must pray five times a day,
    this must be, these times must be calculated astronomically
  \item Aniconism, you can't have statues of paintings
    \rightarrow{} Geometric Motifs / Vegetal Patterns / Calligraphy
  \item There are great buildings that are decorated with a combination of geometric
    patterns

  \item Their mosques were designed to have different levels and colors of light.
  \item There is no evidence of fractals in their math, but their designs show them
  \item These domes have roman style, but muslim touch to it
  \item They learned from Chinese to make tiles, then added them to their
    buildings. Along with adding dyes to these tiles
  \item Materials used: Stone, rubble, baked/unbaked bricks,
    clay, timber, mortar and plaster, Tile
\end{itemize}

\subsection{Muhtasib}
\begin{itemize}
  \item Police the enforcement of Islamic Law
    in particular area
  \item Enforced the correct materials were used
\end{itemize}

\subsection{Roads and Bridges}
\begin{itemize}
  \item trade \rightarrow{} trade routes to gather great wealth
  \item warfare
  \item pilgrimage routes to Mecca \rightarrow{} once one has enough wealth, they must make the trip
  \item Connection to Silk Road \rightarrow{} part of trade routes and enter the global trade center
  \item State postal service (barid) \rightarrow{}
\end{itemize}

\subsection{Water Management}
\begin{itemize}
  \item River Crossing:
    Bridges that were great and still stand today
  \item Irrigation system: they had dry lands
  \item Canals, dams, qanats
  \item Irrigation/water consumption laws and preservation
  \item required great understanding of elevation to make this canal network
  \item The Kebar dam is the oldest arch dam, 26m tall, 55m wide, built around 1300AD
\end{itemize}

\subsection{Surveying}
\begin{itemize}
  \item They had to know lang and long of locations
  \item Especially when it comes to mathematically determining where Mecca is located
  \item Finding Lat is easy, long required a reliable chronometer
  \item you can use Atronomical observations/calculate lat and long
  \item They had clocks before, but pendulum were not good for moving ships
  \item They invented a method to estimate lat and long by observation, based upon
    a single astronomical event
  \item They were able to calculate exactly which time the eclipse would happen.
    Then exactly when the moon would enter the sun's shadow
  \item This was only possible knowing the distance of the moon, they then used this to calculate the distance
  \item They had many scientific projects to help with their accuracy
  \item Public works surveying, they had plumb line with a movable triangle,
    this plumb line's length can scan the elevation between two points
\end{itemize}

\subsection{Air Conditioning}
\begin{itemize}
  \item They did essentially a heat engine, just a passive one
  \item They had hor air come from the ground, go through a Qanat
    (underground water channel), go through a cooled basement,
    then it would escape through Wind Tower
  \item Exact thing can be seen in Zion National Park to show
    they use 70\% less energy than a regular air conditioner,
    including materials to build that infrastructure
\end{itemize}

\subsection{Calendar/ Observatories}
\begin{itemize}
\item  Lunar, but moves inside a solar calendar.
\item They knew that every 3 years, they had to add an extra month
\item Babylonians had to then invent a luni-solar calendar
\item This becomes more problematic during month of fasting,
  especially if it is rainy or cloudy. Then these lunar calendars
  must accurate. Then this is why astrology was so important.
\item Margha Observatory, Iran, (13c.),
  this is a prototype in Islamic lands and India from 13th to 17th centuries
\item There were places to make astronomical tools, smelt metal (for tools), with
  many employes from various islamic territories.

  This was active for 30 years to help with evolutions
\item Huge sextant of Samarkand Observatory
\item Ptolemy postulates that center of Epicycle moves on Deferent
\item Tusi invented a new method based upon that to show that
  there are two circles, one being half the size of the other,
  then the joining point of two circles, it actually moves on a linear line
\item Now there is a new theory to retrograde motion, supposed to explain
  why Mars gets farther and closer, along with moving in different directions
\item Tusi, Copernicus had many similaritis. They then realized that Capernicus was
  not using a Latin model of alphabet, rather than an Arabic model of alphabet.
  Signs show that he had connections
\item Copernicus also believed in the same things as Tusi
\end{itemize}

\subsection{Copernicus}
\begin{itemize}
  \item What we see in Copernicus,
    how he leaarned from his Islamic predecessors
  \item Tusi's couple
  \item Urdi's Lemma
  \item Ibn al-shatir's moon model
  \item Ibn al-shatir's Mercury model
  \item Did Copernicus invent these independently or did he borrow from
    the Islamic world?
\end{itemize}

\newpage

\section{Agricultural \& Military Revolutions In Europe 800s-1500s}
\subsection{New path of Science and Technological Activities}
\begin{itemize}
  \item If an alien were to come during this time, entire globe would have been
    all on the same track
  \item Now that there was more technology, there was a population increase,
    which required more production
  \item The need to produce more, came more technologies to solve practical problems
  \item This led to an agriculture revolution
  \item Main issues with agriculture, plowing was hard since the soil was hard in Europe,
    but there were no issues with water/irrigation
  \item Unique ecological conditions of Norther Europe led to technological innovations
  \item We knew romans had innovations to making plows, but they
    didn't have access to those technologies and were quite
    primitives
  \item The first great invention was Heavy Plow, a metallic cutter (iron),
    they would use animal force to pull it and soften up the soil
  \item This would require a good production in iron. Then having iron easily accessible requires
    a more efficient mining and production system.
  \item There is a butteryfly / domino effect of how more production of Iron can help all other fields
    of civilization

\end{itemize}


\section*{Hellenistic Science}
\begin{itemize}
  \item A period of mixture of Greek and non-Greek cultures with science
  \item Government playing a big role in the development of science
  \item Advancements in mathematics, astronomy, physics, and engineering (all branches of scinece and tech)
  \item Establishment of the Library of Alexandria, which became a center of learning and scholarship

    Many other research centers were established in the Hellenistic world
  \item Euclid's Element, book of geometry and mathematics
  \item Birth of very sophisticated mathematics, such as geometry, algebra, and trigonometry
  \item Ptolemy's Almagest, a comprehensive treatise on astronomy that remained influential for centuries
  \item Euclid's Axioms, built a strong structure of mathematics,
    prove all statements he proposed
\end{itemize}


\end{document}


