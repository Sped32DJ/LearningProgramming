\documentclass{article}
\title{HIST108: Week 3}

\author{Danny Topete}
\date{July 7, 2025}

\begin{document}
\maketitle

\section*{Monday lecture}
July 7, 2025

\section{Continuing from Egyptian Civilization}
\begin{itemize}
  \item There are problems and they must be solved with technology
  \item Frank Muller-Romer: Construction of pyramids took place in several phases,
    therefore we must look at the evolution of the pyramids
  \item The earlier pyramids were step pyramids, then a bent pyramid (failed attempt), and finally the true pyramid of Gyza
  \item The earlier methods were clearly fails and they learned from their mistakes
  \item You can also see the methods used in the construction of the pyramids
  \item Then the Meidum pyramid (collapsed pyramid) was built, which was a step pyramid
  \item He then proposed thgat pyaramids were made in three phases:
    \begin{enumerate}
      \item Laying out the base

        Requires many ramps to be made within the pyramid,
        very step like
      \item Control of the Form of the Pyramid

        Then fill in the slopes or remove the ramps
      \item Movement of the Stones

        Then moving the stones to the top of the pyramid through the ramps
    \end{enumerate}
  \item The bent pyramid was proof of this theory
  \item Alternative Theory: (fill this out)
  \item There was approaches like levers to lift up stones
\end{itemize}

\subsection{Internal Ramp Thoery}
\begin{itemize}
  \item Radical idea from Jean-Pierre Houdin (French architect)
    \begin{enumerate}
      \item A ramp was used to raise the blocks to the top
      \item The ramp was internal and still exists inside the pyramid
      \item The bottom third of the pyramid was built with an external ramp
      \item As the bottom of the pramid was build, a second ramp
        was being built, inside the pyramid, on which the blocks for the top two-thirds
        of the pyramid would be hauled up.
    \end{enumerate}
  \item AI Response below (NOTE: It did autofill his name):
    \begin{enumerate}
      \item He proposed that the pyramids were built from the inside out
      \item He proposed that the ramps were built inside the pyramid
      \item He proposed that the ramps were used to move the stones up
      \item He proposed that the ramps were used to move the stones up to the top of the pyramid
    \end{enumerate}
  \item Proposed there is an internal ramp that snakes around the pyramid
  \item The theory falls with the density of the pyramid at certain points
  \item Jean-Pierre was just a French architect, not an Egyptologist.

    They were all theories without visiting them. The findings came from
    books and other resources
  \item The main problem wit the theory is that the internal ramp would have to be
    very steep, which would make it difficult to move the stones up. There
    would also be no lighting inside these tunnels.
\end{itemize}

\section*{ Egyptian Pyramids cont}
\begin{itemize}
  \item There was a long-lost branch of the Nile that was going along the pyramids
  \item Makes sense that the pyramids were built in that area
  \item There was DNA showing a link between Mesopotamia and Egypt
  \item A theory as to how they made Obelisks:
    \begin{itemize}
      \item They used a ramp to move the stones up
      \item They used a lever to lift the stones up
      \item They used a sled to move the stones
      \item They used water to lubricate the sled
      \item Then from the edge of the cliff, they would drop the obelisk down
        to the ground. Standing straight into a hole
    \end{itemize}
  \item Other drawings show that they were using scaffold to lift the stones up
\end{itemize}

\section*{Vatican Obelisk}
\begin{itemize}
  \item In 1658, Pope Sixtus V ordered the obelisk to be moved to the Vatican
  \item We have documentation of the move
\end{itemize}

\section{Ancient Asian Engineering}
\begin{itemize}
  \item Ancient Asian Architecture is rich and unique
  \item Unique
    \begin{itemize}
      \item Building types
      \item Statues
      \item Gardens and shrines
      \item Tower clocks
    \end{itemize}
  \item The Four Great Inventions: the compass, gunpowder, papermaking, and printing
\end{itemize}

\subsection{Chinese Astronomy}
\begin{itemize}
  \item They were able to observe a star explode (1400 BC)

    Earliest record of supernova explosion
  \item The heavier the star, the heavier the elements it produces
  \item Such as how our star can join protons to create nitrogen, carbon, and oxygen
  \item Some other stars can create heavier elements like iron, nickel, and cobalt.
    At the most, they can make Uranium (heaviest natural element)
  \item Lunar and Solar Calendars
  \item They made the first solar observatory in 2300-1900 BC
  \item The most history we know from science comes from Shang Dynasty (1600-1050 BC)
\end{itemize}

\subsection{Shang Dynasty 1600-1050 BC}
\begin{itemize}
  \item Tools/weapons/ religious objects
  \item IN their architecture, symmetry was very important
  \item Feng Shui: An ancient Chinese prpactice to harmonize people
    with their surrounding of home, office, and land
  \item Chinese architecture has many great example of symmetry such as the typical
    houses of the red chinese temples
  \item Chinese Gardens contain:
    \begin{itemize}
      \item Arcitecture, like building pavilions, bridges, and walls
      \item There are decorative rocks and rock garden
      \item Plants
      \item Trees, flowers, and water element like ponds
      \item Most chinese gardens are enclosed by wall and some have
        winding paths
      \item Chinese gardens you can't see the entire garden at once,
        small scenes are set up so that you can walk through the garden
      \item Chinese Garden The Hunting Library Pasadena, California
        \begin{itemize}
          \item The garden is divided into 5 sections
          \item Each section represents a different element of nature
          \item The garden is designed to be a place of contemplation and relaxation
          \item Every Chinese garden as some type of rock element
          \item Rocks are chosen based on their shape, texture, substance, and color
          \item Miniature mountains are common in Chinese gardens.
            Mountain have symbolic meaning: Stability, and belief of philosophy
          \item Rocks were believed the concentrated amount of natural energy
        \end{itemize}
    \end{itemize}
\end{itemize}

\subsection{Early Magnetism}
\begin{itemize}
  \item Qin Dynasty (221-206 BC) was the first to use a compass
  \item Discovery of Magnetism in loadstone
  \item They discovered magnetism before Greeks
  \item Discovery of simple compass
  \item Magnetic compass was used for a long time
\end{itemize}

\subsection{Great Wall of China 208BC - 1640 AD}
\begin{itemize}
  \item Series of fortifications that were built to protect the Chinese states
  \item Longest buiilt structure in the world
  \item Stretches nearly 4000 miles
  \item Construction began in 208 BC (Qin Dynasty)
\end{itemize}

\subsection{Age of Asian Tech, AD 700-1100}
\begin{itemize}
  \item Science of Traditional China - Joseph Needham
  \item
\end{itemize}

\subsection{Great Silk Road}
\begin{itemize}
  \item Trade and Tech changes, there is mass trade
    network and people connect ideas
  \item Trade of luxury goods, Chine to the West
  \item Export of Chinese silk was stimulus of textile industry in the West
  \item Renaissance was stimulated by the trade of ideas
  \item Giving us more items like paper, printing, and gunpowder
  \item Then this caused people to mine more to afford these items
  \item The rush to mine caused the discovery of new technologies
\end{itemize}

\subsection{William Hardy McNeill}
\begin{itemize}
  \item Raipd expansion of the Iron industry in China
  \item Hebei and Henan provinces were main Iron ore and Coal
  \item Wood or charcoal fuel the furnances was becoming scarce because of deforestation,
    and the expansion of the iron industry
    depended on an increase of use of coal and coke
  \item Uses to make really hot fires in order to smelt iron
  \item Coal needs to be charred due to the impurities, this is coke, it is almost pure
    carbon. Coal without impurities.
  \item Just like how charcoal is made of pure carbon, coke is made of pure carbon
\end{itemize}

\newpage
\section*{Tuesday Lecture:}
July 8, 2025

\begin{itemize}
  \item The test is on Thursday up until what we cover tomorrow
  \item There will be an annoucement on what parts will be on the test
  \item Midterm on Thursday
\end{itemize}


\section*{Cont from last Lecture}
\begin{itemize}
  \item They used coke to smelt iron
  \item Coke was charred coal
  \item Charcoal is from carbonized (charred) wood
\end{itemize}

\subsection{Chinese Iron Industry}
\begin{itemize}
  \item Iron was very heavy, so transporation of iron
    was through rivers and canals
  \item They made canals through rivers to transport goods
  \item This would join two rivers together
  \item Making canals is a big engineering feat,
    you need the elevation to be the same
  \item \textbf{Chinese Hydralic Engineering:} They used pound locks and spillways,
    the same we have in Panama Canal
  \item Tax officials report the output of the iron industry

  \item 90,400 tons in 1064
  \item 125,000 tons in 1078
  \item Liao Threat (and Mongol warriors)
  \item Song China built up enourmous army, exceeding 1 million soldiers
  \item Much iron went to manufacture military equipment
  \item Suits of armor
  \item Arrow-heads, 16 million iron arrow-heads per year for crossbow men
  \item Other uses:
  \item Buddhist temples, bells, statues, bridges
  \item Farmers: Ploughs and farm tools
  \item Government Interference:
  \item Introduction to a new variety of rice from Vietnam
  \item Champa rice > quick-growing rice
  \item Could be planted early, leaving time after it was harvested for a second crop
    to be grown and harvested in the same year
  \item It could grow on land where there waws inssufficient water for other types of rice
\end{itemize}

\subsection{Chinese Ships}
\begin{itemize}
  \item Chienese were very good at making ships
  \item They made the first multi-masted ships, similar
    to the one Columbus used
  \item Chinese on the seas, they always traveled across the coasts to south east Asia,
    to india, to middle east, and to Africa
  \item They traveled to Africa and described giraffes
  \item Water-powered mechanical clocks, 11th century tower clocks

    They would be in city centers. None survived,
    we are only aware of the drawings.
    They have technology state secrets. Making calendars
    and clocks were state secrets since they are related
    to government and taxation
  \item They lost this technology since they never wrote it down
\end{itemize}

\section{Ancient Asian Engineering: Japan}
\begin{itemize}
  \item Japan's enigneering was interesting since
    they were isolated
  \item They became a super power in science and technology by themselves
  \item Written history began in 6th century AD
  \item They were under heavy influence from China
  \item Traditional Japanese science, tech, and medicine are highly influenced by
    Chinese knowledge
  \item They had no universities, but the knowledge was passed down
    from family to family. Such as being a doctor or mathmatician
\end{itemize}

\subsection{Japanese Architecture}
\begin{itemize}
  \item Architecture was based on geography
  \item They had to use bambu since it was abundant
  \item Screens and sliding doors
  \item Tatami: Flooring material traditionally made from rice straw
  \item Verandas: roofed, open-air galleries or porches
  \item Genkan: traditional Japanese entryway area for removing shoes for a house
  \item Relationship with nature
  \item This is why we don't have very old remaining structures (they were quite weak)
  \item They burn quick and were very susceptible to earthquakes
\end{itemize}

\subsection{Shinto}
\begin{itemize}
  \item The Japanese traditional religion
  \item Workship of kami (spirits associated with natural forces and human ancestors)
  \item Kami is inanimate things and influences from Buddhism and Confucianism
  \item Kami are sacred spirits or phenopmena that live everywhere, especially in nature
  \item Humans are able to live by receiving blessings from the kami
\end{itemize}

\subsection{Weapons in Japan}
\begin{itemize}
  \item Japanese Swords: symbol of the samurai class
  \item First use of Katana as a word described a long sword occurs as early
    as the Kamakura Period (1185-1333)
  \item The making of a Katana was a ritual
  \item They had to keep this furnace burning for 3 days
  \item They would sandwich different types of steel together
  \item From Hard steel, medium steel, and soft steel, being
    sanwiched together in layers.

    Today, Japanese knifes are made the same way and also expensive
\end{itemize}

\subsection{Japanese Heredity}
\begin{itemize}
  \item No social role for scientist/researchers in Tokugawa
    period also called Edo period (1603-1867)
  \item There were no universities or higher education
  \item Scientific knowledge was passed down from family to family
\item Nishimura Tosato (astronomer) 1761: "Mathematics was a childish subject
  which people wishing seek fame by constructing impractical theories... will indulge"
\item Yamakawa Kenjiro (physicist) 1877: Unable to study mathematics at school
  becuase "mathematics was despised by samurai as something
  only the merchant should study... The same situation existed in every other region"
\item They closed themselves off from the world and technology stagnated
\item Japan made their first airplane in 1920, then made airplane carriers
  in 1922, and then made a jet engine in 1943
\item They were were very fast at catching up, and joined
  the Axis Powers
\end{itemize}

\section{A glance at African Science and Technology}
\begin{itemize}
  \item Europe's Exploration of Africa disrupted Africa's economy, social, and
    political structures
  \item Story of Africa has been systemcally denied for reasons of slavery and colonialism
  \item Hunting was always available and a part of the culture,
    meanwhile parts of Europe settled with being fishers
  \item Pro-colonial kingdoms of Africa had great trade networks and
    massive wealth
  \item 16th century, brought Colonialism and slave trade
\end{itemize}

\subsection{Kingdom of Kush}
\begin{itemize}
  \item Established in 1070 BC lasting until 350 AD
  \item Ancient Nubian empire reached its peak in the 2nd millenium BC
  \item They had metalurgy (had mirros), pottery, and weaving
\end{itemize}

\subsection{Mali Empire}
\begin{itemize}
  \item Flourished from 1230 to 1670 AD
  \item Very good miners, produced lots of gold
  \item Mansa Musa would make a pilgrimage to Mecca in 1324,
    and dished out so much gold that it created a massive
    10-year recession in all the cities he passed through
  \item Wealthy from trade in gold, salt, and slaves
  \item They had their own accounting system, number theory, etc
  \item Timbuktu was a center of learning and culture
\end{itemize}

\subsection{Kingdom of Zimbabwe}
\begin{itemize}
  \item Medieval kingdom
\end{itemize}

\subsection{Writing System of Africa}
\begin{itemize}
  \item Ancient Meroitic script of Nubia (Kush)
  \item Nsibidi, and old Nibuan used pictograms
  \item Metallurgy
\end{itemize}

\subsection{Metallurgy}
\begin{itemize}
  \item Between 1500-2000 years ago, AFricans living in modern Tanzania
  \item Produced carbon steel
  \item Had furnaces of greater heat than Europe
  \item They had charcoal (wood was abundant)
  \item They used a furnace,
    the base was made of termite mounds,
    surrounded by Red latoric earth
  \item Termite mound was excellent since it didn't absorb moisture
  \item They would have a way to blow air into the furnace to keep burning
  \item Based on their needs and trial and error, they would
    adjust the furnace height and diameter.
    To make them better than European furnaces
\end{itemize}

\subsection{Astronomy}
\begin{itemize}
  \item Began in Africa
  \item Namoratunga: Megalithic site in Kenya
  \item Alignment of 19 Basalt pillars
  \item Making this the oldest observatory in Africa;
    older than stonehenge
  \item 7000 year old stone circle tracked the summer solstice and arrival of annual monsoon season
  \item They were able to track the tilt of the moon, even though it was only 5 degrees
\end{itemize}


\subsection{Mathematics}
\begin{itemize}
  \item Counting and numeration systems
  \item Game and puzzles
  \item Geometry
  \item Graphs
  \item Record-keeping books
  \item Ishango bones are possibly the earliest form of mathematical activity

  They are dated to Upper Paleolithic, around 20,000 years ago.

  The lines on the Ishango bones are prime numbers.
  Some suggest that it represents a six-month lunar calendar
\item Berber numeral, it is vigesimal (base 20)

  West Africa used simple numerical notation among Berbers still use in the 19th century
\item Mathematics hidden in
  \begin{itemize}
    \item Architectural design
    \item Abstract patterms
    \item Geometrical patterns
  \end{itemize}
\end{itemize}

\subsection{Colonial Era}
\begin{itemize}
  \item Colonialism distorted pattern of economic development in many ways
  \item These languages and dialects are now extinct
\end{itemize}

\section{Central and South American}
\begin{itemize}
  \item We know people from Siberia moved to America
  \item This was during the last ice age
  \item Possibly highly developed civilizations from
    Mayans, incas, and aztecs
  \item They had their own astronomy, mathematics, and engineering
  \item They had very advanced irrigation systems
  \item They had many times where water was available, then scarce
  \item Water management system evolved a lot in these communities
\end{itemize}

\subsection{Mayan Civilization}
\begin{itemize}
  \item Had a great calendar
  \item Very good observers of the sky
  \item Astonomy was more accurate than the Europeans at the time
  \item They were not aware of metals, they knew gold and silver
  \item Invented a Technology Complex, they made
    pumps that would raise water

    These pumps would be used for plantation farming
  \item Arrival of Europeans, 1513
  \item Military invasion killed many through violence and brough diseases (smallpox and Measles)
  \item Mexico had 25-30 million people in 1500s and 1567, only 3 million remained
  \item They lost their confidence in their own culture and institutions
\end{itemize}

\subsection{Incan Empires}
\begin{itemize}
  \item They had suspeded bridges
  \item Paper from mulberry barks
  \item There are technologies from China, that could have traveled from Island to island
    until it reached south america
\end{itemize}

\subsection{Tikal}
\begin{itemize}
  \item There was a very interesting water management systesm in this temple
  \item They had carved stones that would channel water
  \item They would collect rainwater into these stones
    and have many reservoirs,
    that would be connected by canals.
    These resevoirs would be connected to each other.
  \item They would use it for daily drinking water and plants
  \item They had a filter system
  \item Maya traveled many miles to collect sand for water filtration
    (sand that came many miles from the coast)
  \item Filtration boxes at entrances of some resevoirs
  \item These cities would be varying elevations to make the irrigation system work
  \item Tikal used quartz and zeolite to remove both heavy metals and biological contaminants
  \item Living in a tropical environment and lack of metals, they had to think about
    these things
  \item They had to be good at figuring out slopes and elevations of variuous parts
    of the city.
\end{itemize}

\end{document}


