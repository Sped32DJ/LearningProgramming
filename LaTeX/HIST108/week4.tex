\documentclass{article}
\title{HIST108 Week 4}
\author{Danny Topete}
\date{July 14, 2025}

\begin{document}
\maketitle

\section*{Monday Lecture}
July 14, 2025

\section{Cont of Ancient Greek}

\subsection{Hellenistic Period 300BC - 200AD}
\begin{itemize}
  \item Represents a fusion of Ancient Greek world
    with that of the Near East, Middle East, and Southwest Asia
    \begin{enumerate}
      \item Massive inter-penetration of Greek and non-Greek ideas
      \item Increasing specialization of sciences
      \item Development of new centers of research
      \item (Skipped slide)
    \end{enumerate}
  \item We really established the foundations of modern science
    \begin{itemize}
      \item Astronomy -> Ptolemy's Almagest
      \item Mathematics -> Euclid's Element
      \item Medicine
      \item Physics
    \end{itemize}
  \item These boooks had very complicated models that were used for the following centuries
  \item This period overlapped with Roman Empire
\end{itemize}

\section{Roman Empire}
\begin{itemize}
  \item Used Greek science extensively
  \item Were the best engineers of the ancient and classical world
  \item They didn't create great scientists or philosophers, they
    just worked upon the Greeks
  \item The only major development that took placew was in medicine
  \item Romans contributed to art, architecture, urban development, warfare, and medicine
  \item Medicine advances due to their wounded soldiers
\end{itemize}

\subsection{Technology}
\begin{itemize}
  \item The Roman Colloseum was built in 80AD
  \item Baths, these were public baths
  \item Indoor plumbing, they were lead pipes
  \item Pompeii Public Fountain
  \item Pont du Gard in France is a Roman Aqueduct built in 19BC
  \item Engineers were good at making big buildings and shaping stones
  \item They would be able to carve stone and use metal or brass to join the stones
  \item Using stones to travel water
  \item They were the best at making roads
  \item These roads still stand today since they had five layers,
    they were well built and took care of water and snow
\end{itemize}

\subsection{Roman Navy}
\begin{itemize}
  \item Became dominant in the Mediterranean
  \item They were good at travel
  \item Roman Concrete was invented in late 3rd century BC

    When the concrete absorbed moisture, it would harden
  \item Great buildings like the Pantheon and Colloseum in Rome
\end{itemize}

\subsection{Roman Science}
\begin{itemize}
  \item In general, Roman science is Greek Science
  \item Rome is hailed for abilities in architecture, engineering, law, urban management,  military,
    and ruling a complex, multiracial, multicultural empire.
  \item Science is almost wholly Greek
  \item They actually destroyed the Library of Alexandria

    Which wasn't brough back until later
\end{itemize}

\subsection{Decline and Fall}
\begin{itemize}
  \item Clear lack of social role for science
  \item Technology was self-sufficient
  \item There was no schools for engineering until 19th century
  \item There was no support for the sciences
  \item No employment for scientists
  \item Seperation of science and technology in antiquity
  \item Role of Religion
  \item Wars and decline of the Roman Empire
  \item Lack of intellectual curiosity,
    they actually used easier versions of Greek texts since
    they were easier to read and would lose meaning
  \item Then the Roman Empire divided into the Eastern and Western Roman Empire
    due to religious differences
  \item West's loss of ancient sciences occured in two stages:
    \begin{enumerate}
      \item Slow decline of quality and quantity of scienctuific works.

        Along with destruction of ancient knowledge centers.
      \item The Western Roman Empire fell in 476AD

      The Eastern Roman Empire (Byzantine Empire) fell in 1453AD
      \item Commentaries, encycloperidias, and general texts
    \end{enumerate}
  \item For about 500 years, nothing happened in the West
    \begin{itemize}
      \item The only thing that happened was the rise of Christianity
      \item The Church was the only institution that survived
      \item They were the only ones who could read and write
      \item They were the only ones who could preserve texts
      \item They were the only ones who could copy texts
      \item Barbarians would invade and destroy the libraries
    \end{itemize}
  \item Rise of Eastern Empire caused the fall
  \item \textbf{Read that last slide carefullly}
\end{itemize}

\section{Ancient and Medival War Technology}
\begin{itemize}
  \item There were many great innovations in war technology
  \item The great driving force for technology
  \item Technological progress, it always happens
  \item Needs / resources, driver for technology
  \item Bottlenecks / End points, limits of technology
  \item Innovations,
  \item Revolutions,
  \item The progress isn't very smooth,
    it is based upon a need and then fulfilling that.
    Then times where the need is not fully meet,
    or requires resources that are not available
  \item Limits were such as when we tried to explain electricity and magnetism
    with newtonian physics and as liquid
  \item The single lense Telescopes have a problem of chromatic aberration,
    which is the inability to focus all colors to the same point
  \item Then there was multiple lenses in telescopes
  \item There would be issues where you need really long telescopes
  \item Then limitations of how the image will ruin over time or
    due to the mass
  \item Espionage Cameras, they were tiny cameras that were used to spy.
    They were the smallest possible cameras that could be used for the time.
    They used the same current technology, such as film and lenses,
    they also understood that smaller leneses would introduce less light,
    meaning they need to compensate with a longer exposure time and higher ISO film.
  \item Then came advancements in solid state electronics,
    which allowed for smaller cameras and better quality images.
\end{itemize}

\subsection{War Technology}
\begin{itemize}
  \item The war technology would be backed by the state budget
  \item Project tile Technology, such as catapult.
    Romans were the first to use catapults
  \item Even books are architecture would have catapults
  \item There was a relationship between scientific knowledge
  \item Romans had many terms that related to war technology such as catapult
  \item When it came to bows, they were concerned with the following:
    \begin{itemize}
      \item The mount of energy that can be stored and what are
        the variables?
      \item Stiffness of the bow (force requiring to bent it)
      \item Length of draw
      \item Both factors were subject
        to human physical limitations
      \item How far can it shoot
      \item How accurate is it
      \item How fast can it shoot
      \item How much force does it have
    \end{itemize}
  \item To solve the elasticity of the bow, they
    used catapult physics in order to store
    this potential energy
  \item \textbf{Counterweight Trebuchet (13th AD)}
  \item A revolutionary invention
  \item The use of gravity and angular momentum
  \item It would use a heavy counterweight,
    this would be very cheap since it can be dirt or rocks
    that would be dropped to launch the projectile
  \item They were invented during the time that Gunpowder came to Europe,
    meaning that they did not have a long life span
  \item Although, they had lots of documentation
    and were used for a long time
\end{itemize}

\subsection{The Greek Fire}
\begin{itemize}
  \item Constantinople used Greek Fire to fight off the Arabs
  \item If Muslims were to pass the Bosphorus and its fortifications,
    it may have just changed the course of history
  \item Greek Fire: Napalm like substance that burned in water
  \item Henri Pirenne said ...
  \item Greek Fire:
    \begin{itemize}
      \item Burns in water
      \item always portrayed as a liquid
      \item shot from tubes or siphons
      \item Appeareance was smoky and loud
    \end{itemize}
  \item We knew that there was some preheating and preassurizing involved
  \item This was a state secret, so we don't know the exact formula
  \item Greek Fire was a weapon system and depended not on a single formula,
    rather an array of knowledge associated with several
    components of the system.
  \item EX: You are want to build a clock, and someone throws all the
    gears, springs, and other components at you. You require a lot of knowledge for this, pertaining
    to how to assemble it and the background knowledge of how clocks work.
\end{itemize}

\subsection{Papyrophillic}
\begin{itemize}
  \item Technology hates paper
  \item No one published the secrets of technology
    that they discovered
  \item Publication is the only mechanism of scientific progress
  \item Papyrophobic means
  \item Espionage technology was used to steal secrets
  \item Eventually came patent systems
    in Europe during thhe 16th century.
    To protect the inventors
  \item Leonardo da Vinci protected all his inventions
    by his unique reverse writting.
  \item Secrecy is more important in military technology
\end{itemize}

\subsection{Compartmentalization}
\begin{itemize}
  \item Science and technology of the atomic bomb were kept secret
    by compartmentalization
  \item Personnel who constructed and oeparted various equipment
    were not told the purpose of the equipment
  \item Coca-Cola technoique is kept secret by having it made only in specific places.
    Then the syrup gets shipped to the bottling plants.
    Only a handful of people know the recipe, and they never travel together.
    Only way to open the vault is to have all people with the keys present.
  \item The recipe of Coca-Cola makes more than 1.9 billion dollars a year
\end{itemize}

\subsection{Secrecy}
\begin{itemize}
  \item When you lose the people who knew the secrets,
    you lose the science
  \item Lost Technology: Stradivarious violins, damascus steel, chinese tower clocks, Greek Fire,
  \item Roman Concrete was rediscovered centuries later by 1710 by French Engineer.
\end{itemize}

\end{document}
