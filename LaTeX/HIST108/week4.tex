\documentclass{article}
\title{HIST108 Week 4}
\author{Danny Topete}
\date{July 14, 2025}

\begin{document}
\maketitle

\section*{Monday Lecture}
July 14, 2025

\section{Cont of Ancient Greek}

\subsection{Hellenistic Period 300BC - 200AD}
\begin{itemize}
  \item Represents a fusion of Ancient Greek world
    with that of the Near East, Middle East, and Southwest Asia
    \begin{enumerate}
      \item Massive inter-penetration of Greek and non-Greek ideas
      \item Increasing specialization of sciences
      \item Development of new centers of research
      \item (Skipped slide)
    \end{enumerate}
  \item We really established the foundations of modern science
    \begin{itemize}
      \item Astronomy -> Ptolemy's Almagest
      \item Mathematics -> Euclid's Element
      \item Medicine
      \item Physics
    \end{itemize}
  \item These boooks had very complicated models that were used for the following centuries
  \item This period overlapped with Roman Empire
\end{itemize}

\section{Roman Empire}
\begin{itemize}
  \item Used Greek science extensively
  \item Were the best engineers of the ancient and classical world
  \item They didn't create great scientists or philosophers, they
    just worked upon the Greeks
  \item The only major development that took placew was in medicine
  \item Romans contributed to art, architecture, urban development, warfare, and medicine
  \item Medicine advances due to their wounded soldiers
\end{itemize}

\subsection{Technology}
\begin{itemize}
  \item The Roman Colloseum was built in 80AD
  \item Baths, these were public baths
  \item Indoor plumbing, they were lead pipes
  \item Pompeii Public Fountain
  \item Pont du Gard in France is a Roman Aqueduct built in 19BC
  \item Engineers were good at making big buildings and shaping stones
  \item They would be able to carve stone and use metal or brass to join the stones
  \item Using stones to travel water
  \item They were the best at making roads
  \item These roads still stand today since they had five layers,
    they were well built and took care of water and snow
\end{itemize}

\subsection{Roman Navy}
\begin{itemize}
  \item Became dominant in the Mediterranean
  \item They were good at travel
  \item Roman Concrete was invented in late 3rd century BC

    When the concrete absorbed moisture, it would harden
  \item Great buildings like the Pantheon and Colloseum in Rome
\end{itemize}

\subsection{Roman Science}
\begin{itemize}
  \item In general, Roman science is Greek Science
  \item Rome is hailed for abilities in architecture, engineering, law, urban management,  military,
    and ruling a complex, multiracial, multicultural empire.
  \item Science is almost wholly Greek
  \item They actually destroyed the Library of Alexandria

    Which wasn't brough back until later
\end{itemize}

\subsection{Decline and Fall}
\begin{itemize}
  \item Clear lack of social role for science
  \item Technology was self-sufficient
  \item There was no schools for engineering until 19th century
  \item There was no support for the sciences
  \item No employment for scientists
  \item Seperation of science and technology in antiquity
  \item Role of Religion
  \item Wars and decline of the Roman Empire
  \item Lack of intellectual curiosity,
    they actually used easier versions of Greek texts since
    they were easier to read and would lose meaning
  \item Then the Roman Empire divided into the Eastern and Western Roman Empire
    due to religious differences
  \item West's loss of ancient sciences occured in two stages:
    \begin{enumerate}
      \item Slow decline of quality and quantity of scienctuific works.

        Along with destruction of ancient knowledge centers.
      \item The Western Roman Empire fell in 476AD

      The Eastern Roman Empire (Byzantine Empire) fell in 1453AD
      \item Commentaries, encycloperidias, and general texts
    \end{enumerate}
  \item For about 500 years, nothing happened in the West
    \begin{itemize}
      \item The only thing that happened was the rise of Christianity
      \item The Church was the only institution that survived
      \item They were the only ones who could read and write
      \item They were the only ones who could preserve texts
      \item They were the only ones who could copy texts
      \item Barbarians would invade and destroy the libraries
    \end{itemize}
  \item Rise of Eastern Empire caused the fall
  \item \textbf{Read that last slide carefullly}
\end{itemize}

\section{Ancient and Medival War Technology}
\begin{itemize}
  \item There were many great innovations in war technology
  \item The great driving force for technology
  \item Technological progress, it always happens
  \item Needs / resources, driver for technology
  \item Bottlenecks / End points, limits of technology
  \item Innovations,
  \item Revolutions,
  \item The progress isn't very smooth,
    it is based upon a need and then fulfilling that.
    Then times where the need is not fully meet,
    or requires resources that are not available
  \item Limits were such as when we tried to explain electricity and magnetism
    with newtonian physics and as liquid
  \item The single lense Telescopes have a problem of chromatic aberration,
    which is the inability to focus all colors to the same point
  \item Then there was multiple lenses in telescopes
  \item There would be issues where you need really long telescopes
  \item Then limitations of how the image will ruin over time or
    due to the mass
  \item Espionage Cameras, they were tiny cameras that were used to spy.
    They were the smallest possible cameras that could be used for the time.
    They used the same current technology, such as film and lenses,
    they also understood that smaller leneses would introduce less light,
    meaning they need to compensate with a longer exposure time and higher ISO film.
  \item Then came advancements in solid state electronics,
    which allowed for smaller cameras and better quality images.
\end{itemize}

\subsection{War Technology}
\begin{itemize}
  \item The war technology would be backed by the state budget
  \item Project tile Technology, such as catapult.
    Romans were the first to use catapults
  \item Even books are architecture would have catapults
  \item There was a relationship between scientific knowledge
  \item Romans had many terms that related to war technology such as catapult
  \item When it came to bows, they were concerned with the following:
    \begin{itemize}
      \item The mount of energy that can be stored and what are
        the variables?
      \item Stiffness of the bow (force requiring to bent it)
      \item Length of draw
      \item Both factors were subject
        to human physical limitations
      \item How far can it shoot
      \item How accurate is it
      \item How fast can it shoot
      \item How much force does it have
    \end{itemize}
  \item To solve the elasticity of the bow, they
    used catapult physics in order to store
    this potential energy
  \item \textbf{Counterweight Trebuchet (13th AD)}
  \item A revolutionary invention
  \item The use of gravity and angular momentum
  \item It would use a heavy counterweight,
    this would be very cheap since it can be dirt or rocks
    that would be dropped to launch the projectile
  \item They were invented during the time that Gunpowder came to Europe,
    meaning that they did not have a long life span
  \item Although, they had lots of documentation
    and were used for a long time
\end{itemize}

\subsection{The Greek Fire}
\begin{itemize}
  \item Constantinople used Greek Fire to fight off the Arabs
  \item If Muslims were to pass the Bosphorus and its fortifications,
    it may have just changed the course of history
  \item Greek Fire: Napalm like substance that burned in water
  \item Henri Pirenne said ...
  \item Greek Fire:
    \begin{itemize}
      \item Burns in water
      \item always portrayed as a liquid
      \item shot from tubes or siphons
      \item Appeareance was smoky and loud
    \end{itemize}
  \item We knew that there was some preheating and preassurizing involved
  \item This was a state secret, so we don't know the exact formula
  \item Greek Fire was a weapon system and depended not on a single formula,
    rather an array of knowledge associated with several
    components of the system.
  \item EX: You are want to build a clock, and someone throws all the
    gears, springs, and other components at you. You require a lot of knowledge for this, pertaining
    to how to assemble it and the background knowledge of how clocks work.
\end{itemize}

\subsection{Papyrophillic}
\begin{itemize}
  \item Technology hates paper
  \item No one published the secrets of technology
    that they discovered
  \item Publication is the only mechanism of scientific progress
  \item Papyrophobic means
  \item Espionage technology was used to steal secrets
  \item Eventually came patent systems
    in Europe during thhe 16th century.
    To protect the inventors
  \item Leonardo da Vinci protected all his inventions
    by his unique reverse writting.
  \item Secrecy is more important in military technology
\end{itemize}

\subsection{Compartmentalization}
\begin{itemize}
  \item Science and technology of the atomic bomb were kept secret
    by compartmentalization
  \item Personnel who constructed and oeparted various equipment
    were not told the purpose of the equipment
  \item Coca-Cola technoique is kept secret by having it made only in specific places.
    Then the syrup gets shipped to the bottling plants.
    Only a handful of people know the recipe, and they never travel together.
    Only way to open the vault is to have all people with the keys present.
  \item The recipe of Coca-Cola makes more than 1.9 billion dollars a year
\end{itemize}

\subsection{Secrecy}
\begin{itemize}
  \item When you lose the people who knew the secrets,
    you lose the science
  \item Lost Technology: Stradivarious violins, damascus steel, chinese tower clocks, Greek Fire,
  \item Roman Concrete was rediscovered centuries later by 1710 by French Engineer.
\end{itemize}

\section*{Tuesday Lecture}
July 15, 2025
\begin{itemize}
  \item Quick recap
  \item Hellinistic period is where greek and non-greek ideas mix together
  \item There was new technology and secrecy
  \item There was a great focus on religion
  \item There was the rise of democracy and a philosophy to discussing these ideas
  \item There was new great ideas in religion
  \item Great pursuit for christianity, how to prove trinity,
    basic ideas of christianity, creation theory, etc
  \item Islam then appears in 610,
    they immediately make a big empire and expand from Southern Spain
    to China
\end{itemize}

\section{Islamic Revolution}

\begin{itemize}
  \item Islam appeared in 610
  \item They were able to conquer two main empires in East and West Mecca
  \item They expnaded their territory to borders of China
  \item They had to create new sciences with their folk medicine, astrology,
    these were people of trade and agriculture
  \item When they moved west, entered Egypt \rightarrow{} Morocco \rightarrow{} Spain, then eventually
    China. They became very wealthy
\end{itemize}

\subsection{Islamic Science}
\begin{itemize}
  \item Translatino Movement, 8th and 10th centuries
    they established the House of Wisdom in Baghdad
  \item They made this city and hired many bilingual people.
    Interesting enough, many were non-muslim. They were CHristian, Jewish, and Zoroastrian (Persian Empire)
  \item Almost all Greek and Hellenistic texts were translated into Arabic
    \begin{itemize}
      \item That being Aristotle, Plato, Euclid, Apollonius, Ptolemy, and Galen
      \item This required to invent new words for arabic. Having new terms
        for astronomical concepts
      \item Algamest was translated at least 5 times due to this issue
    \end{itemize}
  \item After a generation, there was the birth of new scientists
  \item Interestingly enough, we don't have the original of these works,
    but we have the translated pieces that still exists
\end{itemize}

\subsection{Patronage: Science and Technology}
\begin{itemize}
  \item Not only did they preserve science, but
    they added onto it
  \item Islamic fine (prestige) technology:
    \begin{itemize}
      \item Technology connected to gardens
      \item connefcted to astronomy
      \item autonomata
    \end{itemize}
  \item It is said that they paid translated the book's weight in gold
\end{itemize}

\subsection{Islamic Importance}
\begin{itemize}
  \item Islamic civilization had 3 advantages
  \item Connected East to West (in terms of ideas and part of the same civilization; same visa)
  \item Providing direct connect with far east
  \item Preserving Greek materials
\end{itemize}


\subsection{Islamic Architecture}
\begin{itemize}
\item Architecture was a mix of
  Hellenistic, Persian, and loca traditions
\item Impact of Religion, local traditions, geography
\item Expansion happened up to western parts of China that
  are still muslim to this day
\item In China, they learned to make paper
\item Before they were using bark, and more expensive,
  less convenient forms
\end{itemize}

\subsection{Impact of Religion}
\begin{itemize}
  \item All orientation of mosques and shrines would point toward Mecca
  \item Aniconism, no statues or paintings of their idols
  \item Geometric Motifs / Vegetal patterns
  \item Calligraphy
  \item Minerats (Towers)
  \item Fountains / Water \rightarrow{} they would need to wash their hands and
    feet before prayer
  \item Light \rightarrow{} they are a sign of God
\end{itemize}

\subsection{Orientation of mosques}
\begin{itemize}
  \item To find Mecca, it requires mathematics
  \item To find the exact location of Mecca and the angle
    to pray to that direction
  \item Then they must pray five times a day,
    this must be, these times must be calculated astronomically
  \item Aniconism, you can't have statues of paintings
    \rightarrow{} Geometric Motifs / Vegetal Patterns / Calligraphy
  \item There are great buildings that are decorated with a combination of geometric
    patterns

  \item Their mosques were designed to have different levels and colors of light.
  \item There is no evidence of fractals in their math, but their designs show them
  \item These domes have roman style, but muslim touch to it
  \item They learned from Chinese to make tiles, then added them to their
    buildings. Along with adding dyes to these tiles
  \item Materials used: Stone, rubble, baked/unbaked bricks,
    clay, timber, mortar and plaster, Tile
\end{itemize}

\subsection{Muhtasib}
\begin{itemize}
  \item Police the enforcement of Islamic Law
    in particular area
  \item Enforced the correct materials were used
\end{itemize}

\subsection{Roads and Bridges}
\begin{itemize}
  \item trade \rightarrow{} trade routes to gather great wealth
  \item warfare
  \item pilgrimage routes to Mecca \rightarrow{} once one has enough wealth, they must make the trip
  \item Connection to Silk Road \rightarrow{} part of trade routes and enter the global trade center
  \item State postal service (barid) \rightarrow{}
\end{itemize}

\subsection{Water Management}
\begin{itemize}
  \item River Crossing:
    Bridges that were great and still stand today
  \item Irrigation system: they had dry lands
  \item Canals, dams, qanats
  \item Irrigation/water consumption laws and preservation
  \item required great understanding of elevation to make this canal network
  \item The Kebar dam is the oldest arch dam, 26m tall, 55m wide, built around 1300AD
\end{itemize}

\subsection{Surveying}
\begin{itemize}
  \item They had to know lang and long of locations
  \item Especially when it comes to mathematically determining where Mecca is located
  \item Finding Lat is easy, long required a reliable chronometer
  \item you can use Atronomical observations/calculate lat and long
  \item They had clocks before, but pendulum were not good for moving ships
  \item They invented a method to estimate lat and long by observation, based upon
    a single astronomical event
  \item They were able to calculate exactly which time the eclipse would happen.
    Then exactly when the moon would enter the sun's shadow
  \item This was only possible knowing the distance of the moon, they then used this to calculate the distance
  \item They had many scientific projects to help with their accuracy
  \item Public works surveying, they had plumb line with a movable triangle,
    this plumb line's length can scan the elevation between two points
\end{itemize}

\subsection{Air Conditioning}
\begin{itemize}
  \item They did essentially a heat engine, just a passive one
  \item They had hor air come from the ground, go through a Qanat
    (underground water channel), go through a cooled basement,
    then it would escape through Wind Tower
  \item Exact thing can be seen in Zion National Park to show
    they use 70\% less energy than a regular air conditioner,
    including materials to build that infrastructure
\end{itemize}

\subsection{Calendar/ Observatories}
\begin{itemize}
\item  Lunar, but moves inside a solar calendar.
\item They knew that every 3 years, they had to add an extra month
\item Babylonians had to then invent a luni-solar calendar
\item This becomes more problematic during month of fasting,
  especially if it is rainy or cloudy. Then these lunar calendars
  must accurate. Then this is why astrology was so important.
\item Margha Observatory, Iran, (13c.),
  this is a prototype in Islamic lands and India from 13th to 17th centuries
\item There were places to make astronomical tools, smelt metal (for tools), with
  many employes from various islamic territories.

  This was active for 30 years to help with evolutions
\item Huge sextant of Samarkand Observatory
\item Ptolemy postulates that center of Epicycle moves on Deferent
\item Tusi invented a new method based upon that to show that
  there are two circles, one being half the size of the other,
  then the joining point of two circles, it actually moves on a linear line
\item Now there is a new theory to retrograde motion, supposed to explain
  why Mars gets farther and closer, along with moving in different directions
\item Tusi, Copernicus had many similaritis. They then realized that Capernicus was
  not using a Latin model of alphabet, rather than an Arabic model of alphabet.
  Signs show that he had connections
\item Copernicus also believed in the same things as Tusi
\end{itemize}

\subsection{Copernicus}
\begin{itemize}
  \item What we see in Copernicus,
    how he leaarned from his Islamic predecessors
  \item Tusi's couple
  \item Urdi's Lemma
  \item Ibn al-shatir's moon model
  \item Ibn al-shatir's Mercury model
  \item Did Copernicus invent these independently or did he borrow from
    the Islamic world?
\end{itemize}

\newpage

\section{Agricultural \& Military Revolutions In Europe 800s-1500s}
\subsection{New path of Science and Technological Activities}
\begin{itemize}
  \item If an alien were to come during this time, entire globe would have been
    all on the same track
  \item Now that there was more technology, there was a population increase,
    which required more production
  \item The need to produce more, came more technologies to solve practical problems
  \item This led to an agriculture revolution
  \item Main issues with agriculture, plowing was hard since the soil was hard in Europe,
    but there were no issues with water/irrigation
  \item Unique ecological conditions of Norther Europe led to technological innovations
  \item We knew romans had innovations to making plows, but they
    didn't have access to those technologies and were quite
    primitives
  \item The first great invention was Heavy Plow, a metallic cutter (iron),
    they would use animal force to pull it and soften up the soil
  \item This would require a good production in iron. Then having iron easily accessible requires
    a more efficient mining and production system.
  \item There is a butteryfly / domino effect of how more production of Iron can help all other fields
    of civilization

\end{itemize}


\section*{Hellenistic Science}
\begin{itemize}
  \item A period of mixture of Greek and non-Greek cultures with science
  \item Government playing a big role in the development of science
  \item Advancements in mathematics, astronomy, physics, and engineering (all branches of scinece and tech)
  \item Establishment of the Library of Alexandria, which became a center of learning and scholarship

    Many other research centers were established in the Hellenistic world
  \item Euclid's Element, book of geometry and mathematics
  \item Birth of very sophisticated mathematics, such as geometry, algebra, and trigonometry
  \item Ptolemy's Almagest, a comprehensive treatise on astronomy that remained influential for centuries
  \item Euclid's Axioms, built a strong structure of mathematics,
    prove all statements he proposed
\end{itemize}


\section{Wednesday Lecture}
July 16, 2025

\begin{itemize}
  \item Uncrease in population put preassure on food production
  \item Requiring a revolution in agriculture
\end{itemize}

\subsection{Heavy Plow}
\begin{itemize}
  \item There is a domino effect required in order to make
    iron more accessible, such as the fuel required and advances in furnace and smelting
    technologies.
  \item More friction
  \item Needed more traction
  \item Often pulled by as many as eight oxen
  \item How they added more traction to their plow
  \item They moved from oxen to horse with a collar that transfered the weight more efficiently
  \item 3rd innovation, developed a three-field rotation system
  \item Each field would grow a different cop, one would be fallow and the other would be
    wheat or barley. This would keep the soil healthy.
\end{itemize}

\subsection{Consequences}
\begin{itemize}
  \item Social consequences, the deep plow
    made it possible to farm on new lands
  \item Northward shift of agriculture
  \item more food production
  \item There was more product \rightarrow{} more population
\end{itemize}

\subsection{Military}
\begin{itemize}
  \item technological innovation in military affairs
  \item role of military in European feudalism
  \item Military Revolution and Europe's eventual global dominance
  \item Feudalism Pyramid was great in Europe
  \item European could still not use the horsem but they eventually learned about stirrupts through
    Muslim world
  \item Prior, eight century warriors would fight by foot
  \item Stirrupts allowed horsemen to fight to fight more effectively
  \item Chinese invented stirrupt in 5$^{th}$ century CE,
    then eventually diffused to Europe through the Muslim world
  \item They had technology for the armored knight, armored horses and fighter
  \item Knights were expensive, they would fight farmers, collect surplus/tax, and protect the
    higher echelons of society

    They were very important to medieval society
  \item This would help with the feudal setting to take place
  \item No central government was required to manage agriculture
    economy that needed no hydralic infrastructure
  \item Manorial system was well adapted to the European ecologyh
  \item Militaryt Revolution in Europe shifted power from local feudal authorities
    to centralized kingdoms and nation-states.
  \item This made the single kings more powerful then would influence even bigger
    battles
  \item There was the introduction of gunpowder to Europe in 13th century and
    war would change forever
  \item Chinese would use gunpowder for fireworks and make noise. Meanwhile,
    Europeans would make gunpowder based projectiles.
  \item Due to the competition, European got the gunpowder technology and
    immediately improved it.
  \item They had many ideas on ways to improve Chinese and Muslim ideas.
  \item Great example of Skilivmoski and progress first thinking
  \item Gunpowder and early firearms originated in China,
    but Europe made large cannons in 1310-20
  \item Technology spread back to Middle East and Asia,
    in Islam 1330s and China by 1356
  \item Turkish guns deployed siege of Constantinople in 1453
  \item
\end{itemize}

\subsection{Gunpowder Revolution}
\begin{itemize}
  \item The revolution would lead to metal working and mining,
    then making these weapons
  \item Wars were now more deadly and would damage their economies even more than before.
    Along with wiping more of one's population
  \item Bronze cannons were common but after 1541 English mastered casting iron cannons
    under King Henry VIII
  \item Smaller cannons would be mobile on land and put on ships
  \item "gunpowder revolution" undermined military roles of feudal knight
    and feudal lord and replpaced them with enourmousely expensive gunpowder armies
  \item The architecture would need to change, such as castles and fortifications
    since they were not effective against cannons
\end{itemize}

\subsection{Military Budget}
\begin{itemize}
  \item Fortification were built in new ways
\end{itemize}

\subsection{Elements fo Military Revolution}
\begin{itemize}
  \item Reaplcing heavily armored cavalry with infantry
  \item Introduction of gunpowder weapons, most importantly, artillery
  \item Rise in the size of armies.
  \item This was birthing ground for the imperialism and colonialism
  \item They always had better armies that would win against
    everyone else, especially lesser developed nations.
  \item Europe becoming the most powerful political system
\end{itemize}

\subsection{New machines}
\begin{itemize}
  \item New machines and sources of power
    \rightarrow{} wind and water mills
  \item Domesday Book (1086) recorded 6,500 water mills in England
  \item England alone had more production, demand of raw materials,
    trade, and surplus
  \item Europe began to use wind and water power on an unprecedented scale
  \item This helped with developments
    of large cathedrals and buildings
  \item Labor saving machines would let larger scale of all fields of human activity
\end{itemize}

\section{Books}
\begin{itemize}
  \item Charlamegne (r. 768-814) promoted learning and literacy.

    "cathedrawl schools" to make priests literate
  \item Now that there is more surplus and economy, There is now a need for
    educated people. Along with doing tax, calendar, and calculating
    resources along with public infrastructure.
  \item Follow what we knew from Roman Period
  \item Early Middle Ages: Seven Liberal Arts
    \begin{itemize}
      \item Trivium: Grammar, Rhetoric, Logic
      \item Quadrivium: Arithmetic, Geometry, Music, Astronomy
    \end{itemize}
  \item Knowloedge of astronomy for astrological and calendrical purposes,
    especially for setting the date for Easter.
  \item Intellectual emphasis remained on theology and religious affairs rather than on science
  \item Almost no original scientific research took place
  \item Pope Sylvester II 958-1003) was a great scholar and scientist.
    He studied Arab and Greco Roman artihmetic, geometry, and astronomy.

    He reintroduced the abacus to europe and armillary sphere
  \item They transalated knowledge from arabic to latin since they had advancements
    the European world didn't yet have
\end{itemize}

\subsection{European University}
\begin{itemize}
  \item Weakly organized learning in early Middle Ages
  \item Eruption of universities in 12$^{th}$ century the rapid spread of
    higher education across Europe
  \item University of Bologna (1088) was the first university
  \item Paris (1096)
  \item Oxford (1220)
  \item Padua (1222)
  \item And about 80 more universities by 1400s (or 1500s, idk, but its a date)
  \item They got support from the vatican
  \item universities did not depennd on government or church to rule ciruriculum
  \item There required a standardization of higher education, curriculum, and licensing
  \item European universities and Natural Sciences, there was a great focus
    on the natural sciences. In other universities, they would focus on theology
  \item In European universities, students would pay to enroll, meanwhile
    in Bagdad, it was free and they were given a stipend
  \item There was a standadization of books required to read
  \item Medival universities were not primarily research nor science pursued
  \item By 1200 European recoeverd so much ancient science along with
    several centuries of scientific and philosophical development in the Islamic world
  \item Adleard of BNath, translated Euclid's Element (from Arabic)
  \item Garard of Cremona, traveled to Spain to locate copy of Ptolemy's Almagest
    and translated it into Latin.
  \item 12th century: Period of Translation of Greek and Arabic texts into Latin
  \item 13th century: Period of Assimilation and absortion of scientific knowledge
  \item Aristotle had lots of finding about everything and anything. Science to philosophy,
    there is an issue since he believed the Universe was eternal.
    It always existed and will never end. This was a problem for the church.

    Fundamental part of his philosophy was the idea of causality, then this would
    go against the three major religions
\end{itemize}

\subsection{Problems}
\begin{itemize}
  \item Problem of eternity of the universe \rightarrow{} contradicts creationism, base of all major religions
  \item Problem of the soul \rightarrow{} how does the soul interact with the body
  \item problem of cause and effect \rightarrow{} Removes the fact that a God is the
    one deciding our fates. A leaf can not fall down without the permission of God
  \item problem unmoved mover \rightarrow{} God is the first cause of everything
  \item These thinkers had to give approaches to ancient philosophy
  \item It is hard to dismiss Aristotle since he was so influential
\end{itemize}

\subsection{Thomas Aquinas (1224-1274)}
\begin{itemize}
  \item Used Aristotle's logic to prove Theology
  \item There was a mix of pholosophy and theology, that being scholastic philosophy
\end{itemize}

\subsection{Condemnation of 1277}
\begin{itemize}
  \item Bishop of Paris condemned 219 philosophical and theological propositions
  \item There were 219 errors from Aristotle that were against the church, you
    can follow Aristotle, but if you excluse the 219 errors
  \item Anyone who was to teach themk would be excommunicated
  \item In order to teach Aristotle, you would have the fill in the gap with
    new ideas that would abide by the Church
  \item This started a movement of cricitism of Aristotelian ideas to create new ones
  \item Jean Buridan (1300-1358) and Nicole Oresme (1320-82) was a critic of Aristotelian ideas, examined
    the Earth's daily rotation on its axis, then questioned the radial velocity of the universe
    if everything is moving around the Earth in 24 hours. He asked if there are any
    other ways to explain the day/night cycle.

    Oresme also suggested that he didn't believe it since it
    wasn't in the Bible
  \item First chancellor of Oxford argued active investigation of nature
    and on that account sometimes hailed as the father of experimental method in sciences
  \item Jean Buridan: Ideas of Projectiles since there were no vacuums back then.
    Then proposing how there is a vacuum against projectiles. Then questioning motion.

    Gave the first idea of inertia and momentum. The idea that for each motion,
    there requires a change in motion. Which would build Newton's first law of motion
  \item Nicole Oresme: Graphs, to present qualities and qualitive change and geometrically
\end{itemize}

\textbf{Prompts for Final will be posted Thursday}

\end{document}
