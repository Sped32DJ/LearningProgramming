\documentclass{article}
\title{HIST108 Week 5}
\author{Danny Topete}
\date{July 7, 2025}

\begin{document}
\maketitle

\section{Technology and Natural Sciences at the Threshold of the Scientific
Revolution}
\section*{Nautic and Navigation}

\subsection{Ottoman Empire}
\begin{itemize}
  \item A grand new Empire that spread norther Africa, Egypt, Anatolia, Greece to Polan, Balkans, Georgia,
    coast of Arabia,
    and the Causasus.
  \item Ibearian navigators were going based on the $Things worth discovering$
    on sea, distant lands, and at home
  \item Traditional way of oceanic navigation was esssentially an empirical art
  \item Contiguity of coasts and proximity of numerous islands
  \item Visibility of sea, land, and heavens
  \item Using empirical knowledge of seas, winds, tides, landmarks, position of sun and stars
  \item Arabic, Chinese, and European ships
  \item The chinese ships had multiple masts, square sails, and a stern-mounted rudder
  \item Arabic ships were 5x smaller than Chinese ships
  \item Revolution in Iberian shipnilding for multi masts, allowing larger ships to be built

    Can create more ships, more payload, and make them lighter without losing strength (good enforcement)
  \item 1485 cartographers and astronomers employed by Portuguese calculated simplified tables
    to find lat both north and south of equator.
  \item Astronomy quite influencial in navigation on life at sea
  \item Route to India was around the Cape, much more difficult than the route to the Americas

    (No Suez Canal))
  \item Vasco de Gama learned new techniques from Arab navigators, such as the use of the astrolabe.
    Learned from Ahmad Ibn-Majid, an Arab navigator who wrote a book on navigation. He piloted
    across the Indian Ocean to Calicut.

    On his return to Portugalo, Da Gama brough back several $kamals$, instrument
    used in Earth navigation
  \item Skalimovski principles are shown how there were innovations on this.
    The most advanced version of the kamal was Newton's Reflecting Quadrant and
    Hadley's Quadrant, which were used to measure the angle of the sun or stars above the horizon.
  \item This Octant had a accuracy of one mile, very bad, but quite accurate
    given it is just a piece of wood with a mirror and a string.
  \item The Octant would give you an angle, then using a table, you could
    find your latitude.
  \item Pedro Nunes (1502-57) chief cosmographer of King John II of Portugal,
    produced the most infleuncial cosmography of 16th century, the $Tratado da sphere$
  \item Described first time, loxodromic curve, vital math concepot to map projection
  \item Concept of Rhumb Line, a straight line on a map that represents a constant compass direction
    between two points on the surface of the Earth.
  \item Light of Navigation, Dutch sailing handbook, 1608, by Lucas Janszoon Waghenaer
    was the first to use rhumb lines in navigation charts.

    showing compass, hourglass, sea astrolabe, terrestrial and celestial globes, and a compass rose
  \item Cosmography was developing rapidly in Spain,
    Predo de Medina (1493-1567) was the first to use
  \item Compass gave magnetic north, but not true north
  \item 1600s, instruments and charts used Europeans more advanced than Asian navigators
  \item European navigators and nautics had borrowed from Chinese and Arab navigators previously
    about ship construction, navigation, and astronomy
  \item Nautical transmission in future would flow Europe to Asia
\end{itemize}

\newpage

\section{Mechanical clock and self-operating machines (Automata)}
\begin{itemize}
  \item Moving from simple tools to a complex system
    with many small parts that work together
  \item Japan apologized for a train departing 20 seconds late,
    or how a Phillip Patek watch is so precise and cost 1 muillion dollars
  \item Ancient and Medieval Science and Tehcnology
  \item Science was theoretical and philosophical, not practical. Answer $WHY$
  \item Seven liberal arts, Theology, logic, rheotirc, grammar, arithmetic, astronomy, geometry, and music (not medicine)
\end{itemize}

\subsection{Definition of Time}
\begin{itemize}
  \item Measure of length of year (or time between two astronomical events)
  \item Measure small intervals of time
  \item Behaior of celestial bodies is in many ways cyclical: same phenomena repeat themselves with a certain period
  \item Lunar and Solar calendars always existed
  \item You can't really split up years in days, or else you get
    365.2422 days, which causes discrepancies in calendar after some time
  \item Sidereal month, time Moon takes complete full revolution around Earth with respect to stars
  \item Synodic month, or lunar month, average period of Moon's revolution with respect
    to the line joining Sun
  \item 5th cen. BC: Meton calculated 19 tropical years correspond to
    235 synodic months. Called Metonic cycle
\end{itemize}

\subsection{Mechanism}
\begin{itemize}
  \item Antikythera Mechanism Goes to show the many gears and complexity of this mechanism
    making a precise calendar
  \item Stereographic pprojection, mapping projections of a sphere onto a plane,

    This is imagining youy are in the south pole and build up circles of latitude
  \item They were able to invent Astrolabe, originally used in Hellinistic, then
    developed in Islamic world, and then in Europe. Originally by Apollonius of Perga
  \item Complicated, automated astrolabes would come
\end{itemize}

\subsection{Rise of modern Europe from 11th century}
\begin{itemize}
  \item Revolution of agriculture
  \item Population growth
  \item  Developed mining
  \item  Appearance of wealthy courts, nobility, merchants, and knighthood
  \item They wanted mechaniocal clocks, automata, and other devices
  \item (painting) Mechanical universe, winged angels operating the revolving
    machine of teh sky, in a painting showing angels moving the clock.
  \item Innovations during Middle Ages, printing press, mechanical clocks, trebuchet, and automata
\end{itemize}
\subsection{Santa Maria Novella, 1306}
\begin{itemize}
  \item Every day we discover a new art,
  \item One of these new ideas was a clock
\end{itemize}

\subsection{Clock}
\begin{itemize}
  \item "The Clock, not the steam engine, is the key machine of modern industrial age"
    - Lewis Mumford, American historian and philosopher of technology
  \item Giovanni di Donda (1330, 1389)

    Clock developed in Italy, in Florence, by Giovanni di Donda
  \item Richard of Wallingford (1292-1336) was the first to use
    escapement mechanism, which allowed the clock to keep time.
    Mathmatician, made major contributions to astronomy, horology, and astrology.
  \item Escapement important invention for timekeeping, device that transfers energy
    to timekeeping element (the "impulse action") and allows number
  \item De Dondi's Astrarium, mimiced the position of the Sun, Moon, and planets
    in the sky, and was a complex astronomical clock. There were 107 gears
  \item There are still clocks today that have only one hand or just show the position of the sun
  \item Cities that adopted public mechanical clock early
  \item Major places had clocks, such as great civic centres, cathedrals, markets, and monasteries
\end{itemize}

\subsection{How to make a clock}
\begin{itemize}
  \item Scientific knowledge
  \item Design
  \item new materials
  \item new tools
  \item Miniaturization
  \item accuracy
  \item Brought a great outcome of self-operating machines
  \item Brought philosophy of how the unverise works like a clock, it is
    a machine. The creater is a clockmaker, and the universe is a clock.
    The creater is a perfect mathmatician and engineer, and the universe is a perfect machine.
  \item Goes to show how Newton says that using mathematics to find the laws of
    nature, the creator is a perfect mathmatician, and the universe is a perfect machine.
\end{itemize}

\subsection{Automata}
\begin{itemize}
  \item self-operating machines
  \item "acting of one's own will"
  \item They will always work the same way, and will always produce the same result
  \item Linked science, technology, and art
  \item Instroduced new studies in science
  \item Introduces new materials, new tools, and new techniques
  \item People have been writing about automata since Helenic times all the way to Leonardo da Vinci
    and Descartes (bodies of animals are no more than complex machines)
  \item Hero: The steam-powered automaton, the first steam engine, the aeolipile
  \item Temple doors, opened by a fire, which would heat the air inside a chamber,
    causing it to expand and push a piston, which would open the doors.
  \item Islamic period had an automated water distribution system
  \item People have always tried to make a perpetual motion machine (hilarious)
  \item
\end{itemize}
\subsection{Descartes}
\begin{itemize}
  \item Descartes Philosophy: Matter \& Motion

    Appeared during the discovery of new world, corpeniucus helio-centric model, and Galileo's
    physics and telescopic observations,
    Kepler's Discovery of elliptical orbits
    William Harvey's discovery of blood circulation

    The entire universe is a machine, and the creator is a perfect mathmatician and engineer.
    God gave the universe motion, and it will continue to move forever.

    There is Matter and Mind, that of the physical world and that of the spiritual world.

    Matter being the particles causally affects the mind
    The mind/thinking casually affects the body
  \item Descartes and Automata
    \begin{itemize}
      \item Mechanistic Philosophy
      \item Concept of CLockwork universe
      \item Difference between Humans and Animal is Materialism and Vitalism
    \end{itemize}
  \item Germany and France, many clockmakers made androids and automata that repoduced
    as closely as possible the movements of humans and animals
  \item German doctor and Philosophical mechanist, built an entire "artificial man"
    with all internal bodily functions: circulation, respitariom, digestion, and reproduction.
    All except the operations of a rational soul. Dissected live animals to study their movements (Descartes)
  \item First actual anfroid (1738): Jacques de Vaucanson, a French inventor and Flute Player,
    made a life-sized figure of a sheperd that played 12 songs on the pipe.

    Result of 5 year long labor
  \item He had to make flexible lips, moving tongues, soft fingers, and swelling lungs.
    Many things to mimic nature.

\end{itemize}
\end{document}

