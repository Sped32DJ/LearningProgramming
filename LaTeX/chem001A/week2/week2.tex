\documentclass{article}
\title{Chem 001A\\ Week 2}
\author{Danny Topete}


\begin{document}
\maketitle{}
\section{Monday Lecture}
Test
Go back and learn sig fig stuff...

\section{Wednesday Lecture}
\subsection{Beginning of Unit 2: Now more Chem stuff rather than Phys}
\begin{enumerate}
  \item Early ideas in atomic theory
  \item basic atomic structure and symbolism
  \item Avg atomic mass
  \item Representing chemical compounds
  \item Periodic table
  \item Ionic compounds
  \item Nomenclature
    \rightarrow{} Learn this on your own as HW, then
    do example problems in lecture. Less painful.
\end{enumerate}

\subsection{Atomic Theory}
\begin{itemize}
  \item Dalton's Atomic Theory (1805)
  \item Gave experiments that built the foundation of modern chemistry
  \item Mass of Oxygen and Chromium in two samples of Chromium Oxide
    \begin{itemize}
      \item Dalton's experiment where he can predict the mass of oxygen based
      \item THere are ratios in different colors of chromium oxide
      \item Solving for the unknown value
      \item Getting the ratio of red powder and how orange crystals has damn
        near 3x more oxygen for the same mass of chrmium between red powder and orange crystals
      \item Cathode Ray Tubes shoot particles (electrons), discovered electrons
      \item Shows that a positive coil, deflected electrons and negative coils attracted coils
    \end{itemize}
  \item Rutherford Gold Foil Experiment (1911)
    \begin{itemize}
      \item $\alpha{} $ particles, positively charged,
        was shot into a paper thin sheet of gold
      \item 99.99\% particles went straight through, and the rest
        were deflected (implying some were shot at protons)
      \item Proved it is mostly empty space inside the atom
      \item Helped us figure out the nucleus of the atom
    \end{itemize}
  \item Chadwick Beryllium Experiment
    \begin{itemize}
      \item Shot $\alpha{}$ particles through a Beryllium plate,
        then discovered neutrons
      \item Discovered an electron cloud
    \end{itemize}
  \item Doing quantum theory in unit 3 because atoms aren't really made of protons and neutrons with electron clouds
\end{itemize}

\section{Friday Lecture: Oct 10$^{th}$}
\begin{itemize}
  \item Do 70\% of the OLI unit completion in order to retake the Mastery Quiz
  \item We need the room in order to retake the quiz
\end{itemize}
\subsection{Basic Atomic Structure}
\begin{itemize}
  \item The blueberry in a football field represents the size of the nucleus.
    Explains why the alpha particles flew right through. It is mostly
    empty space, but is quite heavy.
  \item Mass of nucleus represents the weight of the atom. Electrons are not accounted for.
  \item Fundamental unit of charge, a constant, something something Couloumbs of charge
\end{itemize}

\subsection{Element differences to each other}
\begin{itemize}
  \item Atomic Number (Z) gives total number of protons. The value of Z represents the element according to periodic table.
  \item You can reference chemicals by their atomic symbols and chemical symbol (H, He, Li, Be...)

    We typically attach atomic number to the chemical symbol
  \item Mass number denoted by superscript on the left.

    The mass number  differences represents an \textbf{Isotope}
  \item \textbf{Isotope}: two (or more) frms of an element that contains equal numbers of protons,
    but different \# of neutrons
  \item Ex: Z = 24

  \item \textbf{Ions}: Charged atoms

    $${}_{2}^{4}HE^{2+}$$
    Here, Helium has a charge of 2. Meaning we lost both electrons. (Funny enough, this is Rutherford's alpha particle)

    $${}_{12}^{24}Mg^{2+}$$
    Here, Mg has 12-2 = 10 electrons
  \item \textbf{Average Atoomic Mass (amu)}
  \item Chlorine consists of 75.77\% Chlorine-35 (34.969 amu) and 24.23\% Chlorine-37 atoms (mass 36.966 amu)

    Atomic mass = 0.7577(34.969 amu) + 0.2423 (36.966 amu) = 35.45 amu

    (This is just the weighted average)
  \item Atomic Mass (average) = $\sum_{n}$ mass$_{n}$ \times{}  Abundance\%$_{n}$
  \item \textbf{Fractional Abundance}:

    Fraction ${}_{1}^{1}H $ = X

    Fraction ${}_{1}^{2}H $ = 1 - X

    Fraction ${}_{1}^{3}H $ $\simeq{}$ X
  \item Understand where the metals are in the periodic table.
    The metalloids, nonmetals, etc.
    Understand which are solid, liquid, or gas in room temperature.

    Remember the alkaline metals, alkaline earth metals, transition meteals, Halogens, and Noble Gases

    Unit 3 explains why the periodic table is arranged as is.
  \item Then study nomenclature over the weekend


\end{itemize}



\end{document}
