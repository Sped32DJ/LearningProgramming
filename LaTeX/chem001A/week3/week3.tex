\documentclass{article}
\usepackage{soul}
\title{Chem 001A Week 3}
\author{Danny Topete}
\date{\today}

\begin{document}
\maketitle

\section{Monday Lecture, 10/13}
\subsection{Sorting Element}
\begin{itemize}
  \item Periodic table sorts elements by metals, nonmetals, metalloids

    (relevant ones for now)
  \item \textbf{Metals:} Has metallic properties, Malluable, good condctors of heat and electricity
  \item \textbf{nonmetals:} dull, poor conductors of heat/electricity
  \item \textbf{metalloids:} has properties of both metals and nonmetals
\end{itemize}

\subsection{Periodic Table}
\begin{itemize}
  \item \textbf{periodic laws} Properties of elements are periodic functions of their atomic numbers
  \item \textbf{Periodic table} arranges elements in increasing order of their atomic numbers and groups
    atoms with similar properties in same vertical column
    \begin{itemize}
      \item Periods: seven horizontal rows
      \item Groups: 18 vertical groups
    \end{itemize}
  \item \textbf{Memorize:} Alkaline Metals (1), Alkaline earth metals (2), transition metals(3-11), halogens(17), noble gases(18)
\end{itemize}

\subsection{Ionic Compounds}
\begin{itemize}
  \item Anything with a charge, typically has to do with the electrons in relation with the protons of the element
  \item Lithium, typically give up an electron. That's a \textbf{cation} (positive charge)
  \item Fluorine is negatively charged and typically gains an extra electron \textbf{anion} (negative charge)
  \item Noble gases, very neutral, stable, don't react. Very noble
  \item Every element wants to gain or lose electrons to look as close to their noble gas since they want to be neutral
  \item Going right to left, they gain electrons
  \item Going left to right, they lose electrons
  \item Example of Na, you can lose an electron to look like Ne, or gain 7 to look like Ar
  \item Transition form many different types of ions, no need to memorize those
  \item \textbf{Monoatomic ions} are ions from only one atom
  \item \textbf{Polyatomic ions} electrically charged molecules
  \item Gemstone sapphire is compound of aluminum cations $AI^{3+}$ and oxygen anions ($O^{2-}$)

    AI$^{3+}_2$
\end{itemize}

\subsection{Nomenclature}
\begin{itemize}
  \item Memory some of the common Polyatomic Ions (check slides)
  \item Magnesium Chloride $Mg^{2+}$ and Cl$^{-}$  = $MgCl_{2}$
\end{itemize}

\subsection{Naming Acids}
\begin{itemize}
  \item Binary acids are composed of hydrogen and non-metals
  \item \textbf{Oxyacids} contain hydrogen and an oxyanion (an anion containing a nonmetal and oxygen)
  \item HNO$_3$ (aq) nitric acid
  \item HNO$_2$ nitrous acid
  \item "I \textbf{ate} more and got s\textbf{ic}k " (ate - 2, ic - 3)
\end{itemize}

\newpage

\section{Wednesday Lecture (Oct 15, 2025):\\ Quantum Theory}
Quantum Model of the Atom
\subsection{Intro Quantum Theory}
\begin{itemize}
  \item Begins as Dirac saying that Chemistry is a solved problem.
  \item The only molecule where we have exact measurements for is Hydrogen. Everything else is an approximation
  \item Albert Enstein didn't like quantum theory, the more it progresses, the sillier it looks
\end{itemize}

\subsection{Light \& Waves}
\begin{itemize}
  \item light is a form of electromagnetic radiation
  \item Consists of oscillating and mutually perpendicular electric and magnetic fields.
  \item Electromagntic wave, electricity and magntic field oscillate perpendicular to each other.
    They are uniform in the way they increase and decrease.
  \item Wavelength ($\lambda{}$) is the distance between the waves
  \item Frequency (nu) is the number of wave crests that pass through a stationary point at a given time.
    (Can be measured in Hz)
  \item Visible light is just a small part of that spectrum
\end{itemize}

\subsection{Relationship between energy, wavelength, and frequency}
\begin{itemize}
  \item c = 2.9979 \times $10^8 $ $\frac{m}{s}$

    = \lambda{}$v$

    = [m] [$\frac{1}{s}$] = [$\frac{m}{s}$]
  \item Interesting because this is a constant (well, up for perception)
  \item \lambda{} is inverse to frequency of light
  \item frequency $v$ is related to Energy (E)

    (Below may be wrong)
    $$ E = \lambda{}v \rightarrow{} E = \lambda{} \frac{h}{v} $$
\end{itemize}

\subsection{Quantum Strangeness with Light}

\begin{itemize}
  \item Properties of light, but we must understand waves first
  \item \textbf{Diffraction:} Wave bends around a slit. Encounters when waves crash to each other
  \item Particle Behavior: particles do not diffract, they either pass through or bounce back
  \item It goes through the slit or it doesn't, waves diffract around the slit
  \item When you have a double slit, and send a wave through it. You get diffraction in both slits

    Creating a new wave front, these two waves interact with each other. They interfere in either
    constructive or destructive form.
  \item Double slit experiment makes waves between light and dark, proof that light behaves as a wave
\end{itemize}

\subsection{PhotoElectric Effect}
\begin{itemize}
  \item Einstein had an electric surface
  \item Connected the positive terminal
  \item Incoming light is transferring energy and kicking out electrons
  \item THe rate of electron ejection increases as increasing intensity of light
  \item After the threshold frequency, you start getting electrons, and as that intensity
    increases, you have more electrons kicked out
  \item Photons are being shot out, they shoot at the metal and electrons get kicked out (almost like shooting a wall and drywall comes out)
  \item Proof that light is diffracting and interfering with as a particle.
    This proves that light is a particle.
  \item After photons are shot out at a sufficient frequency,
\end{itemize}

\subsection{Connecting two experiemnts together}
\begin{itemize}
  \item Photoelectric says light is a particle
  \item Double slit says light is a wave
  \item Light can be seen as a wave and particle at once
\end{itemize}

\newpage
\section{Friday: Oct 17$^{th}$ 2025}

\subsection{Light and Photons being a wave and particle at once}
\begin{itemize}
  \item Retake Unit 2, you need to do 70\% of the module. So study for that!
  \item $v_{photons} > v_{threshold} $
  \item KE (energy of electron) = (Energy of Photon) (energy of ball) - (energy to remove electrons; break drywall)
    $$ KE = hv - \phi{} $$
  \item \textbf{Topic of today:} From yesterday, Light behaves as a particle and a wave
  \item Ocean waves are waves, made up of particles (H$_2$O)
    \begin{itemize}
      \item Saying that Photons only behave like a wave because we have a lot of them
    \end{itemize}
  \item Photons are strange themselves, $c$ is constant and they have no mass.

    Is this exclusive to photons?
  \item Lets say we have an electron gun (shoots electrons, they are as weird as photons; Cathode Ray Gun)
    \begin{itemize}
      \item We shoot it through a small opening, then goes into a detector on the other side of the barrier
      \item We can observe the frequency of impacts for each slit individually. Very predictable with a single slit
      \item When you have two slits, the two bumps, combined, will have odd behavior in what is detected.

        The behavior of summing two waves. The frequency goes to the center between two slits
      \item We can detect the interference pattern, areas with large impacts and high impacts.
      \item Interference pattern similar to double slit experiment
    \end{itemize}
  \item Change of experiment, shooting one electron at a time.
    Observe the behavior
    \begin{itemize}
      \item \st{When you shoot one particle at a time, you treat it as a particle}
      \item Nah, turns out that the distribution pattern shows behavior of a wave
      \item Same applies to protons and neutrons
      \item Even if we shoot atoms
      \item We can keep scaling it up and it keeps showing up the same pattern
      \item Somehow, these electrons go through both top and the bottom slit at once.

        So we just need one detector on one slit. Count the amount of electrons we shoot and detect, we know
        how many go through each
      \item Turns out that when you are \textbf{observing} it, \textbf{electrons behave as a particle}
    \end{itemize}
\end{itemize}

\subsection{De Broglie Relation}
$$ \lambda{} = \frac{h}{mv} $$

(Planks constant  / momentum)

\begin{itemize}
  \item Postulated that everything has a wavelength, since we are all made of molecules, protons, electrons, neutrons (atoms).
  \item Therefore we are all waves (Socrates is a pig due to transitive property)
  \item  wavelength of a basebal (0.145kg) traveling at 1453 m/s
    $$ \lambda = \frac{h}{0.145kg \times 1453 m/s} = 3.14 \times 10^{-36}m$$
  \item That is a tiny wave, quite unobservable. The
  \item We will never observe wave-like phenomena for macroscopic objects
\end{itemize}

\subsection{Emission spectra and the Bohr model}
\begin{itemize}
  \item \textbf{Emission spectrum:} electromagnetic radiation emitted due to an atom or molecule
    making a transition from high energy state to a lower energy state.
  \item Every atom or molecule has it's own discrete (bar code). You can measure the light of what bounces off.
  \item Astronomers use this to figure out the elemental composition of celestial beings
  \item "Molecular Bar-Codes"
  \item You can also burn elements and they produce different colors. Unique to each element
  \item \textbf{Rydberg Equation}
    $$ \frac{1}{\lambda} = R (\frac{1}{n^2_1} - \frac{1}{n^2_2} ) $$
  \item Helps predict the line spectrum for each element
  \item Energy levels are discrete, and we can assume this equation holds true
  \item Next week, we will use Schrodinger's equation to figure out how this works
\end{itemize}

\end{document}
