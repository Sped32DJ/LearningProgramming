\documentclass{article}
\title{Chem 001A Notes\\ Week 4}
\author{Danny Topete}
\date{October 20th, 2025}

\begin{document}
\maketitle
\section{Monday Lecture: Oct 20th}
\subsection{Emissions Spectra and the Bohr Model}
\begin{itemize}
  \item We are now struggling with the emission spectrum of atoms to solve that question
  \item Every atom or molecule has a discrete emission spectrum
  \item They are unique to each atom, treat them as a bar code
  \item Then Rydberg Equation developed a formula to predict the emission spectrum of atomic hydrogen:
    $$\frac{1}{\lambda{}} = R (\frac{1 }{n^2_1} - \frac{1}{n^2_2}) $$
  \item Including Rydberg's constant to make it make sense
  \item Postulate Bohr model to figure out the spectrum model
\end{itemize}
\begin{enumerate}
  \item Called hydrogen atoms to only allow certain energy levels,
    called \textbf{stationary states}.
  \item Atom does not radiate energy while in a stationary state. (violation of classical physics)
  \item Atom changes to another stationary state (another orbit) only by obsorbing
    or emitting a photon. Energy of photon (hv)...
\end{enumerate}
\begin{itemize}
  \item There is something about how atoms absorb photon to increase in energy level
  \item When they emit the photon, they decrease a level of energy (as observed in emission spectrum)
  \item Assuming this is actually happens, you can explain the emission spectrum, hence, why it is a postulate
  \item The energy level can be calculated as
    $$ E = const (\frac{1}{n^2_{final}} - \frac{1}{n^2_{init}}) $$
  \item Looks a lot like Rydberg's equation
    $$ \Delta{}E = E_{final} - E_{init} $$
\end{itemize}

\subsection{Limitations of Bohr Model}
\begin{itemize}
  \item Only predicts transition energyies for atoms
    $$ E = -2.18\times{}10^{-18} J \frac{Z}{n^2} $$
  \item This model requires quantum mechanics to fully understand the gaps
\end{itemize}

\subsection{Quant descryption of hydrogen}
\begin{itemize}
  \item Going from Bohr model of "Orbits"
    \rightarrow{} Quantum model of "Orbitals"
  \item F = ma = m $\frac{d^2x}{dt^2}$, "position function" x(t)

    This is a differential equation (I mean, both of them are).
    This  is a time domain
  \item Wave Function, \Psi{}. H$\psi{} = E\psi{} $

    This is a frequency domain
  \item When it comes to these, there is an infinite number of solutions
  \item Orbitals can be identifies using combination of three integers called quantum numbers
  \item Ech orbital has a methamtically defined shape
  \item Each orbital has a probabilistic "location"(?)
  \item $n$ = Principle Quantum Number

    n = 1, 2,3, .., $\inf{}$
  \item $L$ = Angular Momentum quantum number

    L = 0,,1,2,...,n-1
  \item $m_I$ = Magnetic Quantum Number
    -l, -l, + 1, ..., 0, ..., l - 1, l
  \item Go to office hours to solve these differential equations
  \item When it comes for the \# of orbitals, 2l + 1 = \# of orbitals
  \item l = n-1, 0, -l, +l
\end{itemize}

\subsection{Terms and definitions for QUant}
\begin{itemize}
  \item \textbf{Shell}: Given value of principle quantum number $n$. There are $\inf{} $ \# of shells labelled $n$ = 1,2,..,$\inf{}$
  \item \textbf{Subshell:} each shell divided into subshells, determined by angular momentum quantum number $l$.
    For given shell with principle quantum number $n$, there are $n-1$ different subshells.
    Associated with each subshell is a letter designation.
\end{itemize}

\begin{table}[h]
    \centering
    \begin{tabular}{ccc}
        \textbf{(n, l, m$_I$)} & \textbf{Orbital Name} & \textbf{\# of orbitals} \\
        \hline
        (2,0,m$_I$) & 2s & 1 \\
        (2,1,m$_I$) & 2p & 3 \\
        (4,2,m$_I$) & 4d & 5 \\
        (4,3,1) & 4f & 1 \\
    \end{tabular}
    \caption{Orbital Information}
\end{table}

\begin{itemize}
  \item Electrons go down, orbitals are in a square
  \item Electrons have some mass, they have a charge
  \item Electrons also have a spin
  \item We had no idea where mass came from until a few years ago.

    Mass comes from the large hydron collider, and with sufficient energy.
    We get the Higgs field (Higgs boson). Particles interacting with the Higgs field makes mass.
  \item Spin exists, and use these properties to figure out how to put electrons into orbitals
\end{itemize}

\end{document}
