\documentclass{article}
%\usepackage{lewis}

\title{Chem 001A \\ Week 5}
\author{Danny Topete}
\date{10/27/25}

\begin{document}
\maketitle
\section{Monday: Oct 27th}
\begin{itemize}
  \item Testing center now open
  \item Print out gradescope bubble sheet for section 3 exam (this week in discussion)
  \item Midterm exam next week on Friday
  \item Make sure you have two attempts on unit 3 before you take the midterm
  \item Turn in exam booklets; they will be reused
  \item 10-6-4 are unit 3
\end{itemize}

\subsection{Electron Configuration}
\begin{itemize}
  \item \textbf{More practice in electron configuration!}
\end{itemize}

\newpage
\section{Unit 4: Chemical Bonding + Nuclear Geometry}
\begin{enumerate}
  \item Intro to chemical boding
  \item Intro to Lews Structures
  \item Lewis structures: advance topics
  \item Intro to VSEPR
  \item Advance topics: VSEPR
  \item Molecular Shape and Polarity
\end{enumerate}
\begin{itemize}
  \item No more quantum mechanics, besides the number of valence electrons and e$^-$ shell stuff
\end{itemize}

\subsection{Chemical Bonds}
\begin{itemize}
  \item \textbf{Ionic Bonds:} electrostatic attraction which occurs between oppositely charged ions.
  \item \textbf{Covalent Bonds:} chemical bond that involves \textbf{sharing} electrons between atoms
  \item \textbf{Metallic Bonds:} End up with a sea of electrons that moves over the metal.
    Very nice for the properties of metal such as conductivity.
    Electrons can move across the element to conduct electricity.
\end{itemize}

\subsection{Lewis Structures}
\begin{itemize}
  \item AKA electron-dot structure
  \item molecular representation that shows boding (connectivity)
    between atoms of a molecule, and the ione pairs of electrons
    that may exist in the molecule.
\end{itemize}

\subsection{Bonding Interactions in Lewis Model}
\begin{itemize}
  \item Covalent bonds formed by sharing electron pair between two atom msm, represented by a line in Lewis Model
    $$ H - O - H $$
  \item Ionic Bond
\end{itemize}

\newpage
\section{Wednesday Lecture: 10/29/25}
\subsection{Building Lewis Structures}
%    \lewis{C}{}{}{}{4}
\begin{enumerate}
  \item Find total n of valence e$^-$
  \item  Place least electronegative atom in the center
  \item Place two electrons btween every two adjacent atoms to form chemical bonds
  \item Distribute remaining valence e$^-$ as lone pairs to give octets (or duets for H) to as many atoms as possible
  \item If any atom lacks an octet for double or triple bonds as necessary to give them octets. This is done
    by moving lone electron pairs from terminal atoms into the bonding region with the central atom.
\end{enumerate}

\newpage
\section{Friday Lecture: 10/31/25}

\begin{itemize}
  \item Check Gradescope score on mastery 3
  \item There may be an extra quiz on there or something
  \item Do the damn HW
  \item Study for unit 3 and look at solutions
  \item Figure out testing center, sign ups only at the top of the hour
  \item \textbf{Midterm 1 next Friday}
  \item Unit 3 typically has bad scores, (it is quite abstract; do retakes)
  \item Figure out the shroeder equations and the benefits of understanding them from an ODE perspective
\end{itemize}

\subsection{More Lewis Structures}
\begin{itemize}
  \item Resonance
  \item Showing off our hybrid is our average with molecules
  \item Formal charge = (\# of valence e$^-$) - ($\frac{1}{2}$\# bonding e$^-$) - (\# non-bonding e$^-$)
\end{itemize}

\subsection{Formal Chages}
\begin{enumerate}
  \item Sum of all formal charges in neutral model must be zero
  \item Sum of all formal charges in an ion must equal the charge of the ion
  \item Small (or zero) formal charges on individual atoms are better than large ones.
  \item When formal charges cannot be avoided, negative formal charges should reside
    on most electronegative atoms.
\end{enumerate}

\subsection{Lewis Structure: Exceptions to Octet Rule}
\begin{itemize}
  \item Atoms (stealing electrons) -> Free radicals <- Antioxitants (gives electron)
\end{itemize}

\end{document}

