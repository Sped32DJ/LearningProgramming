\documentclass{article}
\title{Chem 001A\\ Week 6}

\author{Danny Topetee}
\date{Nov 3, 2025}

\begin{document}
\maketitle
\section{Monday: Nov 3, 2025}

\begin{itemize}
  \item Midterm practice is up already!
    Bro said it is easier this year
  \item Check up for Unit 3 retakes
  \item Questions from question banks
\end{itemize}

\subsection{Lewis Structures}
\begin{itemize}
  \item All you need to know are octets, valence,
  \item We will be talking of vesper model and geometries
  \item A free radical (reactive oxygen; odd electron species)
  \item Odd number of electrons in their lewis structures \textbf{free radicals}
    \begin{itemize}
      \item Unstable
      \item Highly reactive
    \end{itemize}
  \item You give that extra electron to the least electronegative atom, make it stable as possible
  \item Antioxidants fuel free radicals and prevent mutations and help you stay young.
    Prevents oxidative stress, we can reduce the free radicals in our body and
    fuel their missing electrons with Antioxidants
  \item Antioxidants are so reactive to the point that you buy a product and you have
    it bottled and shipped to you. The Vitamin C already reacted and oxidized.
\end{itemize}

\subsection{Incomplete Octets}
\begin{itemize}
  \item Borom and Beryllium often have incomplete octents
  \item H*-B*-H -H
\end{itemize}

\subsection{Expanded Octets}
\begin{itemize}
  \item Third row and below have d-orbitals
  \item Close in energy of their valence electrons.

    Presence of d-orbitals provices the possibility of accommodating more than 8 valence electrons.
  \item They can hold more than an octet of 8 electrons, this allows the expanded octet
\end{itemize}

\subsection{VESPR: Valence Shell Electron Pair Repulsion Theory}
\begin{itemize}
  \item VESP tells us shape and geometry
  \item Based upon how electron groups propel each other
  \item Maximizing separation!
\end{itemize}

\pagebreak
\section{Discussion}

\subsection{Democritus}

\begin{itemize}
  \item There is something indivisible, there is a stopping point
  \item Nuclear physics says there is something smaller than atoms, but that's something else
  \item Can't divide matter into infinitely cuttable pieces
\end{itemize}

\subsection{Dalton}

\begin{itemize}
  \item MAtter is composed of indivisibl particles called atoms
  \item Compounds are made of atoms
  \item Figured out the ratios of atoms in compound substances
\end{itemize}

\subsection{Thompson}
\begin{itemize}
  \item Cathode Ray Tube
  \item Emitts light using electrons
  \item Creates lights and makes photons with electron beams
  \item With this theory, we are told that
\end{itemize}

\subsection{Rutherford}
\begin{itemize}
  \item Gold Foil experiment
  \item an \alpha{} partiicle is the nucleus of a helium atom,
    Often release by radiation
\end{itemize}

\subsection{Chadwick Beryllium}
\begin{itemize}
  \item \alpha{} particle emitter, shot at a Beryllium plate
  \item Neutrons were discovered
  \item Some particles went through, some some curved
  \item \alpha{} were shot and the neutrons went through
\end{itemize}

\subsection{Bohr model}
\begin{itemize}
  \item Understanding that there are shells
\end{itemize}

\subsection{Quantum Model}
\begin{itemize}
  \item Understanding that this is a probabilistic approach
  \item Where the orbital is most likely to be
\end{itemize}

\begin{enumerate}
  \item n
  \item l
  \item $m_l$
  \item
\end{enumerate}

\pagebreak
\section{Wednesday Lecture}
\subsection{Back to VESPR}
\begin{itemize}
  \item Linear geometry

  maximal seperationo, has 180 degrees of separation
  \item  Trigonal Planar Geometry: Three electron groups

    If they were in a triangle, the are 120 degrees, when all bonds are equivalent

    When bonds are different, lets say a double bond, it is 121.9 degrees away from the single bonds. Meanwhile
    the single bonds are 116.2 degrees away.
  \item Tetrahedral: Four electron groupss

    109.5 degrees, all equal bonds.
    108.25 degrees for Chloromethane, where Cl - C - H, and Cl is the bigger atom
  \item Molecular vs electron geometry
  \item Ammonia would have varying MG and EG because they have a lone pair
  \item Molecular geometry depends on geometrical arrangements of atoms. If there are no lone pairs, then MG = EG
  \item \textbf{Memorize} Molecular and Electron geometries!!
\end{itemize}

\subsection{Advanced Molecular Geometries}
\begin{itemize}
  \item What if we have 5 groups or 6 groups?
  \item 5 groups

    PCl$_5$, trigonal bypyramidal
    These are 90 degrees away from each other
  \item 6 groups

    One loan pair, square pyramidal molecular geometry
\end{itemize}

\subsection{Introduction to Molecular Shape and Polarity}

Molecular Polarity
\begin{itemize}
  \item Pure covalent bond vs polar covalent bond

    The sharing of electrons isn't very equal sometimes
  \item \textbf{Pure Covalent:} equally shared bonds (non-polar)
  \item \textbf{Polar Covalent:} when there are unevenly sharing bonds
  \item How do we know if these units are equally shared?
  \item Electronegativity grows to the top right, decreases
    bottom left. (you will typically be given a chart of electronegativity)
\end{itemize}


\subsection{Bond Polarity and Dipole Moments}
\begin{itemize}
  \item Dipole Moment are vector quantities
  \item Decides the orientation of molecules
  \item If they are non-polar, there is a net zero dipole
  \item polar, they are nonzero
  \item CO$_2$ has non net dipole momemnts because they are linear. It is a non-polar bond
  \item Given a hydrocarbon, they are pure covalent. You can never have a polar molecule with just carbon and hydrogen.
  \item Water is a polar molecule, has two dipole moment vectors, they have a dipole moment
  \item To solve these problems, you need to know which geomtetric configurations cancel their dipole moments
  \item Check to see if the orientation all point the same direction
  \item Trigonal planar has its values cancel each other out. There are some orientations where they don't really cancel
    each other out
\end{itemize}

\end{document}
